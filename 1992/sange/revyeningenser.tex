\documentclass[a4paper,11pt]{article}
\usepackage{revy}
\usepackage[utf8]{inputenc}
\revyyear{1992}
\melody{Alt det som ingen ser}
\title{Revyen, som ingen ser}

\version{1.0}

\author{?}

\begin
{document}
\maketitle

\begin{song}
Min fyr blev sur, han er gået hjem
Så skal du se revyen helt alene?
Jeg kender ham, han syn's den er for slem
Han er s'gu ked'lig sku' jeg mene!
Hør, sig mig tror du jeg vil høre på, at du sviner fyren til?
Hvad vil du med den drengerøv, han bruger aft'nen på computerspil?

\bigskip

Revyen han ikke ser, har han ikke ondt af.
-- nej, tak, jeg bli'r tilbage!
Det, som han ikke ser, det kan ikke komme ham særligt ved!
ALT det, han ikke ser, har han ikke ondt af
Ah, et kvarter er okay
Det fjols er til computerspil -- hvad ved han om den ægte kærlighed?

\bigskip

Hva' si'r du? -- er revyen noget for dig?
Jeg mener: man ku' gør' så meget andet!
Det frister -- men er det mon klogt af mig,
at droppe grinet og selv fald' i vandet?
Vi går en tur rundt på {\sc diku}s gange for at finde os et sted,
hvor vi har fred for computerfreaks, og hvor der' tid til kærlighed!

\bigskip

Revyen, man ikke ser, har man ikke ondt af
-- Okay, jeg tabte slaget, og de fjolser er til sex og pil, hvad ved
de om den ægte kærlighed?

\bigskip

Hvad gør vi nu\dots min fyr har set os gå fra festen?
Jo, for resten,
jeg ved et sted, hvor 'ingen undres
Kom nu med mig -- ta' og skynd dig!
Revyen starter -- kom med og grin!
Smart tænkt af dig
Så kan vi sæt' os blandt de fulde svin, der ikke sanser dig og mig!

\bigskip

For alt det, som ingen ser, det har ingen ondt af!
Nåh, ja, det er vel okay
Revyen, som alle ser, den kan sagtens komme de andre ved!
Revyen, som alle ser, den har ingen ondt af!
Tys-tys, revyen starter
Revyen som alle ser, skjuler vores ægte kærlighed

\end{song}
\end{document}
