\documentclass[a4paper,11pt]{article}
\usepackage{revy}
\usepackage[utf8]{inputenc}
\revyyear{1992}

\begin{document}

\title{Grundlæggende datalogi: ``BASICviden''}

\author{Jacob, Kristoffer, Gnyf, Ole}

\version{1.0}


\maketitle

\begin{roles}

\role{Interviewer}

\role{Professor Grydeske}



Naur-type med et halvt ton salt.

\role{Mange andre optrændende, bl.a.:}

\role{CPU} Centralenheden. En bogholdertype, der evt. skifter tøj
når operativsystemet skifter kontekst.

\role{RAM}   Indre lager. Lagerarbejder.

\role{Skærm, Printer, Modem, Knap (tastatur)} Ydre enheder, hver med
specielle karakteristika (se nedenfor). Alle skal have en dims at slå
CPU i hovedet med (afbrydelser). Dimsen (evt. en skumgummihammer?)
skal have en størrelse, der afhænger af enhedens prioritet. D.v.s.
Knap,Skærm,RAM,Printer,Modem.

\end{roles}

\begin{sketch}

\says{Interviewer} I serien DR derinde er vi nu i professor
Grydeskes kolonihave for at høre mere om det indre af en datamaskine.
Professor Grydeske er jo kendt fra sin tid som sekretær for ALGOL-42
gruppen, og er i dag foregangsmand indenfor sproget MODULA-17. Han har
desuden skrevet bøgerne ``EBNF på 117 forskellige måder'' og ``En
konfus undersøgelse af paletknive''. Og professor, hvad kan De
egentlig fortælle os?

\says{Professor} Jo, det centrale i en datakantine er jo
paletkniven. Og det kan jeg fortælle meget om. I min bog skriver jeg
på side 397, at man kan bruge en paletkniv på mange interessante og
spændende måder. Man kan f.eks. tage et egern, og stoppe.......

\says{Interviewer} Jo ja øh, men vi ville jo gerne høre noget om
{\em datamaskiner}!

\says{Professor} {\AA}h undskyld, jeg hørte vist galt. Ja, det
centrale i en {\em datamaskine} er centralenheden, også kaldet en
CPU. Der er 2 forskellige typer af CPU'er. Man skal endelig aldrig
komme til at koge dem sammen, fordi så bliver paletkniven ødelagt.

\says{Interviewer} Ja, og de 2 typer kaldes for RISC og CISC, ikke sandt.

\says{Professor} Ja. Til en portion RISC-CPU'er skal man bruge 5
stk, der skal koge i 1 minut. Når man laver CISC-CPU skal man kun
bruge 1, men den skal koge i 5 minutter. Paletkniven skal endelig ikke
koges med i nogen af tilfældende.

\says{Interviewer} Men er der ellers forskel på de 2 typer?

\says{Professor} RISC-arkitekturen udmærker sig ved at have få,
små og hurtige instruktioner. Sammenlignet med CISC-instruktioner,
der er mange, store og langsomme. Jeg har forberedt et lille eksempel.
Normalt vil det kun tage et mikrosekund, men vi har snydt, og viser
det i langsom gengivelse:

\scene
(5 personer (RISC-instruktioner i sportstøj) kommer løbende ind på
scenen i en lang række, hver af dem udfører en lille ting, og
sammenlagt skal de have hægtet et element ind i en liste)

\says{Professor} Se, det var smart, hvis en CISC-instruktion skulle
gøre dette ville det se således ud:

\scene
(2-3 personer, bundet sammen i et ormeagtigt kostume kommer ind på
scenen. De diskuterer højlydt om hvem der skal gøre hvad med hvem,
og foretager indhægtningen)

\says{Professor} Ja, og som enhver kan se er det meget sjovere at
være RISC-instruktion! Der er dog visse problemer med
RISC-processorer, for på en RISC skal data og selve instruktionerne
være det der hedder alignet.

\says{Interviewer} Alignet? Har det noget med lagner at gøre?

\says{Professor} Nej, det kan man vist ikke sige. Det er nemmest at
vise det med et eksempel. For en RISC-maskine skal det være således:

\scene
(På scenen skal der være en låge. Vi har en RISC-instruktion inde
på scenen, som venter på data. Instruktionen skal være polstret,
så den passer præcis til lågen. Efter 3 sekunder kommer en
lagermand med data til instruktionen. Data er i en kasse, der passer
præcist til lågen.)

\says{Professor} På RISC-maskiner er programmerne altså lavet,
så både instruktioner og data ligger på lige adresser i forhold til
ord-bredden. For en CISC-maskine, derimod, er dette ikke nødvendigt.
Der sker nogenlunde dette:

\scene
(Vi har 1 2-mands CISC-instruktion, der godt kan komme igennem, men de
kommer gående skævt ind på lågen. De skilles; halvdelen går
igennem, den ene beskriver en lille cirkel, og går også igennem. De
samles på den anden side af lågen. De venter på lidt data. En
lagerarbejder kommer med en kasse. Kassen skal være delbar, og have
samme størrelse som den første kasse. Kassen kommer skævt ind på
havelågen; det er derfor nødvendigt at dele kassen for at få data
ind på scenen)

\says{Interviewer} Men professor Grydeske, tror De ikke, at dette er
for avanceret for seerne? Kan De ikke forklare om hvordan en
datamaskine fungerer på et lidt højere niveau?

\says{Professor} Det kan da godt være De har ret. Men jeg synes nu,
det er meget simpelt --- og let at forstå. På et højere niveau har
datamaskinen et såkaldt operativsystem. Det er et system, der styrer,
hvordan maskinen opererer --- heraf navnet. Og i den forbindelse vil
jeg gerne anbefale knive fra Raadvad --- de kan dels bruges i en
datakantinge, men også til at foretage forskellige operationer.
F.eks. har jeg udført plastisk kirurgi på min kone, og det hjalp
sgu. Se, jeg startede med at lægge et snit $\ldots$

\says{Interviewer}[Afbryder] Ja, men hvordan fungerer sådan et
operativsystem? 

\says{Professor} Ja, vi kan betragte et lille eksempel. Det kræver en
centralenhed (en såkaldt CPU), det indre lager (kaldet RAM) --- det
kommer egentlig først ved årsskiftet, men vi har snydt lidt; et
tastatur (knap og nap) og en skærm.

\scene
(Når professoren nævner en enhed, kommer den på scenen. Skilte
på maven?)

Et bord mærket CPU og et andet lige ved siden af mærket RAM;
ved hvert bord sidder den pågældende enhed. Ved CPU-bordets
anden side er et skilt, hvor der står "Afbrydelser" med plads
til at hænge et ekstra skilt; fra starten står der "Lukket".
Rundt omkring står de ydre enheder med papskilte på. De
sover. Bagved er et pengeskab.

CPU tager støvlerne af, vender derefter skiltet under afbrydelser så
der nu står "åben". CPU sætter sig ned; de ydre enheder sover.


\says{CPU} Er jeg igang med noget?
\says{RAM} Ja: kommandofortolkeren.
\says{CPU} Javel! (læser lidt) TASTATUR!
\scene 
(TASTATUR vågner, begynder at rende rundt blandt publikum for at
blive trykket på.)

\says{CPU}[afleverer kommandofortolkerbunken til RAM] Er der andet?
\says{RAM} Her: din orm. Den trænger til at bevæge sig lidt.
\says{CPU}[læser lidt. Til RAM] Stik mig lige 0xFFED5546...
\says{RAM} øjeblik, det kræver lige fire cykelklokker 
\says{RAM}[ringer fire gange på en cykelklokke] Her {\em (rækker
en stor * til CPU).}
\says{CPU} Hovsa en asterix ... hvem skal den mon sendes til?
\says{CPU}[vender *en om, hvor der er frimærke og modtager på] Aha! SKÆRM??
\says{SKÆRM}[vågner, gaber, etc. i et stykke tid endnu]
\says{CPU}[afleverer ormen til RAM] Er der mere?
\says{RAM} Nix.
\says{CPU}[fremdrager Playboy og smider stængerne på bordet]
\says{SKÆRM}[kommer hen til CPU og afbryder CPU] Var der noget,
      der sku' vises?
\says{CPU}[lægger sirligt etc., og rækker *en til SKæRM.
Genoptager læsningen]
\says{SKÆRM} Tak. 
\says{SKæRM}[forlader CPU, går rundt blandt publikum og viser
      *en frem]
\says{RAM} Ahem! Hm. øhm. 
\says{CPU} Er du der nu igen.....
\says{RAM} Ja, jeg har en lille primtalsørken, der gerne vil være stor...
\says{RAM}[Rækker processen til CPU]
\says{CPU}[Sukker] Man har da heller aldrig fred til at udvide sin horisont.
\says{CPU}[Lægger Playboy sirligt væk, og læser i stedet for
primtalsprocessen] Suk. Nåh, hvor var det jeg gjorde af min kugleramme?
\says{CPU}[Roder rundt på sit bord, indtil kuglerammen dukker frem
af rodet] Lad mig nu se: 2 op, og 5 i mente. Det giver ..... 8. øh...
Er 8 et primtal? Nej, det kan ikke passe. 4 $\times$ 2 det er 4 + 4,
det er 8. Nej, 8 er ikke et primtal. Nåh, fejl i udregningen,
forfra.. 5 op, og 2 i mente. Det giver .... 41. Det var vist bedre.
Nåh, det må være nok tid til den proces.
\says{CPU}[Afleverer bunken til RAM. Finder Playboy frem o.s.v.]
\says{RAM}[Efter en lille pause prikker RAM CPU på skulderen] Det
er altså dig, der skal lave noget. Det kan ikke være meningen, at
jeg skal holde styr på dig hele tiden. Her: din orm.
\says{CPU}[Læser surt i ormeprocesbunken. Siger til RAM] Stik mig
0xFFED5546... 
\says{RAM} øjeblik, det kræver lige 4 cykelklokker.
\says{RAM}[Ringer fire gange på en cykelklokke] Her {\em (rækker
en stor * til CPU).}
\says{CPU} Det er jo nok til skærmen igen.... SKæRM! {\em
(fremdrager Playboy)} 
\says{SKÆRM}[Vender tilbage til scenen, og afbryder CPU] Har du
mere til mig?
\says{CPU}[lægger sirligt etc., og rækker *en til SKÆRM.
Genoptager læsningen]
\says{SKÆRM} Tak. 
\says{SKÆRM}[forlader CPU, går rundt blandt publikum og viser
      *erne frem] This line intentionally left blank

\scene
(Når nogen trykker på tastaturet, render TASTATUR op på scenen, og
afbryder CPU. TASTATUR stiller sig i en evt. kø ved CPU, og afbryder:
TASTATUR slår CPU i hovedet med sin dims)

\says{CPU}[lægger sirligt Playboy til side o.s.v.] Ja? Er der noget?

\says{TASTATUR}[Afleverer en seddel til CPU] Jeg er blevet trykket
på maven. Her!

\says{CPU}[Læser sedlen] Hm, Hm, Hm....

\says{TASTATUR}[Render ud blandt publikum o.s.v.]

\says{CPU}[Afleverer seddel til RAM, og siger] Stik mig lige
kommandofortolkeren.

\says{RAM} Her!

\says{CPU}[Læser noget i stakken af kommandoer] Lad mig lige få
tegnet igen!

\says{RAM}[Afleverer tegnet]

\says{CPU}[Putter tegnet ind i bunken, og afleverer
kommandofortolkeren til RAM igen]

\says{Professor} Ja, og det bliver bare gentaget og gentaget og
gentaget og gentaget og gentaget og gentaget og gentaget og gentaget
og gentaget og gentaget og gentaget og $\ldots$

\scene 
(Lyset dæmpes omkring det 2. gentaget. Professoren dæmper stemmen i
takt med lyset....)

\end{sketch}

\end{document}
