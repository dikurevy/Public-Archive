\documentclass{article}
\usepackage[utf8]{inputenc}
\usepackage{revy}

\title{Mundtlig eksamen i brugergrænseflader}

\author{Theo Engell-Nielsen}

\version{1.2}


\advance\textwidth by 1cm
\advance\textheight by 1cm
\begin{document}
\maketitle\vspace{-2em}
\begin {roles}
\role {Erik Frøkjær, censor, studerende}\
\end {roles}\vspace{-1em}
\begin{props}
  \prop{Papirer} til lærer og censor --- noter m.m.\
  \prop{Sort lommebog} Frøkærs den lille sorte 
  \prop{Kort} Til at trække eksamensnummer --- (bla. nummer 13).
  \prop{1 overhead} M.e.S.S-logoet med en lille frø. 
\end{props}\vspace{-2em}
\begin{sketch}

  \scene Eksamenssituation. Et stort bord (2 borde sat sammen). Censor
  sidder bag bordet, med siden til publikum. Ind kommer den studerende.
  Erik Frøkjær står parat, så den studerende kan vælge et spørgsmål.

\says{Erik} Du skal bare sige til hvis du er nervøs.

\says{Stud}[Promte] Jeg er nervøs! {\em (Stort smøret grin)}

\says{Erik} Det har du også grund til {\em (Den studerendes grin forsvinder, og nu griner Erik smøret)}

\says{Erik}[atter alvolig] Du skal trække et spørgsmål. {\em (tager et
  kortspil frem og folder kortene ud. Han holder rigtig godt fast i dem
  alle, pånær ét kort der stikker langt frem)}

\says{Stud}[prøver at trække et tilfældigt kort, men må til sidst trække
kortet der stikker langt frem] Det er nummer 13.

\says{Erik} Naej --- så trak du jo spørgsmålet om M.e.S.S. --- {\em
(Reklamemanden kommer frem i ham)} Mit eget Smaddergode Søgeystem!

\says{Studerende} [Sukker] Hvad gør jeg nu?

\scene (Erik smidder en slide på overheaden, hvor der er et X-term-billede
med MeSSs forside med en frø på\ldots)

\says{Erik} Mit første spørgsmål til dig er: Kan man kalde denne her frø
kær?

\says{Studerende} Eh??????

\says{Erik} (ny slide, reklamemanden igen) Kan du fremhæve mindst tre
virkelig gode egenskaber ved Mit eget Smaddergode Søgesystem, MeSS?

\says{Studerende} Næh, det kan jeg ikke!

\says{Erik} Hvad med to?

\says{Studerende} [opgivende] Puha... næh

\says{Erik} \'En så, det kan du da vel?

\says{Studerende} [sukkende] Programmet har en knap der hedder ''Done''\ldots

\says{Erik} [Råber ligesom Per Viking fra Kvit eller Dobbelt] Jamen, er det
ikke fantastisk!

\says{Studerende} Burde den ikke havde heddet ''AFSLUT''?

\says{Censor} [Afbryder] Måske skulle vi vende {\em vores} spørgsmål om,
og sige ''Kan du finde nogle uhensigtsmæssigheder ved MeSS, Mit Eget
Smaddergode Søgesystem!??''

\says{Studerende} Tjaeh, der er jo masser af fejl! Ja, hvor er der ikke
fejl?? Er det i det hele taget ikke bare \'en stor uhensigtsmæssighed??
Jeg tænker også som eksamenspørgsmål, ikke?

\says{Censor} Jo --- du har en pointe der!

\says{Erik} [Snerrende, uden at censor kan høre det] Nu skal jeg sige dig
noget din enfame lille pesticide af en studerende. Det er DIG, der er til
eksamen!!!!! {\em (Højt)\/} Vi går nu over til
almen-viden-spørgsmålene\ldots

\says{Studerende} Åh nej!

\says{Erik} Vær ikke urolig\ldots Hvilke metaforiske begreber ligger til grund
for ordene DynaFuk og MimreMex?

\says{Studerende} Det ved jeg sgu ikke!

\says{Erik} Der er ingen! HA HA HA!!! {\em it's a bloody trickquestion!!!!}
Næste spørgsmål: Hvad hedder skridt på fransk?

\says{Studerende} [måbende] Hva? Hvad mener du?\ldots {\em (opgivende)\/} Hvad
var det nu, det var?? {\em (lyser op)\/} PAS!!

\says{Erik} Det var utroligt, kunne du ikke engang det! Nåh: Hvad hedder
nedsænkningen mellem to bjergrygge?

\says{Studerende} [overrasket og alligvel måbende] Pas!

\says{Erik} Korrekt! Hvad hedder den mest almindelige melding i kortspillet
Whist?

\says{Studerende} [mere overrasket og alligvel måbende] Pas!

\says{Erik} Korrekt! Hvad skal man have med, når man f.eks. skal fra
Danmark til U.S.A?

\says{Studerende} [mest overrasket og alligvel måbende] Pas!

\says{Erik} Korrekt! Hvad hedder standardudvidelsen på pascalprogrammer under 
MS-DOS?

\says{Studerende} [allermest overrasket og alligvel måbende] Pas!

\says{Erik} Korrekt! Jamen er det ikke FANTASTISK!!!! 

\scene (Erik kigger på censor --- Og så på den studerende)

\says{Erik} Jøh --- det bliver vidst et 6-tal!

\says{Studerende} Hvorfor det, jeg kunne jo svare på over 55\% af
spørgsmålene! 

\scene (Scenelys dør ud)

\end{sketch}

\end{document}

