\documentclass{article}
\usepackage[utf8]{inputenc}
\usepackage{revy}
\title{Mænd og deres søgen efter stammekulturen}
\version{1.2}
\author{Anne, Lykke og gnyf efter en idé af Bo Arlif}

\begin{document}
\maketitle

\begin{roles}
\role{En professor} Med hvidslidte cowboybukser og tennissokker der først ses til sidst
\role{En yuppie} Med mobiltelefon
\role{En nørd} Har briller med, som han tager på senere.
\end{roles}

\begin{props}
\prop{En talerstol, en overheadprojektor}\strut
\prop{En overhead} med ordene: ``H.Omo
  S.Apiens --- Cand.med.link og Cand.med.array''.
\prop{En pegepind}
\end{props}

\begin{sketch}
\scene
På scenen står en talerstol. Ved siden af talestolen er en overhead.
Ind kommer en professor-type $<$humanist-klædt med batik-trøje \`a la Henrik
Jeppesen$>$.

\says{Professor} Godaften --- mine Herrer\dots og den dame. Det er mig en
stor ære og fornøjelse at præsentere mig selv: {\em (kunstpause)} Mit navn
er\dots Professor Hans Omo Severin Apiens Cand.med.link mm.\ {\em (ligger
  overheaden på)}

Jeg er i aften kommet på foranledning af formanden for foreningen for
fremme af kendskabet til Danmarks Internationale Kollegie for Ungkarle ---
i daglig tale kaldet DIKU. Temaet for denne så velbesøgte aftenforelæsning
er ``Mænd og deres søgen efter stammekulturen''!.

Det er interessant at observere hvordan mange mænd i høj grad og helt
automatisk søger tilbage til en fælles stammekultur, {\em hvis (benytter
  pegefingeren for at understrege vigtigheden)} de bliver overladt til sig
selv uden de snerende feminine bånd og ugebladenes mandeidealer. Dette er i
praksis afprøvet her på DIKU igennem en længere periode. Man kan se hvordan
de såkaldte Yuppier --- som jo er hårdt præget af den ligelige
kønsfordeling i gymnasiet --- hurtigt ændrer sig til {\em rigtige}
stammemænd. Eller kort sagt: Hvordan de kommer ud af pressefolderne og
tilbage til tennissokkerne!.

Lad mig illustrere: En nyimmatrikuleret. Friskpresset fra gymnasiet.

\scene Ind kommer en Yuppie med en mappe i venstre hånd ---
professoren begynder at pege med en pegepind. Han starter med skoene


\says{Professor} Man ser tydelig hvordan den blankpolerede overflade
afspejler den kvindelige narcissisme, som jo er trangen til at betragte sig
selv i et spejl. Som et korset strammer skoene --- ligeledes ses hvordan de
nystrøgede sokkers bedemandsfarve fuldstændigt er med til at undertrykke
mandens udfoldelsesmuligheder\dots Pressefolderne i bukserne viser tydelig
hvordan manden er foldet sammen til en ubetydelighed {\em (yuppien --- som
  er helt slået ud folder hænderne incl. mappen foran skidtet.
  Kunstpause)}. Slipset signalerer hvordan manden er evigt bundet til
kvinden.  Størrelsen af mandens udfoldelsesmuligheder er jo symboliseret
ved denne lille {\em (smager på ordet)} 'stresscontainer'. {\em (Yuppiens
  mobiltelefon ringer)} I inderlommen ses\dots

\says{Yuppie}[tager forvirret telefonen] Ja \dots jo Sasja --- jeg kommer
hjem nu! {\em (løber ud af sce\-nen, mens professoren kigger lidt undrende
  efter ham)}

\says{Professor} Hmmm --- dette tilsyndeladende håbløse tilfælde er dog
ikke helt fortabt.

Allerede på første år vender de ``back to BASIC'' --- de lærer om træer og
kommer på denne måde tættere på deres rødder. De nye religioner mister
hurtigt deres fascination og pludselig kommer de gamle nordiske guder igen
i højsædet. Den rå vikingetid vender tilbage under de natlige sceancer i
HEL, hvor mænd igen gerne må lugte af mand og ozon!

Efter nogle år blandt ligesindede kan selv en hærdet yuppie fremstå
som\dots 

\scene Ind kommer nørden

\says{Professor}Her har vi et {\em ægte} urmenneske. {\em (igen begynder
  han at pege --- først på skoene)} Man ser tydeligt hvordan de udtrådte
sko giver rig udluftning og mulighed for at sokkernes mandlige duft kan
udbrede sig i lokalet {\em (snuser i nærheden af nørden, og holder derefter
  hurtigt på næsen)}. De hvide tennissokker fungerer som et lærred hvorpå
tilværelsens udfoldelser kan sætte sig spor {\em (Nørden begynder at vende
  sig om. I samme bevægelse tager han brillerne på uden at publikum ser det)}. Buksernes slidte
bagside viser i hvor høj grad individet er faldet til i gruppen og denne
hvide flade anerkendes indenfor stammekulturen som et statussymbol: Jo
hvidere, jo bedre. Stammens kendetegn er tydeligt i ansigtet {\em (nørden
  vender sig igen, så man kan se brillerne, som professoren peger på)},
hvor den rå tilværelse sætter sig spor!.

Mænds fremgang i livet hænger altså sammen med deres evne til at finde
sammen i mandestammer. Jo værre de lugter, jo større succes. Dette
dokumenteres af DIKU's storstillede stammeforsøg {\em (snuser ud blandt
  publikum, og holder hurtigt på næsen)}, som I i aften alle er en del af.

Tak for deres opmærksomhed! {\em (går ud arm i arm med nørden, hvor det
  bliver tydeligt at professoren har hvide tennissokker på (evt.\ spot?))}
  
\end{sketch}

\end{document}

