\documentclass{article}
\usepackage[utf8]{inputenc}
\usepackage{revy}
\title{Grundlæggende datalogi: ``BASICviden''}
\author{Jacob, Kristoffer, Gnyf, Ole, Jørgen}
\version{1.3}
\begin{document}
\maketitle

\begin{roles}

\role{Interviewer}

\role{Professor Grydeske}

Naur-type med et halvt ton salt.

\role{Mange andre optrændende, bl.a.:}

\role{CPU} Centralenheden. En bogholdertype, der evt. skifter tøj
når operativsystemet skifter kontekst.

\role{RAM}   Indre lager. Lagerarbejder.

\role{Skærm, Printer, Modem, Knap (tastatur)} Ydre enheder, hver
med deres spe\-ci\-el\-le  ka\-rak\-te\-ri\-sti\-ka (se
nedenfor). Alle skal have en dims at slå CPU i ho\-ve\-det med
(afbrydelser). Dimsen (evt.  en skum\-gum\-mi\-ham\-mer?)  skal have en
størrelse, der afhænger af enhedens pri\-o\-ri\-tet. Dvs.\
Knap, Skærm, RAM, Printer, Modem.

\role{Rekvisitter:}
  Skilte med {\Large$*$}, egern, evt. reol, dueslag el. lign.
  2 øl, et øljern.

\end{roles}

{\sl\noindent(Starten skal nok ændres fra ``I serien VR
  derinde\ldots'' til noget mere relevant - ellers er den som
  før.)}
\begin{sketch}
\says{Interviewer} I serien DR derinde er vi nu i professor
Grydeskes kolonihave for at høre mere om det indre af en datamaskine.
Professor Grydeske er jo kendt fra sin tid som sekretær for ALGOL-42
gruppen, og er i dag foregangsmand indenfor sproget MODULA-17. Han har
desuden skrevet bøgerne ``EBNF på 117 forskellige måder'' og ``En
konfus undersøgelse af paletknive''. Og professor, hvad kan De
egentlig fortælle os?

\says{Professor} Jo, det centrale i en datakantine er jo
paletkniven. Og det kan jeg fortælle meget om. I min bog skriver jeg
på side 397, at man kan bruge en paletkniv på mange interessante og
spændende måder. Man kan f.eks. tage et egern, og stoppe\ldots

\says{Interviewer} Jo ja øh, men vi ville jo gerne høre noget om
{\em datamaskiner}!

\says{Professor} {\AA}h undskyld, jeg hørte vist galt. Ja, det
centrale i en {\em datamaskine} er centralenheden, også kaldet en
CPU. Der er 2 forskellige typer af CPU'er. Man skal endelig aldrig
komme til at koge dem sammen, fordi så bliver paletkniven ødelagt.

\says{Interviewer} Ja, og de 2 typer kaldes for RISC og CISC, ikke sandt.

\says{Professor} Ja. Til en portion RISC-CPU'er skal man bruge 5
stk, der skal koge i 1 minut. Når man laver CISC-CPU skal man kun
bruge 1, men den skal koge i 5 minutter. Paletkniven skal endelig ikke
koges med i nogen af tilfældende.

\says{Interviewer} Men er der ellers forskel på de 2 typer?

\says{Professor} RISC-arkitekturen udmærker sig ved at have få,
små og hurtige instruktioner. Sammenlignet med CISC-instruktioner,
der er mange, store og langsomme. Jeg har forberedt et lille eksempel.
Normalt vil det kun tage et mikrosekund, men vi har snydt, og viser
det i langsom gengivelse:


\scene (5 RISC-instruktioner kommer ind på scenen i en række. Den
første stiller 1 øl på et bord, den næste lægger et øljern, den
tredje åbner den, den fjerde går smilende hen og drikker den og den
sidste bærer den ud)

\says{Professor} Se, det var smart. Hvis en CISC-processor skulle gøre det
samme, ville det se sådan her ud:

\scene (3 CISC-instruktion-folk kommer ind på scenen med et lagen el.\
lign.\  over sig. Kun to arme er frie (dem i enderne, naturligvis).
Den venstre arm holder en øl, den højre et øljern. Øllen åbnes og
Midten strækker hals og Højre rækker ud efter øllen, ingen af dem når
den.  Venstre smiler stygt og drikker øllen. De går ud.)

\says{Professor} Ja, som enhver kan se, er det meget sjovere at være
en RISC-instruktion.  Der er dog visse problemer med
RISC-processorer, bl.a.\ kræver de flere instruktioner og dermed mere
lagerplads. Lad mig vise et eksempel\ldots

\says{Interviewer}[afbryder] Men professor Grydeske, tror De ikke,
at dette er for avanceret for seerne? Kan De ikke forklare om
hvordan en datamaskine fungerer på et lidt højere niveau?

\says{Professor} Det kan da godt være De har ret. Men jeg synes nu,
det er meget simpelt --- og let at forstå. På et højere niveau har
datamaskinen et såkaldt operativsystem. Det er et system, der styrer,
hvordan maskinen opererer --- heraf navnet. Og i den forbindelse vil
jeg gerne anbefale knive fra Raadvad --- de kan dels bruges i en
datakantinge, men også til at foretage forskellige operationer.
F.eks. har jeg udført plastisk kirurgi på min kone, og det hjalp
sgu. Se, jeg startede med at lægge et snit $\ldots$

\says{Interviewer}[Afbryder] Ja, men hvordan fungerer sådan et
operativsystem? 

\says{Professor} Ja, vi kan betragte et lille eksempel. Det kræver en
centralenhed (en såkaldt CPU), det indre lager (kaldet RAM) --- det
kommer egentlig først ved årsskiftet, men vi har snydt lidt; et
tastatur (knap og nap) og en skærm.

\scene
(Når professoren nævner en enhed, kommer den på scenen. Skilte
på maven?)

Et bord mærket CPU og et andet lige ved siden af mærket RAM;
ved hvert bord sidder den pågældende enhed. Ved CPU-bordets
anden side er et skilt, hvor der står "Afbrydelser" med plads
til at hænge et ekstra skilt; fra starten står der "Lukket".
Rundt omkring står de ydre enheder med papskilte på. De
sover. Bagved er et pengeskab.

CPU tager støvlerne af, vender derefter skiltet under afbrydelser så
der nu står "åben". CPU sætter sig ned; de ydre enheder sover.


\says{CPU} Er jeg igang med noget?
\says{RAM} Ja: kommandofortolkeren.
\says{CPU} Javel! {\em (læser lidt)} TASTATUR!
\scene 
(TASTATUR vågner, begynder at rende rundt blandt publikum for at
blive trykket på.)

\says{CPU}[Afleverer kommandofortolkerbunken til RAM] Er der andet?
\says{RAM} Her: din orm. Den trænger til at bevæge sig lidt.
\says{CPU}[læser lidt. Til RAM] Stik mig lige 0xFFED5546...
\says{RAM} Øjeblik, det kræver lige fire cykelklokker 
\says{RAM}[Ringer fire gange på en cykelklokke] Her {\em (rækker
en stor * til CPU).}
\says{CPU} Hovsa en asterix ... hvem skal den mon sendes til?
\says{CPU}[Vender *en om, hvor der er frimærke og modtager på] Aha! SKÆRM??
\says{SKÆRM}[Vågner, gaber, etc. i et stykke tid endnu]
\says{CPU}[Afleverer ormen til RAM] Er der mere?
\says{RAM} Nix.
\says{CPU}[fremdrager playboy og smider stængerne på bordet]
\says{SKÆRM}[kommer hen til CPU og afbryder CPU] Var der noget,
      der sku' vises?
\says{CPU}[lægger sirligt etc., og rækker *en til SKÆRM.
Genoptager læsningen] 
\says{SKÆRM} Tak. 
\says{SKÆRM}[forlader CPU, går rundt blandt publikum og viser
      *en frem]
\says{RAM} Ahem! Hm. Øhm. 
\says{CPU} Er du der nu igen.....
\says{RAM} Ja, jeg har en lille primtalsørken, der gerne vil være stor...
\says{RAM}[Rækker processen til CPU]
\says{CPU}[Sukker] Man har da heller aldrig fred til at udvide sin horisont.
\says{CPU}[Lægger Playboy sirligt væk, og læser i stedet for
primtalsprocessen] Suk. Nåh, hvor var det jeg gjorde af min kugleramme?

\says{CPU}[Roder rundt på sit bord, indtil kuglerammen dukker frem
af rodet] Lad mig nu se: 2 op, og 5 i mente. Det giver ..... 8.
Øh...  Er 8 et primtal? Nej, det kan ikke passe. 4 $\times$ 2 det
er 4 + 4, det er 8. Nej, 8 er ikke et primtal. Nåh, fejl i
udregningen, forfra.. 5 op, og 2 i mente. Det giver .... 41. Det
var vist bedre.  Nåh, det må være nok tid til den proces.
\says{CPU}[afleverer bunken til RAM. finder playboy frem o.s.v.]

\says{RAM } Her er den næste proces {\em(rækker papir)}.

\says{CPU }[Læser papiret] Hmm\ldots O.K. Stik mig 0xFFAE007 !

\says{RAM } Øjeblik {\em(Fire ringeklokker) (Rækker et egern med
  selvfølgelig mine)} Værs'go!

\says{CPU } Tak ! {\em (undrer sig)} Hmm.  hvad skal jeg\ldots

\says{SKÆRM}[Vendt tilbage, afbryder CPUen] Det var det, har du mere
? {\em(Smiler snedigt til publikum)}

\says{CPU }[Klør sig i hovedet] Jahhh, jeg har et egern {\em(vender sig
mod RAM)} Hvad skal vi med det ???

\says{RAM }[Foreslående] Det skal vel koges\ldots

\says{CPU } O.K. {\em (mod publikum)} Er der en PALETKNIV til stede ???
{\em(Venter)}

\says{RAM }[Efter et stykke tid] Nææhhh, åbenbart ikke!

\says{CPU }[Frustreret] Det' bare for meget! Nå! Nu går jeg
fan'me ned!  {\em(Går ned af scenen, råber imens)} Hvem ind i helvede
har programmeret det lort??

\says{Interviewer }[Vender sig mod Grydeske, skadefro] Nååh, det gik
vist ikke så godt, hvad?

\says{Professor}[Ophidset] Nej, det tror da fanden --- egernet skal
slet ikke koges, det skal hakkes ! HARDWAREFEJL, HARDWAREFEJL,
HARDWAREFEJL\ldots

\says{Interviewer}[Skæver nervøst til Grydeske] Slut herfra, tilbage
til (void)...


\scene TÆPPE.


\end{sketch}

\end{document}


% Local Variables: 
% mode: latex
% TeX-master: t
% End: 
