\documentclass[11pt,danish]{article}
\usepackage{revy}
\usepackage[utf8]{inputenc}
\usepackage{babel}
\usepackage{anysize}

\title{Nyrup's got IT}
\author{lyst, men ej blåt til lyst (don't \textsf{panic})}

\version{0.9 -- færdig} % HUSK AT AJOURFØRE VERSIONSNUMMER!!

\revyyear{1998}

\begin{document}
\maketitle

\begin{roles}
  \role{Nyrup} med laang næse, stor pande, Nyrup-briller,
  Nyrup-leverpostejs-hår, med cykelhjelm, topmave, kedeligt jakkesæt,
  osv. osv.
  \role{Margrethe Vestager} i Margrethe-Vestager-tøj, med en vest
  yderst
  \role{Jan Trøjborg} i Jan-Trøjborg-tøj
  \role{Folkeskolelærer} i striktrøje, fuldskæg og fodformede, eller
  hvad ved jeg
  \role{Forsker} i hvid kittel
  \role{Præst} i præstetøj (kan evt. spilles af lærer-skuespilleren)
\end{roles}
\begin{props}
  \prop{talerstol} til Nyrup
  \prop{papkasser} til universitetsmodeller. Bagpå laves en lille
  hylde/holder til det der skal tages op
  \prop{4 5kkr-sedler} (karton) i stor størrelse (hvad med EURO i stedet???)
  \prop{4 1kr-mønter} (pap) i stor størrelse
  \prop{papdukke} (mandsilhouet) med blinkende pære i stedet for hovede
  \prop{papdukke} med slukket pære i stedet for hovede
  \prop{papdukke} uden hovede
  \prop{papdukke} i engle-form
  \prop{paphuse}
  \prop{diverse skilte} med ``År 2000 Problem -- LØST,'' ``WIN95,''
  ``COBOL,'' ``PC-klon''(overstreget) osv.
  \prop{mikrofoner} en milliard forskellige, med ``TV2'' ``DR1''\ldots 
  skilte på
\end{props}

\begin{sketch}

\scene Nyrup har kaldt til pressemøde og står nu bag talerstolen og
kigger ud over salen. Der er stillet en milliard mikrofoner op foran ham

Til højre for ham står Jan, og længere til højre Margrethe. Jan \&
Margrethe holder en papkassemodel af universitetet mellem sig (den
stiplede streg angiver hvor kassen er delt):
\begin{center}\upshape \unitlength 5mm
\begin{picture}(22,7)(-1,0)
  \put(-1, 4){\line(1,0){22}}
  \put( 0, 0){\line(1,0){20}}
  \put( 0, 0){\line(0,1){4}}
  \put(20, 0){\line(0,1){4}}
  \put(-1, 4){\line(1,1){3}}
  \put( 2, 7){\line(1,0){16}}
  \put(18, 7){\line(1,-1){3}}
  \multiput(10, 7)(-.1,-.2){10}{\makebox(0,0){.}}
  \multiput( 9, 5)(.2,-.3){10}{\makebox(0,0){.}}
  \multiput(11, 2)(-.1,-.2){10}{\makebox(0,0){.}}
  \put(9  ,5.5){\makebox(0,0)[r]{{\LARGE\textbf{\textsf{U}}}}}
  \put(9.5,5.5){\makebox(0,0)[l]{{\LARGE\textbf{\textsf{N\, I}}}}}
  \put(10.5, 2){\makebox(0,0)[rb]{{\Large\strut Forsknings-}}}
  \put(10.75, 2){\makebox(0,0)[lb]{{\Large\strut baseret}}}
  \put(11  , 2){\makebox(0,0)[lt]{{\Large\strut Undervisning}}}
\end{picture}
\end{center}

\says{Nyrup} Det er mig en glæde at kunne byde velkommen til dette så
  velbesøgte pressemøde, hvor jeg sammen med Magrethe Vest-af
  \act{Margrethe tager vesten af} og Jan Legetøjsborg vil præsentere
  regeringens nye plan for fremtidens universitet.
  
  Vi har jo hidtil kendt universitetet for at være et sted hvor vi
  putter gode danske skattekroner ind i den ene ende \act{Margrethe
    putter en 1-krone i kassen, Jan putter en 5kkr-seddel i} og får
  intelligente danske kandidater og kvalitetsforskning ud af den anden 
  \act{Margrethe tager en papdukke med blinkende hovede ud af kassen, 
    Jan et skilt hvorpå der står ``År 2000 problem'' med et ``LØST''
    stemplet henover}.

\says{Nyrup} Nøgleordet, mine herrer og dame, nøgleordet i fremtidens
  universitet er \emph{modulært, objektorienteret typebaseret separat
    oversættelse med ekstern, dynamisk linkning af
    ressourcer}. Regeringen skal ikke lægge skjul på at
  inspirationen\ldots selve kilden til dette nye koncept\ldots er hentet 
  på et af de mest hæderkronede institutter på Københavns Universitet, 
  nemlig \emph{Datalogisk Institut}.

  Jeg har selv været ude at bese dyrestaldene på DIKU, og jeg har haft 
  en kammeratlig samtale med min gode ven, Søren. ``Søren,'' sagde
  jeg, ``hvordan kan det være at I har så megen succes med at uddanne
  genier -- har det mon noget at gøre med regelstyret indlæring?'' Og
  han svarede: ``devaide and kångker,'' eller på godt dansk: det hvide 
  og konkurrér.

  Regeringen iværksætter derfor nu det første trin: vi foretager det
  hvide snit! 
  \act{Margrethe og Jan trækker i hver sin side af universitetet, der
    går midt over. De træder nogle skridt væk fra hinanden mens de
    stadig holder deres kassehalvdele op}. Herved opnås den modulære,
  objektorienterede typebaserede separate oversættelse, og det er nu
  op til de enkelte parter at iværksætte det andet trin: at
  konkurrere, så vi får ekstern, dynamisk linkning af ressourcerne!

\says{Nyrup}[kigger ud i salen] Spørgsmål fra salen? \act{\ldots
  lytter til et fiktivt spørgsmål}
\says{Nyrup} Jeg er glad for at du stiller mig netop dét spørgsmål,
  hvortil jeg må svare: ja, regeringen \emph{har} tænkt på hvordan den
  eksterne, dynamiske linkning af ressourcerne kan se ud, der opereres endda 
  med flere modeller. Vi har først 
  \begin{description}
  \item[skolemodellen] \act{folkeskolelæreren kommer ind med en kasse 
      og holder den op til Margrethes:
\begin{center}\upshape \unitlength 5mm
\begin{picture}(22,7)(-1,0)
  \put(-1, 4){\line(1,0){10.667}}
  \put( 0, 0){\line(1,0){10}}
  \put( 0, 0){\line(0,1){4}}
  \put(-1, 4){\line(1,1){3}}
  \put( 2, 7){\line(1,0){8}}
  \put(10, 7){\line(-1,-2){1}}
  \put( 9, 5){\line( 2,-3){2}}
  \put(11, 2){\line(-1,-2){1}}
  \put(9  ,5.5){\makebox(0,0)[r]{{\LARGE\textbf{\textsf{UDEN VISIO}}}}}
  \put(10.5, 2){\makebox(0,0)[rb]{{\Large\strut Folkeskole-}}}
\end{picture}
\end{center}
    Margrethe putter en 1-krone i og læreren hiver en papdukke ud
      med en slukket pære. Lærer ud.}  der modsvares af
  \item[den decentrale forskningsmodel] \act{forskeren kommer ind med 
      en kasse og holder den op til Jans:
\begin{center}\upshape \unitlength 5mm
\begin{picture}(22,7)(-1,0)
  \put(21, 4){\line(-1,0){11.333}}
  \put(10, 0){\line(1,0){10}}
  \put(20, 0){\line(0,1){4}}
  \put(10, 7){\line(1,0){8}}
  \put(18, 7){\line(1,-1){3}}
  \put(10, 7){\line(-1,-2){1}}
  \put( 9, 5){\line( 2,-3){2}}
  \put(11, 2){\line(-1,-2){1}}
  \put(9.5,5.5){\makebox(0,0)[l]{{\LARGE\textbf{\textsf{F R I}}}}}
  \put(10.75, 2){\makebox(0,0)[lb]{{\Large\strut frihed}}}
\end{picture}
\end{center}
    Jan putter en 5kkr-seddel i og forskeren hiver det ene
      ``Win95''-skilt efter det anden op. Forsker ud.}
  \end{description}
  Der er også
  \begin{description}
  \item[erhvervsmodellen] \act{folkeskolelærer kommer ind med:
\begin{center}\upshape \unitlength 5mm
\begin{picture}(22,7)(-1,0)
  \put(-1, 4){\line(1,0){10.667}}
  \put( 0, 0){\line(1,0){10}}
  \put( 0, 0){\line(0,1){4}}
  \put(-1, 4){\line(1,1){3}}
  \put( 2, 7){\line(1,0){8}}
  \put(10, 7){\line(-1,-2){1}}
  \put( 9, 5){\line( 2,-3){2}}
  \put(11, 2){\line(-1,-2){1}}
  \put(9  ,5.5){\makebox(0,0)[r]{{\LARGE\textbf{\textsf{GOLD VIDE}}}}}
  \put(10.5, 2){\makebox(0,0)[rb]{{\Large\strut Pølsefabrik-}}}
\end{picture}
\end{center}
    Margrethe putter en 1-krone i og læreren hiver en strimmel
      papirmænd ({\unitlength 1.8mm%
      \begin{picture}(17,6)
        \multiput(0,0)(4,0){4}{
          \begin{picture}(4,6)
            \put(0,0){\line( 1,3){1}}
            \put(0,0){\line( 1,0){1}}
            \put(1,0){\line( 1,2){1}}
            \put(3,0){\line(-1,2){1}}
            \put(3,0){\line( 1,0){1}}
            \put(4,0){\line(-1,3){1}}
            \put(0,3){\line( 1,0){1}}
            \put(3,3){\line( 1,0){1}}
            \put(0,4){\line( 1,0){4}}
            \put(2,5){\circle{2}}
          \end{picture}}
      \end{picture}}) ud. Lærer ud.} og selvfølgelig    
  \item[medfinansieringsmodellen] \act{forsker kommer ind med:
\begin{center}\upshape \unitlength 5mm
\begin{picture}(22,7)(-1,0)
  \put(21, 4){\line(-1,0){11.333}}
  \put(10, 0){\line(1,0){10}}
  \put(20, 0){\line(0,1){4}}
  \put(10, 7){\line(1,0){8}}
  \put(18, 7){\line(1,-1){3}}
  \put(10, 7){\line(-1,-2){1}}
  \put( 9, 5){\line( 2,-3){2}}
  \put(11, 2){\line(-1,-2){1}}
  \put(9.5,5.5){\makebox(0,0)[l]{{\LARGE\textbf{\textsf{BETALTE}}}}}
  \put(10.75, 2){\makebox(0,0)[lb]{{\Large\strut lokaler}}}
\end{picture}
\end{center}
    Jan putter en 5kkr-seddel i og forskeren hiver en masse paphuse
      op. Forsker ud.}
  \end{description}
  -- og det er bare nogle af de udgiftsneutrale tanker regeringen
  arbejder med.

\says{Nyrup}[kigger ud på en person i salen] Ja? \act{\ldots
  lytter til endnu et fiktivt spørgsmål}
\says{Nyrup} Ved du hvad\ldots Mads\ldots skal vi ikke lige tage at
  holde foden kold, så vi ikke skyder os selv i hovedet\ldots Hvis
  regeringen ville spare\ldots og jeg siger udtrykkelig ``hvis,'' så
  er der stadig muligheder. Lad mig bare give to eksempler:
  \begin{description}
  \item [juramodellen:] \act{lærer kommer ind med:
\begin{center}\upshape \unitlength 5mm
\begin{picture}(22,7)(-1,0)
  \put(-1, 4){\line(1,0){10.667}}
  \put( 0, 0){\line(1,0){10}}
  \put( 0, 0){\line(0,1){4}}
  \put(-1, 4){\line(1,1){3}}
  \put( 2, 7){\line(1,0){8}}
  \put(10, 7){\line(-1,-2){1}}
  \put( 9, 5){\line( 2,-3){2}}
  \put(11, 2){\line(-1,-2){1}}
  \put(9  ,5.5){\makebox(0,0)[r]{{\LARGE\textbf{\textsf{SPAREMA}}}}}
  \put(10.5, 2){\makebox(0,0)[rb]{{\Large\strut Instruktor-}}}
  \put(1, 2){\makebox(0,0)[lt]{\fbox{{\Large\textbf{JURA}}}}}
\end{picture}
\end{center}
    Margrethe putter tydeligt ingenting i og læreren hiver en
      papdukke op uden pære. Lærer ud.}
    Og den tilsvarende forskermodel er
  \item[industripartnermodellen:] \act{forsker kommer ind med:
\begin{center}\upshape \unitlength 5mm
\begin{picture}(22,7)(-1,0)
  \put(21, 4){\line(-1,0){11.333}}
  \put(10, 0){\line(1,0){10}}
  \put(20, 0){\line(0,1){4}}
  \put(10, 7){\line(1,0){8}}
  \put(18, 7){\line(1,-1){3}}
  \put(10, 7){\line(-1,-2){1}}
  \put( 9, 5){\line( 2,-3){2}}
  \put(11, 2){\line(-1,-2){1}}
  \put(9.5,5.5){\makebox(0,0)[l]{{\LARGE\textbf{\textsf{ENGAGERET}}}}}
  \put(10.75, 2){\makebox(0,0)[lb]{{\Large\strut industri}}}
\end{picture}
\end{center}
    Jan putter tydeligt ingenting i og forskeren hiver skilte med
      ``COBOL'' eller ``WIN95'' op. Forsker ud.}
  \end{description}
  
\says{Nyrup}[kigger ud på en person i salen] Ja? \act{\ldots
  lytter til endnu et fiktivt spørgsmål}
\says{Nyrup} Nej, vi har i regeringen drøftet at samle det hele under
  Margrethes ministerium, men ulemperne er simpelthen for store -- vi
  kunne så komme ud for følgende situation:
  \begin{itemize}
  \item \act{Præst kommer ind med:
\begin{center}\upshape \unitlength 5mm
\begin{picture}(22,19)(-1,0)
  \put( 4, 0){\line(1,0){6}}
  \put( 4, 0){\line(0,1){7}}
  \put( 4, 7){\line(1,0){6}}
  \put( 4, 7){\line(1,4){3}}
  \put(10, 7){\line(-1,4){3}}
  \put(10, 7){\line(-1,-2){1}}
  \put( 9, 5){\line( 2,-3){2}}
  \put(11, 2){\line(-1,-2){1}}
  \put(9  ,5.5){\makebox(0,0)[r]{{\Large\textbf{\textsf{CEREMO}}}}}
  \put(10.5, 2){\makebox(0,0)[rb]{{\Large\strut Andagts-}}}
\end{picture}
\end{center}
    Margrethe putter en 1-krone i og præsten hiver en engel
      op.} eller denne  :
  \item \act{Præst kommer ind med:
\begin{center}\upshape \unitlength 5mm
\begin{picture}(22,19)(-1,0)
  \put(10, 7){\line(1,0){6}}
  \put(10, 7){\line(1,4){3}}
  \put(16, 7){\line(-1,4){3}}
  \put(10, 0){\line(1,0){6}}
  \put(16, 0){\line(0,1){7}}
  \put(10, 7){\line(-1,-2){1}}
  \put( 9, 5){\line( 2,-3){2}}
  \put(11, 2){\line(-1,-2){1}}
  \put(9.5,5.5){\makebox(0,0)[l]{{\LARGE\textbf{\textsf{GUDELIG}}}}}
  \put(10.75, 2){\makebox(0,0)[lb]{{\Large\strut religion}}}
\end{picture}
\end{center}
    Jan putter en 5kkr-seddel i og præsten hiver et skilt op hvorpå
      ordene ``PC klon'' er streget over med rødt. Præst ud.}
  \end{itemize}

\says{Nyrup} Tak. Der er ikke tid til flere spørgsmål, for jeg fik et
  computerspil med da jeg besøgte Sørens Institut, så nu skal jeg hjem 
  at spille ``a Lone in the Dark 3.''

\scene Lys ud.
\end{sketch}

\end{document}
% Local Variables: 
% mode: latex
% TeX-master: t
% End: 
