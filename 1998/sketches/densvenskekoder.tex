\documentclass[danish]{article}
\usepackage{revy}
\usepackage[utf8]{inputenc}
\usepackage{babel}
\usepackage{a4wide}

\usepackage{pifont}
\usepackage{calc}

\newsavebox{\fminibox} \newlength{\fminilength}
\newenvironment{fminipage} [1][\linewidth] {
\setlength{\fminilength}{#1-2\fboxsep-2\fboxrule}%
\begin{lrbox}{\fminibox}\begin{minipage}{\fminilength}}
{\end{minipage}\end{lrbox}\noindent\fbox{\usebox{\fminibox}}}

\newenvironment{todo}{%
  \renewcommand{\labelitemi}{{\LARGE \ding{46}}}%
  \renewcommand{\labelitemii}{{\LARGE \ding{46}}}%
  \renewcommand{\labelitemiii}{{\LARGE \ding{46}}}%
  \renewcommand{\labelitemiv}{{\LARGE \ding{46}}}%
          \begin{fminipage}\begin{itemize}}{\end{itemize}\end{fminipage}}

\title{Den Svenske Koder}
\author{Sidsel, Bobo, Duck, Uphfe}

\version{1.0}
   % HUSK AT AJOURFØRE VERSIONSNUMMER!!
   % - Ok, ska' nok!

\revyyear{1998}

\begin{document}
\maketitle

\begin{roles}
  \role{K} Den Svenske Koder -- Kokkekluns og med armene bundet ind
  til siden over albuerne
  \role{O} Oprydder -- samler kort op undervejs. (Nå, måske ikke
  alligevel) 
  \role{V} Voice -- styrer koderens stemme.
\end{roles}

\begin{props}
  \role{1 bord}
  \role{1 stykke sort stof} hæftet på bordet
  \role{3 gryder}
  \role{2 sedler} med sort tusch påskrevet \emph{INSORT} og
  \emph{MERGESORT} 
  \role{3 spil kort}
  \role{1 økse eller kødhammer}
  \role{Den svenske kok melodi} Enten brændt eller live
  \role{Quake rocket launcher lyd} Forhåbentlig brændt
  \role{Skæg} Til at dække koderens mund med så det ikke ses at der er
  voice på.
\end{props}

Der står et bord på scenen med de tre gryder med et spil kort foran
hver gryde. I de to første gryder ligger de to sedler. \emph{INSORT} i
den første og \emph{MERGESORT} i den anden. Øksen er placeret under
bordet. Med lidt held kan det hele bæres ind i een omgang.

\begin{sketch}

\scene Den svenske kok melodi starter.

\says{V}[hele tiden, i takt med musikken, pånær når der skal siges
noget] Hey-dy-bøy-di-bøy-di-muyf-muyf-muyf-muyf \act{nynner som kokken
gør}

\scene Koderen kommer ind, stiller sig ved bordet. Han fægter med
armene og ser ud til at sige sine karakteristiske lyde.

\scene Koderen tager en seddel op af den første gryde og viser den
frem. Viser folk at der står \emph{INSORT}.

\says{V} Insort!

\scene Koderen krøller sedlen sammen og smider den over skulderen,
tager nu et kort fra bunken op i hånden. Herefter tager koderen et
mere og et mere og et mere og stopper ind mellem de andre mens det ser
ud som om han tæller og sorterer. Får undervejs sat kortene op som en
korthånd. Ved det fjerde ser koderen helt forvirret ud og ryster på
hovedet og kaster det hele over skulderen. Også bunken med kort!

\scene Koderen går nu videre til næste gryde. Tager en seddel op af
gryden og viser folk at der står \emph{MERGE SORT}.

\says{V} Merge sort!

\scene
Koderen tager hele bunken op og deler
den over i to halve, en i hver hånd. Koderen prøver nu at flette
kortene ind i hinanden over gryden ved at nærmest kaste dem sammen mod
hinanden. Dette går også galt og han ryster på hovedet og kaster gryden
væk!

\says{V}[mumlende] Meget sort.

\scene Koderen går nu videre til den sidste gryde. Vender den om så
bunden er opad. Lægger bunken af kort ovenpå. Bukker sig ned under
bordet og finder øksen/kødhammeren frem. Ser vild ud og begynder at
banke på kortene og gryden. Stopper så op.

\says{V}[meget tydeligt] QUAKESORT.

\scene Koderen tæsker videre, Tæppe for ret hurtigt.
 
\scene Efter tæppet er kommet for kan man høre en rocket launcher. Der
skal høres et skrig.

\end{sketch}

\end{document}
