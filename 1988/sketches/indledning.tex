\documentclass[a4paper,11pt]{article}

\usepackage{revy}
\usepackage[utf8]{inputenc}
\usepackage[T1]{fontenc}
\usepackage[danish]{babel}


\revyname{DIKUrevy}
\revyyear{1988}
% HUSK AT OPDATERE VERSIONSNUMMER
\version{1.0}
\eta{$n$ minutter}
\status{Færdig}

\title{Indledning}
\author{MA, KHH, KM, m.fl.}

\begin{document}
\maketitle

\begin{roles}
\role{J}[] En journalist
\role{JC}[] ?
\role{St}[] Stemme
\role{Sp}[] Speaker
\end{roles}

\begin{sketch}

  \scene{Publikum mumler og hujer og råber "`Ekstranummer."'  Over
    højttalerne høres ouverturen til "Carmen.  En stemme isger: "`Ih
    nej, P1"' Derefter scala-spoling på en radio med et udvalg af
    lyde.  Til sidst høres afmeldingen: "`-og det var radioavisen"'.
    Plimmelimmelim.}

  \says{Sp} Nu følger udsendelsen en arbejdsplads i Danmark.  I denne
  uge skal vi sammen med Eliza MIPS besøge Datalogisk Institut.

  \scene{Kendingsmelodi for en arbejdsplads i DK og J kommer på scenen
    i fikst firsertøj med en båndoptager og fortæller lidt om DIKU:}

  \says{J} Ja, velkommen her til østerbro, hvor jeg i øjeblikket står
  foran data-logisk institut, som er et af Københavns Universitets
  nyeste institutter.  Instituttet blev oprettet i 1970 under ledelse
  af den...

  \scene{Under denne smøre går J hen og sætter sig på RESET-knappen,
    hvorefter lyset går ud.  Mens der er mørkt smutter J ud af scenen.
    Derefter begynder kendingsmelodien forfra, der kommer lys på
    scenen og J kommer på scenen i fikst firsertøj med en båndoptager
    og fortæller lidt om DIKU:}

  \says{J} Ja, velkommen her til østerbro, hvor jeg i øjeblikket står
  foran data-logisk institut, som er et af Københavns Universitets
  nyeste institutter.  Instituttet blev oprettet i 1970 under ledelse
  af den verdenskendte datalog Peter Naur, som blev hentet hertil fra
  det private erhvervsliv.  Indtil for få år siden havde man til huse
  i en gammel fabriksbygning på Nørrebro, men siden
  nittehundredeogseksogfirs har insituttet bot her lige ved siden af
  Fælledparken i dejlige grønne omgivelser.  Der er ansat ? mennesker
  på instituttet, deraf ? forskere, mens resten klarer for eksempel
  administrationen.  Der starter hvert år 120 nye studerende på
  instituttet og efter nogle problemer for et par år siden er man nu
  fuldt på højde med situationen.

  \says{J} Og vi er nu klar til at gå gennem hoveddøren.

  \scene{J ruser i en imaginær dør henne ved en højttaler e.lign.}

  \scene{Lydeffekt: Ruske, ruske.}

  \says{J} Hovsa, der er vist låst.

  \scene{Lydeffekt: Ruske, ruske.}

  \says{J} Øh, vi stiller om til studiet og spiller lidt musik.

  \scene{Der spilles She Came in Through the Bathroom Windows.  J går
    ud af scenen og over højttalerne høres lyden af en journalist med
    en båndopgaver over skulderen, der kravler gennem et
    kældervindue.}

  \says{J} Ja, efter at være kravlet ind gennem et åbentstående
  kældervindue er vi nu kommet op på institutbestyrerens
  penthouse-kontor.  Her er døren heldigvis imødekommende åben -- men
  hov her er jo ikke nogen -- jo, der kommer netop institutbestyreren.

  \says{JC} Jamen goddag -- vil du mig noget?  Hvis du er kommet for
  at bidrage til de smukke 12\% af studenterne, så er du kommet til
  den rette.

  \scene{Under denne replik fæstner JC en hvid ponpon til J's bagdel.}

  \says{J} Goddag, jeg kommer fra en arbejdsplads i DK.  Ja, du er jo
  bestyrer her på stedet.  Hvor længe har du været det?

  \says{JC} Jo, jeg er netop blevet valgt; vi plejer at lade det gå på
  omgang, for det tager jo tid fra min forskning.

  \says{J} Aha, du har måske også haft andre opgaver her på instituttet?

  \says{JC} Ja, udover min forskning var min sidste opgave at udarbejde
  en nye studieplan for vores førstedel.

  \says{J} Du vil måske forklare lytterne, hvad en studieplan er?

  \says{JC} Joh, studieplanen er den måde, man fordeler undervisningen
  imellem lærerne, så de også får tid til deres forskning.

  \says{J} Ja, det lyder jo spændende, men er det ikke svært sådan at
  få det hele til at hænge sammen?

  \says{JC} Jo, men den veldefinerede opslitning i små moduler giver
  en større grad af sammenhæng, især mellem hver enkelt lærers
  undervisning og forskning.

  \says{JC}[råber ud til resten af sangerne] Hej kolleger (/drenge) --
  er der sammenhæng?

\end{sketch}
\end{document}
