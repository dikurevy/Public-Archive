\documentclass[a4paper,11pt]{article}

\usepackage{revy}
\usepackage[utf8]{inputenc}
\usepackage[T1]{fontenc}
\usepackage[danish]{babel}

\revyname{DIKUrevy}
\revyyear{2003}
% HUSK AT OPDATERE VERSIONSNUMMER
\version{0.9}
\eta{-}
\status{Færdig til brug som tæppenummer eller revyhed}

\title{MARS (tæppenr.)}
\author{marvin}

\begin{document}
\maketitle

\begin{roles}
\role{O}[Adam] Opblæser
\end{roles}

\begin{props}
\prop{Ingenting}
\end{props}

\scene{Kan bruges som tæppenummer eller som en del af revyhederne} 
  
\begin{sketch}

\says{O} 
En ny, smitsom virus har set dagens lys. Forskere har givet den navnet 
"Meget Alvorligt Rum-Syndrom", MARS.

MARS blev første gang opdaget på Astrologisk Observatorium, men er siden 
observeret flere andre steder på NBIfAFG.

De lokale myndigheder forsøgte først at skjule den nye sygdom, men nu er 
over halvdelen af alle fysigere påvirket af MARS.

VerdensVirusVogter-organisationen, W3.org, betegner MARS som "yderst 
farlig" og Statens Sirup Institut fraråder al unødig udrejse til HCØ.

Symptomerne på MARS er åndedrætsbesvær, tør hals og en følelse af at 
være langt hjemmefra.


\end{sketch}
\end{document}

%%% Local Variables: 
%%% mode: latex
%%% TeX-master: t
%%% End: 

