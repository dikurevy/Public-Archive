\documentclass[a4paper]{article}
\usepackage[utf8]{inputenc}
\usepackage[T1]{fontenc}
\usepackage[danish]{babel}
\usepackage{charter}
\usepackage{revy}

\revyname{DIKUrevy}
\revyyear{2003}

% HUSK AT AJOURFØRE VERSIONSNUMMER!
\version{1.5}
\eta{x min.}

% OG STATUS!
\status{Færdig}

\title{Opkald under revyen 1 A}
\author{Martin Parm \& Neger-Uffe \& Ulla Tordenskjold \& Heidi}

\begin{document}
\maketitle

\begin{roles}
\role{S}[Søren] virkelig person (Voice over)
\role{L}[Adam] Sludderkvinde (Voice-over)
\end{roles}

\begin{sketch}
\scene Lyset slukket, lærredet nede

\act{Lyden af en mobiltelefon afspilles over højtaleren.}

\says{S} Ja, det er Søren.

\act{Kort pause}

\says{S} Hej skat.

\act{Kort pause}

\says{S} Hvad siger du?

\act{Kort pause}

\says{S} Lise? Det lyder da hyggeligt.

\act{Kort pause}

\says{S} Hva'? Nej, det har jeg altså ikke tid til nu. Jeg sidder lige og ser
DIKUREVY.

\act{Kort pause}

\says{S} Åh skat, det er jeg sørme ked af, men jeg sidder altså og ser
DIKUREVY. Så det kan altså ikke blive til noget med den trekant lige nu.

\act{Kort pause}

\says{S} Hva'? Bagefter? Jamen, da skal jeg jo til sommerfest.

\act{Kort pause}

\says{S} Ja, det ved jeg ikke. Men sent bli'r det i hvert fald.

\act{Kort pause}

\says{S} Ja.

\act{Kort pause}

\says{S} Ja, skat, det må I gøre. Jeg elsker også dig. I må hygge jer,
ikk'.

\act{Kort pause}

\says{S} Hej.
\end{sketch}
\end{document}
