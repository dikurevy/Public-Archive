\documentclass[a4paper,11pt]{article}

\usepackage{revy}
\usepackage[utf8]{inputenc}
\usepackage[T1]{fontenc}
\usepackage[danish]{babel}

\revyname{DIKUrevy}
\revyyear{2003}
% HUSK AT OPDATERE VERSIONSNUMMER
\version{0.71}
\eta{5 min}
\status{Færdig}

\title{Den store brugervenlige sketch}
\author{parmus, marvin, nokando}

\begin{document}
\maketitle

\begin{roles}
\role{G}[Peter W] Gammel, bitter datalog
\role{U}[Parmus] Ung, håbefuld datalog
\role{GS}[Andr\'e] Georg Strøm
\role{DDD}[Adam] Drabelige Danny Død, DIKU Linebacker
\role{G1}[Mads] Grøn mand \#1
\role{G2}[Uffe C] Grøn mand \#2
\end{roles}

\begin{props}
\prop{1 bord}
\prop{2 stole}
\prop{1 øl, plåk} malet en farve, så man kan se den.
\prop{Badering} til GS
\prop{Svømmevinger} til GS
\prop{svømmefødder til GS}
\prop{dykkermaske til GS}
\prop{sydvest til GS}
\prop{regnjakke til GS}
\prop{grønt tæppe} til G1+2 på skovtur, til at sidde på
\prop{grøn kurv} til G1+2
\prop{grøn flaske} til G1+2
\end{props}

\begin{sketch}
  \scene{Sceneopstilling: bord med stol på hver side placeret i højre
    side af scenen. I venstre side lige bag tæppet står en øl-flaske.}
  
  \scene{Rekvisitter er på plads, men tæppet går fra til synet af en
    mørk scene kun oplyst af en svag grøn spot, der lyser på et grænt
    kærestepar (G1 og G2). De er på skovtur: udbredt tæppe, kurv,
    vin.  G2 sidder op, og G1 ligge på ryggen med hovedet i skødet
    på den G2.}
  
  \scene{Efter ikke alt for lang tid, går ``sketchen igang''. G og U
    kommer ind, sætter sig på hver sin stol.}
  
  \scene{Lys skal forsøge at få fokus på ``sketchen'' og fjerne fokus
  fra G1 og G2.}

  \says{G} Nå, så du læser datalogi? Det var interessant, jeg er jo
  selv datalog.
  
  \says{U} Nåda, hvor langt er du så?
  
  \says{G} Pedantisk rettende, lettere overbærende] Nejnej, jeg sagde,
  at jeg er \emph{datalog}. Ikke datalogi\emph{studerende}.
  \act{Storsnudet} Der \emph{er} jo en vis forskel\ldots
  
  \says{U}[Undskyldende] Åh, undskyld.
  
  \says{G}[Nedladende] Men sig mig så, lille ven, hvad lærer I så på
  DIKU nu om stunder?
  
  \says{U}[Lyser op] Vi lærer bare de svedigste ting! Lige nu har jeg
  for eksempel et kursus om \act{stolt} \emph{multimedier} !
  
  \says{G}[Skeptisk] Det lyder da ikke svedigt. \act{Vemodigt} Næ, der
  var derimod sved, dengang vi skrev kerne\ldots \typeout{Look out!
    There's a tiger behind you!}
  
  \says{U}[Ivrig] Nej, den er god nok. Vi lærer alle mulige ting:
  \act{remser op} HTML, \LaTeX , Matlab, og en hel masse om lyde!

  \says{G}[Rystet] Hvad!!?! Dengang, \emph{jeg} var ung, der lærte vi
  virkelig noget. Den slags pjat lærte vi os selv i vores fritid. Og
  de eneste lyde, vi bekymrede os om, var lyden af eksploderende
  radiorør!
  
  \says{U} Jamen, vi lærer også om HCI. Brugerinteraktion og
  sådan\ldots
  
  \says{G}[Fnyser] Ha! \act{Vrænger} Brugere! \act{Bitter} Sådan nogen
  kan bare\ldots \act{Afbrydes af GS}
  
  \says{GS}[råber] \act{Kommer ind på scenen vildt faktende med armene
    og iført regnjakke.For hver gang han kommer ind, er han iført
    noget nyt badeudstyr} Stop, stop, stop. Sådan der kan I ikke lave
  en sketch.
  
  \says{U}[Glædeligt overrasket] \act{Peger på GS med udstrækte arme}
  George Strøm!!!
  
  \says{GS} Den sketch er jo helt ubrugelig. Den sketch er jo slet
  ikke brugervenlig.
  
  \says{G}[Forvirret og skeptisk] Øh, hvad mener du? Det er jo bare en
  sketch.
  
  \says{GS}[Forarget] Bare en sketch?!?!? Der er ikke noget som hedder
  ``bare en sketch''!!! Jeg bliver jo nødt til at rette op på den HELT
  fra bunden. For det første : Hvem er jeres målgruppe?
  
  \act{G og U kigger forvirret på hinanden}
  
  \says{G} \act{Peger ud mod publikum} Øh, dem der.
  
  \says{GS} \act{Kigger ud på publikum og derefter tilbage på G og U}
  Og de er?
  
  \says{U} Spritstive dataloger.
  
  \says{G}[Bistert blik på U] Ahrøm!
  
  \says{U}\ldots og datalogi\emph{studerende}.
  
  \says{GS} Og har I måske tager hensyn til det? Har I overhovedet
  undersøgt jeres brugeres stærke og svage sider?
  
  \says{G og U}[Ligeglade] Næh...
  
  \says{GS} Nej, det tænkte jeg nok. Heldigvis ved jeg noget om
  spritstive dataloger \act {G stirrer olmt på GS} fordi jeg har
  engang mødt en mand, som engang kendte en datalog, som engang var
  spritstiv.  For eksempel er spritstive datalogers fatteevne jo
  stærkt bedøvet, så jeres replikker skal være ekstra tydelige og
  artikulerede.
  
  \says{G}[Utålmodigt] Jaja, det skal vi nok. \act{Vender sig mod U
    for at fortsætte sketchen}
  
  \says{GS}[Afbryder hurtigt] Og hvad så med rekvisiternes placering.
  Har I måske tænkt over det?
  
  \act{G åbner munden og skal lige til at svare, men bliver afbrudt af
    GS}
  
  \says{GS} Nej, det har I ikke. Det bord \act{Peger på bordet} skal
  over til venstre \act{Peger mod højre side}.
  
  \says{U}[(be)undrende] Hvorfor det?
  
  \says{GS} Så sketchen følger læseretningen naturligvis.
  
  \act{G trækker på skuldrene, U signalerer, at nu har han forstået
    det}
  
  \says{G} OK, så flytter vi bordet.
  
  \act{GS ud. G og U tager fat i bordet og løfter det tilpas langsomt
  til at det passer med de grønne mænd over til venstre}
  
  \scene{G1 skal være VILDT balletagtige. G1 har set en blomst under
    bordet og er gået hen for at plukke den. Han bukker sig ned netop
    som bordet bliver løftet og båret hen over ham. Bordet har netop
    passeret ham, idet han igen rejser sig op og går tilbage til G2.}
  
  \says{U} Nåh, nu må sketchen være klar.
  
  \says{G}[Tilbage i rollen, overdreven tydeligt] Ha! \act{Vrænger}
  Brugere! \act{Bitter} Sådan nogen kan bare\ldots
  
  \says{GS}[Kommer ind og afbryder igen, med mere badeudstyr] Nej,
  nej, der er stadig noget galt. Sketchen mangler jo fokus på det som
  er vigtigt på scenen, nemlig den som taler. Det er jo ikke til for
  publikum at se, hvem af jer, de skal holde øje med. Det må vi gøre
  noget ved. \act{Ser sig om} Lad mig nu se.... \act{Får øje på
    ølflasken bag tæppet.} Det kan bruges. Den, der har øllen, har
  fokus. Den skal nok fange publikums opmærksomhed!
  
  \act{GS giver flasken til U, som smiler. G ser sur ud.}
  
  \says{GS} Sådan! Så kan publikum få øje på dig når du taler.
  
  \says{G} Jamen, hvad så, når jeg taler?
  
  \says{GS} \act{Tager øllen fra U og giver den til G. U rækker
    fortvivlet ud} Jamen, så skal du naturligvis ha' flasken.
  
  \says{U} Mener du virkelig at vi skal sidde og gi' en flaske frem og
  tilbage hele tiden?
  
  \says{GS} Nej, du har jo ret... \act{Står et øjeblik og klør sig
    betænksomt i skægget} Det er jo slet ikke nok. Vi må også gruppere
  de ting, som hører sammen. Den gamle datalog og bordet hører jo
  naturligt sammen, så du skal \act{Tager fat om Gs stol og flytter
  den ud i højre side af scenen} skal sidde herovre.
  
  \scene{G1 ser stolen, sætter sig på den for at undersøge sin sko. G
    går forvirret hen til sin stol og skal til at sætte sig (på den
    grønne mand)\ldots{} G ryster på hovedet, går om bag stolen,
    kigger hen på GS}
  
  \scene{G1 er færdig med at kigge sig under skoen, rejser sig op og
    gåt venstre om hen til G2}
  
  \scene{G tager stolen med sig hen imod GS mens han siger:}
  
  \says{G} Jamen \act{peger på stolen, han er igang med at trække
    efter sig} virker det ikke mærkeligt.
  
  \says{GS} Nej, nej, nu er sketchen ordenlig og brugervenlig. Jeg er
  sikker på at den er meget bedre nu. Prøv at fortsætte.
  
  \scene{GS smutter ud og tager mere udstyr på. G og U ser
    opgivende på hinanden, trækker på skuldrene. Så sætter G sig til
    rette.}

  \scene{G vifter opgivende med den frie hånd, går tilbage med stolen
    og stiller denne på plads}
  
  \scene{G sætter sig. Sketchen fortsætter. G og U spiller og snakker
    overdrevet artikuleret, omend knap så entutastiske.}
  
  \says{G}[monotont] Det lyder da ikke svedigt. Næ, der var derimod
  sved, dengang vi skrev kerne\ldots \typeout{Look out! It is still
    there!}
  
  \act{G rejser sig med flasken, går over til U med den, og går
    tilbage}
  
  \says{U}[Lige så monotont] Nej, den er god nok. Vi lærer alle mulige
  ting: \act{remser op, keder sig} HTML, \LaTeX , Matlab, og en hel
  masse om lyde!
  
  \act{U rejser sig, går over til G med flasken}
  
  \scene{G1 og G2 er på nippet til at blive gået ned. Det går op for
  dem, at der er for trangt på scenen. Rejser sig op med hænder på
  hofte og strittende albuer\ldots{} indsamler tæppe etc. og uddanser
  i protest. Dette skal går forholdsvis hurtigt.}
  
  \says{G}[Irriteret over flasken] Hvad!!?! Dengang, \emph{jeg} var
  ung, der lærte vi virkelig noget. \act{Kigger på flasken. Får en ide
    og kaster den til U}
  
  \says{U}[Ikke spor ivrig] Jamen, vi lærer også om HCI.
  Brugerinteraktion og sådan\ldots
  
  \act{U kaster flasken til G}
  
  \says{G}Ha! Brugere! Sådan nogen kan bare\ldots
  
  \says{GS}[Kommer ind og afbryder] Ja, det er meget bedre, men vi får
  ikke rigtigt understreget magtforholdet mellem den gamle og den
  unge.  \act{Henvendt til U}. Kan du ikke flytt din stol bagud på
  scenen?
  
  \scene{U flytter sin stol, den vender stadig mod G. Sætter sig. GS
  ud efter mere udstyr}
  
  \says{G}Ha! Brugere! Sådan nogen kan bare\ldots
  
  \says{GS}[Afbryder] Nejnej, det er forkert. Publikum føler sig
  udenfor. I må dreje Jer mod publikum!
  
  \act{G og U drejer irriteret deres stoloe mod publikum. Nu kan G
    ikke se U}
  
  \says{G}Ha! Brugere! Sådan nogen kan bare\ldots læse manualen. \act
  {Kaster flasken hårdt mod U}
  
  \says{U}[Bittert, hadefuldt, ironisk] Neeeejda. Brugere er bare
  \emph{så} vigtige. Det er nok sådan noget, vi bruger allermest tid
  på\ldots
  
  \act{U vil kaste flasken til G, men G kigger stadig mod publikum. U
    må rejse sig, gå hen og prikke G på skulderen og overdrevent
    høfligt/ironisk ``servere'' flasken til G. U går tilbage.}
  
  \says{G}[nedladende] Ungdommen har det tydeligvis nemt nu om
  stunder. Næh, dengang, \emph{jeg} var ond, der måtte vi \emph{bide}
  vores programmer i hulkort, vi \emph{selv} havde lavet af gamle
  aviser!
  
  \act{G triller flasken til U}
  
  \says{U}[forsigtig] Jo, men brugere \emph{er} jo vigtige\ldots Jeg
  er i hvert fald glad for at læse multimedie!
  
  \act{U lægger an til at tyre flasken i nakken på G}
  
  \says{GS} \act{Kommer ind og bryder pludseligt ud i vild begejstring
    og klapper vildt og voldsomt} Bravo, bravo!!! Det er jo fantastisk
  godt.  \act{Kigger ud mod publikum} Synes I ikke også det var
  utroligt?
  
  \says{U}[Harm] Jamen, det var jo fuldstændigt åndsvagt!
  
  \says{GS} Det er jo ligemeget. \act{Løfter pegeren belærende} Det
  var brugervenligt. Kom igen.
  
  \scene G rejser sig og tager flasken fra U. Giver flasken til en
  undrende GS, tager den fulde flaske fra GS, drikker begærligt på vej
  tilbage til sin stol.
  
  \says{G}[Ligeglad, halvsnalret] Brugere er bare \emph{så} vigtige.
  
  \says{U}[Forvirret, venter på flaske, men G gider ikke. U fortsætter
  uden og tager G's replik] Måske har du ret. Åh, gid det var mig!
  
  \says{GS}[Tilfreds] Jaaaaah! Det \emph{er} nu bedre på denne måde.
  Nu mangler vi bare nogle farver til at illustrere sindstilstand hos
  skuespill\ldots
  
  \scene DDD farer ind og tackler GS.
  
  \says{DDD}[Råber] Hvad ind i helvede er det du laver, din lusede
  humanist! De dér homo-farver kan du godt stikke skråt op. Og hold så
  nallerne fra scenen. Du har været med i én sketch allerede, og det
  var allerede for meget. Hvis du vil gøre dig nyttig kan du jo lære
  at sige \LaTeX , din skvatpisser.
  
  \scene G og U blander sig og sparker til GS
  
  \says{DDD} Det kan måske lære dig, din vandmand! Ikke mere HCI og
  HTML her. Vi vil have hardware og hårde algoritmer! 
  
  \scene Tæppe for, mens DDD skælder og fades ned
\end{sketch}

\end{document}
