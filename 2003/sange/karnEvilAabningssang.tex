\documentclass[a4paper]{article}
\usepackage[utf8]{inputenc}
\usepackage[T1]{fontenc}
\usepackage[danish]{babel}
\usepackage{charter}
\usepackage{revy}

\revyname{DIKUrevy}
\revyyear{2003}

\version{1.0}
\eta{-}

\status{Færdig}

\title{Åbningssang}
\author{Torben Mogensen}
\melody{Emerson, Lake og Palmer: Karn Evil 9, First Impression, part 2}

\begin{document}
\maketitle

\begin{roles}
\role{S}[Jakob J] Sanger
\role{K1}[Uhd] Kor
\role{K2}[Søren] Kor
\end{roles}

\begin{props}
\prop{Kjole og Hvidt}
\prop{Høj Hat}
\prop{Stok m. diamant}
\prop{Glimmer parykker x3}
\prop{Tylskørt x3}
\prop{Stram sort top ala variete show x3}
\end{props}

\begin{song}

\sings{S}Sæt jer ned,
\sings{K1+2+3} Hallo!
\sings{S}Vi har alletiders show,
Det er DIKU fyldt med "Go",
\sings{S\&K}Kom og se! kom og se!

\sings{S}Vi har dans og sang,
Og et band af første rang,
Med den allerbedste klang,
\sings{S\&K}Kom og lyt! Kom og lyt!

\sings{S}Vær beredt, vort show det sparker røv.  
det er så I næsten bliver døv'.
Garanter't en kæmpe sensation,
Som vil vække hele vor nation!

\sings{S}Men husk nu tøs og gut:
At ild og røg og krudt,
Ja, alt der siger "fut",
Det er strengt forbudt! Slut!

\sings{S}Hvis der sku' bli' brand,
Skal I gå med takt og sang,
Til den grønne nødudgang,
\sings{S\&K}Nødudgang, nødudgang.

\sings{S}Vær beredt, vort show det sparker røv.
det er så I næsten bliver døv'.

\sings{S}Men husk nu tøs og gut:
At ild og røg og krudt,
Ja, alt der siger "fut",
Det er strengt forbudt! Slut!

\scene Mellemspil 

\sings{S\&K1+2+3}Det er vores sag,
Når vi klæder DIKU af,
At det sker i bedste smag,
Ingen snag! Ingen snag!

Ingen hensyn dog,
Når vi tager ud på rov,
Ind i fysikernes skov,
Den bli'r sjov! Den bli'r sjov!

Revy! Revy! Revy!
Sommerfest!

Ikke nok med det:
Vi har multimedie,
En stor overraskelse,
Vent og se! Vent og se!

Alt det vil I få,
Når vi lader tæppet gå,
Til vore brædder skrå,
Glæd jer så! Glæd jer så!

Kom til sommerfest!
Kom til sommerfest!
Kom til sommerfest!
Sommerfest!

\end{song}
\end{document}
%%% Local Variables: 
%%% mode: plain-tex
%%% TeX-master: t
%%% End: 
