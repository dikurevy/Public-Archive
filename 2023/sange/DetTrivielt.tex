\documentclass[a4paper,11pt]{article}

\usepackage{revy}
\usepackage[utf8]{inputenc}
\usepackage[T1]{fontenc}
\usepackage[danish]{babel}

\revyname{DIKUrevy}
\revyyear{2023}
\version{1.0}
\eta{Estimeret tid}
\status{Almost Done}

\title{Det' Trivielt}
\author{Lise Bruun}
\melody{Obviously af McFly}

\begin{document}
\maketitle

\begin{roles}
\role{Pawel 1}[Lise] 
\role{Pawel 2}[Lotte]
\role{Elev}[Lukas]
\role{X}[Bjørn]
\role{N}[Eva]
\role{N}[Jonatan]
\role{N}[Jeppe]
\role{N}[Simon]



\end{roles} 

\section*{Noter}
Pawel 1 kloner sig selv i forsketchen så der en Pawel 1 og en Pawel 2.
Sidste omkvæd kan synges to gange så kan Pawel og Pawel klonen synger disse sammen)

\begin{props}
\prop{Tavle}
\prop{Stol til Pawel}
\prop{}
\prop{}
\prop{}
\end{props}

\begin{sketch}
\scene{Pawel sidder ned. Elev kommer ind.}
\says{Pawel} Velkommen til eksamen!
\says{Elev} Hej Pawel... hvordan trækker jeg et spørgsmål?
\says{Pawel} Nårh det ved jeg ikke... lad os bare sige at du trækker Grafteori.
\says{Elev} Vent, er det en del af pensum? 
\scene{Pawel hører ikke efter fordi han roder igennem nogle papirer}
\says{Pawel} Du går bare i gang.
\says{Elev} Men Pawel, skal vi ikke vente på sensor?
\says{Pawel} Skal vi nu også have det? Urgh, okay så, hvem kunne det nu være...
\scene{Akavet pause}
\says{Elev} Jeg kunne jo også bare komme tilbage i morgen?
\says{Pawel} Nejnejnej. Men det er jo lidt svært at finde nogen, der er så god til grafer som jeg. Hmmm. Nej vent
\scene{Pawel griner lidt.}
\says{Pawel} Jeg kender den perfekte! Vent der
\scene{Pawel rejser sig og åbner scene tæppet lidt. Pawel går ind.}
\scene{AV: et standard Windows-skrivebord, hvor der er en genvej til en mappe, der hedder Pawel. Der bliver højreklikket og der vælges \textit{Kopiér}. \textit{Sæt ind} og der kommer en ny mappe "Pawel (1)".}
\scene{Teknik: Røg og mekaniske lyde/larm når der bliver kopieret.}

\scene{Pawel 1 og 2 kommer ud fra hver sin side af scenetæppet. Elev virker bange.}

\says{Elev} øøøøøøh
\says{Pawel 2} Hvad med at du beviser… Dijkstra’s Algoritme?
\says{Elev} *mere insisterende* øøøøøøh
\says{Pawel 1} Altså, det er jo trivielt! Hør nu...

\scene{Sangen starter.}
\end{sketch}

\begin{song}
\sings{Pawel 1} Hele mit kursus
Var I så stille!
Spurgt’ ikke ind!
Jeg troed’ at I
Forstod mig
\sings{Pawel 2} Jeg var så grundig
Med alle beviser!
\sings{Pawel 1} I hvert fald med dem,
\sings{Pawel 2} Der ik’ super nem’
\sings{Pawel 1} Og kedlig’
\sings{Pawel 2} Men til eksamen,
Ka’ du ik’ bevis’
Grafalgoritmer!

\sings{Pawel 1} For det’ trivielt,
Det skriver sig selv
Beviset er nemt
Og du sir det er glemt
Men jeg tror, du aldrig har forstået et ord
\sings{Pawel 2} Nej nej
\sings{Begge} du har aldrig forstået et ord
\sings{Pawel 1}Kan du ik’ Dijkstra's?
		Dit bevis er rent bras, yeah
\sings{Pawel 2} Din intuition
\sings{Pawel 1} Den er sgu ik’ god
\sings{Pawel 2} Du sejler (Zoom fucker da TeXnikken sejler)
\sings{Stud}Jeg syn’s du afbryder (Pawel 1: Jeg syn’s du afbryder)
\sings{Begge}Er du så dum som du nu lyder? 

\sings{Pawel 2}For det’ trivielt,
Det skriver sig selv
Beviset er nemt
Og du sir det er glemt
Men jeg tror, du aldrig har forstået et ord
\sings{Pawel 1} Nej nej
\sings{Begge} du har aldrig forstået et ord	
\scene{C-stykke}
\sings{Pawel 1}Du ryster på hånden
		Og har dårlig tavleorden
\sings{Begge} Og du sta-sta-sta-sta-stammer jo helt vildt
For du stammer jo helt vildt
 Stammer jo helt vildt

\sings{Pawel 1} For det’ trivielt,
Det skriver sig selv
Beviset er nemt
Og du sir det er glemt
Men jeg tror, du aldrig har forstået et ord

\sings{Pawel 2} For det’ trivielt,
Det skriver sig selv
Beviset er nemt
Og du sir det er glemt
Men jeg tror, du aldrig har forstået et ord

\sings{Pawel 1} For det trivielt
Det skriver sig selv
\sings{Begge} Og jeg spilder min tid
For du jo halvt invalid
Og eksamen den er slut, 
\sings{Pawel 1} du dumper med 00?
\sings{Pawel 2} 0 0!
\sings{Begge} Det er slut, du dumper med 00
\end{song}

\begin{sketch}
\scene{Sangen er slut. Slutsketch nu.}
\says{Elev} Mange tak.
\says{Pawel} Skulle det være en anden gang.
\scene{Pawel 1 og 2 griner og Elev går ud.}
\scene{lys ned}

\end{sketch}



\end{document}
