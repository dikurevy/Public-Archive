\documentclass[a4paper,11pt]{article}

\usepackage{revy}
\usepackage[utf8]{inputenc}
\usepackage[T1]{fontenc}
\usepackage[danish]{babel}

\revyname{DIKUrevy}
\revyyear{2023}
\version{1.0}
\eta{Estimeret tid}
\status{Almost Done}

\title{Rundvisning for Tilvalgsstuderende}
\author{Ejnar, Mathilde og Jeppe}

\begin{document}
\maketitle

\begin{roles}
\role{R}[Jeppe] Rundviser
\role{I1}[Sia] Instruktor 1
\role{I2}[Nicholas] Instruktor 2
\role{S1}[Linus] Studerende 1 
\role{S2}[Jonatan] Studerende 2 
\role{X}[Elbo]
\end{roles} 

\section*{Noter}
Hvem der skal have hvilke roller specifikt er ikke blevet fastlagt. Linus og Frederikke skal nok være de letteste, da de ikke kan være ligeså meget med
\begin{props}
\prop{2 x Stol}
\prop{6 x kitler}
\prop{6 x sweater}
\prop{4 x fez}
\prop{6 x Blyanter}
\prop{1 x B dunk}







\end{props}



\begin{sketch}
\scene{lys op}
\says{R}Velkommen alle sammen.
\says{R}I har udtrykt interesse for at læse nogle datalogiske fag på tilvalg,
\says{R }så vi vil lige vise jer, hvad I kan forvente af undervisningen.

\says{R}	Herinde har vi nogle biologer, der er ved at lære om datastrukturer.
\scene{Biologer sidder på stole}
\says{I1} Hvilken farve har et træ?
\says{S1}	brunt?
%\scene{I1 Slår S1}
\says{I1}   NEJ! Hvilken farve har et træ?
\says{S1}	grøn?
%\scene{I1 Slår S1} 
\says{I1}   NEJ! Lad os prøve et lettere spørgsmål:
\says{I1}   hvilken vej vokser et træ?
\says{S1}	Opad?
%\scene{I1 Slår S1}
\says{I1}nej, hvilken vej vokser et træ?

\scene{Fysiker sidder i stol}
\says{R}	Fysikerne er her også, de har CompSys.
\says{I2}	Beskriv C’s hastighed
\says{S2}	Lysets hastighed i et vakuum. Den øverste kendte grænse for hastighed…
\says{I2}	Forkert! Assembler er hurtigere.

\scene{Fysiker sidder i stol}
\says{R}	Fysikerrus er her også, de har POP.
\says{I1}	Hvad er det objektivt bedste programmeringsprog?
\says{S1}	Altså Matlab er meget sejt og også Maple. Der kan jeg også lave plotte pæne grafer.
\says{I1}	Nej! Svaret er.. Jamen det er faktisk lidt svært. Men det er i hvert fald ikke fucking Maple.

\scene{Kemikere sidder i stol}
\says{R}	Der er sågar nogle få kemikere, de har altid så svært ved SU.
\says{S2}	Jamen jeg sværger! Der står altså det samme på begge sider af lighedstegnet.
\says{I2}	Så har du jo tydeligvis gjort noget galt. Forfra!

\scene{Matematikere sidder i stol}

\says{R} og her har vi så nogle Matematikere, de har DMA
\scene{$x=x+1$ på overtex}
\says{I1} Hvad er x lig med nu?!
\says{S1} 2
\says{I1} Jaaa, hvad er x så nu?!
\says{S1} Øøh x
\says{I1} Nej for fanden hvad er værdien af x
\says{S1} så 2?
\says{I1} Nej det var sidste gang! Det er 3, det kan du vel nok se!

\says{R} Ja hvis blot matematikere kunne tælle.

\scene{Historiker sidder i stol}
\says{R}	Historie hører til her, de har LinAlgDat
\says{I2}	Importér numpy!
\says{S2}	Jamen, kan vi overhovedet stole på kilden?
\says{I2}	Man skulle tro I kunne finde ud af at bruge et bibliotek. Importér numpy!
\says{S2}	Kan jeg ikke udføre min egen fortolkning?
\says{I2}  NEJ!
\says{I2} Selvom jeg er glad for, at du endelig har forstået at python er et interpretet sprog!
Du kan ikke lave din egen implementation af Numpy.
Det er skrevet på et sprog, du slet ikke forstår!
\says{S2}	Latin? Aramæisk?
\says{I2}	Nej, C.

\scene{RUC'er sidder på gulvet}
\says{R}	Vi har også folk uden for universiteterne. F. eks. RUC.
\says{I1}	For sidste gang! Sådan behandler man ikke et hash-table!

\says{R}    Nå, nu er klokken da også blevet 17.
\scene{Alle tilvalgsstuderende frigives, de takker instruktorerne og går glade afsted.}

\says{R}    Og nu sidder I jo nok og tænker:
\says{R}   "Jamen de var jo helt fortabte, havde mistet alt håb og var lige ved at miste deres værdighed"
\says{R}	Hah! Har du set jobcentrene! og hvordan de behandler folk dernede?
\scene{lys ned}
\end{sketch}

\end{document}
