\documentclass[a4paper,11pt]{article}

\usepackage{revy}
\usepackage[utf8]{inputenc}
\usepackage[T1]{fontenc}
\usepackage[danish]{babel}

\revyname{DIKUrevy}
\revyyear{2008}
% HUSK AT OPDATERE VERSIONSNUMMER
\version{1.1}
\eta{$2.5$ minutter}
\status{Færdig}

\title{Velkomst}
\author{Munter}

\begin{document}
\maketitle

\begin{roles}
\role{K}[Munter] Konferencier
\role{LE}[Hanne] Lene Espersen
\end{roles}

\begin{props}
\prop{Cola på en fiskestang}[Rekvisitholdet] Lene Espersen
\prop{Ministertøj, jakkesæt}[Hanne] Lene Espersen
\prop{Jakkesæt}[Munter] Konferencier
\end{props}

  
\begin{sketch}

\scene{Beskrivelse} 

\says{K} Godaften mine damer og herrer, og velkommen til DIKU revy 2008!

\says{K} I år har revyen gjort et stort stykke arbejde for at følge regeringens og rektoratets opfordring om at skabe 
en grøn og bæredygtig revy. Men inden vi kunne gøre det, blev vi nødt til at finde ud af hvad det egentlig betyder.

\says{K} Der er naturligvis ikke tale om øl. I så fald vil vi fra DIKUrevyens side kraftigt opfordre til at 
konceptet omdøbes til gyldent. \act{Tager en GT frem og skåler}

\says{K} Vi kan naturligvis også antage at der med grønt universitet og bæredygtig udvikling menes miljø og genbrug. 
Og her er DIKU jo allerede virkelig langt fremme i skoene... Altså hvis man ser bort fra at vi skifter institutbestyrer oftere end en fysiker skifter underbukser.

\says{K} Nogle få eksempler på genbrug på DIKU er: Genbrug af gammel kode, genbrug af studerende og genbrug af 
eksamensopgaver og opgaveformuleringer.

\says{K} Revyen har også selv gjort en del for at genbruge dårlige ordspil, nødugangene, som IGEN i år er placeret 
der, der, der, der, der og der, ligesom sidste år.

\says{K} Det fremgår altså tydeligt at DIKU er frontløberen indenfor miljø og genbrug. Og det er jo også virkelig 
nødvendigt at dataloger gør hvad de kan for at bruge færre ressourcer i en tid hvor Windows Vista gør alt for at bruge 
flere.

\says{K} Der har også været meget snak om fortætning fra rektoratet.
\says{K} Her er DIKU naturligvis endnu engang gået forrest. Vi har haft op til flere forskere i gang med at optimere vores forbrug af plads. Løsningen viste sig at være at fjerne de selvsamme forskere.
\says{K} Derudover har vi også afskaffet nogle lokaler som alligevel stod fuldstændigt ubrugte hen, og bedt de studerende om at sidde på skødet af hinanden for at udnytte den eksisterende plads bedre.

\scene{En cola på en fiskestang kommer ud gennem fortæppet og svinges lokkende frem og tilbage.}

\says{K} Nej, en cola! \act{rækker ud efter colaen}

\scene{LE springer frem fra fortæppet. B får et chock.}

\says{LE} DER FIK JEG DIG!

\says{K} Ja det må du nok sige. Men vi er altså lige i gang med en revy og du forstyrrer faktisk.

\says{LE} Jeg forstyrrer ikke! Jeg lokker! Og nu har jeg lokket dig! \act{Svinger colaen ud over forreste række med 
den ene hånd mens hun laver 
en kom nærmere fagt med den anden hånd}

\says{K}[Tøvende] Ok... Hvem er du?

\says{LE} Jeg er Lene S. Espersen, Konservativ folkepartis førstedame, ophavskvinde til lokningsbekendtgørelsen og 
justitsmor i Danmark.

\says{K} Jamen hej Justitsmor. Jeg tror altså ikke det er den slags lokning der er omtalt i bekendtsgørelsen om 
DATAlogning. 

\says{LE} Jamen er det her ikke dataLOGisk institut?

\says{K} Nej, det er datalogisk institut. Os kan du ikke logge, vi bruger kryptering.

\says{LE} Kryptering? Hvad er det I prøver at skjule?

\says{K} Vi har skam intet at skjule.

\says{LE} Ja det siger alle terrorister jo. Det her skal jeg nok komme til bunds i!

\act{Lene Espersen forlader scenen}

\says{K} Det må I undskylde. Og lad os så få startet DIKUrevyen 2008: "Intet at skjule".

\scene{Lys ud}

\end{sketch}
\end{document}

%%% Local Variables: 
%%% mode: latex
%%% TeX-master: t
%%% End: 

