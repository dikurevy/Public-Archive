\documentclass[a4paper,11pt]{article}

\usepackage{revy}
\usepackage[utf8]{inputenc}
\usepackage[T1]{fontenc}
\usepackage[danish]{babel}

\revyname{DIKUrevy}
\revyyear{2008}
% HUSK AT OPDATERE VERSIONSNUMMER
\version{1.2}
\eta{$4$ minutter}
\status{Færdig}

\title{Datalogisk cuisine}
\author{Daniel og Uffe \& Uffe Productions\texttrademark}

\begin{document}
\maketitle

\begin{roles}
\role{K}[Dirk] Kok
\role{A}[Allan] Assistent
\role{S}[Bo Elling] Sovende studerende
\end{roles}

\begin{props}
\prop{Papstykke, hvidt, ca. 80x80cm eller flip-over.}[Rekvisitgruppen] Der skal være to motiver, som begge skal være inddelte trekanter ala kostpyramider. Ene kostpyramide: Nederste lag: Fløde, smør, olie. Mellemste lag: Steg, lammekølle. Øverste lag: Kylling. Anden kostpyramide: Nederst: 3 stykker bacon. Midten: 2 stykker bacon. Øverst: 1 stykke bacon.
\prop{Urimelige mængder bacon - fx. 5kg}[Festbosser]
\prop{Fløde, 4 stk. halv liter piskefløde.}[Festbosser]
\prop{1 stk. 0.5L Coca Cola}[Rekvisitgruppen]
\prop{1 stk. 0.5L Coca Cola Light}[Rekvisitgruppen]
\prop{1 stk. 0.5L Pepsi Max}[Rekvisitgruppen]
\prop{Viskestykke}[Rekvisitgruppen]
\prop{1 stk. kokkehue}[Munter]
\prop{1 stk. kokketøj}[Munter]
\prop{Forklæde}[Munter]
\prop{Tjenertøj - hvid skjorte, sorte bukser}[Allan]
\prop{2 gryder - en stor og en mindre}[Rekvisitgruppen]
\prop{Stort piskeris eller grydeske}[Festbosser]
\prop{Sofa til sovende studerende}[Rekvisitgruppen]
\prop{Kantinebagstykke}[Rekvisitgruppen]
\end{props}


  
\begin{sketch}

\scene{TV-Køkken setup. TV-Køkken musik spiller.} 
\scene{Assistenten står og gestikulerer alt hvad kokken siger.}
\scene{Der ligger en studerende og sover i hjørnet.}

\says{K}[Med fransk accent]\footnote{Stumme h'er, artikulering, kradsende r'er, stemme s'er og z'er, etc.} Bonsoir og velkommen til denne udgave af
datalogisk cuisine - det er fedt! Vores mål er som altid at højne standarden af mad
der er tilberedt her på DIKU. Alt for ofte man ser de nouvelle
studerende kaste ulækkert DIKUsplat i sig.

\says{K} For at alle kan være med i aftenens udsendelse, vi starter med
det helt basale. Min smukke assistent nu viser den datalogiske kostpyramide.

\scene{Assistenten fremviser en kostpyramide i 3 lag. Nederste lag: Fløde, smør, olie. 
Mellemste lag: Steg, lammekølle. Øverste lag: Kylling}

\says{K} Altså. I den nederste lag vi har de basale flødevarer. Det, der skal give 
datalogen næring, saft og kraft til at klare dagens gerninger. Det er altså, fløde, smør 
og olie samt alle andre former for fedtstof, allerhelst animalsk.

\says{K} I den mellemste lag 
vi finder den gode kød. Det er her vi finder okse og lam og andre proteinholdige dyr som 
ikke er for tynde. 
\says{K} Sidst, og det vi skal have mindst af, de hurtige kalorier som kun er 
med, for at give en smule smag. Det er de dejlige kyllinger, kalkuner, fasaner, påfugle, 
strudse, med andre ord alt der skal plukkes -- også kaldet ``frugt''!

\act{K gestikulerer, at han nu skal til at i gang med den helt store gastronomiske udledning, men afbrydes af A, der rømmer sig og vifter med endnu et stykke pap}

\says{K} Ahh, men vi glemmer helt at intet måltid er komplet uden krydderi. Og derfor vi har 
vores smagspyramide.

\act{A viser smagspyramiden inddelt i tre lag. Nederst: 3 stykker bacon. Midten: 2 
stykker bacon. Øverst: 1 stykke bacon.}

\says{K} Nederst vi har substansen i smagen: godt med Bacon. I midten vi har kødets eget 
krydderi: Bacon. Og øverst vi har den man kun skal have lidt af for smagens skyld: Bacon!
Men nok teori, lad os lave dejlig mad!

\says{K} I aften vi skal lave en zyper-såås. Denne såås baserer sig på den velkendte 
læresætning, at mere fedt giver bedre mad. For at starte denne zyper-såås, vi tager et pund 
fløde. 

\scene{A præsenterer et pund fløde og giver til K. K hælder fløde i.}

\says{K} Dernæst vi tager tre pund bacon og rører godt i le såås. 

\scene{A giver K bacon. K hælder i. A rører rundt}

\says{K} Til slut vi topper le såås med to pund zyper-såås. 

\scene{A tager en gryde op og hælder i fra gryde til gryde.}

\says{K} Et voilá -- en fantastisk zucces.

\scene{A nyder duften af den dejlige sauce.}

\says{K} Bemærk, at siden vi nu har delt opskriften med jer alle, den er nu også en open-såås.

\says{K} Til vores zyper-såås jeg vil anbefale, at man serverer en dejlig afkølet Coca-Cola 
årgang deux-mil otte eller deux-mil syv. 

\scene{A præsenterer på ægte tjeneragtig maner en flaske Cola.}

\says{k} Har man perverse gæster, kan man vel servere en cola light \act{spytter over 
skulderen} 

\scene{A præsenterer lidt ligegyldigt en flaske Cola light.}

\says{K} Eller i det ekstreme tilfælde en Pepsi Max \act{gør korsets tegn} Mon 
Dieu!

\scene{A holder en flaske Pepsi Max ud i en strakt arm med kun to fingre. Kigger væk fra 
flasken og holder sig for næsen.}

\says{K} Som hors deuvre inden måltidet jeg kan anbefale et god gang Bacon i Baconsvøb. 
Eller evt. bacon naturelle garneret med røget bacon på pind med en lille tallerken bacon i 
tern ved siden af.

\says{K} Det var alt for i aften. Men følg med i næste uge, hvor vi gennemgår en række 
dejlige desserter -- blandt andet vores lækre flødebaserede økologiske baconsmoothie.

\scene{Bukker høfligt, og bakker ud af scenen.}

\end{sketch}
\end{document}

%%% Local Variables: 
%%% mode: latex
%%% TeX-master: t
%%% End: 

