\documentclass[a4paper,11pt]{article}

\usepackage{revy}
\usepackage[utf8]{inputenc}
\usepackage[T1]{fontenc}
\usepackage[danish]{babel}

\revyname{DIKUrevy}
\revyyear{2008}
% HUSK AT OPDATERE VERSIONSNUMMER
\version{0.3}
\eta{$4$ minutter}
\status{Ikke færdig}

\title{Help Desk}
\author{Revygruppen}

\begin{document}
\maketitle

\begin{roles}
\role{S}[Jakob] Sælger
\role{K}[Troels] Køber
\role{KL}[Guldfisk] Kæmpeklips
\end{roles}

\begin{props}
\prop{Dansk ordbog}[Done]
\prop{Engelsk ordnog}[Done]
\prop{Tysk ordbog}[Done]
\prop{Fransk ordbog}[Done]
\prop{Densk/Engelsk ordbog}[Done]
\prop{Flipover}[]
\prop{Kuglepind}[Jakob]
\prop{Blyant}[Jakob]
\prop{Limstift}[Jakob]
\prop{Rettelak}[Jakob]
\prop{Saks}[Jakob]
\prop{Klipsekostume}[]
\prop{Sælgekostume}[Jakob]
\prop{Navneskilt}[Jakob]
\prop{Studerendekostume}[Troels]
\prop{Bord}
\end{props}

  
\begin{sketch}
\scene{K har en sæk hvor alle objekter sælgeren sælger ham bliver placeret i.
Hver gang der købes noget nyt skal sælgeren også have penge, så K skal
have en stor bunke penge}

\says{K}      Undskyld hvad er det her for et sted?
\says{S}      Det er en boghandel
\says{K}      Godt, så er jeg kommet det rigtige sted hen
\says{K}      Jeg skal til skriftlig eksamen på DIKU. Jeg har fået af vide, at
alle elektroniske ydre enheder er bandlyste.
\says{S}      Så ingen linieskriver?
\says{K}      Ingen linieskriver. Hvordan skal jeg nogensinde kunne skrive noget uden?
\says{S}      Jo, nu skal du se her \act{viser ham en kuglepen}. Prøv engang at skrive med den.
\says{K}[dobbeltklikker på kuglepennen, så stiften ryger ud
  og ind igen]. Hvordan åbner man den. Dobbeltklik virker ikke.
\says{S}      Jamen ser du. Du skal kun klikke een gang.
\says{K}[prøver at gøre det, bliver lidt flov og kværulerer]  Det er
jo inkonsistent. Man åbner altid med et dobbeltklik. Det er imod alle
HCI-principper om vanedannelse. Den var aldrig gået hos Georg.
\says{S}      Sa så, indlæringskurven er kun lidt stejlere end Emacs. Prøv
nu bare at skrive lidt.
\says{K}      [begynder at skrive, og siger derefter bebrejdende] Denne font
er jo ulæselig. Hvordan vælger jeg en anden?
s:      [overpædagogisk] Det er MEGET besværligt at installere en ny
font. Det jo en opgradering. Det kræver flere uger på aftenskole.
\says{K}      Det har jeg virkelig ikke tid til. Lad mig lige komme i
gang.\act{begynder at skrive, begynder at lyse lidt op} Nå, sådan her. Ah,
det er sgu' da meget let \act{kigger ikke på papiret, mens han skriver
pludselig ud over sidekanten, opdager det pludselig}. Hey, hvor blev
de sidste tegn af?
\says{S}      Man har mange flere muligheder når man skriver på papir, så du
skal selv sørge for at gå en linie ned.
\says{K}[bevæger kuglepenne lodret ned og skriver videre ud over sidekanten]
\says{S}      Nej ikke bare ned, du skal bruge en vognretur \act{skubber hans håd tilbage}
\says{K}      Ahh, ok. \act{Skriver videre} Hov, er der ikke nogen stavekontrol?
\says{S}     Jojo. Vi har endda flere forskellige sprogversioner. Her er en
dansk, her er en tysk, her er en engelsk og her en Fransk. Hvor mange
vil du have? \act{Klasker tykke bind op på disken}
\says{K}      Jeg tror bare jeg nøjes med den danske.
\says{S}      Er du nu helt sikker, vi har også oversættermoduler \act{hiver
dansk/engelsk ordbog frem}
\says{K}      Nej, det der oversætter - det er jeg færdig med. Jeg bestod med et
stort 2-tal.
       Det minder mig forresten om; jeg får nok brug for rette lidt til
rundt omkring. Hvad med         undo og slette-funktionerne?
\says{S}      Jo da, så skal du enten opgradere til en model blyant, eller
bruge udvidelsespakken rette-lak.
\says{K}      Jamen, lad mig så få det med \act{køber blyant og rette-lak og S
kaster det i sækken}.
\says{K}      Det er meget godt alt sammen. Det vigtigste for mig er at jeg
kan cut'n'paste. Det er sådan jeg laver al min kode.
\says{S}      Det er ikke noget problem. Så skal du anskaffe dig dette
eksternt klip-modul \act{finder en saks og viser den frem, tager
  papiret og klipper afsnittet ud. Smider derefter saksen i sækken}
\says{K}      Nåå, på \_den\_ måde \act{tager afsnittet og sætter det et andet sted på
papiret, hvorefter det falder ned}. Hov, vent, det virker jo ikke.
\says{S}      Nååå, du vil også indsætte tekst. Jamen, så skal du bruge
klister-udvidelses-modulet \act{finder en limstift frem og limer afsnittet
fast på papiret}.
\says{K}      Der er godt nok meget man skal lære. Kan jeg bare kontakte dig
hvis jeg skal bruge hjælp?
\says{S}      Nej, men jeg har noget meget bedre til dig 
\scene{En kæmpeklips KL kommer ind}
\says{K}      Nooooooo
\says{S}[Råber efter ham] Husk at tage sikkerhedskopier. Jeg har hørt der
er bogorme i omløb
\scene{Tæppe}

\end{sketch}
\end{document}

%%% Local Variables: 
%%% mode: latex
%%% TeX-master: t
%%% End: 

