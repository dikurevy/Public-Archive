\documentclass[a4paper,11pt]{article}

\usepackage{revy}
\usepackage[utf8]{inputenc}
\usepackage[T1]{fontenc}
\usepackage[danish]{babel}

\revyname{DIKUrevy}
\revyyear{2008}
% HUSK AT OPDATERE VERSIONSNUMMER
\version{1.1}
\eta{$3.5$ minutter}
\status{Færdig}

\title{VW-V}
\author{Uffe, Uffe og Johan}

\begin{document}
\maketitle

\begin{roles}
\role{A}[Johan] Nørd (med laptop)
\role{B}[Allan] Nørd (med laptop)
\end{roles}

\begin{props}
\prop{Laptop 1}[Rekvisitgruppe]
\prop(Laptop 2)[Allan]
\end{props}

  
\begin{sketch}

\scene{De to nørder sidder med hver deres laptop. Når de snakker sammen
kigger de mest på deres laptops og taster lidt.}

\says{A} Hey, har du set den nye Folkevogn 5?

\says{B} Hvad? Nej, er der kommet en ny?

\says{A} Ja, de kalder den Volkswagen V, altså VW-V.

\says{B} Nåh okay. Er den lækker?

\says{A} Ja for pokker mand. Har du virkelig ikke set den? Du kan se den på
deres hjemmeside...

\says{B} Okay. Kan du give mig deres adresse?

\scene{Den følgende URL udtales langsomt og tydeligt og med distinktion
mellem "enkelt-v" og "dobbelt-w".}

\says{A} Åh ja hvad var det nu... \act{kigger op tænksomt} Jo... Ok...
www.vw-v.vw.de/vw-v/v2 \act{vender tilbage til laptoppen}

\says{B}[knapper det ind, rynker på brynene] Hm... Jeg får bare en 404 på den.

\says{A} Underligt lad mig lige tjekke... \act{knapper på laptoppen} Nårh klart.
De har jo lige indført deres nye portal for alle Folkevognsmodellerne:
Volkswagen Welt.

\says{B} Nårh ja det er der da også et link til heroppe.

\scene{URLen stadig langsom og tydelig.}

\says{A} Ja, og de har nok også opgraderet til ny fancy flash forside... Så
det bliver www.vww.v3.vw-v.vw.de/vww/vw-v

\says{B}[har lidt svært ved at følge med] Okay. Yes har den. Wow den er sgu
lækker! Hold kæft nogle forlygter! \_Den\_ skal bookmarkes!

\says{A}[kigger på sin laptop] Ah vent vent vent! Lad være med at bookmarke
den adresse. Her står der at de er ved at indføre en nice-URL standard
på deres site, så fra i morgen af skal du erstatte alle skråstreger
med "W".

\says{B}[kigger mistroisk på ham] "W"? Virkelig?

\says{A}[tjekker laptoppen igen] Ja, det står der her. Det burde virke allerede nu.

\scene{Nu sættes hastigheden på URL-oplæsning lidt op.}

\says{B} Jeg prøver lige... Så... www.vww.v3.vw-v.vw.deWvwwWvw-w ? \act{taster}
Hm... Den kunne Firefox ikke lide...

\says{A} Ah, nej, DNSen afviser den, så det første slash skal jo være
normal indtil de får deres standard gennem W3C. De kalder den IPw9. Så
burde W transparent virke som slash på alle websites.

\says{B} Så bliver slashdot til... Wdot.org?

\says{A} Ja, eller... nej... En del af IPw9 er at dot bliver til "v". Så det
bliver Wvvorg ... Øhm...

\says{B} Så vil jeg kunne finde Folkevogn V siden på wwwvvwwvv3vvw-vvvwvdeWvwwWvw-w ?

\says{A}[tænksomt] .... Jaeh, det lyder rigtigt!

\scene{URLen læses nu hurtigt og casual op.}

\says{B} Nå jeg prøver lige med første slash normalt og normale dots...
www.vww.v3.vw-v.vw.de/vwwwvw-w ... Det virker altså stadig ikke...

\says{A}[kigger over på B's laptop] Må jeg se?

\says{B}[holder sin laptop op] Her...

\says{A} Nååååårh... Det er jo fordi slash skal erstattes af store
"W". IPw9 er jo case sensitiv!

\says{B}[målløs] Nu giver jeg altså snart op...

\says{A} Arh come on... Så svært er det da heller ikke. \act{kigger på sin
laptop igen} ... Hey, har du set det her link til Volkswagen Welt
Worldwide Wireless WAP Workshop Weekend?
\scene{Tæppe}

\end{sketch}
\end{document}

%%% Local Variables: 
%%% mode: latex
%%% TeX-master: t
%%% End: 

