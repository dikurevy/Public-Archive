\documentclass[a4paper,11pt]{article}

\usepackage{revy}
\usepackage[utf8]{inputenc}
\usepackage[T1]{fontenc}
\usepackage[danish]{babel}

\revyname{DIKUrevy}
\revyyear{2010}
% HUSK AT OPDATERE VERSIONSNUMMER
\version{1.2}
\eta{$4$ minutter}
\status{Færdig}

\title{Lesserwisser}
\author{Johan og Munter}

\begin{document}
\maketitle

\begin{roles}
	\role{D}[Ronni] Datalog
	\role{S}[Dirk] Sekretær
	\role{B}[Mark] Datalog i baggrunden
	\role{(X)}[Munter] Instruktør
\end{roles}

\begin{props}
\prop{Computer-skærm-ramme på piedestal}[Rekvisitgruppen] D og S skal stå og pege på en ``computerskærm'', som blot er en træ- eller solid papramme ud til publikum. Dvs. den skal være stor nok til at to personers ansigter kan ses igennem af publikum, måske 80cm i bredden og 50cm i højden. Piedestalen skal være i cafebordshøjde, dvs. sådan at når man står op ved den, er ens ansigt på højde med skærmen. Evt. kan et cafebord fra forhallen bare bruges?
\prop{Tastatur}[Ronni] Gerne et gammelt, larmende et.
\prop{Usb-pen}[Ronni] Eller noget der ligner. Skal kunne være i en lomme
\prop{Gummimus}[Dirk] her menes en gummi-version af en RIGTIG mus.
\prop{Ballontip-lyd}[Troels] 
\end{props}



\begin{sketch}

\scene{Lyset blænder op for JobService. S står irriteret og hjælper B. D kommer ind.}

\says{D} Ja dav, er det her JobService?

\says{S} Det er det. Hvad kan jeg hjælpe med?

\says{D}[Opgivende] Mit job er blevet outsourcet til Indien på grund af finanskrisen og jeg er blevet sendt herned for at sende jobansøgninger før jeg kan få dagpenge.

\says{S}[Ruller med øjnene] Javel. Endnu en datalog. Nu skal jeg være der.

\says{D} Jeg behøver altså ikke rigtig hjælp. Den eneste grund til, at jeg er her er, at du skal overvåge mig mens jeg sender ansøgninger.

\says{S} Se DET er jobsikkerhed! Skal vi komme i gang?

\says{D} \act{sukker}

\says{S} Du kan stille dig ved den ledige computer her og gå i gang med at indtaste dit CV.

\says{D} Jeg har allerede mit CV med. \act{Hiver en USB-pen op ad lommen}

\says{S} Ja, det siger de allesammen. Så lad os udskrive det og se hvad du kan.

\scene{D sætter USB-pen i computeren og venter... og venter...}

\says{D} KOM NU!

\says{S} Tænkte jeg det ikke nok. Det er det samme med alle jer dataloger. I ødelægger altid jeres USB-penne.

\says{D} Den virker fint! Det er Windows der ikke forstår EXT4 på CryptFS!

\says{S} Ja undskyldninger er der nok af. Se så at komme i gang!

\says{D} Måske virker det hvis jeg genstarter maskinen.

\says{S} Hvad regner du med at opnå med en genstart?! Computeren opfører sig nøjagtig ens ved hver opstart. Hvis du forventer forskellig opførsel hver gang så er du simpelthen for dum.

\says{D} Men \act{S afbryder her} jeg har hørt at Windows \ldots

\says{S}[afbryder] \ldots desuden så er USB plug and play, så det virker fint hvis du ikke selv ødelægger ting. Og kom så i sving!

\says{D} Ja ja! Det tager jo lidt tid når man ikke har kommandolinje.

\says{S} Du kan jo bare bruge musen.

\says{D} Musen? \act{løfter gummimusen op i halen og kigger skeptisk på den} Jeg troede det var en kaffekopholder. Nå, men hvor er ikonet til OpenOffice henne?

\says{S} Den slags stads skal vi ikke have indenfor murene. Du kan bruge Wørd ligesom alle de andre sociale udskud. Tryk der!

\scene{D klikker demonstrativt på musen}

\says{D} Ok, så er vi i gang. Så tror jeg godt du kan tage en kop kaffe.

\says{S} Ja det havde jeg så sandelig også tænkt mig!

\scene{S begynder at gå væk}

\scene Lyd: Ballontip.

\says{D} HOV! Der dukkede noget op på skærmen!

\scene{S sukker dybt og vender tilbage}

\says{S} Det er et ballontip.

\says{D} Skal jeg så genstarte nu?

\says{S} NEJ! Klik der.

\says{D} Jeg skal lige læse hvad der står.

\says{S} Kom nu videre! \act{Tager musen og klikker ballontippet væk}

\scene{S går ud efter kaffe, men når ikke langt}

\says{D} CV. Gem... Hey, vent lige.

\says{S} Hvad nu?

\says{D} Jeg prøver at gemme dokumentet, men jeg kan ikke signere det.

\says{S} Du skal altså først underskrive CV'et når du \emph{har} printet det ud.

\says{D} \act{Ryster på hovedet}. Nu har jeg gemt. Skal jeg så genstarte?

\scene{S skuler ondt}

\says{D} Nej, se nu der? Jeg prøver at copy/paste med din elskede mus, men det virker ikke.

\says{S} Du glemmer at trykke Ctrl+C \act{læner sig ind over tastaturet}

\says{D} NEJ STOP! \act{skubber S væk} Er du gal mand? Så afbryder du jo programmet! Så bliver jeg nødt til at genstarte.

\says{S} Hold nu op med det pjat. Copy/paste har altid været Ctrl+C, Ctrl+V \act{S demonstrerer}

\says{D} Det giver jo ingen mening. Det næste du fortæller mig er vel at Ctrl+Z ikke suspender programmet.

\says{S} Ctrl+Z er Undo. Sig mig hvad i alverden er det i lærer på datalogi nu om dage?

\says{D} Ikke PC-kørekort i hvert fald.

\says{S} Så tror da pokker at I alle er arbejdsløse!

\scene{S går ud efter kaffe}

\says{D} \act{taler til sig selv} Nå nå nå, bitre madame. 

\scene Lyd: Ballontiplyd

\says{D} Hov! Nåh, ballontip igen... sikkerhedsadvarsel... root exploit!

\scene{D lyser op og hakker for vildt i tastaturet i 5 sekunder}

\scene{S kommer tilbage idet D trykker hårdt på enter}

\says{S} Nå, hvordan ser det ud?

\says{D}[Går hen mod printeren] Jeg er færdig. Mit CV er på vej ud af printeren!

\says{S} Nå, du kan alligevel noget når du bliver hold lidt i ørerne. \act{Går hen til computeren. Opdager hvad der sker på skærmen} Hov, hvad søren sker der her?

\says{D}[Taler hurtigt. Ligeglad med om S forstår det, kigger fraværende ned på sit nyprintede CV.] Jeg fik et ballontip med en sikkerhedsadvarsel om et root exploit, så jeg udnyttede det beskrevne bufferoverflow, fik root, installerede cygwin, og så er resten jo trivielt.

\says{S}[Kigger målløs på D] Øh, nå. Hvordan kommer jeg tilbage?

\says{D}[På vej ud. Gestikulerer henkastet med hånden] Du skal bare genstarte!

\scene{Lys ned, tæppe for}

\end{sketch}

\end{document}

%%% Local Variables: 
%%% mode: latex
%%% TeX-master: t
%%% End: 

