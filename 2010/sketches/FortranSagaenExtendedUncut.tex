\documentclass[a4paper,11pt]{article}

\usepackage{revy}
\usepackage[utf8]{inputenc}
\usepackage[T1]{fontenc}
\usepackage[danish]{babel}

\revyname{DIKUrevy}
\revyyear{2010}
% HUSK AT OPDATERE VERSIONSNUMMER
\version{1.0}
\eta{$1$ minutter}
\status{Færdig}

% Fortransagaen - The Golden Years - Extended Un-cut Edition
\title{Fortransagaen ...}
\author{Sune}

\begin{document}
\maketitle

\begin{roles}
\role{VO}[Spectrum] Voiceover
\role{D1}[Bo Elling] Datalog 1
\role{D2}[Jakob] Datalog 2
\role{D3}[Munter] Datalog 3
\role{K}[Ronni] Ko
\role{A}[Kristine] Afbryder
\end{roles}

\begin{props}
\prop{Kikset 70'er tøj med seler til D1, D2 og D3}[] Se evt. Fortransagaen-optagelsen fra 2007
\prop{Én eller flere piper}[]
\prop{Hulkortstrimler}[Kan evt. findes i Harlem] Skal koordineres med Datababy, hvor de også bruges (de bliver trukket ud af datamaten under datamat-strip-scenen, og blive liggende på gulvet i pausen.
\prop{blyanter}[]
\prop{3 stole}[]
\prop{Ko-stume}[Kan evt. købes i BR?]
\end{props}

  
\begin{sketch}

\scene{Beskrivelse} 

\says{VO} Efter tre års hårdt arbejde er det lykkedes DIKU revyen at fremskaffe tre timers, aldrig før hvist, materiale fra serien hvis gigantiske fanskare er repræsenteret på alle kontinenter, I alle lande, og på alle studier, nu med endnu flere tabte blyanter og uendeligeløkker.

\says{VO} DIKU revyen er stolte af at kunne præsentere "Fortransagaen - The Golden Years - Extended Un-cut Edition"

\scene{Lys op. To stole står foran tæppet, drejet lidt væk fra hinanden. På stolene sidder D1 og D2 og prikker i hulkortstrimler}

\scene{D1 taber sin blyant og samler den op, meget langsomt}

\scene{A kommer ind på scenen}

\says{A} Ja tak det er fint, kan i så komme ud. Vi har ikke tid til det her.

\says{D2} Jamen vi er slet ikke færdige

\says{D1} Vi har materiale til tre timer

\says{A} Meget muligt, men vi skal altså videre her, kan I så komme ud!

\scene{Alle forlader scenen, D1 og D2 er meget triste}

\scene{Tæppe fra, næste sketch starter}

\end{sketch}
\end{document}

%%% Local Variables: 
%%% mode: latex
%%% TeX-master: t
%%% End: 

