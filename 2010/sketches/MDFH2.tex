\documentclass[a4paper,11pt]{article}

\usepackage{revy}
\usepackage[utf8]{inputenc}
\usepackage[T1]{fontenc}
\usepackage[danish]{babel}

\revyname{DIKUrevy}
\revyyear{2010}
% HUSK AT OPDATERE VERSIONSNUMMER
\version{1.4}
\eta{$1$ minut}
\status{Brugbar}

\title{Manden der forklarer humor}
\author{Brainfuck, Phillip}

\begin{document}
\maketitle

\begin{roles}
\role{M}[Brainfuck] Manden der forklarer humor
\role{K}[Naja] Kvinden der forklarer humor
\end{roles}
  
\begin{sketch}

\scene{M kommer ind på scenen.}

\says{M} Se, det der var humor.  Men også den sværeste form for humor at 
forstå; \emph{sort} humor.  Dette emne er så komplekst at vi endnu ikke har
formået helt at forstå det.  Men lad mig gøre et forsøg.

\scene{K kommer ind på scenen.}

\says{K}[Mobset] Står du og forklarer min sketch?

\says{M} Det er jo ikke sikkert at folk forstår den!  Jeg leverer en public
service til folket.  Jeg er humorens DR.

\says{K}[Det går op for hende hvad der er sjov ved M] Ah, nu kan jeg se hvad
det er du laver!  Det sjove ved dig er jo at du prøver at forklare humor til
et publikum der sagtens forstår det, mens du ignorerer dem og lader som om de
slet ikke forstår.

\says{M} Ej, hør, \emph{jeg} skal aldeles ikke forklares.  Det er \emph{mig}
forklarer ting her.  Det sjove ved dig er, atdu kommer ind og prøver at 
forklare mig, imens jeg står og forklarer den sketch du er med i.

\scene{Beat.}

\says{K} Det er jo ikke sjovt.

\scene{Lys ud.}

\end{sketch}
\end{document}

%%% Local Variables: 
%%% mode: latex
%%% TeX-master: t
%%% End: 
