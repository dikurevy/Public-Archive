\documentclass[a4paper,11pt]{article}

\usepackage{revy}
\usepackage[utf8]{inputenc}
\usepackage[T1]{fontenc}
\usepackage[danish]{babel}

\revyname{DIKUrevy}
\revyyear{2010}
% HUSK AT OPDATERE VERSIONSNUMMER
\version{1.0}
\eta{$2.5$ minutter}
\status{Færdig}

\title{JesusPad}
\author{Munter}

\begin{document}
\maketitle

\begin{roles}
	\role{F}[Klaes] Fanatiker. giPad-bærende fanboy i yuppiesmart skjorte, slips og moderigtige briller med skyklapper.
	\role{S}[Allan] Skeptiker. Afslappet klædt. Ikke nørd.
	\role{(X)}[Ejnar] Instruktør
\end{roles}

\begin{props}
	\prop{giPad}[] En iPad i en størrelse hvor en person kan bære den som et kors der låser armene helt i yderposition.
	\prop{Lyserød skjorte}[] Moderigtig. Til Fanatikeren.
	\prop{Lyserødt/gråt slips}[] Moderigtigt. Til Fanatikeren. Må ødelægges!
	\prop{Tykke briller med skyklapper}[] Moderigtigt stel hvor man ikke kan se til siderne.
	\prop{mobiltelefon}[] Til Skeptikeren
	\prop{Lyd: Kustig shutter-lyd}[] Den slags som mobiltelefoner laver når de efterligner lyden på et spejlreflekskamera.
\end{props}

\scene Regi: giPad udtales på engelsk "djaiPad", forkortelse af gigantisk.

\scene Sketchen foregår foran tæppet.
  
\begin{sketch}

\scene S står på scenen og trykker på sin mobil. F er bag tæppet.

\scene Lys: lamper foran tæppe. Ikke spot.

\says{F}[bag tæppet] Hey du der!

\scene S ser sig forvirret omkring

\says{F} Se! Jeg har en giPad!

\scene Tæppe åbner sig nok til at man kan se F i åbningen.

\says{F} En JesusPad! \act{Går frem foran tæppet}

\scene Tæppet lukker

\says{F} Nuj, hvor er jeg bar lækker og oppe på beatet. Jeg har sagt det én gang og jeg siger det igen \ldots

\says{S}[afbryder] Du har en iPad?

\says{F} Bedre endnu. En giPad! \act{viser sig frem}

\says{S} Det ligner nu mest af al bare en forvokset mobiltelefon.

\says{F}[Overbærende] En telefon? Nej nu må du lige. Det er slet ikke det den er beregnet til, så man kan ikke ringe fra den.

\says{S} Ok. Men jeg kan godt se det er smart med en så stor skærm til at se alle de billeder man tager med den.

\says{F}[Mindre overbærende] Man kan se billeder, men ikke tage dem. Steve Jobs har selv sagt at det er bedste sådan. Vi kan jo ikke have hvem som helst til at rende og tage billeder. Det forringer jo bare kvaliteten af billeder hvis det ikke er Applegodkendte fotografer der tager dem.

\says{S} Det lyder da ikke særlig smart. Jeg synes da det er meget rart at man kan tage billeder. \act{Rækker telefonen frem og tager et billede af F}

\scene Lyd: Shutterlyd

\says{F} Den slags anarkisme er der slet ikke brug for. Spørg selv Steve Jobs. Eller endnu bedre, køb en giPad, så kan du selv se.

\says{S} Jeg kan altså ikke rigtig se hvad man kan med den der iPad\ldots

\says{F}[afbryder] giPad\ldots

\says{S} \ldots som man ikke kan med min telefon.

\says{F} Er du da helt bims? Kan du slet ikke se det? Den er jo totalt lækker!

\says{S} Ja ja. Men kan man feks. proppe den i lommen? \act{Putter demonstrativt sin mobiltelefon i lommen}

\says{F} Selvfølgelig ikke! Så ville hele pointen med giPad jo være forsvundet.

\says{S} Og det er?

\says{F} At blive set! Og så kan man browse på den. Kan man måske det på din gamle brugte sag der?

\says{S} Ja. Det virker fint.

\says{F}[Nedladende] Ja fint og fint. Det er jo ikke en særlig fed oplevelse når man har så lille en skærm og ikke kan bruge touch til navigation.

\says{S} Der er touch.

\says{F} Det er jo ikke rigtig touch hvis man ikke har en giPad. Det er simpelthen den mest intime browsingoplevelse du nogensinde har oplevet. \act{smyger sig op af sin giPad}

\says{S} Jeg har det helt fint med ikke at drikke dus med min mobiltelefon.

\says{F}[Nedladende] Det er et taberargument fra en stenaldermand. Indse at du lever i fortiden og at man ikke kan noget med din forældede mikrobe af en telefon. Der er jo ikke én eneste ting du kan nævne som jeg ikke kan bedre med min giPad!

\says{S} Jeg kan nu godt komme i tanke om en enkelt ting.

\says{F}[Ironisk] Én ting. Wow. Det må jeg se før jeg tror det.

\says{S} \act{Hiver en saks op af lommen, går hen til F og tager fat i hans slips}

\says{F} Hov, nej, stop!

\says{S} \act{Klipper Fs slips af, stopper det i lommen og smutter ind bag tæppet igennem midten}

\says{F}[Meget vred] Din taber! Kom her! \act{forsøger at løbe efter S, men kan ikke komme igennem tæppet fordi giPad'en er så stor}

\says{F} Argh! Det var da\ldots Grrr\ldots

\says{F}[Råber] Tæppe fra! NU! Tæppe! Jeg vil ud!

\scene Lyd ned.

\scene Tæppe lidt fra så F kan komme ud mens publikum forhåbentlig griner og klapper.
\scene Tæppe for.

\end{sketch}
\end{document}

%%% Local Variables: 
%%% mode: latex
%%% TeX-master: t
%%% End: 

