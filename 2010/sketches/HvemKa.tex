\documentclass[a4paper,11pt]{article}

\usepackage{revy}
\usepackage[utf8]{inputenc}
\usepackage[T1]{fontenc}
\usepackage[danish]{babel}

\revyname{DIKUrevy}
\revyyear{2010}
% HUSK AT OPDATERE VERSIONSNUMMER
\version{1.3}
\eta{$2.5$ minutter}
\status{Helt færdig}

\title{Hvem ka'}
\author{Wikipedia, Munter, Phillip, Sune, Ejnar}

\begin{document}
\maketitle

\begin{roles}
	\role{S}[Dirk] Entusiastisk, begejstret og lidt børnetimeagtig storvildtjæger
	\role{B0}[Valiant] Burkaklædt kvinde
	\role{H1}[Allan] Hjælper
	\role{H2}[Jakob] Hjælper
	\role{B1}[Mark] Kampberedt burkakvinde med gevær
	\role{B2}[Naja] Kampberedt burkakvinde med gevær
	\role{(X)}[Munter] Instruktør
\end{roles}

\begin{props}
	\prop{Lyd af gevær}[Troels] 
	\prop{Junglelyde og fuglekvidder}[Troels]
	\prop{Storbylyde}[Troels] Fx. biler, larm, politisirener.
	\prop{3 Burkaer}[Rekvisitgruppen] Forskellige farver. Dog ikke rød eller sort.
	\prop{Nettokurv}[Rekvisitgruppen] Fyldt med varer
	\prop{Bedøvelsespil}[Rekvisitgruppen] 
	\prop{Luftgevær}[Dirk] Til jægeren
	\prop{2 AK47}[Rekvisitgruppen] Til B1 og B2
	\prop{Radiosender}[Rekvisitgruppen] Dims med antenne, som er stor nok til at man kan se den. Tape/velcro/sele til påmontering omkring ben eller arm.
\end{props}

  
\begin{sketch}
\scene Lys op, tæppe fra. Bag tæppet står en Nettokurv med varer.

\scene Lyd: Junglelyde og fuglekvidder.

\scene Bagtæppet bliver trukket langsomt fra og en burkaklædt kvinde sniger sig frem til kurven. Hun roder i kurven.

\scene Kvinden opdager publikum og farer sammen og er klar til flugt.

\scene Der lyder et skud

\scene Kvinden får et chock, løber lidt rundt om sig selv, nu med en bedøvelsespil i ryggen. Hun trækker pilen ud og kigger på den, hvorpå hun falder om.

\scene En Storvildtjæger lister ind på scenen som om han er bange for at kvinden vil rejse sig og angribe. Han opdager publikum.

\says{S}[Meget pædagogisk] Hej! \act{Tjekker lige at B ligger stille}

\says{S}[Stadig snigende, listende] Bare rolig, hun er kun bedøvet. 

\scene S gør tegn til sine hjælpere udenfor scenen

\scene Hjælperne kommer ind og begynder at sætte noget på benet af B. De fortsætter feltarbejdet med at måle og notere i en notesbog mens S taler videre.

\says{S} Jeg er burkaforsker på Københavns universitet. 

\says{S} Som forsker på Biologi har jeg fået øremærket basismidler fra  øvrige studier, således at vi fokusere på hunnens migreringsmønstre hos denne indvandrede art. Og her ser man faktisk en interessant tendens.

\scene H1 og H2 forlader scenen. Man ser en radiosender sidde på B0's ben.

\says{S} Ved hjælp af de radiosendere, vi har udstyret hunnerne med \act{Peger på B0's ben}, har vi nu i længere tid kunnet spore en unaturligt organiseret samling og fortætning af arten.

\says{S} Vi er på Nørrebro.

\scene Lyd: Maskingevær skyder. Biler kører forbi og dytter. Lydkulisser skifter generelt til bylyde.

\says{S} Her i betonjunglen bygger hunnerne rede, passer unger og samler lokale frugter og grøntsager mens hannerne går på shawarmabar for at skaffe kød til at forsørge familien.

\says{S} Dog er der somme tider nogle få familier der forsøger at flytte rede og migrere til mere lovende områder, hvor temperaturen ikke er nær så høj på grund af politiske klimaforandringer.

\scene Lyd af maskingevær, højere, tættere på

\says{S} Imidlertid har min forskningsgruppe observeret at disse emigranter desværre ikke når særlig langt fra Nørrebro. Deres nomadiske træk hindres af den naturlige ruse, der omkranser området i form af ensrettede gader.

\scene Lyd af maskingevær, højere, tættere på

\says{S} Og ny forskning viser faktisk at arten, på grund af barrieren og de beskyttede omgivelser, har ynglet tilstrækkeligt til at opnå kritisk masse og muligvis udvikle selvstændig intelligens!

\scene Lyd af maskingevær, højere, tættere på

\says{S} Hvis dette sker, så er der ingen der ved hvad resultatet vil være, og hvilke konsekvenser det vil få for lokalbefolkningen og den eksisterende biologiske balance! 

\scene Lyd af maskingevær, meget tæt på. S falder om

\scene B1 og B2 kommer ind på scenen og går hen til B0.

\scene B1 skubber til B0 med foden.

\scene B1 og B2 kigger på hinanden og nikker.

\scene B2 tager Nettokurven og de går begge ud af scenen.

\scene Lys ned


\end{sketch}
\end{document}

%%% Local Variables: 
%%% mode: latex
%%% TeX-master: t
%%% End: 

