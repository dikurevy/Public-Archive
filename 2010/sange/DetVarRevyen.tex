\documentclass[a4paper,11pt]{article}

\usepackage{revy}
\usepackage[utf8]{inputenc}
\usepackage[T1]{fontenc}
\usepackage[danish]{babel}

\revyname{DIKUrevy}
\revyyear{2010}
\version{1.0}
\eta{$3$ minutter}
\status{Færdig}

\title{Det var revyen}
\author{Munter}
\melody{Folkesang: ``Hava Nagila''}

\begin{document}
\maketitle

\begin{roles}  
\role{S1}[Jakob] Sanger
\role{S2}[Kristine] Sanger
\role{D1}[Mark] Danser i burka
\role{D2}[Naja] Danser i burka
\role{D3}[Valiant] Danser i burka
\role{D4}[Mikkel] Danser i spændetrøje
\end{roles}

\begin{props}
\prop{3x burka}[Hvem ka'] Kostumer fra Hvem ka' genbruges
\prop{Spændetrøje}[Værdikamp] Kostume fra Værdikamp genbruges
\prop{2x stol}[] Skal kunne løftes op og danses rundt med, mens der sidder en person i den. Ligesom til jødiske bryllupper. En stol fra kantinen er nok passende, men det skal lige testes.

\end{props}

\scene Sangen gentages 3 gange med mellemspil.

\scene S går ud foran tæppet alene. Sangen startes meget langsomt og uden for meget band så publikum kan høre den komplekse tekst.

\begin{song}
  \sings{S} 'Vyen, det var re-
vyen, det var re-
vyen, det var den sidste sang

'Vyen, det var re-
vyen, det var re-
vyen, det var den sidste sang

Det var den sidste sang
Nu er der fest og larm
Dans publikummer, natten lang

Det var den sidste sang
Nu er der fest og larm
Dans publikummer, natten lang

fe-sten, fe-sten starter
festen starter på Caféen?
festen starter på Caféen?
festen starter på Caféen?
festen starter på Caféen?

Gå til fest!
Gå til fest!

Dans nu natten lang

\scene Mellemspil. Tæppet går fra. Bag tæppet danser 4+ ortodokse jøder kædedans.

\scene Musikken øger tempoet.

\sings{S} 'Vyen, det var revyen...

\scene Mellemspil. Bagtæppet går fra. Bag tæppet står alle skuespillerne. Der danses kædedans over hele scenen.

\scene Musikken øger tempoet

\sings{S} 'Vyen, det var revyen...

\end{song}

\end{document}

