\documentclass[a4paper,11pt]{article}

\usepackage{revy}
\usepackage[utf8]{inputenc}
\usepackage[T1]{fontenc}
\usepackage[danish]{babel}

\revyname{DIKUrevy}
\revyyear{2017}
% XXX: HUSK AT OPDATERE VERSIONSNUMMER
\version{4.0}
\eta{$3$ minutter}
\status{Som den blev opført}

\title{Niels Brohr-monologen}
\author{Sebastian, Spectrum, Søren Pilgård, Johan, Ejnar, Thomas}

\begin{document}
\maketitle

\begin{roles}
  \role{I}[Andreas]  Interviewer; TV-vært
  \role{NB}[Ronni] Niels Brohr
  \role{X}[Caroline] Instruktør
\end{roles}

%\begin{props}
%
%\end{props}

\begin{sketch}

\scene{Nyhedsprogram; interviewer sidder midt for på bordet og kigger ud mod
       publikum, á la et nyhedsprogram, med Niels siddende som gæst ved siden af.}

\says{I} Den nye Niels Bohr-bygning er delt i to af Jagtvej. Som vi lærte til sidste
         års DIKUrevy er disse to dele forbundet med en såkaldt ``Skywalk''. Med i studiet har vi
         lederen af byggeprojektet; Niels Bohrs bror, Niels Brohr.

\says{I}[Til Niels] Niels; hvorfor konstruere en skywalk, når der nu også bygges en tunnel?

\says{NB} Alle ved at tunnelbyggeri er lyssky, undergravende virksomhed.
          Enhver der vil bygge en tunnel må have noget at skjule. Såsom Higgs partiklen!

    Næ, broer ligger dybt i den danske kulturarv.
    Betragt blot Danne\emph{bro}g! Vi ser to hvide broer der skyder over et rødt landskab.
    En der forener venstrefløjen med højrefløjen, og en der bygger bro
    mellem toppen og bunden af klassehierakiet. Forenet i et stort dansk
    \emph{bro}derskab!

\says{I} Det er meget godt, men hvordan har det på nogen måde relevans for byggeprojektet?

\says{NB} Ja, forestil Dem blot Dannetunnel! Et pure-rødt flag --- Sig mig, de er vel ikke kommunist!?

\says{I} Nej, bevares.

\scene{NB træder frem; spot på NB}

\says{NB}
    Lad os kigge videre til dengang København blev grundlagt i \emph{bro}nzealderen.
    Vi kan tydeligt se på Københavns stednavne, Nørre\emph{bro},
    Øster\emph{bro}, Vester\emph{bro}, at broer er Danmarks fundament!
    De har vel hørt om kontinentaldriften, har De ikke?
    Havde vi ej broer til at sammenholde Danmarks øer, da ville de drive fra
    hinanden. Se blot Grønland! Og de Vestindiske Øer!

    Ja, broer er vejen frem, og man skal jo tænke på fremtiden - og børnene!
    \act{Læg tryk på hvert ord i følgende sætning.}
    \emph{Alle}. \emph{børn}. \emph{\bf elsker}. \emph{broer}!
    Det hedder jo ikke ``Tunnel, Tunnel, Brille'', vel?
    Og ``Jeg gik mig under sø og land'' - det er jo det rene vrøvl!

    Jeg har en drøm! I fremtiden vil det være muligt at gå en Eulertur igennem
    \emph{hele} verden! Broer fra kyst til kyst, fra Holste\emph{bro} til \emph{Bro}nholm!

    Det vil også gavne vores \emph{bro}ttonationalprodukt!
    Tunneller? Næææh! De er slet ikke \emph{bro}gervenlige nok!
    De skal væk! Vi må være hårde og \emph{bro}tale!

\scene{I rejser sig og afbryder NB.
       Lys tilbage til normalt scenelys.}

\says{I} Tak, tak. Det var desværre alt vi havde tid til i aften.
         Tak til Niels Bohrs bror, Niels Brohr.
         Følg med i næste uge hvor vi tager et kritisk blik på mordet af Georg Mohrs mor, Georg Mor.

\scene{Lys ned}

\end{sketch}
\end{document}

%%% Local Variables:
%%% mode: latex
%%% TeX-master: t
%%% End:
