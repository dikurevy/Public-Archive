\documentclass[a4paper,11pt]{article}

\usepackage{revy}
\usepackage[utf8]{inputenc}
\usepackage[T1]{fontenc}
\usepackage[danish]{babel}
\usepackage[hidelinks]{hyperref}


\revyname{DIKUrevy}
\revyyear{2017}
\version{1.1}
\eta{$3.5$ minutter}
\status{Faktisk sjov}

\title{Science of Nature Science of the Sciences}
\author{Niels, Bitre-Mikkel, Simon, Sebbe}

\begin{document}
\maketitle

\begin{roles}
\role{S}[Vivien] Studerende der lige har skrevet sin første videnskabelige artikel
\role{A}[Kim] Agent/sælger der udgiver artikler mm.
\role{X}[Simba] Instruktør
\end{roles}

\begin{props}
\prop{Selvskreven videnskabelig artikel}[] Bare nogle papirer
\prop{En rulle citeringer}[] Skal fylde meget; se sketchkrop
\end{props}


\begin{sketch}

\scene{Lys op.  A står i et meget ekstravagant og billigt udseende jakkesæt
(tænk Donald Trump) og tjekker sin datafon.}

\scene{S kommer ind med sin artikel under armen.}

\says{S}[frisk, til A] Goddag!  Jeg, en ph.d.-studerende, har lige skrevet min
første \emph{videnskabelige artikel}.  Nu vil jeg være IT-branchens Lady Gaga!

\scene{S giver artiklen til A.}

\says{A}[bladrer løst i artiklen] Du er kommet til den rette!  Her hos Science
of Nature Science of the Sciences Publishing er du i gode hænder.  Så lad os
lægge en plan for hvordan du bliver IT-branchens, øh, René Diff.

\says{A} Nu har jeg læst din artikel, og jeg tror godt vi kan få den udgivet.  I
``IEEEE Transactions on Computers'' \act{siges hurtigt}.

\says{S} ``I Triple-E Transactions on Computers''!  Super!

\says{A} Nejnej, ``I \emph{Quadruple}-E Transactions on Computers'', det polske
tidsskrift.  Det tredje E er for ``extra''.  Tænk på det som videnskabens
Ekstra-Bladet!

\says{S} Men jeg vil gerne have at Mark Zuckerberg fra Silicon Valley ser min
artikel!

\says{A} Nej, du skal være fremsynet.  Amerika er færdigt, kaput.  Du kan
\emph{opstarte} Warszawa Valley -- Polen er et helt friskt marked.  Du bliver en
\emph{trendsetter}!

\says{S} Men er Polen ikke gold og øde?

\says{A} Jo, men det er ligesom at være den første mand på månen!

\says{S} Månen er også gold og øde.

\says{A} Æhh...  Nej, den \emph{anden} side af månen.

\says{S} Der er altså heller ikke noget.

\says{A} DET ER DET DE VIL HAVE DIG TIL AT TRO!

\scene{A fanger sig selv i at være gået lidt for langt.  A sunder sig og indser
at han jo skal sælge et produkt, og retter på sit slips.}

\scene{A tager armen rundt om skulderen på S.}

\says{A} Men nu skal du huske: Det der \emph{virkelig} betyder noget i akademia
er \textbf{citeringer}.

\says{S}[bliver drømmende] Ja, jeg kan se det for mig: En citering af \emph{min}
artikel i en banebrydende artikel fra ICFP-konferencen...  Øjj!

\says{A} Nej, \emph{tusinde} citeringer!

\says{S}[mere drømmende] Orv, TUSINDE citeringer af min artikel...

\says{A} Ja, \emph{tusinde} citeringer i clickbait-artikler på Facebook!

\says{S} Hvad, Facebook?  Jeg tænkte på peer reviewede artikler i videnskabelige
tidsskrifter.

\says{A} En like er jo \emph{også} en slags peer review.

\says{S} Ahrrr, er du sikker på det?  Jeg spørger lige min vejleder.

\says{A}[fast i stemmen] Nejnej, det er ikke nødvendigt.  \act{beroliger S} Hør
her, nu ringer jeg til en\ldots øh\ldots \emph{kollega} og hører hvad han
siger.

\says{A}[tager sin smartphone op] Ja, hallo?  Jeg har den her artikel med meget
tekst\ldots der er en overskrift, vent, \act{bladrer} \emph{fem overskrifter},
og et par citeringer.  Er den ikke fin?

\scene{A laver thumbs up til A.}

\says{A}[i telefonen, fortsat] Jo, der er i hvert fald tre illustrationer i.

\says{A}[holder telefonen væk fra munden og taler til S] Ja, han sagde at den er
lige til en Oscar!

\scene{A tager telefonen op igen.}

\says{A}[i telefonen, fortsat] Godt, jeg glæder mig til at du kommer ud!

\says{S} Vent, kommer ud af hvad?

\says{A} \act{lægger på} Øh, et møde med... Steve Jobs!

\says{S} Wow, hils!

\scene{A kigger ned på sin telefon, som han jo lige har lagt på.}

\says{A} Nå, men du ville gerne have 1000 citeringer af din artikel.  Ved du
hvad, kan du ikke lige citere min gode kollega Troels' artikel, så glider
processen ligesom lidt nemmere.

\says{S} Tjooh, men hvad handler Troels' artikel om? Er det \act{er excited}
nyskabende compilerforskning eller \act{er skeptisk} måske bare et funktionelt
arraysprog til GPU'en?

\says{A} Æhh, nej, det er faktisk en opskrift.  På banankage.  Den er ret god!
Du kan bare sætte den ind mellem reference 17 og 18, så ser folk det ikke.

\says{S} Jeg har kun 4 referencer.

\says{A} Ja, men jeg har en hel liste her, så det får vi lige ordnet.

\scene{A tager en rulle citeringer op af lommen og taper den fast til artiklen.
Han slipper rullen, og den ruller hen ad gulvet.}

\says{S} Åhrja, fedt mand, nu er den meget længere!

\says{A} Ja, angående længden, så må den jo maks være tyve sider.

\scene{A river de forreste sider af, så der kun er referencerullen tilbage.}

\says{S}[kigger på de afrevne sider] Men det er jo det \emph{jeg} skrev!

\scene{Sketchen slutter.}

\scene{Forslag: Der afspilles en punchlinejingle, og S og A
bukker.  Jinglen skal være den fra lasagnacat-YouTube-kanalen:}

\noindent\url{http://harlem.dikurevy.dk/~niels/dikurevy2017/av/jingle.flac}

\scene{Alternativt kan en bedre punchline indsættes.  Alternativt kan A gøre
ekstra opmærksom på at S's artikel er dårlig og at det er derfor den bliver
revet af.}

\scene{Lys ned.}

\end{sketch}
\end{document}
