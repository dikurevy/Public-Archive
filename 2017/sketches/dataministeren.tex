\documentclass[a4paper,11pt]{article}

\usepackage{revy}
\usepackage[utf8]{inputenc}
\usepackage[T1]{fontenc}
\usepackage[danish]{babel}


\revyname{DIKUrevy}
\revyyear{2017}
\version{1.01}
\eta{$4$ minutter}
\status{Ikke medister}

\title{Dataministeren}
\author{Mikkel Kjær Jensen, Niels Gustav Westphal Serup, Simon S. Linneberg,
Nana Girotti, Caroline Miller, Brainfuck}

\begin{document}
\maketitle

\begin{roles}
\role{VO}[René] Voice over
\role{DM}[Romeo] Dataminister
\role{J}[Brandt] Journalist
\role{D}[Kim] Datalog
\role{X}[Simba] Instruktør
\end{roles}

\begin{props}
\prop{Et bord}[]
\prop{En kasse med plads til to mapper}[]
\prop{En fysisk mappe med navnet ``Dataminister''}[]
\prop{En fysisk mappe med navnet ``Sigurd''}[]
\prop{En dildo}[]
\end{props}


\begin{sketch}

\says{VO} Folketinget TV præsenterer nu, fra statsminister Lars Løkke Rasmussens
ottende regering i denne regeringsperiode, Danmarks nye dataminister!

\scene{Lys op.  J og DM står ved et bord.}

\says{J} Velkommen til.  Det er jo spændende at have denne nye ministerpost i
nationen.  Er du sådan en rigtig data-fan?

\says{DM} \act{slår ud med armene} Ja, data-dit og data-dut, det er lige mig!
Mange siger at data er det nye sort.  Og hvad er også sort?  Olie!  Data er jo
det nye olie.  Det er derfor at regeringen har erstattet Miljøministeriet med
Dataministeriet!

\says{J} Hm.  Hvad siger du til kritikken om at regeringen mest har skabt din
stilling for at distrahere privatlivsfortalere?

\says{DM} Tak for spørgsmålet.  Det er utroligt vigtigt at vi passer på vores
resurser.  Der skal jo også være data til vores børn og børnebørn.  Det skulle
nødigt være dataen der presser under studenterhuen.

\says{J} Øh, hvaad?  Data under--\act{bliver afbrudt}

\says{DM} Vi sammenligner jo ofte os selv med Norge.  Norge var først ude med at
\act{bliver lidt indespist} bøffe Nordsøolien, men nu, \act{bliver opstemt}
\textbf{nu} skal vi være first-movers når det kommer til at håndtere borgernes
data!

\says{J} Hvis det er din plan, hvad mener du så om at Norge lige har oprettet
verdens største datacenter med CO2-neutral hydroenergi på 120.000 kvadratmeter,
og hvordan vil I konkurrere med det?

\says{DM} Pißß\ldots Cloud?

\says{J} Men hvor?

\says{DM} Joh, jeg indrømmer gerne at jeg er ``ny i job'', men jeg har set at vi
i ministeriet bruger det her smarte \act{siges med tyk dansk accent}
``Dropbox'', og så tænkte jeg at det kan man da bare rulle ud på nationalt plan.

\says{J} Øh, hvordan vil det virke?  Kan du uddybe det?

\says{DM} Ja.  \act{træder frem, er stolt} Vi laver Dropbox til heeeele
Danmark!

\says{J} Så I køber seks millioner Dropbox-konti?

\says{DM} Nej da, vi har én, og så deler vi kodeordet, ligesom vi gør med vores
FE-mappe i ministeriet (udtales F E, som i ``Forsvarets Efterretningstjeneste'').

\scene{D kommer trampende ind på scenen og skubber J væk fra bordet.  J trækker
på skuldrene og går ud af scenen.}

\says{D} HVAD!  Du kan da ikke putte danskernes data i en Dropbox-mappe!

\says{DM} Hvem er du?

\says{D} Jeg er datalog!  Så jeg ved hvad jeg taler om!  Er du skør?

\says{DM} Nejda, tænk bare på hvor meget af den danske infrastruktur vi kan
integrere i Dropboxen: Patientjournaler, skatteoplysninger, feriebilleder -- og
vi slipper for at svare på aktindsigter, for borgerne kan bare finde sagsfilerne
selv!

\scene{D facepalmer.}

\says{D} Nejnejnej, vores allesammens private data i en Dropbox\ldots I kunne i
det mindste have lagt det i git!

\scene{D tror at han er klog.}

\says{D} Hvordan skal en borger i jeres Dropbox overhovedet tilgå sine filer?

\says{DM} Simpelt.  Hver borger har sin egen mappe!  Så kan du bare lægge
feriebillederne direkte i din mors mappe.

\says{DM} Lad mig illustrere hvordan Dropbox fungerer.  \act{tager kassen frem}
Vi har en kasse, den kalder vi ``Dropbox''.  Inde i den der har vi en masse
mapper med navne på.

\scene{DM tager sin mappe op.}

\says{DM} Her har vi for eksempel min egen mappe.  Den indeholder alle de
vigtige love jeg har været med til at skrive.

\scene{DM stiller sin egen mappe på bordet og tager Sigurds mappe op.}

\says{DM} Og her har vi så Sigurds mappe. \act{åbner den} Data-cadabra, og vi
kan se at han var inde hos lægen i går og få fjernet fodsvamp!  Og der er hans
browserhistorik, uha den frækkert.

\scene{D tager ministerens mappe op og kigger lidt på den.  Der ryger en dildo
ud.}

\says{DM} Hvad er det du laver?  Det er jo privat!

\says{D} Nånånå, så I forhindrer ikke folk i at kigge i andres private mapper?
I skulle have hyret mig, så havde hver person haft sin egen git branch, og, og
\act{bliver vildt spændt} man ville kunne lave pull requests hos hinanden!  I
har jo slet ikke tænkt det her igennem!

\says{DM} Det er jo det rene hurlum-hack, man kan i hvert fald ikke kigge i
andres mapper, for det er nemlig ulovligt!  Og vi har skam tænkt over hver en
detalje.  I tror mig nok ikke, men i ministeriet arbejder der faktisk to
personer -- der begge hedder Preben!  \act{er helt forundret} Og så var det at
vi tænkte: Vi kan ikke have at folk har samme mappe bare fordi de deler navn --
det ville jo være noget... rod!  Derfor bruger vi også efternavn.  Hasta la
tasta og problemet er løst!

\says{D} \act{er lidt kølet ned, scroller på sin smartphone} Ej, men inde på
Danmarks Statistik står der at der er en hel del personer med navnet Rikke
Pedersen--\act{afbrydes}

\says{DM} Ahrmen, så må Rikke da lige snakke med \act{gestikuler med at det er
en anden Rikke} Rikke om at dele deres mappe, ikke?

\says{D} Men 1600 Rikker, hvordan vil du dog--\act{afbrydes}

\says{DM} Nåååhrr, vent lidt, jeg får en idé!  I mappenavnet skriver vi også
bare\ldots deres CPR-nummer!

\scene{Sluk bål, mammut for.  Måske facepalmer D igen!}

\end{sketch}
\end{document}
