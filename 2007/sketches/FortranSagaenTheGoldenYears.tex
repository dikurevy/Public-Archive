\documentclass[a4paper,11pt]{article}

\usepackage{revy}
\usepackage[utf8]{inputenc}
\usepackage[T1]{fontenc}
\usepackage[danish]{babel}

\revyname{DIKUrevy}
\revyyear{2007}
% HUSK AT OPDATERE VERSIONSNUMMER
\version{1.2$\delta$}
\eta{$2$ minutter}
\status{Færdig}

\title{Fortran Sagaen - The Golden Years}
\author{Munter og Uffe}

\begin{document}
\maketitle

\begin{roles}
\role{VO}[Uffe] Voiceover
\role{D1}[Johan] Datalog
\role{D2}[Allan] Datalog
\end{roles}

\begin{props}
\prop{2 stole}[Rekvisitgruppen] Placeholder 2 og 3
\prop{Hulkortsstrimler}[Rekvisitgruppen] Placeholder 2 og 3
\prop{bukser der kan trækkes godt op om livet}[bo ælling] bo ælling
\prop{seler}[Bo ælling] bo ælling
\prop{bælter}[bo ælling] bo ælling
\prop{1 par hvide tennissokker}[bo ælling] bo ælling

\prop{bukser der kan trækkes godt op om livet}[johan] johan
\prop{seler}[johan] johan
\prop{bælter}[johan] johan
\prop{1 par hvide tennissokker}[Johan] Johan

\end{props}

  
\begin{sketch}
\scene{D1 og D2 skal gerne have tøj på der er så 70'er-nørde-agtigt
  som muligt}

\scene{Tæppe for. Lys slukket}

\says{VO} Hvem kender ikke den fantastiske spændingsserie, Fortran
Sagaen?\act{Pause mens fysikerne råber} Med uforglemmelige øjeblikke som ``Tabte Blyanter'', ``Terminerende
eller ikke terminerende programmer'' og ``Hvem tager i døren?'' har
serien formået at skabe en enorm fanskare på tværs af kontinenter,
religioner, etniciteter og studier. DIKU revyen er derfor stolte af at
kunne præsentere et lykkeligt gensyn med et potpourri af de tidligste
afsnit under titlen: ``Fortran Sagaen - The Golden Years''.

\scene{Lys op. To stole står foran tæppet, drejet lidt væk fra hinanden. På stolene sidder D1 og D2 og
  prikker i hulkortsstrimler}

\says{D1}\act{Prikker et hul i hulkortsstrimlen} Arjj, nu kom jeg til
at stikke den i det forkerte hul.

\scene{Pause. Vent på at publikum råber ``Det sagde hun også igår''}

\scene{D1 og D2 drejer \emph{langsomt} hovederne mod hinanden}

\says{D2} Det kom jeg også til at gøre i sporvognen i morges.

\scene{Pause. D1 og D2 drejer sig tilbage. D1 drejer sig mod D2 og
  fremviser sin \emph{lange} hulkortsstrimmel.}
\says{D1} Se lige denne programstrimmel.
\says{D2}\act{drejer sig mod D1} Den er godt nok lang.

\scene{Pause. Vent på at publikum råber ``Det sagde hun også igår''}

\says{D1} Det er mine sokker også\ldots  blevet.

\scene{D1 og D2 drejer tilbage, væk fra hinanden.}

\says{D2} Er det dig der har lavet denne implementation af algoritmen fra
Turings artikel?

\says{D1}``Über die Wesen des primtalalgoritmus des kvadratischer bubblesort
  anno 1943''?

\says{D2}[hurtigt] Det sagde hun også igår!

\scene{Lys ud, hurtigt, efterfølgende sangnummer igang med det samme}


\scene{Tæppe}

\end{sketch}
\end{document}

%%% Local Variables: 
%%% mode: latex
%%% TeX-master: t
%%% End: 

