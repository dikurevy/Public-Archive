\documentclass[a4paper,11pt]{article}

\usepackage{revy}
\usepackage[utf8]{inputenc}
\usepackage[T1]{fontenc}
\usepackage[danish]{babel}

\revyname{DIKUrevy}
\revyyear{2007}
% HUSK AT OPDATERE VERSIONSNUMMER
\version{19.2418$\diamondsuit$}
\eta{$4.5$ minutter}
\status{Færdig}

\title{Det Private DIKU}
\author{marvin}

\begin{document}
\maketitle

\begin{roles}
\role{I}[Kristine] Introduktør
\role{SB}[Marvin] Institutboss
\role{V}[Johan] En VIP
\end{roles}

\begin{props}
\prop{1 bord}[Rekvisitgruppen] Placeholder 2
\prop{2 stole}[Rekvisitgruppen] Placeholder 2 og 3
\prop{tøj}[Rekvisitgruppen] Marvin, institutleder-agtigt
\prop{tøj}[Rekvisitgruppen] Johan, lektor-agtigt
\prop{tøj}[Rekvisitgruppen] Kristine, introduktør-agtigt
\prop{Nogle papirer i sagsmappe}[Rekvisitgruppen] Placeholder 2
\prop{Avis}[Rekvisitgruppen] Placeholder 2
\prop{Lydeffekt: Bank på dør}[farmand] Johan
\end{props}

  
\begin{sketch}

\scene{Tæppet er for. Foran tæppet kommer I ind.} 

\says{I} Højtærede publikum. Mange mener, at ansatte på DIKU lever i en
støvet, beskyttet verden -- fjernt fra virkeligheden, og endnu fjernere end de
hårde vilkår, der gælder i det private erhvervsliv.

Intet kunne være længere fra sandheden. Udadtil ser det måske ud til, at DIKUs
forskere er samling uengagerede døgenigte, der bare venter på pension -- men
bag kulisserne fungerer alting fuldstændigt som det i 
private erhvervsliv.

Ved hjælp af moderne overvågningsudstyr kan vi nu præsentere Jer for et
eksempel på, hvad der i virkeligheden foregår bag de lukkede døre. Vi stiller
om til institutlederens kontor.

\scene{I laver præsenterende fagter, mens tæppet går fra. I forlader
  scenen diskret. Bag tæppet sidder SB på en stol bag et bord. På den anden side af
  bordet står en tom stol.}

\says{SB}[Med høj, klar røst] Kom ind!

\scene{V kommer ind på scenen. Han ser lidt betuttet ud}

\says{V} Du ville tale med mig?

\says{SB} Ja. Tag plads.

\scene{V sætter sig på den ledige stol}

\says{SB}[Bladrer lidt i nogle papirer. Kigger op.] Vi er ikke helt tilfredse
med din indsats på det seneste. Du har faktisk været en rigtig dårlig
medarbejder.

\says{V}[Skamfuld og bange] Det\ldots   Det forstår jeg ikke. Jeg har arbejdet
hårdt, og gjort præcis, som du har sagt.

\says{SB} Jeg er bange for, at det ikke er godt nok. Du har brudt reglerne.

\says{V}[Ser ned i gulvet] Jeg ved det godt. Jeg skulle ikke have hacket NSA.

\says{SB}[Vredt] Det er slet ikke det, det handler om! \act{Smider avis på
  bordet} Du har udtalt dig til pressen!

\says{V} Jamen, jeg offentliggjorde jo bare resultatet af vores forskning. En
videnskabelig artikel\ldots


\says{SB}[Holder et papir op] Ved du, hvad jeg har her? Det er noget, du har
skrevet under på\ldots

\says{V} Ja, jo\ldots

\says{SB} Det er en non-disclosure-agreement. En NDA! Du har selv skrevet
under på, at du slet ikke må fortælle offentligheden om, hvad du laver her på
DIKU.

\says{V} Det blev jeg jo ligesom nødt til at skrive under på, for at få
jobbet\ldots

\says{SB} Er du slet ikke klar over, hvad der vil ske, hvis vores konkurrenter
får nys om, hvad vi går og laver her?

\says{V} Nej, hvad?

\says{SB} Spil ikke uskyldig. Du ved udmærket godt, hvad der sker! De vil
fluks gå til politikerne. Alle de problemer, vi har løst for længst:
Talegenkendelse, kunstig intelligens,
det perfekte brugerinterface, sorteringsalgoritmer, der kører i kvadratisk
tid, løsning af NP-komplette problemer i negativ lineær tid -- de vil finde ud
af, at det hele allerede \emph{er} løst. Og så får vi ikke forskningspenge til
at løse dem!

\says{V} Nå, jo. Det kan jeg godt se. Så går alle pengene bare til dansens
æstetik og historie eller sådan noget.

\says{SB}[Peger i papirer] Og hvad er det her? Et kursus? Du har udbudt et
kursus!

\says{V}[Lidt stolt] Ja, jeg tænkte, at det ville være godt at lære de
studerende lidt om\ldots

\says{SB} Sig mig, læser du slet ikke det, du skriver under på? Det er jo den
anden paragraf i din NDA: Ingen. Kurser!

\says{V} Jo, men officielt har vi jo pligt til at undervise.

\says{SB} Det er jo derfor der er den bestemmelse. Du må kun udbyde kurser,
der er kedelige, ubrugelige, eller begge dele. Ellers går det kun en vej!

\says{V}[Modløs] Jaja, jeg ved det. Interessante kurser giver bare flere
studerende.

\says{SB} Ja, og hvornår skal vi så få tid til at forske. Der er en grund til,
at vi har vores NDA!

\says{V} Jeg synes altså ikke om det. Jeg tror, jeg vil finde et andet job.

\says{SB}[Kattevenlig] Virkelig? Jamen, det synes jeg bare, du skal
prøve\ldots

\says{V} Jeg har faktisk søgt flere steder allerede. Men ingen vil have
mig. Det er meget underligt. Der er trods alt mangel på dataloger\ldots

\says{SB}[Ser påtaget uskyldig ud, kigger op] Ja, det virker da\ldots 
underligt. Næsten\ldots  sært.

\says{V}[Mistænksom, let anklagende] Ja, meget. Man skulle næsten tro, at der
var en aftale om, at ingen måtte ansætte mig\ldots

\says{SB}[Forsvarende] Hvad, jobkarteller? Det bruger vi ikke her!

\says{V} Det er heller ikke det, jeg har fundet ud af.

\says{SB} Det er der desuden ikke noget galt med. Ministeren siger, det er
helt i orden.

\says{V} Men andre instituter bruger det jo ikke?

\says{SB}[Fortsætter sin egen tale] Faktisk er det jo \emph{godt} for
arbejdsmarkedet med jobkarteller. 

\says{V} Hvordan kan det være godt for noget.

\says{SB} Og det er en kæmpe fordel for de ansatte!

\says{V}[Rejser sig og slår i bordet] Altså! Med den slags kan jeg jo blive
hængende her, til jeg skal på pension!

\scene{SB fløjter uskyldigt, undlader at svare.}

\says{V}[Vredt] Er det det, du er ude på? At jeg skal bruge min tid her, uden
åbenlyst engagement, uden at se ud som om, jeg laver noget, uden at udbyde
ordentlige kurser -- og så bare sidde på min flade indtil pensionen?

\says{SB}[Lokkende] Det virker for alle dine kollegaer\ldots

%\says{V}Jeg finder mig ikke i det. Så hellere være arbejdsløs! \act{Vender sig
%  om og går}
%
%
%\says{SB}[Råber efter ham] Pas nu på, du ikke kommer ud for et uheld!
%
\scene{Tæppe}

\end{sketch}
\end{document}

%%% Local Variables: 
%%% mode: latex
%%% TeX-master: t
%%% End: 

