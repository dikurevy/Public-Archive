\documentclass[a4paper,11pt]{article}

\usepackage{revy}
\usepackage[utf8]{inputenc}
\usepackage[T1]{fontenc}
\usepackage[danish]{babel}

\revyname{DIKUrevy}
\revyyear{2007}
% HUSK AT OPDATERE VERSIONSNUMMER
\version{1.0}
\eta{$5.5$ minutter}
\status{Ikke færdig}

\title{DIKU Bli'r}
\author{Uffe og Uffe Productions\texttrademark, samt marvin, Munter og Guldfisk}

\begin{document}
\maketitle

\begin{roles}
\role{A1}[Hanne] Autonom
\role{A2}[Kristine] Autonom
\role{A3}[Daniel] Autonom
\role{P1}[André] Politi
\role{P2}[Munter] Politi
\role{P3}[Guldfisk] Politi
\role{M}[Uffe] Pegemand
\end{roles}

\begin{props}
\prop{3 politiskjorter}[Rekvisitgruppen] P1, P2, P3.
\prop(sorte bukser)[André] P1, André
\prop(sorte bukser)[Munter] P2, Munter
\prop(sorte bukser)[Guldfisk] P3, Guldfisk
\prop{3 politihjelme}[Rekvisitgruppen] P1, P2 og P3.
\prop{3 politistave}[Rekvisitgruppen] P1, P2 og P3.
\prop{Sort hættetrøje}[Hanne] A1, Hanne
\prop{Sort hættetrøje}[Kristine] A2, Kristine
\prop{Sort hættetrøje}[Daniel] A3, Daniel
\prop(bukser)[Hanne] A1, Hanne
\prop(bukser)[Kristine] A2, Kristine
\prop(bukser)[Daniel] A3, Daniel
\prop(partisanertørklæde)[Hanne] A1, Hanne
\prop(partisanertørklæde)[Kristine] A2, Kristine
\prop(partisanertørklæde)[Daniel] A3, Daniel
\prop(elefanthue)[Rekvisitgruppen] P1, André
\prop(elefanthue)[Rekvisitgruppen] P2, Munter
\prop(elefanthue)[Rekvisitgruppen] P3, Guldfisk
\prop{1 brosten}[Rekvisitgruppen] A1, Hanne
\end{props}

  
\begin{sketch}

\scene{A2 og A1 sidder på hver sin ølkasse. De har autonom-agtigt tøj og elefanthuer på. Dæmpet belysning, evt. lille spot kun på de to}

\says{A2}[Hviskende, surt] Hvor fanden bliver de af? Det kan ske når
som helst nu. Ethvert øjeblik kan det myldre ind med betjente. Vi har brug for folk til at kæmpe imod den autoritære fasciststat.

\says{A1}[Hviskende, overbevist] Ja, klart! Ned med svinene!
\act{pause, bliver usikker} \ldots  men, øhhh\ldots nu har de truet
med at smide os ud i årevis. Og der er stadig ikke sket
noget. \act{bliver mere sikker} De tør jo ikke. De ved, at der bliver
ballade.

\says{A2} Nå, men har du husket huskelisten for besættere:

\says{A2} 1. Ikke tage stilling til flytning
\says{A1} Check!
\says{A2} 2. Ikke forberede noget
\says{A1} Check!
\says{A2} 3. Bliver overrasket over deadline
\says{A1} Ja, klart. Den har vi i hvert fald styr på!
\says{A2} 4. Blive forurettet og besætte huset
\says{A1} Det gør vi nu.
\says{A2} 5. Kræve nye bygninger for 1 krone
\says{A1} Kan vi ikke vente med det til de rent faktisk tilbyder os en
bygning?

\says{A2} Arh ok. Har du banneret klar?

\says{A1} Jep!

\scene{A1 folder et banner ud med teksten ``Til salg inklusive 1024
  blegfede WOW-spillende dataloger fra Hel''}

\says{A2} Nej, nej, nej! Jeg sagde at der skulle
stå "Til salg inklusive 300 stenkastende voldspsykopater fra helvede".

\says{A1} Jeg synes det andet passede bedre.

\says{A2}[Afbryder hviskende] Shhh! Der kommer nogen!

\says{P1}[Bag fra scenen] Denne forsamling er ulovlig. Passér studiet!

\says{A1}[Råber] Aldrig i livet! DIKU Bli'r!

\scene{A2 lusker ud af scenen. P1 kommer ind på scenen og Benny Hill
  delen af sketchen kan begynde.}

\scene{Se kort for beskrivelse af steder}
\scene{iLøber betyder løber i auditoriet}
\scene{uLøber betyder løber udenfor auditoriet}
\scene{Says numrene angiver hvilken rundgang dette foregår i}

\scene{Start: A1+P1 på SC, A2+P2 på SH, A3+P3 på MH.}

\says{0} A1+P1 iLøber rundt på SC og slutter af med at iLøbe ud til SV.

\scene{kort pause}

\says{1-2} A1+P1 iLøber til BH.

\scene{kort pause. PM træder ind på scenen. Når musikken starter peger
PM på MH.}

\says{3} A3+P3 iLøber til MV. P1 uLøber til MH.

\scene{PM peger på SH}
\says{4} A2+P2 iLøber til MH.

\says{5} A3 iStiller sig på MV og rækker tunge af P2, P2 iStiller sig på MH
og kigger søgende rundt i auditoriet. Halvvejs opdager P2 A3 og iLøber
til MC, A3 slutter af med at uStille sig på MV. A2 uLøber til BV.

\says{6} A1 iStiller sig på BH og larmer af P2 som er nået næsten til MV og
som så vender sig om og begynder at iLøbe mod BH. A1 slutter af med at
uStille sig på BH.

\says{7} A2 uStiller sig på BV og larmer af P2 som er nået næsten til
trappen og som så vender sig om og begynder at iLøbe mod BV. A2+P2
slutter af med at iLøbe ud til BV. A1 uLøber til MH.

\says{8} A3 iStiller sig på MV, P1 iStiller sig på MH. A3 fniser, P1 søger.
A3 uStikker armen af MV, A1 iStikker armen af MH og prikker P1 på
skulderen. P1 vender sig om og vil ud af døren, men A1 lukker døren og
holder den. Samtidig lukker A3 døren og holder den mod P3 som prøver
at komme ind. Der afsluttes med at begge døre åbnes og A3 og P1 iLøber
ud. P2 uLøber til MV, A2 uLøber til MH.

\says{9} P3+P2 iLøber mod SH, lige inden trappen opdage P2 at det ikke er en
autonom han forfølger og vender sig om. P1 uLøber til MV.

\says{10} A3 iStiller på MV og P2+A1+A2 iLøber mod MV, lige inden P2+A1+A2
når A3 iLøber A3 mod SV og P2+A1+A2 iLøber efter dem.

\says{11} A3+P2+A1+A2+P1 iLøber til SV.

%\includegraphics{benny_hill_graf.gif}

\newpage

\scene{A1}

\scene{Starter på scenen sammen med P1.}

\says{0} Du bliver jagtet rundt på scenen af P1 og slutter af med at løbe af scenen.
\scene{(kort pause)}
\says{1-2} Du bliver jagtet op fra scenenedgangen, henover midtergangen og op af den fjerne bagtrappe.
\scene{(kort pause)}
\says{3-5} Du bliver stående udenfor døren.
\says{6} Du stiller dig indenfor døren og laver provokerende fagter af P2 som er næsten ovre ved den fjerne midterdør. Du slutter af med at stille dig uden for døren igen.
\says{7} Du løber udenfor salen hen til den nærmeste midterdør.
\says{8} På cue stikker du armen inden for døren og prikker P1 på skulderen. Idet han opdager dig og vil løbe efter dig smækker du døren i og holder den. Du slutter omgangen af med at åbne døren så P1 kan komme ud.
\says{9} Du bliver stående uden for døren.
\says{10} Du løber efter P2 og med A2 i hælene henover midtergangen\dots{}
\says{11} \dots{}og ned af trappen mod scenenedgangen og ud.


\newpage
\scene{A2}

\scene{Starter til højre for scenen (hvor der ikke er trapper op til scenen), skjult for publikum, sammen med P2.}

\says{0-3} Du bliver stående, skjult.
\says{4} Du bliver jagtet af P2 op af trapperne og ud af den nærmeste midterdør.
\says{5-6} Du løber udenfor salen hen til den fjerneste bagdør.
\says{7} Du stiller dig indenfor døren og laver provokerende fagter af P2 som er på midtergangen ca. ved trappen op til den anden bagdør. Du slutter af med at løbe ud af døren igen med P2 lige i hælene.
\says{8-9} Du løber udenfor salen hen til den fjerneste midterdør.
\says{10} Du løber efter A1 (som er lige i hælene på P2) henover midtergangen\dots{}
\says{11} \dots{}og får P1 i hælene idet i løber ned af trappen mod scenenedgangen og ud.


\newpage
\scene{A3}

\scene{Starter til højre for scenen udenfor midterdørene, sammen med P3.}

\says{0-2} Du bliver stående udenfor døren.
\scene{(kort pause)}
\says{3} Du bliver jagtet af P3 henover midtergangen og ud af døren der.
\says{4} Du bliver stående udenfor døren.
\says{5} Du stiller dig indenfor døren og laver provokerende fagter af P2, som står ved den modsatte midterdør. Han sætter efter dig og du slutter af med at træder ud af døren igen.
\says{6-7} Du bliver stående udenfor døren.
\says{8} Du stiller dig indenfor døren og fniser af P1 som står ved modsatte midterdør og ikke kan se dig. Du stikker armen ud af din dør (og i modsatte side kommer en anden arm ind og prikker P1 på skulderen). Idet P1 vender sig om og prøver at komme ud skal du (og armen i modsatte side) smække døren(e) i og holde dem. Udenfor din dør står P3 og buldrer på din dør. Til slut åbnes begge døre og alle forlader salen.
\says{9} Du bliver stående udenfor døren.
\says{10} Du stiller dig indenfor døren og laver provokerende fagter af P2, som kommer løbende med A1 og A2 i hælene. Lige inden de når dig\dots{}
\says{11} \dots{}løber du med alle dem i hælene ned af trappen mod scenenedgangen og ud.


\newpage
\scene{P1}

\scene{Starter på scenen sammen med A1.}

\says{0} Du jager A1 rundt på scenen og slutter af med at løbe af scenen.
\scene{(kort pause)}
\says{1-2} Du jager A1 op fra scenenedgangen, henover midtergangen og op af den fjerne bagtrappe.
\scene{(kort pause)}
\says{3} Du løber udenfor salen hen til den nærmeste midterdør.
\says{4-7} Du bliver stående udenfor døren.
\says{8} Du stiller dig indenfor døren og søger efter en autonom. Du kan ikke finde nogen på trods af at A3 står ved døren lige modsat. Pludselig bliver du prikket på skulderen af A1 og vender dig om og prøver at komme ud af døren hvor prikket kom fra. Den bliver dog smækket i af A1 og du må buldre på døren. Til slut bliver døren åbnet og du kommer ud.
\says{9-10} Du løber udenfor salen hen til modsatte midterdør.
\says{11} Hold døren meget svagt på klem og idet du kan se at hele toget af A3+P2+A1+A2 er kommet forbi flyver du ind af døren efter resten og ned af trappen mod scenenedgangen og ud.


\newpage
\scene{P2}

\scene{Starter til højre for scenen (hvor der ikke er trapper op til scenen), skjult for publikum, sammen med A2.}

\says{0-3} Du bliver stående, skjult.
\says{4} Du jager A2 op af trapperne og ud af den nærmeste midterdør.
\says{5} Du træder ind af døren igen med det samme og kigger søgende rundt i auditoriet. Cirka halvvejs gennem rundgangen opdager du A3 ved modsatte midterdør og begynder at løbe mod ham. Når rundgangen slutter er du nået næsten over til ham.
\says{6} A1 træder ind ved fjerneste bagdør og provokerer dig til at vende om og løbe mod ham. Når rundgangen er slut er du lige ved trappen op til ham, eventuelt et trin oppe.
\says{7} A2 træder ind ved den anden bagdør og provokerer dig til at vende om og løbe mod ham. Du løber hele vejen op af hans trappen og ud.
\says{8} Du løber udenfor salen hen til nærmeste midterdør.
\says{9} Du jager P3 henover midtergangen, idet I næsten har nået døren opdager du der er noget galt og standser op udenfor døren. P3 fortsætter ned af trapperne.
\says{10} A3 træder ind ved modsatte midterdør og du sætter efter ham. Ind af døren bag dig kommer A1 og A2 efter dig, men det ser du ikke.
\says{11} Du jager A3 ned af trapperne mod sceneudgangen og ud.


\newpage
\scene{P3}

\scene{Starter til højre for scenen udenfor midterdørene, sammen med A3.}

\says{0-2} Du bliver stående udenfor døren.
\scene{(kort pause)}
\says{3} Du jager A3 henover midtergangen og ud af døren.
\says{4-7} Du bliver stående udenfor døren.
\says{8} A3 stikker armen ud og smækker kort efter døren i. Når han har gjort det buldrer du på døren i forsøg på at komme ind. Når rundgangen er næsten slut åbnes døren og A3 kommer ud.
\says{9} Du bliver jagtet af P2 henover midtergangen, P2 stopper op, men\dots{}
\says{10} \dots{}du fortsætter ned af trappen og ud.


\end{sketch}
\end{document}

%%% Local Variables: 
%%% mode: latex
%%% TeX-master: t
%%% End: 

