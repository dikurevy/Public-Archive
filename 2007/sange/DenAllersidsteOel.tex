\documentclass[a4paper,11pt]{article}

\usepackage{revy}
\usepackage[utf8]{inputenc}
\usepackage[T1]{fontenc}
\usepackage[danish]{babel}

\revyname{DIKUrevy}
\revyyear{2007}
\version{1.0}
\eta{$3$ minutter}
\status{Mangler mellemspil}

\title{Den allersidste øl}
\author{Guldfisk og Munter}
\melody{Kai Normann Andersen Andersen - ``Den allersidste dans''}

\begin{document}
\maketitle

\begin{roles}  
\role{D}[Bo Ælling] Dreng
\role{P}[Hanne] Pige
\end{roles}

\begin{roles}  
\prop{Hat}[Hanne] D, Bo ælling
\prop{Guld Tuborg}[Rekvisitgruppen] D, Bo ælling
\prop{Kjole}[Hanne] P, Hanne
\prop{Barstol}[Rekvisitgruppen] D, Bo ælling
\end{roles}

\begin{song}

\sings{D} Aldrig har jeg set så smuk en pige
Aldrig var min lykke nær så stor
Den lille skæve tand
derinde i din mund
aldrig har jeg set en bag så rund

\sings{D} Den allerførste øl
har gjort dig smuk
Og drikker jeg lidt fler'
bli'r du et hug
endnu er natten ung
så grib din pengepung
og køb en ekstra
guldtuborg
før vi går

\sings{D} Et møde med din mund,
før vi går hjem
Og flet nu vore tog-
skinner min ven
men du forstår mig ej
det rør' dog ikke mig
at du er nok så udenlandsk
for du dufter så dejligt af gammeldansk

\scene{Parret danser vals.}
\scene{Mellemspil. Indsæt generel naturlideskabelig samtale.}

\sings{D} Den allersidste dans
Før vi går hjem
Jeg ved godt andre si'r,
at du er nem
Caféen? lukker nu

\scene{Tæppet trækkes langsomt for så man kun kan se de to på scenen}
\scene{Scenelyset går ud, spottet fokuserer på parrets overkrop}

\sings{D} Så kys mig lille du
det slutter ikke her i nat
for det er kun begyndelsen kære skat
og nu bunder jeg lige den sidste sjat. \act{S tager hatten af og holder for deres ansigter til et kys.}

\scene{Spottet bliver helt småt på hatten, tæppet går for, lys ud.}

\end{song}

\end{document}
