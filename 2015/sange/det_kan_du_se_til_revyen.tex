\documentclass[a4paper,11pt]{article}

\usepackage{revy}
\usepackage[utf8]{inputenc}
\usepackage[T1]{fontenc}
\usepackage[danish]{babel}


\revyname{DIKUrevy}
\revyyear{2015}
\version{1.0}
\eta{$2$ minutter}
\status{Ikke færdig}

\title{Det kan du se til revyen}
\author{Nana, Phillip, Troels}
\melody{Jullerupfærgeby: ``Det kan du få for en krone''}
% http://harlem.dikurevy.dk/~roschnowski/det_kan_du_faa_for_en_krone.mp3

\begin{document}
\maketitle

\begin{roles}
\role{S}[Mia] Sanger, antagelivis
\role{D}[Jenny] Danser der kommer ind
\role{X}[Klaes] Instruktør
\end{roles}

%Idéen med denne sang er at den fungerer som åbningsnummer til revyen,
%hvor vi synger om de ting publikum vil få at se.  Det er således
%nødvendigt at omskrive visse af versene, men det kan gøres efter
%materialeudvælgelsen, såfremt denne sang medtages.
%\hfill\\
%\hrule
\begin{song}
\scene{Lys op og bandet begynder at spille, herefter kommer Mia ind på scenen og synger.}

\sings{S}
Compilering af Gentoo,
flytteplaner for DIKU
og en brugt-på-druk-SU
ka' du se til revyen

Nog'n der vil på Absalon,
Dataloger på et ton
og en masse Disney-sang'
ka' du se til revyen

SCIENCE-IT-uniform,
nul trafik på eduroam
og en slacker som er dov'n
ka' du se til revyen

Mad på fad og en GT,
sange om ML og C
-plus plus, Java og VB
ka' du se (plus plus) til revyen

Lyd og lys fra TeXnikken,
film med Torben Mogensen
og tylklædte kjolemænd
ka' du se (plus plus) til revyen

Rustur versus prodekan,
musikere i et band
og OR XOR eller AND
ka' du se (plus plus) til revyen

Folk der koder kernen selv,
ord der rimer på sig selv
og et gammelt C-ordspil
ka' du se (plus plus) til revyen

\end{song}

\begin{sketch}

\scene{D kommer ind.}

\says{D} Ej Mia, publikum er slet ikke med på 1970erne længere.  Folk vil have noget meget mere moderne.

\scene{Bandet begynder at spille næste nummer.}

\end{sketch}

\end{document}

