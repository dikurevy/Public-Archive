\documentclass[a4paper,11pt]{article}

\usepackage{revy}
\usepackage[utf8]{inputenc}
\usepackage[T1]{fontenc}

\revyname{DIKUrevy}
\revyyear{2015}
\version{1.0}
\eta{$7$ minutter}
\status{Faerdig}

\title{Hvad har du lavet på din datamat?}
\author{Bette-Mikkel, Niels, Troels, Phillip}
\melody{Cirkusrevyen 1967: ``Hvem har du kysset i din gadedør?''}
%\footnote{se texfil for repræsentativ version af sangen}
% redigeret version som passer til sketchens flow på: http://harlem.dikurevy.dk/~storgaard/hvem_har_du_kysset.mp3
% SNAK MED A/V
\begin{document}
\maketitle

\begin{roles}
\role{BI}[Jenny] Iben, klædt i natkjole og nathue
\role{BE}[Brainfuck] Betjenten, uniformeret
\role{P1}[Maya] Postbud, postuniform
\role{P2}[Simon] Postbud, postuniform
\role{X}[Bette-Mikkel] Instruktør
\end{roles}

\begin{props}
\prop{Pizzabakke}
\prop{Pizza}
\prop{Seng m. dyne og pude}
\prop{Mappedatamat}
\prop{Et nummer af PROSAbladet}
%\prop{Jewel case m. cd i}
\prop{Dommerfløjte}[]
\prop{Cykel x2}[begge med ringeklokker]
\prop{Ubuntulyd}[http://www.youtube.com/watch?v=CQaEXZ-df6Y] %SNAK MED A/V
\prop{Griegs 'Peer Gynt No. 1 Morning Mood'}[http://www.youtube.com/watch?v=JBxBAXVP8IA] %SNAK MED A/V
\end{props}

\begin{sketch}

\scene{Tæppe går fra, AV spiller de første 10 sekunder af Griegs 'Morning', mens lyset på scenen blænder op ligesom en sol. HøjTeX viser en blå himmel med skyer og en stor gul sol.}

\scene{BI sætter sig op i sin seng, strækker sig, gaber, og smasker tilfreds.}

\says{BI} Åhr, hvor er det en dejlig morgen. I dag vil jeg skrive endnu mere på mit DEJLIGE speciale! 
          - men først, mor-gen-mad!
\scene{IB går beslutsomt i køleskabet, og tager en pizzabakke frem. Hun placerer bakken på sit bord}
\says{BI} Aaaaah. Morgenmaden \textit{er} jo det vigtigste målti-
\scene{P1 ringer med cykelklokken off-stage og kommer ind, vinkende til BI}
\says{P1} Godmorgen godmorgen, BI! 
\says{BI} Godmorgen, Post! 
      Har De post med til mig, Post? 
\says{P1} Nej, beklager, BI. 
       Men jeg tog nu omkring alligevel, ifald det ændrede sig undervejs! 
\says{BI} Hvor fint! Kunne De ikke tænke dem en kop ka- 
\scene{P2 ringer med cykelklokken off-stage og kommer ind, vinkende til BI}

\says{P2}Godmorgen, godmorgen, BI! 
\says{BI}Godmorgen, Post! 
\says{P2}[givende et venligt hilsende nik til P1] Godmorgen, Post! 
\says{P1}Godmorgen, Post!
\says{P2}[giver en kuvert til P1] Jeg har post til dem, Post!
\says{P1}[tager imod] Tak for posten, Post!  \act{Åbner kuverten og finder endnu en kuvert indeni.}  Næh, post til omdeling!
\says{P1} Det er til Dem, Jenny!  Jeg har det nye nummer af PROSAbladet med til Dem, frøken! 
\says{P1}[let undrende]Nå nå? Men udkommer det ikke først i morgen? 
\says{P1}Ganske så, men der har jeg fri!
\scene{P1 og P2 slår en hjertelig latter op}
\says{P1} Værsågod, frøken Iben! Så vidt vides er der endda frit programmel vedhæftet til læseren.
\scene{P1 rækker IB et nummber af PROSAbladet}

\says{P1} Javel. Jamen så kan jeg jo tage hjem og slappe af!
\says{P2} NÅ?! Det gjorde du da også i går.
\says{P1} Ja. men der blev jeg ikke færdig.
\scene{P1 og P2 slår atter en hjertelig latter op}
\says{BI}[grinende] Nu er det vist godt med Jer, Post!
\says{P1 og P2}[let klukkende] Åhja. Asti-afsted med os.
\says{Alle}Farvel farvel..!
\scene{P1 og P2 ringer med cykelklokken, og trækker cyklen med ud af scenen. IB vinker efter dem} 
\scene{IB sætter sig på sin stol, og læser i PROSAbladet}

\says{BI} Hillemænd - En ny udgave af Tor! Det skal på min datamat, lige på en studs.

%\scene{BI vender avisen et par gange for at finde det CD Jewel Case, der er tapet fast i avisen. BI tager caset ud, og lægger det på bordet.}
%
%\says{BI} Hvor var det nu jeg lagde min datamat?
%
%\scene{BI løfter nathuen, klør sig i hovedbunden, og sætter huen på igen.}
%
%\says{BI} Nååhr, den er stadig i Sleep Mode. Den syvsover, haha!
%
%\scene{BI går hen til sengen, flytter hovedpuden, og vinker til sin datamat, som ligger under puden.}
%
%\says{BI} Godmorgen datamat!
%
%\scene{BI vinker til datamaten, som spiller Ubuntus startlyd. BI samler datamaten op, sætter den på bordet og sætter sig foran den.}
%
%\says{BI} Så, lad os nu se hvordan der hersens går for sig.
%
%\scene{AV spiller tikkende ur, mens BI venter i fem sekunders tid, og kigger på sit lomme-ur.}
%
%\says{BI} Så, nu er den lige på trapperne. Iiih, hvor er det spændende!
%
%\scene{BI klapper i sine hænder, og hæver sin pegefinger en halv meter over datamatens enter-knap. BI tæller ned, mens hans finger nærmer sig enter-testen.}
%
%\says{BI} Eeeeen, tiiiiiii, elllllleve, NU! Enter!
%
\scene{BI lægger PROSAbladet ved datamaten.}

\scene{Det banker meget alvorligt på døren. Ind træder BE, med komisk strækmarch!}

\says{BE} Holdt! Det er forbudt!

\says{BI} Nej, det er det da vel ik-øøøøøhh!

\says{BE} Jo ho. Dét der Tor? Det MÅS man ikke!

\says{BI} Jo!

\says{BE} Nej, fordi det er ulovligt! 

\scene{BE klapper datamaten hårdt sammen.}

\end{sketch}

\begin{song}
\sings{BI}
Jeg er 30 år og bachelor,
her på data( - )logi
\scene{mellem anden og tredje stavelse i ordet datalogi, udfører BI et kækt dansetrin, i stedet for at synge en stavelse}
\sings{BE}
Vores log har set du brugte Tor, 
her til morgen, klokken ni.   
\sings{BI}
Hr. Betjent, det må man gerne
Har du måske glemt din hjerne?
    
\sings{BI og BE}
Hvad har du (jeg) lavet på din (min) datamat, 
i dit (mit) hackeri?
\sings{BI}
Jeg har blot øvet på min kandidat
Det kan jeg godt li'!
    
\sings{BE}
Næh, du vil bare surfe ud'n kontrol
Så du kan se på død og vold
Fodre et kæledyr med alkohol
Og andet svineri!
        
\sings{BI}
Kan du så gå din vej med dit forhør?
SÅ unødvendig' 
\scene{BI trækker opgivende på skuldrene og ruller med øjnene}

\sings{BI og BE}
Hvad har du lavet på din (min) datamat
I dit hackeri?!
\sings{BE}
Fyyyyh..!
\end{song}
\scene{mellemspil start, taledel starter}
\begin{sketch}
    \says{BI}
Nej, men du forstår slet ikke hvad jeg har gang i. 
Det er en BOMBE under den nuværende infrastruktur.
jeg kommer til at sprænge muren for effektivitet!
\says{BE}En BOMBE??!!
\scene{mellemspil slut, taledel slutter}

\end{sketch}
\begin{song}
\sings{BE}
Du skal ikke bombe noget som helst
Det er regulær terrór
\sings{BI}
Hr. Betjent, du misforstår mig vidst
Det var blot en metafor!
\sings{BE}
Nej, men det kan vær' det samme
Metaforer de er klamme!

\sings{BE og BI}
Hvad har du lavet på din datamat
I dit hackeri?
\sings{BE}
Hvem så dig sidste ug' i Labitat?
Det gjord' politi!
\end{song}
\scene{mellemspil start, taledel starter}
\begin{sketch}
\says{BE}Nå, og hvad har vi så her?
Det er måske en... naaaarko-pizza?
\says{BI}Nej, den er fra Duniyas, det er bare mug.
\says{BE}[Rynker på næsen]Nå, men det bliver man vel også høj af!
\end{sketch}
\scene{mellemspil slut, taledel slutter}
\begin{song}
\sings{BE}
  Jeg tror at du har hentet programmel 
  fra www.kriminel (udtales v - v - v (dot) kriminel)
  Den slags man bruger til at slå ihjel 
  Oh fy, oh føj, forbudt! 

  Deeeeeeeeet har man bestemt derned' i Bruxelles 
  Fine lille snut! 
\sings{BE og BI}
  Hvad har du (jeg) kodet på din datamat? 
  I dit (mit) hackeri? 
  \sings{BE}[får en meget god, meget ond idé]
BE: Nååååå...! 
\end{song}
\scene{mellemspil start, taledel starter}
\begin{sketch}
    \scene{BE laver et par fakter, hvor han prøver på at få startet en sætning, 
   og finder til sidst den rette grimasse. 
BE taler nu med den blideste, mest pædagogiske betjent-stemme.}

\says{BE} Må jeg ikke se engang, hvad det er for et fint og storartet 
projekt, De er i gang med? 
%\scene{BI undrer sig, men accepterer hurtigt at BE har ændret sin attitude 
%fuldstændigt - nu har hun jo en chance for at vise lægmand noget datalogisk.}
%
%\says{BI}Jo Hr. Betjent! Se her engang: 
%         Jeg har manipuleret med nogle kodestykker, som jeg lånte af en ven. 
%         Simpelthen har jeg forbedret noget af den ganske danske infrastruktur - 
%         I fremtiden vil ingen danskere have problemer, når de stempler 
%         ind eller ud, 
%         på hverken lokomotiv, S-lokomotiv, i tunnelbanen eller på omnibussen! 
\says{BI}
      Jo, ser De, Hr. Betjent: 
      Et. Fungerende. Rejsekort. 

      \scene{BI afslører et kæmpe Rejsekort, som hun hæver stolt i vejret. 
    Pasbilledet er af BI, som både har flot hat på, smiler et kæmpe 
tandsmil, og vender lidt med siden til, og viser thumbs-up.}


\scene{BE sætter sin fløjte for munden, puster hårdt og aggressivt og 
peger på bandet.}

\says{BE}[skarpt kommanderende] MUSIK! 
\scene{mellemspil slut, taledel slutter}
\end{sketch}
\begin{song}
\sings{BE}
Du er sand'lig haft det ganske nemt, 
\sings{BI}[tilbagelænet, udglattende]
Jeg har fikset Rejsekort 
\sings{BE}[gradvist mistende besindelsen]
Du har forbrudt dig på danske softwarepatenter. Det må man ikke! IH! DET.. DET ER IHVERTFALD BARE MEGA ULOVLIGT!
\scene{musikken sænker tempoet og går i stå i løbet af BEs raserianfald, og spiller videre når BE når til næste linje}
\sings{BE}[Prøvende på at genfinde melodien]æh .. har du overhod'et spurgt?
\scene{BE genvinder besindelsen, og retter på sin hat}
\sings{BI}
Hr. Betjent, du må vær' ra-ar 
Lad mig bare kort forklare. 
\sings{BI}
Det, som jeg laved' på min datamat, 
i mit hackeri 
var blot at fixe hele Rejsekort, 
så, tag du bare fri 

\sings{BE}
Men den slags rebeller kan vi ikke li' 
I vores demo-bureaukrati, 
og inden at du har fået talt til ti 
Da bli'r du sat fast! 
Ingen' dataloger de må røre ved 
Statens kort af plast! 

\sings{BI og BE}
  Så nu er du (jeg) anholdt i dit (mit) hackeri 
  Du' under arrest! 

\sings{BI}
BI: ÆV! 
\end{song}
\scene{BE lægger BI i håndjern, og bandet spiller BI og BE ud, som 
marcherer i takt, og vinker storsmilende til publikum.}

\scene{HøJTeX: "BI blev siden idømt i alt 6 års fængsel, for brud på 
Straffelovens §263 (Hackerparagraffen) stk. 2, §263a , Patentlovens §57 og Ophavsretslovens §76."}
\\\\\
NOTE: de relevante paragraffer skal i det trykte program 
\end{document}
