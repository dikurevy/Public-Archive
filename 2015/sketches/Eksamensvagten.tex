\documentclass[a4paper,11pt]{article}
\usepackage{revy}
\usepackage[utf8]{inputenc}
\usepackage[T1]{fontenc}
\usepackage[danish]{babel}

\revyname{DIKUrevy}
\revyyear{2015}
% HUSK AT OPDATERE VERSIONSNUMMER
\version{1.0}
\eta{$3.5$ minutter}

\status{Færdig}
\title{Eksamensvagten}
\author{Mikkel, Troels, Nana, Spectrum, Ronni, Søren, Sebastian, Niels}

\begin{document}
\maketitle
\begin{roles}  
  \role{S1}[Nanna] Studerende nr. 1  
  \role{S2}[Jonas] Studerende nr. 2  
  \role{E}[Maya] Eksamensvagt
  \role{R}[Bitre-Mikkel] Rigtig eksamensvagt
  \role{N}[Niels] Ur-ninja
  \role{X}[Troels] Instruktør
\end{roles}

\begin{props}
  \prop{Eksamensvagtjakke}[Maya]
  \prop{2 borde}
  \prop{En mobil telefon}[Troels]
  \prop{En skoletaske}[Maya]
  \prop{Et æble}[Mikkel]
  \prop{En cola}[Mikkel]
  \prop{3 kuglepenne}[Nanna]
  \prop{Et penalhus}[Nanna]
  \prop{Massere og massere af papir, nej, mere end det}[Maya]
  \prop{Lydeffekt af et analogt ur der går}
  \prop{Et stort analogt hur (tænk bornholmerur) hvor man kan stille på viserne.}
  \prop{En pakke cigaretter}
  \prop{Toiletpapir}[Troels]
\end{props}

\begin{sketch}
  \scene{E har autoritetsjakke på og står med papir og eksamenssæt, S1
    og S2 sider ved deres borde.}

  \scene{Ur: 9}

  \says{E} Velkommen til den skriftlige 4 timers eksamen. Husk at
  holde mobiltelefoner slukkede, åben ild er ikke tilladt,
  nødudgangene er der et sted.

  \says{E} Klokken er nu 9, og I må vende jeres papirer \ldots nu.

  \scene{S1 og S2 vender deres papirer og begynder febrilsk at
    arbejde}

  \scene{Lyd effekt: lyden af tiden der går (Tik-Tok).}

  \scene{Ur: 9:10}

  \scene{E går og tripper, ser på sit ur, det skal se ud som om der er
    gået meget tid}

  \says{E} Klokken er nu 10 minutter over 11.

  \scene{Pause. S1 og S2 kigger på hinanden og går i panik.}

  \says{E} GMT + 3, dvs. klokken er 10 minutter over 9 i Danmark.

  \scene{S1 og S2 sukker og kigger træt på E.}

  \says{E}[til S1] Papir?

  \says{S1} Ja tak!

  \scene{E giver S2 en masse papir indtil hun bliver viftet væk.}

  \says{E}[til S2] Papir?

  \says{S2} Nej!  Jeg har rigeligt!

  \says{E} Skal du så ikke med ud og ryge, jeg gir' \ldots du har
  masser af tid!

  \says{S2} Nej, jeg ryger ikke.

  \says{E} Kom nu, den første er gratis.

  \says{S2} Nej.

  \scene{E går forbi S1 og kigger ned i hans papirer - E fniser højt
    og fortsætter hen til S2}

  \scene{E sukker og giver S2 mere papir. S2 åbner sin taske og sætter
    et æble på hans bord, vender sig ned mod tasken for at tage en
    Cola frem.  Idet S2 vender sig væk snupper E æblet, tager en bid,
    og smider det ud bag scenen. S2 tager sin Cola frem og er ved at
    sætte den, da han opdager at æblet er væk. S2 ser forvirret rundt,
    under bordet osv.  Kigger derefter anklagende på E, som blot
    trækker på skuldrene.}

  \scene{Tiden går.}

  \scene{Ur: 9:30}

  \says{E} Der er nu 3 timer og 30 min tilbage af eksamen -- dvs. 210
  minutter -- dvs. 11010010 i binær (Big Endian), i Little Endian er
  det naturligvis 01001011, omkring MDCXXV UNIX time i romertal, og
  322 i Octal.

  \scene{E's mobiltelefon ringer, E ser først olmt på S1 og S2, men
    opdager derefter det er hans, S1 og S2 går lidt i panik da den
    ringer, før de opdager det er hans, hvorefter de ser olmt på ham}

  \says{E} Ja? Oh, hej skat, jeg er til eksamen. \act{Sætter sig på
    S1's bord, og på S1's eksamens papir, S1 prøver febrilsk at hive
    papiret ud under E, papiret må meget gerne gå i stykker} Ja jeg er
  til eksamen, jo det var i dag. Jeg \textit{er} altså til eksamen.

  \scene{E nærmest støder mobiltelefonen ind i S1's hoved}

  \says{E} Sig du er til eksamen! Sig det!

  \says{S1} Øhh, jeg er lige til eksamen?

  \says{E} Se, se! Ja jeg kan godt købe ind. Aha, aha, vent et øjeblik.

  \scene{E stjæler S1's kuglepen og går hen til S2 og tager det papir
    S2 skriver på - E skriver på papiret, S1 ser paf ud og finder
    derefter en ny kuglepen i sit penalhus, S2 prøver forgæves at få
    papiret tilbage, E skal vifte ham af som om han var et irriterende
    insekt} Ja, jeg kan godt nå det, de giver nok snart op her.

  \scene{S1 rækker hånden op, E sukker, slukker for mobil telefonen og
    går hen og giver ham mere papir, S1 ryster på hovedet og hopper op
    og ned på stolen, han skal tydeligt på toilettet, E gå ud fra
    scene og kommer tilbage med en toiletrulle som han rækker S1}

  \scene{E går over på den anden side af scenen.}

  \says{E} Kender i det når man virkelig har vildt dårlig
  mave\ldots\act{E skal tale indtil S1 bryder ind}

  \says{S1}[banker toiletpapirsrullen i bordet og bryder ind i E's
  talestrøm] Vil du ikke være sød og være stille? Det er faktisk ret
  svært det her.

  \scene{E sukker dybt og keder sig åbenlyst.}

  \scene{Ur: 11:00}

  \scene{Efter et øjeblik går der en prås op for ham og han går hen og
    skruer på uret i baggrunden som får tiden til at gå meget
    hurtigere, lydeffekter skal gå hurtigere og hurtigere. Den skal
    slutte ved at der først kommer et ding, og så fortsætte et eller
    to sekunder. S1 og S2 ser bestyrtede ud og arbejder endnu mere
    febrilsk.}

  \says{E} Ja, så er tiden gået, faktisk er der gået 10 min. ekstra,
  men det vil jeg se igennem fingre med.

  \scene{S1 og S2 afleverer deres papirer og er på vej ud. E begynder
    at samle papirer sammen, imens han mumler}

  \scene{E sætter uret tilbage til 11:00, sætter sig som en studerende
    med papirene og tager sin jakke af.  E skriver sit navn på
    papirene.  R kommer ind.}

  \says{R} Er de andre allerede gået?  Der er jo to timer tilbage!

  \says{E} Ja, der er kun mig.  Og jeg vil faktisk også gerne aflevere nu!

  \scene{E rækker R papirene.}

  \scene{Lys ned}
\end{sketch}
\end{document}
%%% Local Variables: %%% mode: latex%%% TeX-master: t%%% End:
