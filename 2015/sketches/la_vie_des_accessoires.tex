\documentclass[a4paper,11pt]{article}

\usepackage{revy}
\usepackage[utf8]{inputenc}
\usepackage[T1]{fontenc}
\usepackage[danish]{babel}


\revyname{DIKUrevy}
\revyyear{2015}
% HUSK AT OPDATERE VERSIONSNUMMER
\version{1.0}
\eta{$5$ minutter}
\status{Færdig}

\title{La Vie Des Accessoires}
\author{Mikkel, Niels}

\begin{document}
\maketitle

\begin{roles}
\role{B0}[Jonas] Bord
\role{B1}[Simon] Bord
\role{S}[Brandt] Stol
\role{L}[Jenny] Lampe
\role{V}[Nanna] Vase
\role{P}[Sebastian] Person + Voiceover
\role{I}[Brainfuck] Visionær instruktør
\role{X}[Bette-Mikkel] Instruktør
\end{roles}

\begin{props}
\prop{Bord}[] Et almindeligt firebenet bord
\prop{Stol}[] En almindelig firebenet stol
\prop{Lampe}[] En almindelig standerlampe, med lampeskærm, og træksnor til at tænde og slukke lyset med
\prop{Vase}[] En fin vase
\prop{Tre roser}[] Tre smukke roser
\prop{Ringbind fyldt med løse ark}[]
\prop{Båthorn}[]

\prop{Musik0}[https://www.youtube.com/watch?v=Fo6aKnRnBxM\#t=23]
\prop{Musik1}[https://www.youtube.com/watch?v=j6nY7A6UI5Q\#t=106]
\prop{Musik2}[https://www.youtube.com/watch?v=7YwNiFICaH8\#t=30]
\prop{Musik3}[https://www.youtube.com/watch?v=IKpk51PnjcA]
\prop{Eksplosion}[http://www.freesfx.co.uk/rx2/mp3s/10/11273\_1396702916.mp3]
\prop{Cancan}[https://www.youtube.com/watch?v=SDbFbZCWhj4\#t=38] Den oprindelige version er lidt lang, så vi kan sagtens skære i den 
\end{props}
Samtlige roller spilles af sceneninjaer, som hverisær styrer sin respektive rekvisit.\\
Når en rekvisit ikke skal bevæge sig, er dens ninja ikke til stede på scenen.\\
Alle ninjaer bærer en hvid handske på højre hånd. Denne handske er "rekvisittens hånd," således rekvisitten kan gestikulere som en del af dens optræden.
\begin{sketch}
\scene{Scenen er mørk}
\says{P}[Voiceover] Rekvisitten har ofte brokket sig over, at være i skyggen under hele revyen.
Derfor har de fået deres helt egen sketch.

\scene{Der kommer dunkelt blåt og hvidt lys på scenen, samt teaterrøg.\\\\
\textbf{Musik0 spilles}, mens B stolt bevæger sig mod midten af scenen.\\
B stiller sig på midten af scenen, og musikken stopper.}

Der er lav belysning på scenen.
\scene{\textbf{Musik1 går i gang}, og spottet følger nu V, som danser yndigt omkring på scenen, indtil den indfinder sig på midten af B.}
\scene{\textbf{Musik2 går i gang}.\\L jokker lidt arrogant ind ved siden af B, spankulerer omkring, og teaser med sin træksnor.\\\\
Mens buildet vokser, bliver L mere og mere vild i sin tease/dans, og lige inden droppet ved 0.56, stiller L sig i Jesus-positur!

Musikken stopper.}
\scene{\textbf{L trækker dramatisk i sin lampesnor!} \\
Der kommer optimal belysning på scenen,  \textbf{Eksplosion og Musik3} afspilles på samme tid.\\\\
\textbf{Mens Musik3 afspilles}, kommer S gående overlegent ind på scenen. S går eventuelt hen og flirter med nogen på første række, og finder til sidst sin plads bag B.}

\says{P}[Voiceover] Og nu en person som kan bruge rekvisitterne.
\scene{P vader ind på scenen, mens han båtter i sit båthorn - båt båt båt!}
\scene{Der lyder rumsteren backstage, og I styrter ind på scenen, med ringbind og papirer flagrende til alle sider}
\says{I}[pegende vredt på publikum] I -- ØDELÆGGER -- MIN VISION!!
\scene{Det flyver omkring med papirer}
\scene{I forlader demonstrativt scenen}
\says{I}[i en ikke-slukket mikrofon backstage] Så går jeg altså over til FysikRevyen.
\scene{Rekvisitterne kigger sig betuddede omkring, og forlader hurtigt scenen.}
\scene{Lyset er nede i lidt tid.}
\scene{Lyset er nede i lidt mere tid.}
\scene{Cancan-musikken begynder, og rekvisitterne kommer én efter én tilbage på scenen, indtil alle er klar.\\
Rekvisitterne danser lystigt cancan, og forlader igen scenen!}
\end{sketch}
\end{document}
