\documentclass[a4paper,11pt]{article}

\usepackage{revy}
\usepackage[utf8]{inputenc}
\usepackage[T1]{fontenc}
\usepackage[danish]{babel}


\revyname{DIKUrevy}
\revyyear{2015}
% HUSK AT OPDATERE VERSIONSNUMMER
\version{1.0}
\eta{$3$ minutter}
\status{Første udkast}

\title{Tryk ``Start'' for at starte}
\author{Andreas, Simon L., Sebastian}

\begin{document}
\maketitle

\begin{roles}
\role{S}[Niels] Sigurd
\role{P}[Sebastian] Pawel
\role{R}[Brandt] Rengøringsmedarbejder
\role{L}[Ejnar] Lig
\role{H}[Mia] Humanist
\role{X}[Amanda] Instruktør
\end{roles}

\begin{props}
\prop{Stole}[]
\prop{Eksaminationsbord}[]
\prop{Tavle}[]
\prop{OverTeX;\@ se sketchen}[]
\end{props}


\begin{sketch}

\scene{På scenen til højre ser vi S. I venstre side står Pawel veden tavle. I midten ligger L og er død. Kun S er oplyst.}

\scene{S sidder på en stol og bladrer monotont i en bog i 5--10 sekunder.}

\scene{OverTeX viser teksten ``Råb mellemrum for at skippe cutscenen.''}

\scene{Publikum råber. Teksten på OverTeX lyser grønt.}

\scene{P går hen til S, således at begge er i lyset. P banker på sit armbåndsur med pegefingeren.}

\scene{S pakker sammen og følger efter P over mod eksaminationslokalet. Lyset følger dem. De stopper ud for L. Scenen fryser.}

\scene{OverTeX viser teksten ``Råb F for at vise din respekt for den faldne rus.''}

\scene{Publikum råber. Teksten på OverTeX lyser grønt. S gør honnør, og R fejer L ud. S og P fortsætter til eksaminationslokalet. P sætter sig ned. Fuldt lys på scenen.}

\scene{S tager fat i kridtet på bordet, men løfter det ikke. OverTeX viser teksten ``Bliv ved med at råbe X for at samle kridtet op''.}

\scene{Publikum råber. S kæmper med at løfte kridtet -- undervejs må hånden gentagne gange falde ned mod bordet -- men rækker til sidst triumferende kridtet i vejret, mens teksten på OverTeX lyser grønt. Pawel kigger skeptisk på ham og rømmer sig, hvorefter han går op og tegner nogle punkter på tavlen og peger på dem.}

\scene{S vender sig mod tavlen og holder kridtet op mod den. OverTeX viser teksten ``Brug piletasterne til at tegne det konvekse hylster for de givne punkter.''}

\scene{Publikum råber. S ``prøver at følge publikums anvisninger'' og fejler. Teksten på OverTeX lyser rødt.}

\scene{P kigger ud på publikum og ryster på hovedet. Derefter giver han giver S en lussing og visker krusedullen ud. S sætter igen kridtet mod tavlen.}

\scene{OverTeX viser teksten ``Råb Enter for at tegne det konvekse hylster.''}

\scene{Publikum råber. S tegner et rigtigt konvekst hylster. Teksten på OverTeX lyser grønt.}

\scene{P tegner nu et nyt punkt uden for hylstret og gestikulerer.}

\scene{OverTeX viser teksten ``Råb Å for at undvige Pawels spørgsmål.''}

\scene{Publikum råber. S gemmer sig bag tavlen. Teksten på OverTeX lyser grønt.}

\scene{P hiver S frem og peger insisterende på punktet.}

\scene{OverTeX vister teksten ``Skynd dig at råbe `Firkant Firkant Trekant Cirkel R2 ü þ Firkant' for at bevise det.''}

\scene{Publikum råber desperat, men teksten bliver rød. P overfalder S med algoritmebogen.}

\scene{OverTeX viser teksten ``Brug L1 og R1 til at undvige Pawels feedback.''}

\scene{Publikum råber vildt, men S' forsøg på at undvige resulterer i, at han rammer direkte ind i hvert slag. Der er samlet omtrent 5--6 slag. S falder død om.}

\scene{OverTeX viser teksten ``YOU DIED'' (fra Dark Souls). Efter nogle sekunder kommer teksten ``Råb mellemrum for at prøve eksamen igen på en lavere sværhedsgrad'' frem nedenunder.}

\scene{Publikum råber. Efter 2--3 sekunder fades der til sort. S sætter sig tilbage på stolen. Spotlightet er igen på ham som i starten. Han bladrer i bogen.}

\scene{OverTeX viser teksten ``Råb mellemrum for at skippe cutscenen''.}

\scene{Publikum råber. H går hen til S og lægger en hånd på skulderen af ham. Han har et diplom i den anden hånd.}

\scene{OverTeX viser teksten ``Råb på en vilkårlig tast for at blive humanist.''}

\scene{Publikum råber. H overrækker S diplomet og giver ham hånden.}

\end{sketch}
\end{document}
