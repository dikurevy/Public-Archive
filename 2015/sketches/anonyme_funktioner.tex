\documentclass[a4paper,11pt]{article}

\usepackage{revy}
\usepackage[utf8]{inputenc}
\usepackage[T1]{fontenc}
\usepackage[danish]{babel}


\revyname{DIKUrevy}
\revyyear{2015}
% HUSK AT OPDATERE VERSIONSNUMMER
\version{0.2}
\eta{$3$ minutter}
\status{Færdig}

\title{Anonyme Funktioner}
\author{Simon Shine, alle mulige revytter}

\begin{document}
\maketitle

\begin{roles}
    \role{P}[Niels] Psykolog
    \role{A}[Simon] Anonym funktion, bærer et skilt
    \role{C}[Jonas] PHP-funktion, bærer et skilt
    \role{O}[Maya] Binær operator, bærer et skilt
    \role{I}[Mia] Identitetsfunktion, bærer et skilt
    \role{X}[Troels] Instruktør
\end{roles}

\begin{props}
\prop{Skilt med \texttt{gets()}}
\prop{Skilt med \texttt{mysql\_real\_escape\_string()}}
\prop{Skilt med $\lambda x . x$}
\prop{Skilt med \texttt{<}\texttt{<}\texttt{.\&.\~}}
\prop{En lille risbold - undtagelsen}
\prop{Falsk skæg}
\end{props}

\begin{sketch}

\scene{P er klædt ud som psykolog (grå tweed og briller), de andre
       almindeligt tøj. Muligvis t-shirts med lambda-udtryk på (i hvert fald
       I burde have $\lambda x.x$ på sin trøje og O burde nok også have
       typesignaturen \texttt{'a -> 'a -> 'a}).}

\scene{Alle undtagen I starter inde på scenen. P og funktionerne sidder ned.
       Der er en tom stol til I. D1 og D2 står i det meste af sketchen i baggrunden
       og ser seje ud.}

\says{P} Hej, og velkommen til Anonyme Funktioner, støttegruppen for
funktioner uden navne. Vi har fået et nyt medlem i dag, og, ja, jeg
tænkte på om ikke du vil introducere dig selv?

\says{A}[Lidt nervøs] Hej med jer. Ja, jeg hedder... øh... ikke noget.

\scene{De andre siger i kor: ``Hej... øh...''}

\says{A} Jeg er en funktion.

\scene{De andre klapper, A nikker sørgmodigt}

\says{A} I dag er det... 14 dage siden jeg sidst evaluerede. Og ja... det
         har været svært.

\says{C}[skubber til O med albuen] 'Kæft han ser også ud som om han er homomorf.

\says{O}[forarget] Og hvad så?  Jeg er da selv en polymorf operator!

\says{C} Ad!  Jeg troede bare du var binær-curious!

\says{P} Hov hov!  Det er er et frit rum, med lager til alle.  Her kan
man frit addressere enhver funktion, men man skal huske den høflige
kaldkonvention.

\says{A} Og det er jo ikke som om vi får æren for arbejdet! Al æren
går til de store kanoner i den højeren orden som \texttt{map} og
\texttt{fold}...

\scene{De andre funktioner nikker anerkendende}

\says{O} Altså, nu har jeg jo som infix-operator været udsat for lidt
af hvert igennem tiden.  Jeg synes altid programmørerne sætter mig i
bås mellem to operander.

\says{C} Uuuuhhhh, se mig! Jeg er binær! Hvordan kan du måske være infix,
         hvis du er anonym, hva'?!

\says{P} Arj, nu kaster jeg altså lige undtagelsen.

\scene{P kaster en lille risbold til O.}

\says{O} Hvis jeg ikke er anonym, hvad hedder jeg så måske?

\says{C}[vender O's skilt] Øøøh....

\scene{I kommer ind på scenen.}

\says{P} Hov, vi har vist en der kommer lidt sent.  Velkommen til
Anonyme Funktioner!

\says{I} Selv velkommen til Anonyme Funktioner.

\says{C}[sleazy] Vil du op på min kaldstak?

\says{I}[nedladende] Ville \textit{du} op på din kaldstak?

\says{A}[vender skiltet på I og alle gisper] Den funktion er ikke en
skid anonym!  Det er bare identitetsfunktionen!

\says{I}[usikker] Du er ikke en skid anonym... du er også bare identitetsfunktionen...

\says{C} La' mig lige se...

\says{A} ...plus plus...

\scene{C vender A's skilt og afslører \texttt{gets()}.  Alle gisper.}

\says{C} Jeg er ikke engang SELV anonym!

\scene{C vender sit skilt og afslører at han er $Y$.  Alle gisper.}

\says{P}[alvorlig] Ja... der har været mange afsløringer i dag.

\scene{Lys ned og spot på P.}

\says{P}[tager sit falske skæg af] Faktisk... er jeg slet ikke psykolog.

\says{P} Og... det her ikke engang en sketch.  Det er bare et
sceneskift.  Og jeg først med i næste nummer.

\scene{Næste nummer går i gang.}

\end{sketch}
\end{document}
