\documentclass[a4paper,11pt]{article}

\usepackage{revy}
\usepackage[utf8]{inputenc}
\usepackage[T1]{fontenc}
\usepackage[danish]{babel}

\revyname{DIKUrevy}
\revyyear{2009}
% HUSK AT OPDATERE VERSIONSNUMMER
\version{1.2}
\eta{1 minut}
\status{Færdig}

\title{Standuptragikeren}
\author{Troels}

\begin{document}
\maketitle

\begin{roles}
\role{T}[Troels] Standuptragiker
\end{roles}

\begin{props}
\prop{Piratkostume}
\end{props}

  
\begin{sketch}


\says{T} [Voice-over, med sin egen stemme] Tag godt imod....DIKUs Standuptragiker!

\scene{T kommer ind fra siden, i piratkostume fra sidste sketch}

\says{T} Kender I det at man synger en piratsang og der ikke er nogen der gider høre den?

\says{T}[Desperat] Så sad jeg igår og skrev et instant messenger system i Erlang... ..og jeg kom i tanker om at den ost jeg lugtede som havde de på tilbud i Netto...

\says{T}[Manisk] Kender I det med at man har fundet en stol\act{Gestikulerer voldsomt}...og en solid loftsbjælke og et godt reb og stillet sig på stolen med løkken om halsen, og så ringer det på døren og når man åbner op står Jyrki der...

\act{T}

\says{T}[Friskfyrsagtigt] ...og det var alt hvad jeg havde for idag, tak allesammen! I har været et dejligt publikum! Bortset fra jer fra HCØ... I er nederen.
\act{T går ud igen}

\end{sketch}
\end{document}

%%% Local Variables: 
%%% mode: latex
%%% TeX-master: t
%%% End: 

