\documentclass[a4paper,11pt]{article}

\usepackage{revy}
\usepackage[utf8]{inputenc}
\usepackage[T1]{fontenc}
\usepackage[danish]{babel}

\revyname{DIKUrevy}
\revyyear{2009}
% HUSK AT OPDATERE VERSIONSNUMMER
\version{1.2}
\eta{5 minutter}
\status{Færdig}

\title{PhD-forsvaret}
\author{Guldfisk, Phillip, Ejnar, Mikkel og Brainfuck}

\begin{document}
\maketitle

\begin{roles}
\role{K}[Farmand] Kandidat
\end{roles}

\begin{props}
\prop{Lærred}
\end{props}

  
\begin{sketch}
Idéen er en kandidat, der vågner op i auditoriet til sit Ph.D.-forsvar efter en våd nat, hvor han vistnok kom til at fifle lidt med sin powepoint-præsentaton...
"PP" angiver, hvad der skal ske på storskærmen med pp-præsentationen bagved kandidaten.

\scene{Kandidat vågner op til sit Ph.D.-forsvar - opdager pludselig, at der er publikum i salen og smiler anstrengt. Dykker hurtigt ned - dukker op igen - holder hånden for øjnene og håber at publikum er gået væk - kigger forsigtigt ud gennem fingrene. Stønner. Ned igen. Pludselig dukker han op igen med et stort påtaget smil, overfriskt! Hans tøj sidder helt forkert - som om, han har skyndt sig at klæde sig på efter}

\says{K} Godaften. Jeg drak vist lidt finsprit med nogle biokemikere i går, men jeg kan da se, at jeg ankom til mit forsvar ... til tiden.... før... tiden...  Fiksede lige powerpointen i nat, kan ikke rigtig huske... Mit emne er... Almene teori... almene teoridann...øøhh : \act{kan ikke huske det, starter powepoint}

\says{PP} Titel: Almene teoridannelser. 

\says{K}   - som jo er et fint og.. præcist emne. \act{trykker igen på fjernbetjeningen}

\says{PP} om generisk programmering

\says{K} øøøh \act{trykker igen}

\says{PP}    Gradvist kommer præsentationen op med den fulde titel, som er: Almene Teoridannelser om Generisk programmering af nummeriske løsninger til sædvanlige og partielle differentialligninger med specile henblik på udbyttelse af standard Fortran 76 og dennes henvisninger til de videre resultater beskrevet i turings artikel, "uber die Wesen des primtalalgorimus des kvadratischer bubblesort anno 1943" bilag F.

You may be a victim of software counterfeiting" popper up

\says{K}[ingnorerer popupen - eller ser den måske slet ikke]. Mit valg af platform er:

\says{PP}[ny slide] Valg af platform: C?

\says{K} Ups, Det mente jeg selvfølgelig ikke - jeg mente natuligvis C\# Man kan jo ikke kode på Caféen! \act{griner anstrengt}

\says{PP}[Ny slide] Problem: Implementeringer af grafteoretiske paradigmatiske problemfelter.

\says{K} Nemlig. Det drejer sig om, at beregne en passende rute mellem to eller flere punkter i et tredimensionalt rum. 

\says{PP}[msn-popup]: Tobias skriver "Tak for i går. Det var dejligt ;-*".

\says{K} [ømmer sig i måsen og dækker diskret over msn.]

\says{PP}[Ny slide] En jordklode med to punkter: A og B.

\says{K} Af alle mulige veje fra punkt A til punkt B er der X! veje. Vi ser her punkt A og B. \act{trykker på remoten}

\says{PP} Diku markeres som punkt A.

\says{K} ... som så er... DIKU ... Jeg vil nu illustrere mit første eksempel på en implementering af en algoritme \act{trykker på fjernbetjening}

\says{PP} Sjoveste vej algoritmen

\says{K} Øh \act{trykker}

\says{PP} Viser en knude med "Dilans" et tilfældigt sted på jordkloden.

\says{K} Øh \act{trykker}

\says{PP} Næste knude er Large Hadron Collider

\says{K} Øh \act{trykker}

\says{PP} Næste knude er Zoo med et billede af Baltazaar og sidste knude er Caféen, som så er Punkt B

\says{K} Jeg var måske lidt ...vidtløftig i går.... men nu kommer det seriøse. Lad kigge på den næste slide som er ... meget seriøs... siger han og trykke med gru på fjernbetjeningen.

\says{PP} [slide] "Amortiseret køretidsanalyse af kompressionsalgoritmen." \act{meget seriøs slide.}

\says{K} [kigger på den - overvejer - er lettet. Det er ikke umiddelbart pinligt] Præcis. Meget seriøs. Det er trivielt åbenlyst, at denne algoritme vil køre med den fremragende køretid

\says{PP} MSN -> Tobias: Hvorfor svarer du ikke?

Kandidataen: [trykker på knap]

\says{PP}    $$a^n*logn + max(k_1, k_2, ... , k_n)$$

\says{K} ...ja, og $a$ er så en konstant, som har en åbenlys værdi, som ... er trivielt udledt fra $k$, som er \act{trykker}.

\says{PP} , hvor $k_i$ = antal genstande ved den $i$'te knude

\says{K} [Suk]

\says{PP} Pludselig er der billede af Jyrkii hoppende over skærmen: HALLOOOOO.

\says{K} Jeg tror måske, jeg syntes, det var sjovt i går... Lad mig præsentere min vejleder - Jyrki.

\says{PP} [Torrent popper up med færdigt pornodownload] NiceRack.AVI has finished downloading.

\says{K} Ahem... Jeg søgte på noget stof til noget billedbehandling

\says{PP} Pornobillede af patter pop up

\says{K} [Prøver at redde sig selv med et dårligt ordspil] Ja, vi var jo ude i noget PATTERN MATCHING!

\says{PP} MSN -> JEDI siger: DIKUREVY 1973

\says{K} og jeg har foretaget en del dybdeborende, gennemtrængende undersøgelse af \act{trykker}

\says{PP} Billede af Julia Lawall

\says{K} af emnet, som IKKE er Julia Lawall - \act{bliver pinlig} Jeg søgte ikke efter "Julia Lawall" på Google i kombination med <indsæt klam ting>....

\scene{Under det følgende prøver kandidaten desparat at dække over alle de pop ups, der kommer... Desværre har han en hvid t-shirt på...}

\says{PP} MSN -> Tobias siger: Du har ellers en fræk numse ;-) Der er go' plads. LOL!

\says{PP} PopUP ->  Your virus database has been updated.

\says{PP} Jyrkii hoppende over skærmen: HALLOOOOO!

\says{PP} MSN -> Tobias siger: Betød det virkelig ikke noget for dig?

\says{PP} MSN -> Tobias siger:  Svar nu!

\says{K}  [opgivende] Men alt i summer alt dette op til den trivielle konklusion \act{trykker}

\says{PP}  [ny slide] "Konklusion: P=NP"

\scene{K komme rmed en raktion på dette TODO: Skal der siges noget) K trykker febrilsk på sin slideskifter hvilket få tæppet til at gå for (K står foran tæppet). Han trykker febrilsk videre, men tæppet går for og tilsidst begynder lyset at tændes og slukkes i salen.}

\says{K}[opgivende] Suk. Jeg drikker aldrig igen
 
 \scene{fortsættes direkte i næste sang}
\end{sketch}
\end{document}

%%% Local Variables: 
%%% mode: latex
%%% TeX-master: t
%%% End: 

