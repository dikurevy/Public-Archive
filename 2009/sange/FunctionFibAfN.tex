\documentclass[a4paper,11pt]{article}

\usepackage{revy}
\usepackage[utf8]{inputenc}
\usepackage[T1]{fontenc}
\usepackage[danish]{babel}

\revyname{DIKUrevy}
\revyyear{2009}
\version{1.0}
\eta{3 minutter}
\status{Færdig}

\title{function fib(n)}
\author{Jakob og Kristine}
\melody{Sarah Brightman og Andrea Bocelli: ``Time To Say Goodbye''}

\begin{document}
\maketitle

\begin{roles}  
\role{S}[Bo Elling] Sanger
\role{K}[Pigekor] Kor
\end{roles}

\begin{song}
  \sings{S} Læser man på diku skal man kunne programmere rekursioner
Man kan gøre det i matlab kun med ganske få allokationer

  \sings{S} let som en leg
kod med mig, med mig

  \sings{S} tænd for din laptop
slå nu guien fra med matlab no desktop
vi kan programmer'
og kalkuler'
fibonaccis talrække

  \sings{K} 
  function fib(n)
  p = 0;
  A = 1
  for j = 1 : n
    temp = A;
    A = A + p
    p = temp;
  end
 
\sings{K} og programmet er slut
  
  Skulle man være i tvivl om hvordan omkvædet skal synes, kan noderne findes i filen FunctionFibAfN\_noder.pdf :-)
\end{song}

\end{document}

