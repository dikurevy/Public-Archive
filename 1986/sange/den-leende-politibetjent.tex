\documentclass[a4paper,11pt]{article}

\usepackage{revy}
\usepackage[utf8]{inputenc}
\usepackage[T1]{fontenc}
\usepackage[danish]{babel}


\revyname{DIKUrevy}
\revyyear{1986}
\version{1.0}
\eta{$n$ minutter}
\status{Færdig}

\title{Velkomst}
\author{Karsten Kynde}
\melody{``Den leende politibetjent''}

\begin{document}
\maketitle

\begin{roles}
\role{P1}[Inger] Politisanger 1
\role{P2}[Jan] Politisanger 2
\end{roles}

\scene{Noter vedr. afsyngningen: Visen kan synges af to personer. som skiftevis
  synger en tekstlinie og ``ha, ha'' i kor. B personen, der ofte har punkterende
  pointer må gerne have en Troels-Triersk irriterende, gennemtrængende stemme.

Nogle nødrim har fået to stavelser. Da stryger man bare et ``ha'' fra næste
linie -- det kræver blot begge er forberedt. I et af versende går det helt amok
`(``dejl-igt øl'') og man stryger to ``ha''er i \underline{den} \underline{ene}
rimlinie.

Første vers synges strengt rytmisk og ikke for hurtigt. Senere kan man gradvist
flippe lidt og stedvist kløjs i ``ha, ha''erne. \underline{Sidste}
\underline{vers} starter igen meget langsomt og rytmisk idet publikum opfordres
til at synge med. I anden linie sættes tempoet i vejret. I tredje linie bliver
latteren hysterisk fra de to scneoptrædendes side for i fjerde og sidste linie
at kamme over i en vanvittig rallen. De to flytemænd kommer ind og fjerner blidt
men med magt de syngende.}

\begin{song}
\sings{Nogen}
Studenten bor på tankens slot
Ha, ha, ha, ha, ha, ha!
hvis han har fået mer' end ott'
Ha, ha, ha, ha, ha, ha!
På DIKU kan du komme frit
Ha, ha, ha, ha, ha, ha!
Hvis du har 12 i gennemsnit
Ha, ha, ha, ha, ha, ha!

På 2.del du vælger 10 fag
ha, ha, ha, ha, ha!
når du har fundet dig et bifag
ha, ha, ha, ha, ha!
På bifaget man kræve vil
Ha, ha, ha, ha, ha, ha!
du har festået 2.del
Ha, ha, ha, ha, ha, ha!

Vort ny' hus er en kæmpesag
Ha, ha, ha, ha, ha, ha!
men lærer' er der ingen af
Ha, ha, ha, ha, ha, ha!
Næh, faen sku' være lærer når
Ha, ha, ha, ha, ha, ha!
de får en månedsløn pr. år
Ha, ha, ha, ha, ha, ha!

Det skyldes, hvad I nok forstår
Ha, ha, ha, ha, ha, ha!
at Danmark mangler datalo'er
Ha, ha, ha, ha, ha, ha!
så når du målet langt om længe
ha, ha, ha, ha, ha!
så får du røven fuld af penge
ha, ha, ha, ha, ha!

Mens humanister står i kø
Ha, ha, ha, ha, ha, ha!
og sulter og er ved at dø
Ha, ha, ha, ha, ha, ha!
De sku' ha' valgt et bedre mål
Ha, ha, ha, ha, ha, ha!
Cand.mag.er' læser nu dat 0
Ha, ha, ha, ha, ha, ha!

Mens datalogens dag er rig
Ha, ha, ha, ha, ha, ha!
han programmerer stjernekrig
Ha, ha, ha, ha, ha, ha!
Når de får kraft i Tjernobyl
Ha, ha, ha, ha, ha, ha!
så er det datalogers skyld
Ha, ha, ha, ha, ha, ha!

Han skaffer credit til din bank
Ha, ha, ha, ha, ha, ha!
din lønningscheck, den gør han blank
Ha, ha, ha, ha, ha, ha!
Når der er valgkan fletal rykke
ha, ha, ha, ha, ha!
hvis stadset går i uend'lig løkke
ha, ha, ha, ha, ha!

En sættekasse er til nips
Ha, ha, ha, ha, ha, ha!
en gartner stoppes ned i chips
Ha, ha, ha, ha, ha, ha!
En datamat kan brygge dejligt øl
ha, ha, ha, ha!
og træsprit ved en indeksfejl
Ha, ha, ha, ha, ha, ha!

Men ka' han osse følge med?
Ha, ha, ha, ha, ha, ha!
Som fyrreårig går han ned
Ha, ha, ha, ha, ha, ha!
Fra terminalen på hans bord
Ha, ha, ha, ha, ha, ha!
syntetisk høres disse ord
\act{syntetisk stemme:} Ha, ha, ha, ha, ha, ha!

Ha, ha, ha, ha, ha, ha, ha, ha
Ha, ha, ha, ha, ha, ha!
Ha, ha, ha, ha, ha, ha, ha, ha
Ha, ha, ha, ha, ha, ha!
Ha, ha, ha, ha, ha, ha, ha, ha
Ha, ha, ha, ha, ha, ha!
Ha, ha, ha, ha, ha, ha, ha, ha
Ha, ha, ha, ha, ha, ha!
\end{song}

\end{document}

