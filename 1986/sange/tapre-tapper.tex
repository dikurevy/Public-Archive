\documentclass[a4paper,11pt]{article}

\usepackage{revy}
\usepackage[utf8]{inputenc}
\usepackage[T1]{fontenc}
\usepackage[danish]{babel}


\revyname{DIKUrevy}
\revyyear{1986}
\version{1.0}
\eta{$n$ minutter}
\status{Færdig}

\title{Den tapre tap(per)}
\author{Charlotte og Karen}
\melody{``I den gamle pavillon''}

\begin{document}
\maketitle

\begin{roles}
\role{S}[Lotte] Tap
\end{roles}


\begin{song}
\sings{S}
Jeg er tap og gammel DIKU-støtte
jeg var så ked af
at vi sku' fltte
Ak nu er jeg mange mange mile
fra DIKU's hvile
og UNI$\bullet$C
Jeg er ny på stedet
kender meget lidt
ovre her på UP1

Jeg drømmer mig tilbage til det gamle institut
til kontoret hvor jeg sad
jeg skrev på min maskine
og laved' noget uafbrudt
og nød at være' i Sigurdsgad'
jeg ænsed' ikk' at timerne de spænte
og allered' vare arbejdsdagen slut
jeg traved hjemad mens jeg gik og tænkte
på det gamle institut

Her på UP1 er en kantine
hvor man må tie
og ikke grine
her er også førstedels-studenter
og de forventer
jeg rydder op
når jeg koger karklud'
kan jeg let bli' svedt
ovre her på UP1

Så sender jeg en tanke til det gamle institut
til kantinen hvor vi sad
jeg drak en masse kaffe
fik en sjælden gang lidt sprut
jeg var altid mægtig glad
af sommerfester havde vi så mange
og de var med revy og megen fut
i fejemøget, mange lod sig fange
af det gamle institut
\end{song}

\end{document}

