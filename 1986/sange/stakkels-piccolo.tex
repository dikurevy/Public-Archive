\documentclass[a4paper,11pt]{article}

\usepackage{revy}
\usepackage[utf8]{inputenc}
\usepackage[T1]{fontenc}
\usepackage[danish]{babel}


\revyname{DIKUrevy}
\revyyear{1986}
\version{1.0}
\eta{$n$ minutter}
\status{Færdig}

\title{Stakkels Piccolo}
\author{DIKUrevyen 1986}
\melody{``Just a Gigalo''}

\begin{document}
\maketitle

\begin{roles}
\role{L0}[Ragnar] Lærer
\role{L1}[Jan] Lærer
\role{L2}[Mia] Lærer
\role{SM}[Lars] Sprechstallmeister
\role{F1}[Jacob] Flyttemand
\role{F2}[Flemming] Flyttemand
\role{P}[] Piccolo (en maskine)
\end{roles}

\begin{sketch}
\scene{\textbf{Note fra digitalisator:} Mange stavefejl i den oprindelige.}

\scene{Fortsat fra Reklamesang.}

\says{L0} Ho, ho, man forbereder sig altså ikke til en Dat-0 forelæsning, det
ved alle.

\says{SM} Udmærket, puplikum er dit. Farvel.

\scene{SM går ud. Flyttemand bærer ruller ind en Piccolo.}

\says{L0} Hvad.., jeg.., du kan ikke............

\scene{L0 kigger forvirret ud over puplikum. Flyttemænd går ud.}

\says{L0} Huhumm.... I dag snakker vi om begre.., en maskine der kaldes en
computer, \act{udtale overdrivet}, eller på dansk, en datamat. Her kan i se den
nyeste og mest perfekte udgave af en datamat. Det er den så kaldte
Piccalo. \act{Peger på Piccoloen}. I dette kursus lærer i at programmere i
programmeringssproget Pascal. Jeg ved godt i hører disse begrebure for første
gang, men i kommer til med at forstå dem i vinterens løb. I for uddelt en
compiler, på danske, en oversætter. \act{Ta'r en tyk bog som ligger på
  Piccoloen}. Jeg har her i hånden en manual, en brugervejledning til den
oversætter i kommer til at bruge. Den er meget koncis og let forståelig og i får
den uddelt sammen med oversætteren. Denne oversætter kaldes Compassion, som
bekendt betyder sympati. Vi starter.

- Afsnit eet; tænd for maskinen...., nej, nej... \act{blader i bogen og læser
  højt}

- Der skal indtastes ordet compassion og så kan oversætteren bruges.

Ja, så enkelt er det. \act{Demonstrerer}. Vi går videre.

- Men glem ikke at trykke på return.

Ja, hummm..., altså. \act{Trykker på return}.

- Det er yderst vigtigt at der sidder en diskette i diskette-stationen.

\act{Ta'r en diskette. Det ta'r et stykke tid at finde ud af hvor den skal
  sidde. Efter nogle mislykkede forsøg lykkes det.}

- Der skal tændes for diskettestationen.

Nu skulle alt være parat.

- I har vel husket at tænde for maskinen som blev beskrevet i afsnit eet?

Nå.., ja..... Altså, det er faktisk meget vigtigt at tænde for
maskinen. \act{Tænder og siger til sig selv} ohhh, shit...

- Nu er alt parat og oversætteren kan bruges. Husk at sluke og fjerne disketten
når i forlader maskinen.

\act{Slukker og fjerne disketten.}

- Nu kan man direkte indtaste f. eks. et Pascal program. \act{Prøver på at
  indtasten noget men intet sker. Kigger igen i brugervejledningen og læser for
  sig selv} Hvis der er nogle problemer så start forfra. \act{Højt} Det er nok
bedst at vi gennemgår vejledningen igen for at alle kan være sikre på, hvordan
man gør. \act{Starter igen på en tilsvarende måde.  Når han skal til at sætte
  disketten på plads giver han op}

- Afsnit eet; tænd for maskinen...., nej, nej...

\act{Blader i bogen og læser højt}

- Der skal indtastes ordet compassion og så kan oversætteren bruges.

Ja, så enkelt er det. \act{Demonstrerer}. Vi går videre.

- Men glem ikke at trykke på return.

Ja, hummm..., altså. \act{Trykker på return}.

- Det er yderst vigtigt at der sidder en diskette i diskettestationen.

\act{Ta'r en diskette. Det ta'r et stykke tid at finde ud af hvor den skal
  idde. Efter nogle mislykkede forsøg lykkes det.}

- Der skal tændes for diskettestationen.

Desværre er vores tid udløbet i dag.  Er der nogle spørgsmål? (Hvis der er,
så improvisere). Så siger jeg tak for i dag. \act{Går ud og Piccoloen begynder
  at synge.}

\scene{\textbf{Note fra digitalisator:} Der står ikke hvem der oprindeligt
 sang det næste.}
\end{sketch}

\begin{song}
\sings{P}
Stakkels Piccolo
Arme Piccolo
Hvorfor kan jeg ikke græde?
Hvorfor blev jeg til?
Ved de hvad de vil?
De som nu er her til stede.
Hvad skal man dog tro
er man gigolo
drejer det sig kun om penge?
Eller er jeg et svar
på de spårgsmål men'sker har?
Er jeg tekknikens gøgler?

Store Piccolo
flotte Piccolo.
Jeg er universets nøgler.
Hvis i si'r hvordan
ved jeg at jeg kan
løse alle jeres tøjler
og fordi jeg ved 
at samvittighed
er religionens søjler
si'r jeg Gud ved hvordan
man skal bruge sin forstand
han er jo kun en gøgler.

Men'ske Piccolo?
Maskine Piccolo?
Hvor er mon min Piccoline?
Hvis hun kommer snart
ku' det være rart
hvis hun også kunne grine!
Det ku' være smart
det er ganske klart
hvis hun drikker Ballantine
og hvis hun bli'r for tør
og altså at hun tør
så bru'r vi brillantine.
\end{song}

\begin{sketch}
\scene{Flyttemænd kommer ind.}

\says{F1} Tænk, at en maskine kan være klogere end en lærer.

\says{F2} Hvorfor synes du det er mærkeligt, du ved vel, at vi begynder os på
DIKU.

\act{Fjerner Piccoloen.}
\end{sketch}

\end{document}

