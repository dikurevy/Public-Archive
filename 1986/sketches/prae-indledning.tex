\documentclass[a4paper,11pt]{article}

\usepackage{revy}
\usepackage[utf8]{inputenc}
\usepackage[T1]{fontenc}
\usepackage[danish]{babel}


\revyname{DIKUrevy}
\revyyear{1986}
% HUSK AT OPDATERE VERSIONSNUMMER
\version{1.0}
\eta{$n$ minutter}
\status{Færdig}

\title{PRÆ-indledning}
\author{Vilmar}

\begin{document}
\maketitle

\begin{roles}
\role{F1}[Jacob] Flyttemand
\role{F2}[Flemming] Flyttemand
\role{K1}[Vilmar] Revykommisær iført agentfrakke
\role{K2}[Mia] Revykommisær iført agentfrakke
\role{K3}[Charlotte] Revykommisær iført agentfrakke
\role{K4}[Inger] Revykommisær iført agentfrakke
\role{K5}[??? (Ce)] Revykommisær iført agentfrakke
\role{Pianist}[???] Pianist
\role{SM}[Lars] Sprechstallmeister
\role{Offer}[Lotte] En ikke særlig stor person
\end{roles}

\begin{props}
\prop{Stor (meget stor) papkasse}[]
\prop{Mindre papkasse}[]
\prop{2 øl}[]
\prop{Vandpistol}[]
\prop{Stor rekvisit fra næste nummer}[]
\end{props}


\begin{sketch}

\scene{F1 og F2 kommer ind i salen i musiksiden oppe fra midtergangen. Pianisten
sidder på en af de yderste stole på 3. eller 4. række. Flyttemændene slæber på
et eller andet, som skal bruges på scenen i et efterfølgende nummer.}

\says{F1} Guud, de er allerede kommet.  Bare vi når det inden revyen begynder!

\says{F2} Det er det sædvanlige, alting skal absolut klares i sidste øjeblik.

\scene{Mere brok, imens de går ned og stiller genstanden på sin plads på
  scenen.}

\says{F1} Pyh', det var hårdt. Sku' vi ikke tage os en bajer. \act{hiver 2 øl
  frem, knapper op, sidder på scenekant og drikker}

\says{F1} Ahhhhh! Det' en flot stor scene. Hvad med en lille sang?

\says{F2} Go ide, jeg finder lige en pianist. \act{henter pianist, bærer ham hen
  til piano}

\says{F1 + F2} 1 -- 2 -- 3 -- 4

\end{sketch}


\begin{song}
\scene{Melodi:Den ta'r vi også med}

\sings{F1} DIKU's lokaler var blevet for små
\sings{F2} vi var får mange så I kan nok forstå
\sings{F1} At vi sku' flytte og det i en fart
\sings{F2} og nu er det overstået -- snart

\sings{F1 + F2}
Vi måtte samle det hele og stuve det meste ned
den tog vi også med, den tog vi også med
bøger, kontorer, et helt trykkeri
da vi flytted' over til MAI.

\sings{F2} Her var en ... \act{afbrydes, musik stopper}
\end{song}


\begin{sketch}

\says{F1} MAI ! Det hedder da ikke MAI længere!

\says{F2} Det er da her, ikke ??  Det tidligere medicinsk-anatomisk institut,
altså, ikke?

\says{F1} Ja ja, men nu er det DIKU -- og man kan da ikke kalde DIKU MAI, vel?

\says{F2} DIKU ? Jamen -- det er da i Sigurdsgade, ikk' ?

\says{F1} Joh -- øhh -- næh -- altså, hør nu her: I gamle dage var DIKU DIKU og
det her var MAI. Men så blev DIKU for lille og MAI flyttede til Panum og så
kunne DIKU flytte fra DIKU til MAI og RECKU kunne få DIKU. Men da MAI var
flyttet kunne man jo ikke kalde MAI MAI længere nu hvor DIKU skulle have MAI,
vel, så derfor kaldte man det Fælledannekset.

\says{F2} DIKU ?

\says{F1} Nej dit fæ, MAI!  Og i overgangsperioden hvor DIKU bor både på DIKU og
i Fælledannekset kalder man det så UP1 -- altså MAI -- og Sigurdsgade -- altså
DIKU. Men snart er DIKU flyttet helt og så hedder det her DIKU.

\says{F2} Nåhh. Og hvad så med DIKU? Jeg mener Sigurdsgade.

\says{F1} Den overtager RECKU.

\says{F2} Skal RECKU så hedde DIKU ?

\says{F1} Nej nej nej, RECKU skal hedde UNI-C, det hedder det faktisk
allerede. Så vi kalder det her for UP1. \act{Musik starter}

\end{sketch}


\begin{song}
\sings{F2} Her var en lejlighed til at begynde igen
\sings{F1} få startet påny, hvor vi nu er flyttet hen
\sings{F1 + F2}
Vi har et stort auditorium med plads til vor menighed
den flytted' også med, den flytted' også med
Men her i revyen har det ikke været let
prøv I at sku' rime på UP1

\act{nusser lidt rundt, tager en tår øl}

\sings{F1} Revyens rekvisitter de blev pakket ned
\sings{F2} noget smed vi ud -- det meste tog vi med
\sings{F1} Kulisser og fortæppe passer ej mer
\sings{F2} alligevel må vi spille her

\sings{F1 + F2}
Imange år var revyen på gulvet i 'ABC
men det ku' vi ik' ta' med, det ku' vi ik' ta' med
Revygruppens folk måtte knokle sig svedt
og bygge en scene på UP1

\sings{F1} Vi hamred' og skrued' på søm og på skruer
\sings{F2} og se resultatet -- vi håber at den duer

\sings{F1 + F2}
For selve ånden fra DIKU og fra 18ABC
den har vi taget med, den har vi taget med
den skal de bruge på scenen om lidt
revyen er flyttet til UP1

\scene{SM ind, genner flyttemændene ud og får musikken til at holde op}
\end{song}

\begin{sketch}
\says{SM} Ja undskyld mange gange at vi roder lidt, og så disse personer men det
er jo vanskeligt at nå det altsammen når man både vil blæse og godaften mine
damer og herrer og rigtig hjertelig velkommen, vi skulle jo gerne have
arrangementet her til at forløbe gnidningsfrit og det hele til at KLAPPE

\scene{Her ventes lidt mens folk klapper}

\says{SM} Inden revyen starter er der lige et par praktiske bemærkninger om
nogle få, helt basale forhold, som vi lissom er nødt til at have op at
vende. Som nogen måske har bemærket holder vi i år revyen et andet sted, end vi
har gjort tidligere år. Dette har visse sociale følger, f.ex. er der i dette
beskedne lokale kun ringe mulighed for bunkepul på de forreste rækker (ja, vi er
kede af det, Carsten), lissom sæderne må forventes at være noget umagelige at
falde i søvn i. Dette sidste skulle dog ikke være noget problem for de blandt,
som inden for de senere år har fulgt et førstedelskursus; ej heller for de, som
har deltaget tilstrækkeligt aktivt i middagen.

Jeg vil også kort forklare lidt om de fysiske omgivelser. De, der har været her
i lokalet før, vil -- hvis de er nogenlunde ædru endnu -- bemærke, at det jeg
står på \underline{ikke} er det sædvanlige bord i dormitoriet. Dette omtalte
bord forekom upassende at holde revy på -- vi var bange for at falde i
håndvasken og drukne -- så vi har bygget denne konstruktion henover bordet. Og i
den forbindelse vil jeg gerne rette en varm tak til tømrermester Kjeld Petersen,
som var så venlig kun at lade \emph{nogen} af plankerne være fejlmålt -- og de
var heldigvis for lange. Et andet forhold, som er let ændret fra de sidste mange
års revyer, er fortæppet. Det hidtidige fortæppe passer ikke helt her i
dormitoriet. Imidlertid har vi brug for et tæppe når der skal laves lange
kedelige sceneskift, så derfor har vi fundet på følgende nemme løsning: Oppe til
højre (eller venstre, afh. af hvor den nu hænger) hænger der en lille røde
pære. Når den er tændt, betyder det at tæppet er for, og at I ikke kan se, hvad
der foregår på scenen. For at der ikke er nogen, der skal snyde, skal alle lukke
øjnene, når pæren tændes -- og første åbne dem igen når pæren slukkes. Simpelt,
ikke?

Skal vi prøve? 1 -- 2 -- 3 \act{pære tændes} ja, det ser nogenlunde ud, der er
vist nogen oppe på fjere række som snyder og ja, så er prøven forbi \act{pæren
slukkes} og I må godt åbne øjnene igen. Jeg bemærkede, at der var nogen, som
\textbf{SNØD!} Den går ikke! Vi laver en prøve til, og denne gang vil der være
en række revykommisærer diskret tilstede blandt publikum, som vil kontrollere at
alle har lukket øjnene mens tæppet er for. Klar ? Ok, læg nu mærke til tæppet.
\act{venter, pære tændes efter et stykke tid} Ja, det ser meget godt ud, det
hjalp nok med kommisærerne, hva? Kontrol er nu bedre end tillid.

\says{K1} Her er en med et øje på klem !

\says{SM} ET ØJE PÅ KLEM !!! Ok, giv hende/ham en påmindelse !

\scene{K1 hiver en vandpistol frem og giver den ulykkelige et splat i sylten.}

\says{SM} Nå, der kan I vel bare se -- ja det vil sige, det kan I jo ikke, VEL?

\says{K2} Her er en med \textbf{begge} øjne helt åbne!!

\scene{De andre kommisærer styrter til undsætning}

\says{SM} HVAFORNOGET !!  Begge øjne åbne !!  Her må vist mere end en påmindelse
til.  Bring ham/hende herned, og I andre kan godt åbne øjnene \act{pæren
  slukker}, tæppet er fra nu.

\scene{Kommisærerne bringer ofret -- som bør være en ukendt person fra
  revygruppen -- ned på scenen}

\says{SM} Nå, det var ikke så godt, vi må vist sørge for at du ikke ``kommer
til'' at åbne øjnene på forkerte tidspunkter. Bring kassen ind !!

\scene{Flyttemænd bringer lille kasse ind, monterer den på ofrets hoved}

\says{SM} og for at du ikke skal komme for skade at tage den af vil du indtil
videre blive anbragt i denne boks.

\scene{Flyttemænd kommer ind med den store kasse, som er forsynet med et hul,
  hvorigennem ofrets hoved kan stikke ud. Ofret puttes i den store kasse, som
  lukkes. Det skal nu blot ligne en stor kasse med en lille kasse stående
  ovenpå.  Kassen skal være forsynet med en oplukkelig bagside, og stilles et
  sted, så ofret ubemærket kan åbne bagsiden og kravle ud og væk.}

\says{SM} Nå, det var det. Inden vi går videre til DIKU-revyen 1986 er vi
desværre nødt til at afvikle et par andre arrangementer, vi har måttet påtage os
nogle forpligtelser for at få lov til at låne dette prægtige lokale til at holde
revy i.  DIKU's bestyrelse og lignende myndigheder, som råder over vort ve og
vel, og som venligt har stillet f.eks. dette dormitorium til vor rådighed, har
betinget sig at vi \act{remser op}:
\begin{itemize}
\item gør reklame for instituttet
\item afholder ekstraordinær undervisning
\item omtaler vore donatorer virkelig pænt
\item sørger for en evakueringsplan for dormitoriet
\end{itemize}

Evakueringsplanen er optrykt i programmet, hvor alle flugtveje også er vist på
side 7. Endvidere har vi lavet en beredskabsplan, som skal træde i kraft i
tilfælde af det værst tænkelige uheld. Den vil jeg kort gennemgå nu.

Enhver tilskuer er delt op i 5 grupper: Lærer, TAP'er, studerende, kæreste eller
andet.  Hvis man er kæreste til en person i en af de andre grupper, hører man
til i denne gruppe, hvorimod hvis man er kæreste til en anden person i gruppen
kæreste hører begge til under gruppen ``andet''. Så er vi nede på 4 grupper. I
tilfælde af det værst tænkelige uheld skal alle lærere straks og uopholdeligt
give sig til at forske og undervise uden i øvrigt af den grund at udvise nogen
panik, selv om det er en uvant situation. Alle studerende skal falde i søvn
øjeblikkelige, idet det anses for den mest typiske aktivitet for studerende i
dette lokale, mens TAP'er som sædvanlig skal sørge for at det hele bare fungerer
og i øvrigt gøre hvad de plejer. Alle andre skal lade sig gribe af panik. I
tvivlstilfælde kan man henvende sig hos en af kommisærerne, som vil være
behjælpelig med råd og vejledning.

Og så skulle vi være klar til reklameindslaget.

\end{sketch}

\end{document}
