\documentclass[a4paper,11pt]{article}

\usepackage{revy}
\usepackage[utf8]{inputenc}
\usepackage[T1]{fontenc}
\usepackage[danish]{babel}


\revyname{DIKUrevy}
\revyyear{1986}
% HUSK AT OPDATERE VERSIONSNUMMER
\version{1.0}
\eta{$n$ minutter}
\status{Færdig}

\title{Dispensationsbutikken}
\author{JC}

\begin{document}
\maketitle

\begin{roles}
\role{S}[Ragnar] Studerende
\role{E}[Mia] Ekspedient
\end{roles}

\begin{sketch}
\says{S} Goddag -- jeg ville gerne ha' en dispensation

\says{E} Javel -- havde du tænkt på nogen speciel dispensation? Har du et
problem, der sal løses hvor vi skal levere bergrundelsen for dispensationen?
eller har du en god grund, der bare venter på at blive brugt i en
dispensationssag?

XeSJaa -- jeg ved ikke rigtig -- jeg ville egentlig godt se jeres udvalg inden
jeg bestemmer mig.

\says{E} Dispensationsudvalget? Det er Jens Damgaard A -- Han bor i Sigurdsgade.

\says{S} Nej, Nej -- udvalget af mulige dispensationer!

\says{E} Du ka' jo kigge på hylderne -- der ligger alle de gamle
dispensationssager fremme og de er gode at få forstand af.  Men vi har jo ikke
alt liggende fremme --åmkse ku jeg vise dig et udpluk fra vort store postordre
katalog?

\says{S} Vil du virkelig -- Åh det ville være skøønt!

\says{E} Lad os starte forfra: hvad med en dispensation for adgangskravene til
studiet -- du kan komme ind selve med et bestået første år af gammelsprogligt
gymnasium.

\says{S} Nej, nej; jeg er jo immatrikuleret!

\says{E} Nåh, ja --em n hvad så med dispensation fra dat.0 -- eller dele af det?

\says{S} Jamen, jeg har bestået dat.0 -- jeg har faktisk bestået hele den
datalogiske 1.del.

\says{E} Nåå -- det begrænser jo muligheden stærkt. Så kan vi heller ikke bruge
den med beståelse 5 i skriftligt og 6 eller derover i rapporterne. Og den med
mere tid eller 5. forsøg til skriftlig eksamen er også yt -- den er ellers meget
populær.  Men det er vel ikke så heldigt, at du er immatrikuleret i 1982 eller
senere.

\says{S} Jo -- jo. Jeg er fra 1982.

\says{E} Jamen så har vi lige noget for dig! Dispensation fra reglen om, at
datalogisk 2. del kun kan påbegyndes hvis ``andet bifag'' er bestået.

\says{S} Men jeg har bestået matematik-bigaf her i sommmers.

\says{E} Tja -- der røg den mulighed. og vi kan heller ikke dispensere fra
matematikbifaget eller matematik D -- det er ellers også noget, vi sælger
tit. Nu må vi tænke os godt om -- du er sikker på, at du ikke selv har en god
grund?

\says{S} Nææh ....

\says{E} Jamen -- jeg kommer lige i tanke om en nylig gennemførm sag, hvor en
student udbag sig dispensation fra en del af 3.delsstudiet, fordi han syntes,
han var god nok. Den må da kunne bruges ... ?

\says{S} Jeg har faktisk gennemført de krævede mundtlige og skriftlige timer og
er midt i indskrivelsen af specialet.

\says{E} Jamen så forstår jeg ikke rigtig sagen -- du vil ikke ha' dispensation
for dit køn, din alder, din hudfarve eller manglende børnesygdome? -- eller har
du AIDS?

\says{S} Nej, Nej -- jeg vil ha' en faglig dispensation.

\says{E} Ja så er jeg bange for, at vi ikke kan hjælpe dig. Det var sørens --
den situation troede jeg ikke, vi ku' komme i. Men du er jo fandme en helt
normal standardstudent -- hvad vil du egentlig med en dispensation.

\says{S} \act{bryder sammen, hulkende} -- det var bare sådan -- snøft -- at
forleden sad vi i kantinen, sådan nogle stykker -- og da gk det op for mig i al
min gru, at jeg var en paria, en udskudt, et misfoster -- jeg var den
\underline{eneste}, der ikke havde en dispensation. (Hulk, snøft) Åh hjælp mig
dog -- du kan da ikke afvise mig -- jeg kaster mig ud fra toppen af køleanlægget
i UP1 -- jeg besætter bestyrerens kontor, jeg tager dig som gidsel indtil
studienævnet finder på en dispensation jeg kan få, jeg ...

\scene{to bomstærke trykkere kommer ind og fjerner den skrigende student}

\says{E} Jeg tror der er basis for en ny vare i sortimentet -- en dispensation
fra at være en student der ikke følger normalstudieplanen. Det kræver en fiks
lille omdefinition af hvad der er normalt -- men hvad -- det er jo ikke unormalt
i vore dage.
\end{sketch}

\end{document}
