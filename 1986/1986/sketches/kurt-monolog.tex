\documentclass[a4paper,11pt]{article}

\usepackage{revy}
\usepackage[utf8]{inputenc}
\usepackage[T1]{fontenc}
\usepackage[danish]{babel}


\revyname{DIKUrevy}
\revyyear{1986}
% HUSK AT OPDATERE VERSIONSNUMMER
\version{1.0}
\eta{$n$ minutter}
\status{Færdig}

\title{Kurts monolog om revy på UP1}
\author{Vilmar}

\begin{document}
\maketitle

\begin{roles}
\role{Holm}[Holm] Holm
\end{roles}

\begin{sketch}
  \scene{Dette nummer er ment som det absolut alleraller sidste i revyen. Scenen
    er forholdsvis tom fra starten, ind kommer Vilmar med en guitar. En
    flttemand sørger for een barstol og en mikrofon på stativ. Det skal se ud
    som om Vilmar skal lave en sang alene. Indledningsvis skal der være noget
    snak, mens guitaren og publikum stemmes, noget om relationer til tidligere
    års revyer og sådan noget, og det kan afsluttes med at sige noget om, at nu
    kommer der en af de gamle stjerner og så kommer Holm ind med sin barstol og
    en mikrofon mens Kurt-temaet begynder.}

\says{Holm}
``Universitetsparken, linie 43'' annoncerer chaufføren i bussens
høkttaleranlæg. Kurt rejser sig, trykker på knappen og går hen til
udgangen. ``Fedt nok'' tænker Kurt, jeg er lige i humør til at gå til
sommerefest i dag. bussen holder og venter for rødt lys, og Kurt benytter
lejlighdenen til at børste et par sorte hår af sine nydelige hvide
bukser. Satans kat ! Kurt stiger af ved stoppestedet og skrår tværs over krydset
-- der er jo ikke meget trafik sådan en lørdag ved sekstiden. ``Det var
skidegodt sidste år til revyen'' tænker Kurt ude midt i krydset, da han
pludselig hører nogen råbe sit navn. Elegant undgår han en gammel kone på en
grøn Puch-maxi og skynder sigind på fortovet ovre ved telefonboksen. Det er
Preben, der står ovre ved Zoologisk Museum og råber. Kurt synes det lyder som
``Hvor skal du hen Kurt'', hvilket han er fuldstændig uforstående
overfor. Preben ved da godt, at der er revy i aften. Det er ikke mere end 4
dagesiden, at Kurt var inde og meldte Preben og sig selv til revy og sommerfest
-- for sent. Heldigvis kender Kurt godt Ilse, så han fik kringlet 2 billetter
alligevel. Så Preben burde da vide hvor Kurt skal hen. Han skal jo selv med.

Preben kommer over til Kurt. ``Hvor skal du hen Kurt'' spørger han lidt
stakåndet. Kurt forstår stadig ikke. Når han tænker godt efter, er han osse ret
sikker på, at han og Preben snakkede om det i går efter kampen, hvor de sad og
fejrede Danmarks sejr over/druknede sorgene over Danmarks nederlag til
Vesttyskland. Men han husker det ikke tydeligt, hvilket, hvis han skal være helt
ærlig, kan skyldes den mængde alkoloiske drikke, de nåede at indtage. Let
bebrejdende -- Preben burde da også kunne huske det -- siger Kurt ``Jeg skal til
revy''. Preben fniser. Kurt forstår ikke hvorfor. Preben fniser endnu
mere. ``Prøv at se på din billet'' siger han til Kurt. Kurt ser på sin
billet. ``DIKUREVY 1986'' står der med store, lidt klodsede bogstaver -- og så
er der nogle ubehjælpsomme tegniger. Kurt ved godt hvad tegningerne
betyder. Maskerne betyder revy, den lille cirker med barnligt tegnet bestik ved
siden af skal forestille en tallerken, og den betyder at man er tilmedt
spisningen, og tingesten foroven er et glas og betyder, at Kurt skol have vin
til maden. ``Jeg ved da godt, hvad det betyder'' siger Kurt forarget til Preben,
som jo ikke behøver at få at vide, at Ilse måtte forklare Kurt det før han
forstod det -- to gange endda. ``Maskerne betyder revy, tallerkenen og bestikket
betyder ...'' ...  ``Det er ikke det jeg mener'' afbryder Preben -- lidt
uhøfligt, synes Kurt -- ``Prøv at se oppe til venstre''. ``Lørdag d. 14. juni,
Middag kl. 18, Revy kl. 20'' står der. Kurt ser på sit avancerede digitalur, som
ud over klokken også kan vise ham ugedag, dato, hvad tid han skal op om
morgenen, hvad klokken er i Mexico City og mange andre ting, som Kurt heller
ikke har brug for. Det er den 14. og klokken er lidt i seks om aftenen. Kurt
forstår stadig ingenting. ``Hvor er du på vej hen'' spørger Preben
storfnisende. Kurt bliver pludselig ildrød i hovedet. ``På UP1'' står der jo.

Preben griner højt nu, og det værste er, at Ingelise, som Kurt har været lun på
et godt stykke tid snart, er kommet og står og fniser med. ``Jeg øh, oh øh, jeg
skulle lige over og gente et par øller i automaten, vil I have nogen med''
improviserer Kurt lynhurtigt. Det er godt nok en lidt tynd undskyldning, men
hva'. Nu er både Preben og Ingelise ved at komme om af grin. ``Velkommen til
1986, Kurt'' får Ingelise fremstammet mellem to latterhulk. Kurt ved godt, at
der er 1986, men han kan ikke se, hvorfor det skulle være særlig morsomt. ``Hvor
tit kommer du på DIKU'' spørger Preben sødt. Kurt kan ikke se, hvad det har med
sagen at gøre. For eksempel har han jo lige været inde at købe billetter for 4
dage siden -- og gangen før -- jah, det var vel -- øhh -- til julefrokosten. Men
det behøver de vel ikke at bore i, bare fordi man ``forsker hjemme''. ``Jeg
kommer da herinde regelmæssigt'' siger Kurt.  ``Ja, til sommerfest og til
julefrokost'' siger Preben, ``det er over 4 måneder siden at hele kantinen
flyttede over til UP1, kom så med dit fæ''. Kurt må kapitulere og lader sig
blidt føre tilbage over Jagtvej og over mod UP1. ``Satans'' tænker Kurt, ``nu
syns Ingelise nok, at jeg er et værre kvaj, jeg har nok forspildt min chance for
i aften''. Men heldigvis virker det ikke som om Ingelise har noget imod, at Kurt
har kvajet sig, tvært imod, hun stikker sin arm ind under Kurts og går og nynner
lidt. Kurt kender godt melodien. Han kender også godt teksten. Den handler om
ham. Den handler også om Preben og om Ingelise. Den handler også om dig. Den
lyder sådan her:
\end{sketch}

\begin{song}
\sings{Holm}
Du kan få, hva' du ka' li'
bare du læser datalogi
Du kan få, hva' du ka' li
der' ingen grænser or hva' du ka' bli'
bare begynd, det' ikke så svært
og efter nogle år får du det sikkert lært
at du kan få, hva' du ka' li'
bare du lser datalogi
\end{song}

\begin{sketch}
\scene{Her holdes en lille pause, mens folk klapper}

\says{Holm}
Der er nogen, der rusker i Kurts skulder. ``La' vær''' mumler han gnavent. Det
bliver ved. Kurt prøver at dreje sig lidt og trække skulderen til sig. Det
hjælper ikke. Nu er der osse nogen, der siger ``Kurt'' lige ind i øret på
ham. Pludselig er Kurt lysvågen. Han sidder jo i det store auditorium på UP1 og
ser revy, og det er Ingelise, der sidder og rusker i ham. ``Gamle fulde
apparat'' siger hun kærligt til ham, da hun ser, at han er vågen, ``det er snart
forbi''. ``FORBI !?'' Kurt er rystet, han er jo lige kommet. ``Vi er jo lige
kommet'' får han med besvær fremmumlet. Det er lidt svært at styre tungen, synes
Kurt. Ingelise ryster på hovedet og smiler. ``Du har sovet siden midt i første
akt. Ca. en time''. Kurt er målløs -- bortset fra en irriterende hikken af og
til. Så skulle han alligevel have nøjedes med to flasker vin. Satans. Nå, men
han må jo hellere nyde det, der er tilbage af revyen, så kan han da i det
mindste snakke med om \underline{det} bagefter. Kurt retrer sig op i sædet og
kigger ned på scenen. Der sidder to fyre, den ene spiller guitar og den anden
snakker. De ser ud som om de hygger sig. Kurt kender dem godt, det er nogle
gamle nisser, der har læst i mange herrens år -- længere end Kurt i hvert fald
-- og de har også været med til de andre revyer, Kurt har set. Kurt prøver at
høre efter. Han synes han kender melodien. Jo, det er jo den, der handler om
ham.  Tænk at sidde her og vågne op, og så handler det om en
selv. Fantastisk. Det er sådan en underlig rekursiv fornemmelse -- et nummer der
handler om, at man sidder og ser et nummer, der handler om at man sidder og ser
et nummer, der handler om -- Kurt bliver svimmel og må lukke øjnene et
øjeblik. Men nu -- nu handler det ikke om Kurt længere ! Hvad er det de sidder
og siger?  De siger, at nu er det altså helt forbi, at der ikke kommer flere
ekstranumre og at dormitoriet skal evakueres igen at vi allesammen skal gå op i
kantinen og golde sommerfest. Kurt er skuffet. Han havde håbet på, at der kom
mere, men det gør der altså ikke. Han ville ellers så gerne have klappet og
buh'et og sunget med og -- men nu, nu , NU synes Kurt at det er FOR MEGET, Kurt
vil synge og han rejser sig op og synger så højt han kan
\end{sketch}

\begin{song}
Du kan få ....
\end{song}

\end{document}
