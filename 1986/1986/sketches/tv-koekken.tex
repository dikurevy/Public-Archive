\documentclass[a4paper,11pt]{article}

\usepackage{revy}
\usepackage[utf8]{inputenc}
\usepackage[T1]{fontenc}
\usepackage[danish]{babel}


\revyname{DIKUrevy}
\revyyear{1986}
% HUSK AT OPDATERE VERSIONSNUMMER
\version{1.0}
\eta{$n$ minutter}
\status{Færdig}

\title{TV-køkken}
\author{Vilmar, Lars m.fl.}

\begin{document}
\maketitle

\begin{roles}
\role{I}[Lise] Interviewer
\role{F1}[Jacob] Flyttemand
\role{F2}[Flemming] Flyttemand
\role{U1}[Karen] UNI-chef
\role{U2}[Sanne] UNI-chef (Aksel)
\end{roles}

\begin{sketch}

\scene{Overtages uændret fra Lokalnetsang.  Flyttemændene rydder lidt efter lidt
  alt overflødigt væk.}

\says{I} For at nedkøle gemytterne ldit bringer vi nu et beroligende og
søvndysende indslag fra TV-køkkenet, hvor 2 Bang-kokke fra UNI-C -- som vi
omsider har fået forbindelse til -- vil demonstrere UKUXNAD-lavning.

\says{U1} Ja, og vi skal i dag se på tilberedning af et menustyret system.  Det
er som bekendt meget søvndyssende. Aksel, vil du læse menuen op?

\says{U2} Vi starter med en lille appetitvækker -- en såkaldt appetizer --
bestående af chips med stærk hvidløgsdressing.  Derefter tager vi en kold start,
som vi koger suppe på, indtil det går i fisk.  Hovedretten er en CAD/CAM med
grise-kode og saftige fejl stegt i bitfedt i en mikro-ovn.

\says{U1} Jeg kan lige indskyde, at hvis man ikke kan få grise-kode, så kan man
nøjes med den lidt billigere spaghetti-kode, men det \underline{er} jo vigtigt
ved menu-styrede systemer som ved al anden UKUXNAD-lavning at have gode
råvarer.  F.eks. bør man altid benytte det allerbedste smør.

\says{U2} Dertil serveres en båndsalat med grofthakkede kvadratrødder og
sortrete agurk.  Desserten er en 7-lags CICS'e-kage med card jam -- pyntet med
flag.
\says{U1} Det er jo dejligt, det du der disker op med. Hvad drikkevarer angår,
kan man enten hente sig en øl i den endelige automat -- eller nøjes med en tynd
kop te.  Til desserten vil vi anbefale en port-vin. Se godt på etiketten.

\says{U2} Da enhver ved, at det tager lang tid at tilberede et menu-styret
system, vil vi springe det rutine-prægede over -- det kan jo findes i ethvert
programbibliotek -- og gå direkte til desserten: JUMP DIRECT DESSERT.

\says{U1} Ja, desserter er ikke sådan at løbe fra. Man skal i hvert fald løbe
langt -- f.eks. ud i sandet.  Desserten er som sagt en 7-lage CICS'e-kage efter
den berømte OSI-opskrift.  Den er meget vanskelig at lave: Hvis det ikke
CICS'er, går det i kage.

\says{U2} Så man må sørge for at være i form.  Først tager man en kraftig
kærnemælk, og...

\says{U1} Er du gal mand?  Ikke noget om kerner. Det skal være beroligende det
her!

\says{U2} Øh, nå nej, undskyld mit lille uheld -- jeg mener hændelse. Nå, men så
tager man noget andet og blander det godt sammen og drysser med hakkede
nøddekerner -- nej, nej, jeg mener -- krydrer med kommentarer.

\says{U1} Og så smelter man noget chokolade ned over...

\says{U2} Hov! Ikke noget om nedsmeltning! Ikke et ord! Vi, øh, man, øh, vi
antager, at vi har noget cacao-pulver, som er flydende, og som ikke lige
pludselig står op i en sky. Nej, det er der slet ikke noget , der gør. Det ville
være fuldstændig usandsynligt. Det sker kun en gang hver 65.535. gang.

\says{U1} Vi kommer det bare i formen, så det ikke slipper ud, og...

\says{U2} Der er ikke noget, der slipper ud. Ikke den mindste bit, øh, den
mindste krumme. Der er krummer i det her. Der sker ikke noget.

\says{U1} Nej, slet ikke noget.

\says{U1+U2} Her sker slet ikke noget. Slet ikke noget. Slet ingen ting...

\says{I} Nu har i vist fordærvet dette her tilstrækkeligt. Exit
kokke. \scene{Flyttemænd rydder scenen for kokke m.v.} Det gik jo lidt i kage
det her, så vi bringer i stedet et be-åndet -øh- båndet indslag om hverdagslivet
på DIKU.

\end{sketch}

\end{document}
