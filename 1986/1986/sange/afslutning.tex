\documentclass[a4paper,11pt]{article}

\usepackage{revy}
\usepackage[utf8]{inputenc}
\usepackage[T1]{fontenc}
\usepackage[danish]{babel}


\revyname{DIKUrevy}
\revyyear{1986}
\version{1.0}
\eta{$n$ minutter}
\status{Færdig}

\title{Afslutning}
\author{Vilmar}
\melody{``Farewell, farewell''}

\begin{document}
\maketitle

\begin{roles}
\role{S}[Mia] Sanger
\end{roles}

\scene{Efter at have kigget hele ``revyen'' iennem kunne jeg altså ikke nære mig
for at skrive et vammelt afslutningsnummer. Både fordi vi har et vammelt
indledningsnummer (velkomstsangen) og fordi dispensationsbutikken ikke er noget
særlig egnet nummerat slutte revyen af med. Min ide med dette nummer er, at
ekspedienten fra disp. butikken synger nedenstånde lille yndige sang efter at
studenten er forsvundet ud af scenen. Melodien hedder - meget passende --
``Farewell farewell'' og er af Fairport Convention.  Teksten er et sammenkog af
brokker fra tidligere års afslutningsnumre. Under soloen kommer resten af
revygruppen ind med lommetørklæder og tørrer en tåre væk, og under sidste
``farvel farvel'' viker alle og der SÅ sørgelit altsammen.}

\begin{song}
\sings{S}
Farvel, farvel -- for denne gang
vi synger nu på sidste vers
mod enden det lakker igen så småt
så ha' det rigtig godt

Farvel, farvel -- en afskedssang
det sidste blad har vi nu vendt
vi takker fordi at i gad høre på
nu skal vi til at gå

\scene{Solo}
\sings{S}
Farvel, farvel -- et ømt farvel
revyens stjerner år i hi
vi håber, vi ikke har spildt vores krudt
for nu' revyen slut, Farvel farvel.
\end{song}

\end{document}
