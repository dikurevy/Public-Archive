\documentclass[a4paper,11pt]{article}

\usepackage{revy}
\usepackage[utf8]{inputenc}
\usepackage[T1]{fontenc}
\usepackage[danish]{babel}


\revyname{DIKUrevy}
\revyyear{1975}
\version{0.1}
\eta{$n$ minutter}
\status{Ikke faerdig}

\title{Ilwo valsen}
\author{Tom Skovgaard (også melodi), Jens Hammerum}
\melody{Original melodi}

\begin{document}
\maketitle

\begin{roles}
\role{S1}[Karsten Kynde] Sanger
\role{S2}[Erik Meiling] Sanger
\end{roles}

\begin{song}
(Omkvæd; synges efter hvert vers)
Dagen begynder i DIKU-land
først når der kommer en ILWO-mand.

Timerne tidligt om morg'nen
er værre end man tror
man prøver forgæves at tænke
mens maverne rumler i kor
Ved indkøb af snitter og basser
det sikres skal at det passer
sådan at man morg'nen derpå
en lille bid brød kan få

Ole og Peter de springer
når klokken kvart i ti ringer
så sidder de bare og titter
og venter på basser og snitter

Om morgenen fimperne vinder
end ikke en knægt sidst man finder
men kommer der to kasser basser
så går de på konger i masser.
\end{song}

\end{document}

