\documentclass[a4paper,11pt]{article}

\usepackage{revy}
\usepackage[utf8]{inputenc}
\usepackage[T1]{fontenc}
\usepackage[danish]{babel}


\revyname{DIKUrevy}
\revyyear{1975}
\version{0.1}
\eta{$n$ minutter}
\status{Færdig}

\title{Ankers Pris}
\author{Peter Johansen}
\melody{"`Ankers Pris"' m. violon og englekor}

\begin{document}
\maketitle

\begin{roles}
\role{S1}[Erik Meiling] Sanger
\role{S2}[Karsten Kynde] Sanger
\role{K1}[Anker Helms Jørgensen] Kor
\role{K2}[Jan Tovgård] Kor
\end{roles}

\begin{song}
  Gartner Anker går sin runde
  alle blomster kender ham
  gartner Anker gi'r dem vand
  han holder ømt sin lille spand

  Gartner Anker vander blomster
  gartner Anker planter om
  Hvorfor Neriens blomster kom
  (ja) det ved Anker alting om

  Plantens nød er Ankers smerte
  plantens lyst er Ankers fryd
  heftigt banker Ankers hjer-
  te, når han skuer DIKU's pryd

  Anker passer vores blomster
  deres tarv er Ankers mål
  DIKU's blomster er ham tro
  og kun for ham vil planten gro

  Hvis vor gartner os forlader
  blomster visner, blade dør
  Hvem skal så os gi' musik
  fra blomsterblades mosaik?

  Plantens nød er Ankers smerte
  plantens lyst er Ankers fryd
  heftigt banker Ankers hjer-
  te, når han skuer DIKU's pryd
\end{song}

\end{document}
