\documentclass[a4paper,11pt]{article}

\usepackage{revy}
\usepackage[utf8]{inputenc}
\usepackage[T1]{fontenc}
\usepackage[danish]{babel}


\revyname{DIKUrevy}
\revyyear{1975}
\version{0.1}
\eta{$n$ minutter}
\status{Ikke faerdig}

\title{Tag en formel...}
\author{Peter Johansen}
\melody{Hen til kommoden}

\begin{document}
\maketitle

\begin{roles}
\role{S}[Mette Staugaard] Sanger
\end{roles}

\begin{song}
Børn nu skal I høre
hvor'n I skal opføre
jer hvis I datalogi studere vil
I skal kunne mere
end blot programmere
Knuth og Hoare og Naur de skal også til
Og så må I kunne
se så nogenlunde
vågne ud, selv om I sover indeni
I må end'lig ikke
snorke eller nikke
selv om Nils Andersel bli'r ved evindelig
Tag en formel og sæt i den
visse størrelser som nok skal passe
Hen til kommoden og tilbav's igen
så har I et program af første klasse.

I skal op hver morgen
høsten eller våren
meget tidligt, det er ikke altid let
Skønt man har kvarteret
til at blive placeret
er der tit lidt tomt i auditoriet
Ja, og Anton Jensen
ville gå til grænsen
men han står og mumler og er gåt i stå
LIMP er ingenting til
beviset halter indtil
en fra sidste række tør foreslå:
"`Tag en formel og sæt i den
infinitterne sku' nok ku' passe
Hen efter svampen og tilbav's igen
så har du et bevis af første klasse."'

Det kan gerne være
at man kunne lære
ting i datalogi, som ku' bruges, men
der var ladrig nogen
der ku' hold' sig vågen
mer end et kvarter af forelæsningen
Så fik de den dille
alting at indspille
Stands blot båndet, når I ondt i ho'det får
Arbejdskraft det sparer
især når man erfarer
at de samme bånd nu kør' på ti'nde år
Tag maskinen, læs kort i den
Når lampen lyser, så er RECKU oppe
Hen til maskinen og tilbav's igen
for også denne kørsel må vi droppe.

Matematikerne
abstraherer gerne
men om anvendelser hør' I ikke't kuk
og pædagogikken
bliver ladt i stikken
man sku' sikkert hellere ha' læst på RUC
Men på datalogisk
er man mer' metodisk
prøverne er rationaliseret væk
Kan I se komikken?
Eksamensmekanikken
kører efter det velkendte princip:
Tag et skema og fyld det ud
blyantskrydserne skal nok ku' passe
Hen til eksamen og tilbav's igen
så har du et bevis for første klasse.
\end{song}

\end{document}

