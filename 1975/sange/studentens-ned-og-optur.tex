\documentclass[a4paper,11pt]{article}

\usepackage{revy}
\usepackage[utf8]{inputenc}
\usepackage[T1]{fontenc}
\usepackage[danish]{babel}


\revyname{DIKUrevy}
\revyyear{1975}
\version{0.1}
\eta{$n$ minutter}
\status{Ikke faerdig}

\title{Studentens ned- og optur}
\author{Erik Meiling}
\melody{Tom Dooley}

\begin{document}
\maketitle

\begin{roles}
\role{S}[Erik Meiling] Sanger
\end{roles}

\begin{song}
Troede da jeg started
at det her var lige sagen
troede det var lykken
at gå og hulle hele dagen.

Men jeg blev hurtigt klogere
fik det hurtigt lært
dat0 er ikke morsomt
det er bare fandens svært.

Lavede så programmer
lærte at stå i kø
kom dog aldrig længere
end hullestuen, HCØ.

Så var jeg helt alene
stod der uden ven
læste så et opslag
nu ved jeg hvor jeg går hen.

For redningen er kommet
trøst for hvert et tab
kom nu alle venner
ta' og meld jer ind i FLAB!

For FLAB tager dig i hånden
giver dig et tiltrængt hvil
dulmer dine nerver
aldrig mere Restenil.

Programmerne kører bedre
instruktoren giver dig følgeskab
læreren smiler til dig
tag og meld dig ind i FLAB.
\end{song}

\end{document}

