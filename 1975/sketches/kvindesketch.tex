\documentclass[a4paper,11pt]{article}

\usepackage{revy}
\usepackage[utf8]{inputenc}
\usepackage[T1]{fontenc}
\usepackage[danish]{babel}


\revyname{DIKUrevy}
\revyyear{1975}
% HUSK AT OPDATERE VERSIONSNUMMER
\version{0.1}
\eta{$n$ minutter}
\status{Ikke færdig}

\title{Kvindesketch}
\author{Kvindegruppen}

\begin{document}
\maketitle

\begin{roles}
\role{F1}[Anker Helms Jørgensen] 1. fyr
\role{F2}[Erik Meiling] 2. fyr
\role{P}[Mette Staugaard] Pigen
\end{roles}

\begin{sketch}

\says{F1} Hvad laver I egentlig i den der kvindegruppe?

\says{P} Jo, ser du altså sådan set... vi sidder og hyggesnakker, og
så snakker vi om de problemer, man har som kvindelig
datalogistuderende (bl.a. som dat0'er).

\says{F2} Hvad mon det skulle være for problemer?

\says{F1} Det er da selvfølgelig deres frustration, når de skal
beslutte sig for at give efter for deres masochistiske trang til at
udfylde en maternalistisk rolle, hvormed de fornægter deres akademiske
evne, hvor lille den end måtte være.

\says{F2} Nåh, du mener, at deres naturlige dovenskab forhindrer dem i
både at studere og få børn?

\says{F1} Næe, du udtrykker dig uden finesse.  Hvad jeg mener er, at
det selvfølgelig er deres frustration, når de skal beslutte sig for at
give efter for deres masochistiske trang til at udfylde en
maternalistisk rolle, hvormed de fornægter deres akademiske evne, hvor
lille den end måtte være.

\says{P} Vi har f.eks. det problem, at vi aldrig får lov til at sige
noget i forsamlinger.
\end{sketch}
\end{document}
