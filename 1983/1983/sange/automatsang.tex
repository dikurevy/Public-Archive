\documentclass[a4paper,11pt]{article}

\usepackage{revy}
\usepackage[utf8]{inputenc}
\usepackage[T1]{fontenc}
\usepackage[danish]{babel}


\revyname{DIKUrevy}
\revyyear{1983}
\version{0.1}
\eta{$n$ minutter}
\status{Ikke færdig}

\title{Kantine-automatens sang til Kurt}
\author{?}
\melody{Sebastian: ``En kos bekendelser''}

\begin{document}
\maketitle

\begin{roles}
\role{A}[] Automat
\end{roles}

\begin{song}
\sings{A} Jeg'r den lille automat,
          din kantines surrogat
          for en køledisk med øl og vand'er.
          Prop en femmer i og stræk
          hånden ud -- tryk -- og træk,
          når du hører, at din femmer lander!

\sings{A} Sådan står jeg i kantinen.
          Du kommer fornøjet og glad.
          Larm og brum fra ølmaskinen,
          så nøj's du med kaffe og mad!
          Å' syn's du, fornemmelsen er -- flad?

\sings{A} Jeg'r den lille ølregent,
          står i hjørnet som betjent,
          vogter konsumeringen af øller.
          Ølrationen laves om,
          pluds'lig er min maves dom:
          Ikke flere øl til fulde bøller!

\sings{A} Sådan står jeg i kantinen.
          Du kommer fornøjet og glad.
          Larm og brum fra ølmaskinen,
          så nøj's du med kaffe og mad!
          Å' syn's du, fornemmelsen er -- flad?

\sings{A} Jeg'r den smarte mekanik,
          der kan skabe vild panik,
          når jeg stopper dine bajerdrømme.
          Kløj's i mønterne og går
          i uend'lig løkke, så
          vand'er vælter ud i stride strømme!

\sings{A} Sådan står jeg i kantinen.
          Du kommer fornøjet og glad.
          Larm og brum fra ølmaskinen,
          så nøj's du med kaffe og mad!
          Å' syn's du, fornemmelsen er -- flad?

\end{song}

\end{document}

