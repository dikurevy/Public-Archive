\documentclass[a4paper,11pt]{article}

\usepackage{revy}
\usepackage[utf8]{inputenc}
\usepackage[T1]{fontenc}
\usepackage[danish]{babel}


\revyname{DIKUrevy}
\revyyear{1983}
% HUSK AT OPDATERE VERSIONSNUMMER
\version{0.1}
\eta{$n$ minutter}
\status{Mere underlig end del 1}

\title{Kurts studiestart, del 2}
\author{?}

\begin{document}
\maketitle

\begin{sketch}

Kurt har nu læst et par uger og fundet både HCØ og auditorium 1. Vandrehallen er nu ligegodt ikke særlig rar, synes Kurt. Masse mennesker -- og så er der så stort, koldt og upersonligt. ``Jeg kender jo ikke et øje,'' tænker Kurt en dag i frokostpausen i kantinen. Det er nu ikke helt i overensstemmelse med sandheden, når Kurt tænker lidt efter. Der var en dag, hvor han havde sagt hej til en pige, der havde gået en klasse over ham i gymnasiet i Holbæk, og en fyr der også læser datalogi havde snakket med Kurt et par gange efter DAT-0 forelæsning. Kurt betaler sin mad og ser sig om i kantinen. Den pige fra gymnasiet er der ikke og ham fra forelæsningen -- jo, der sidder han. Kurt sætter sig hen til ham. ``Hej Kurt,'' siger fyren, Preben hedder han. De snakker lidt om maden. Kurt synes den er meget god, men Preben siger, at hos Aksel får man meget bedre smørrebrød. De aftaler at købe frokost hos denne Aksel næste dag, hvor de skal til øvelser.

Næste dag mødes Kurt med Preben udenfor DIKU. De går over til Aksel, som viser sig at være den slagter, Kurt var inde hos den første dag. Heldigvis er den lille hvidhårede dame der ikke. Der er en sorthåret der ekspederer. ``Jeg vil godt ha' 3 stykker,'' siger Kurt og kigger sig omkring. Det er svært at vælge, synes han, der er jo så meget. ``Nå, bliver det snart til noget, jeg har ikke hele dagen,'' siger damen til Kurt, som får fremstammet ``Har-har de oksebryst?'' Nej, det har jeg ikke, ser jeg måske sådan ud, knægt?'' siger damen brysk. Kurt er rystet. Så foretrækker han alligevel den hvidhårede. Kurt får bestilt sine madder og sammen med Preben går han tilbage til DIKU. ``Det var dog mærkeligt,'' synes Kurt, da han får pakket maden ud og kigger på et stykke med oksebryst, ``hun sagde da at hun ikke havde noget.'' Preben ryster på hovedet og siger overbærende:

\begin{quote}
    Du kan få, hva' du ka' li' \\
    bare du læser datalogi \\
    Du kan få, hvaø du ka' li' \\
    der' ingen grænser for, hva' du ka' bli' \\
    bare begynd, det' ikke så svært \\
    og efter et par uger har du fået det lært \\
    Du kan få, hva' du ka' li' \\
    bare du læser datalogi 
\end{quote}

\end{sketch}
\end{document}
