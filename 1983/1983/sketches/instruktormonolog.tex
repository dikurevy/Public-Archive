\documentclass[a4paper,11pt]{article}

\usepackage{revy}
\usepackage[utf8]{inputenc}
\usepackage[T1]{fontenc}
\usepackage[danish]{babel}


\revyname{DIKUrevy}
\revyyear{1983}
% HUSK AT OPDATERE VERSIONSNUMMER
\version{0.1}
\eta{$n$ minutter}
\status{Færdig}

\title{Kurts instruktormonolog}
\author{?}

\begin{document}
\maketitle
\begin{sketch}
\begin{quote}
    Du kan få, hva' du ka' li' \\
    bare du læser datalogi \\
    Du kan få, hvaø du ka' li' \\
    der' ingen grænser for, hva' du ka' bli' \\
    bare begynd, det' ikke så svært \\
    og efter et par uger har du fået det lært \\
    Du kan få, hva' du ka' li' \\
    bare du læser datalogi 
\end{quote}

Nu har Kurt været instruktor i 3 år, og han er ved at være godt og grundigt træt af det. Der kommer højst halvdelen af holdet til øvelser, og de, der kommer, har aldrig forberedt sig en brik. Og rapportopgaverne! Nutidens unge mennesker kan hverken stave eller formulere sig! Det der med at formulere sig er det nu heller ikke alle lærere, der er supergode til. I de år, Kurt har været instruktor, har der været 3 lærere på hans kursus. Den ene var meeeeget god, men de andre...

\end{sketch}
\end{document}
