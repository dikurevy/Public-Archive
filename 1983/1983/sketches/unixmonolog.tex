\documentclass[a4paper,11pt]{article}

\usepackage{revy}
\usepackage[utf8]{inputenc}
\usepackage[T1]{fontenc}
\usepackage[danish]{babel}


\revyname{DIKUrevy}
\revyyear{1983}
% HUSK AT OPDATERE VERSIONSNUMMER
\version{1}
\eta{$n$ minutter}
\status{Færdig}

\title{Kurt's unixmonolog}
\author{?}

% Hvorfor hedder den Kurt's unixmonolog? Det er ikke Kurt der snakker, og unix bliver aldrig nævnt...

\begin{document}
\maketitle

\begin{sketch}

Kurt har aftalt at mødes med Preben og en pige der hedder Anne-Marie, i kantinen på DIKU og spise inden øvelser. De har lige fået deres første rapportopgave og vil godt diskutere den sammen. Kurt går ind til Aksel -- han er stadig ikke helt tryg ved damerne -- og be'r om 2 stykker. 1 med karrysalat og 1 med noget, Anne-Marie havde kaldt ``hurrasalat''. ``Har du ingen tænder knægt, skal vi også tygge kødet for dig?'' spø'r den lyshårede. Kurt griner, han har efterhånden fundet ud af at de i virkeligheden er enormt søde. Da han kommer over på DIKU går han op på første sal. Der kommer han pludselig i tvivl; skal man ind ad døren til venstre eller døren til højre for at komme ind for at komme ind i kantinen? Kurt går over til højre. Der lyder både snak og kaffekopsraslen -- og latter derinde. ``Det må være kantinen'', tænker Kurt, det lyder jo nærmest som Vin og Ølgod. Kurt går derind.

\end{sketch}
\end{document}
