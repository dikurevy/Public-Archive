\documentclass[a4paper,11pt]{article}

\usepackage{revy}
\usepackage[utf8]{inputenc}
\usepackage[T1]{fontenc}
\usepackage[danish]{babel}


\revyname{DIKUrevy}
\revyyear{1983}
% HUSK AT OPDATERE VERSIONSNUMMER
\version{1}
\eta{$n$ minutter}
\status{Færdig}

\title{Dat-0 Gymnastik}
\author{?}

\begin{document}
\maketitle

\begin{roles}
\role{I}[] Instruktor
\role{D0-2}[] Dat-0'ere
\end{roles}

%\begin{props}
%\prop{Rekvisit}[Person, der skaffer]
%\end{props}


\begin{sketch}

\scene{D følger I's instruktioner}

\says{I} Velkommen til øvelser, vi starter med DIKU's begyndergymnastik. Varm op med let løb på stedet. Løb nu hen til piccoloen og sæt Jer foran ``dyret'', hø hø.

\says{I} Tag disketten frem med højre hånd \act{tager en imaginær diskette frem}. Før venstre hånd skråt ned til venstre og tænd for controlleren og diskettestationen. Flyt blidt disketten fra højre hånd over til venstre hånd, og hold den med tommel- og pegefinger, ja såååådan. Put disketten ind i diskettestationen og luk lugen FORSIGTIGT! Pas endelig på fingre, slips og andre løshængende genstande. Før venstre hånd ned og tryk på ``RESET''.

\says{I} Mens vi booter skal vi have nogle fingerøvelser -- vi skal nemlig prøve at skrive et program.

\says{I} Vi varmer op med at sætte fingrene sammen. Tryk til -- slap af. Og en gang til, tryk -- slap af. 

\says{I} Knyt nu hænderne -- og stræææk ud, og en gang til, knyyt og strææk.

\says{I} Og nu til programmet. Først skal vi ind i dat-0 systemet. Tast: d, a, t, 0 og tryk på ``return''.

\says{I} Ja, og så skulle vi være klar til selve programøvelsen: Begin \act{D taster lystigt}

\scene{PUNK starter}

\end{sketch}
\end{document}
