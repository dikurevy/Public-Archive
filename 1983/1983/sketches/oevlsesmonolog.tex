\documentclass[a4paper,11pt]{article}

\usepackage{revy}
\usepackage[utf8]{inputenc}
\usepackage[T1]{fontenc}
\usepackage[danish]{babel}


\revyname{DIKUrevy}
\revyyear{1983}
% HUSK AT OPDATERE VERSIONSNUMMER
\version{0.1}
\eta{$n$ minutter}
\status{Færdig}

\title{Kurts øvelsesmonolog}
\author{?}

\begin{document}
\maketitle

\begin{sketch}

Kurt synes, der er vældig hyggeligt i DIKUs kantine. Det er en skam at dat-0'erne ikke rigtig har noget at gøre der, for MAI er der ikke meget ved. Kurt er på MAI, når han skal have øvelser; det har han prøvet to gange, han synes hans instruktor er rigtig sød. Marianne hedder hun. Og hun er dygtig, Kurt kan aldrig følge med til forelæsningerne, men når Marianne forklarer ved tavlen, går det meget bedre, Kurt kan næsten forstå det. Og når man spø'r om man ikke kan få det gennemgået en gang til, så smiler hun sødt og siger:

\begin{quote}
    Du kan få, hva' du ka' li' \\
    bare du læser datalogi \\
    Du kan få, hvaø du ka' li' \\
    der' ingen grænser for, hva' du ka' bli' \\
    bare begynd, det' ikke så svært \\
    og efter et par uger har du fået det lært \\
    Du kan få, hva' du ka' li' \\
    bare du læser datalogi 
\end{quote}

\end{sketch}
\end{document}
