\documentclass[a4paper,11pt]{article}

\usepackage{revy}
\usepackage[utf8]{inputenc}
\usepackage[T1]{fontenc}
\usepackage[danish]{babel}


\revyname{DIKUrevy}
\revyyear{1983}
% HUSK AT OPDATERE VERSIONSNUMMER
\version{0.1}
\eta{$n$ minutter}
\status{Færdig}

\title{Kortspil sketch}
\author{Mads Tofte og Knud Henriksen}

\begin{document}
\maketitle

\begin{roles}
    \role{ON}[] Ove Nathan
    \role{BH}[] Bertel Haarder
    \role{CR}[] Christian Rovsing
    \role{ES}[] Edda Sveinsdottir
    \role{HO}[] Henrik Olsen
    \role{S}[] Sekretær
\end{roles}

\begin{props}
\prop{Spillebord}[Person, der skaffer]
\prop{Navneskilte}[Person, der skaffer]
\end{props}

\begin{sketch}

\scene{Scenen består af et spillebord, og på bagtæppet er der markeret, hvad de enkelte personer hedder, og hvor de sidder.}

\scene{BH, ON, CR og HO kommer ind fra forskellige sider af scenen og omfavner hinanden under højlydte hilsner.}

\says{BH} Nå, det gik ellers ikke så godt sidste gang, hva'? Ha, ha, ha...

\says{ON} Nej, det var sgu' noget være noget, at du snuppede humaniora.

\says{CR} Nå, skal vi se at komme igang?

\says{BH} Ja, ja. Skru' lige den pæne mine på, så får vi lige et par sjusser. \act{S kommer ind og deler rundt} Nå, lad os så se at komme i gang. Har du kortene med, Nathan?

\says{ON} Gud nej, dem har jeg vist glemt.

\says{BH} Nå, du tager måske ikke disse forhandlingsrunder alvorligt?

\says{ON} Jo, naturligvis, jeg beder ministeren undskylde...

\says{BH} Ja, ja, du skal ikke beklage, men nu må vi jo se \act{klapper ON på skulderen} om du ka' beholde dit selvstyre -- sidste gang var den jo tæt på! \act{til HO} Hør, jeg synes ikke jeg har set Dem før; det er måske vores studenterrepræsentant?

\says{HO} Jo, det ka' jeg jo ikke løbe fra.

\says{BH} Nå, udemærket. Har DU måske kortene med?

\says{HO} Ja. \act{tager kortene op ad lommen}

\says{BH} Udmærket. Sig mig, hvad er det egentlig du hedder?

\says{HO} Henrik Olsen

\says{BH} Ja, det er jo ikke sikkert vi ses igen, men rigtig held og lykke.

\says{CR} Jeg vil gerne henstille, at vi indleder forhandlingerne snarest muligt. Jeg har et ganske vigtigt møde om halvanden time i TV-byen.

\says{BH} Nå, det er måske med denne her dame, de kalder Manuella, har, har, har. Hør hvor bliver Edda af?

\says{ES}[Kommer ind] Neej, her sidder I rigtig nok og hygger Jer.

\says{BH} Godaaaawooouuu \act{stor omfavnelse}.

\says{ES} Tænk, har I virkelig holdt den her plads til mig? Er I gået i gang?

\says{BH} Nej, vi ventede selvfølgelig på dig.

\says{CR} Undskyld, må jeg henstille, at vi snarest begynder forhand...

\says{HO} Ja, jeg er helt enig. \act{Alle STIRRER på HO -- pinlig tavshed. De tager øjnene til sig, og BH giver kort.}

\says{ON} [Rømmer sig] Nå, lad os se, hvad vi har at spille om.

\says{CR} Ja, jeg mangler kandidater.

\says{ES} Jeg har ingen penge!

\says{HO} Der er for mange studerende til for få lærere.

\says{ON} Jeg har mit selvstyre at spille om.

\says{BH} Ja, selv spiller jeg ud fra devisen: ``Flere stole -- nul gysser''. Så er det tid til indsatser.

\says{CR} Gratis lærere.

\says{ES} Ja, jeg prøver med planudvalget.

\says{HO} Den går jeg med på.

\says{ON} Jeg støtter planudvalget.

\says{BH} Flere stole, nul gysser og gyldne løfter, der rydder bordet!

\says{HO} Satans, gyldne løfter er trumf!

\says{BH} Ha, ha, ha. Ja, det gik jo meget nemt.

\says{ES} Mit udvalg!

\says{BH} Så, så, lille Edda. Et enkelt udvalg fra eller til...

\says{ON} Jamen Bertel, gamle ven. Du kan da ikke sende mig hjem og sige, at jeg har tabt mit selvstyre. Vis nåde!

\says{BH} Nå, ja, ja. Vi politikere kan da også vise storsind -- tør du være med på en kvit eller dobbelt?

\says{ON} Ja, jeg har vel ikke noget valg.

\says{BH} Nej, såmænd du har ej. Det er dig der gi'r. \act{ON giver kort}

\says{CR} Min indsats bliver: FLERE gratis lærere.

\says{HO} Vi fordobler planudvalgets tal: 40 studerende på hovedfag.

\says{ES} Jeg går op til 70 studerende på hovedfag.

\says{ON} Ja, jeg må jo følge med.

\says{BH} 200 studerende, skrappe optagelsesbetingelser, nul gysser og oceaner af gyldne løfter!

\says{ON} Det er BLUFF Bertel!

\says{BH} Fandens også. Nå, lad os se kortene.

\says{CR} Royal Straight Flush i klør.

\says{ON} Royal Straight Flush i ruder.

\says{BH} Royal Straigh Flush i SPAR.

\says{ES} Royal Straigh Flush i HJERTER.

\says{HO} BAR RØV!

\says{BH} Nå, det var vel nok synd for dig; men du ka' få lov til at se på.

\says{ON} Puh, den var vel nok tæt på; men jeg klarede den da denne gang.

\says{CR} Udmærket, men det betyder kandidater til mig.

\says{ES} Og jeg får gysserne, Bertel!

\says{BH} Ja, ja, det er jo ikke MINE penge, og jeg får da en pæn del politisk prestige plus et par hundrede stole.

\says{HO} Jamen, hva' med os? Vi er for mange studerende allerede og hvad bliver det for nogen der kommer ind i fremtiden, og hvad med alle vores lærere, der rejser... \act{fortsætter med ``og hvad med...'' spørgsmål indtil han bliver afbrudt}

\says{Alle} Bertel barberer bare bedre!

\scene{Tæppe}

\end{sketch}
\end{document}
