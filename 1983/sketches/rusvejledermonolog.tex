\documentclass[a4paper,11pt]{article}

\usepackage{revy}
\usepackage[utf8]{inputenc}
\usepackage[T1]{fontenc}
\usepackage[danish]{babel}


\revyname{DIKUrevy}
\revyyear{1983}
% HUSK AT OPDATERE VERSIONSNUMMER
\version{1}
\eta{$n$ minutter}
\status{Færdig}

\title{Kurt's rusvejledermonolog}
\author{?}

\begin{document}
\maketitle

\begin{sketch}

En dag opdager Kurt et skilt i kantinen, ``Rusvejledere søges'', står der. ``Det ku' da være meget sjovt at prøve,'' tænker Kurt. Han kan huske hvor sjovt det var da han selv var på ruskursus, det vil han godt prøve igen. Kurt går til det første møde for at se hvad det egentlig er for noget. Han synes folk snakker så meget på mødet og han forstår ikke rigtigt hvad de snakker om, men Kurt synes det lyder meget imponerende. Kurt tør ikke sige noget til mødet men beslutter sig alligevel til at være med til at lave ruskursus. Kurt kommer på et hyttehold og i løbet af sommeren får de lavet noget der ligner et ruskursus. Kurt føler at han er blevet meget bevidstgjort af at lave ruskursus og han føler at han virkelig har noget fornuftigt at fortælle de nye dat-0'ere:

\begin{quote}
    Du kan få, hva' du ka' li' \\
    bare du læser datalogi \\
    Du kan få, hvaø du ka' li' \\
    der' ingen grænser for, hva' du ka' bli' \\
    bare begynd, det' ikke så svært \\
    og efter et par uger har du fået det lært \\
    Du kan få, hva' du ka' li' \\
    bare du læser datalogi 
\end{quote}

\end{sketch}
\end{document}
