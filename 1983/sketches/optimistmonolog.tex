\documentclass[a4paper,11pt]{article}

\usepackage{revy}
\usepackage[utf8]{inputenc}
\usepackage[T1]{fontenc}
\usepackage[danish]{babel}


\revyname{DIKUrevy}
\revyyear{1983}
% HUSK AT OPDATERE VERSIONSNUMMER
\version{0.1}
\eta{$n$ minutter}
\status{Færdig}

\title{Kurts optimistmonolog}
\author{?}

\begin{document}
\maketitle

\begin{sketch}
\begin{quote}
    Du kan få, hva' du ka' li' \\
    bare du læser datalogi \\
    Du kan få, hvaø du ka' li' \\
    der' ingen grænser for, hva' du ka' bli' \\
    bare begynd, det' ikke så svært \\
    og efter et par uger har du fået det lært \\
    Du kan få, hva' du ka' li' \\
    bare du læser datalogi 
\end{quote}

Det går ret godt for Kurt nu. Han er lige blevet instruktor og han har vundet to af de mest eftertragtede kurser i lodtrækningen på 2. del. Godt nok er det lidt hårdt og stresset, når man nu også sidder i institutråd og arbejder 30 timer om ugen hos Christian Rovsing, men Kurt mener, at han nok kan nå det, edb er jo trods alt fremtiden, ikke? Det får Kurt også at vide hver gang, han er på besøg hos sin familie, selvom de ikke kan forstå at det skal tage så lang tid at blive færdig. Kurt har også boet sammen med Anne Marie i halvandet år nu, men hun har det ikke så godt med DIKU og edb...

\end{sketch}
\end{document}
