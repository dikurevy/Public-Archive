\documentclass[a4paper,11pt]{article}

\usepackage{revy}
\usepackage[utf8]{inputenc}
\usepackage[T1]{fontenc}
\usepackage[danish]{babel}


\revyname{DIKUrevy}
\revyyear{1983}
% HUSK AT OPDATERE VERSIONSNUMMER
\version{1}
\eta{$n$ minutter}
\status{Underlig}

\title{Kurts studiestart, del 1}
\author{?}

\begin{document}
\maketitle

%\begin{roles}
%\role{A}[Skuespiller] Rolleforklaring
%\end{roles}
%
%\begin{props}
%\prop{Rekvisit}[Person, der skaffer]
%\end{props}


\begin{sketch}

``Universitetsparken, linie 43'' annoncerer chaufføren i bussens højtaleranlæg. Kurt retter sig op; det må være nu, han skal af. Kurt trykker på stopknappen og kæmper sig vej gennem horder af pensionister hen til døren. Bussen stopper, og Kurt stiger af. Han ser sig lidt omkring. Han kan genkende zoologisk museum fra udflugten med gymnasiet, og gaden må vel så være Universitetsparken. Det ligner nu nærmest en byggeplads, tænker Kurt ved sig selv, og hvor i alverden ligger mon dette ``Sigurdsgade''? Kurt får øje på nogle butikker på den anden side af lyskrydset. Måske ved de det derovre, tænker han, det kan da aldrig skade at spørge. Kurt går over til butikkerne og ind i en af de nærmeste, en slagter. Der står en lille venligt udseende, hvidhåret dame bag disken. ``Dav min dreng, er du blevet sulten?'' spørger hun Kurt i et imødekommende tonefald. Kurt får røde ører. Sådan taler butiksdamerne hjemme i Asnæs i hvert fald ikke til kunderne; desuden kan han ikke lide at blive kaldt ``min dreng'', han er trods alt lige startet på universitetet. ``N-n-nej, jeg skal ikke have noget,'' fremstammer Kurt, ``jeg ville bare spørge, om De kunne fortæller mig hvor Sigurdsgade ligger?'' Damen fortæller Kurt, hvor Sigurdsgade ligger, og Kurt går derover. Han finder sedlen frem fra lommen. Nr. 41 er det. Kurt finder nr. 41. D-I-K-U står der på et grønt skilt til højre for porten. Kurt går gennem porten. Der står en masse cykler inde i porten. Måske skulle jeg også anskaffe mig en cykel, tænker Kurt, eller måske en knallert føjer han til, efter kort at have overvejet afstanden fra Valby til Sigurdsgade.

Kurt går ind gennem en dør til højre i porten. Der sidder en venligt udseende, mørkhåret dame inde bag en glasrude. Kurt går ind til hende. ``Hvad kan jeg hjælpe dig med?'' spørger hun Kurt, som med tilfredshed noterer sig, at her bliver han ikke kaldt ``min dreng''. Kurt spørger om, hvor han kan melde sig til dat-0 øvelseshold og bliver vist ovenpå. Kurt går op ad vindeltrappen og ind på 1. sal og ganske rigtigt. Der er en dør, hvor der står ``Dat-0 Adm''. Kurt går ind gennem døren. ``Hov, kan sådan en skvat ikke lære at banke på inden han går ind?'' siger en -- ellers nydelig ung pige til Kurt på klingende jysk. ``Jo, undskyld,'' siger Kurt lidt beskæmmet og forskrækket, ``men kan jeg få en dat-0 øvelsesholdstilmelding?'' Pigen smiler sødt -- virkelig sødt -- til Kurt og siger:

\begin{quote}
    Du kan få, hva' du ka' li' \\
    bare du læser datalogi \\
    Du kan få, hvaø du ka' li' \\
    der' ingen grænser for, hva' du ka' bli' \\
    bare begynd, det' ikke så svært \\
    og efter et par uger har du fået det lært \\
    Du kan få, hva' du ka' li' \\
    bare du læser datalogi 
\end{quote}

\end{sketch}
\end{document}
