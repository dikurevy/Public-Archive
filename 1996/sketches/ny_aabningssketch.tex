\documentclass[10pt]{article}
\usepackage[utf8]{inputenc}
\usepackage{revy}
\title{Revyen i revyen}
\author{Theo Engell-Nielsen}
\version{1.2}
\parindent0pt
\parskip 1ex minus 1ex
\flushsingsright

\begin{document}
\maketitle

\begin{sketch}

\begin{roles}
  \role{3 Revyttere} ved et bord.
  \role{2 Revybosser} bag samme
\end{roles}

\scene På scenen står et bord - omkring det sidder 3 revyttere og
hænger/sover. Revybosserne kommer ind.

\says{Revyboss/2 1} [Let ophidset/irriteret] Det her er det tredjesidste
revymøde inden revyen. Er der virkelig ingen der har lavet nogle tekster siden
sidst?? Er der ingen med nogle gode id\'eer??

\says{Revyt 1}[Træt, gabende] Jeg synes, vi skal have nogle sange med.
\says{Revyt 2}[Vågner op] Ja, nogen med nogle sjove tekster\ldots!
\says{Revyt 3} Det går bare ikke, det nytter ikke noget, spild af kræfter, folk er alligevel så fulde, at de ikke fatter en kæft. 

\says{Revyt 1} Hvad med nogle sjove sketches så?
\says{Revyt 3} Nej, det er sgu' da det samme\ldots
\says{Revyt 2} Hvad om vi tager og laver en sketch, hvor kantinevagterne
fylder cola-automaterne op med Pepsi?
\says{Revyt 3} [Chokeret] Tag dig sammen mand, hvad sjov er der ved det?
\says{Revyt 1} Hvad så om vi sviner DKD'erne til, det er da poulært, ja, og
fysikerne??
\says{Revyt 2} Der er måske noget der---fysikerne---det ku' vi bruge, nej, det
er for let\ldots

\says{Revyboss/2 1} Vi skal også passe på at revyen ikke kommer til kun at
handle om at svine fysikerne til\ldots Der skal også være plads til
f.eks. biokomikerne. Jeg tror vi skal {\em prøve} om vi kan lade være med
at lave mudderkastningsnumre i år, {\em prøve hårdt}!!

\says{Revyt 1} Vi skal ihvertfald ikke have noget shu-bi-dua med, det er så
kedeligt, det er ligeså kedeligt som viser og bandet vil heller ikke spille
det. Vi skal have noget {\em hævyyyyy (Slår i bordet)!!}

\says{Revyt 2} Det havde vi sidste år. Jeg syn's vi skal prøve nogle
litterære værker, nogle gamle viser med Oswald Helmuth, det er klassikere\ldots

\says{Revyt 3} Ølhunden glammer, den kan vi bruge {\em (Får en id\'e og
  begynder straks at grifle ned på et stykke papir)}

\scene (De andre begynder at få interesse for hvad han skriver, men han
skjuler papiret på bordet med sin arm og skriver løs mens han dækker med
overkroppen. Revybosserne, der står bagved, kigger over skulderen uden at det
bliver opdaget)

\says{Revyt 1} [Mens revyt 3 grifler løs] Hvorfor Oswald Helmuth, hvorfor
ikke Morten Korch, det er mere oppe i tiden. Vi kunne lave en
trivial-forelæsning med Gregers Korch\ldots

\says{Revyt 3} [Afbryder]Hold kæft, bland dig uden om\ldots

\says{Revyt 2} Det var da ellers en god id\'e\ldots

\says{Revyt 3} Nej!! Folk er for fulde\ldots

\says{Revyt 1}[Får en god id\'e] Hvad med at opføre en musical som sidste
år? Vi kunne lave Stjernekrigen som musical {\em (henvendt til publikum):}
Eller Star Trek?

\says{Revyboss/2 1} Det kan vi ikke for fysikrevyen brugte
Stjernekrigen i deres revy i år, og du så selv hvordan det gik!


\says{Revyt 1} Hvad med at få folk til at sige ``buuuuuh!'' når vi siger
``institutbestyrer'' eller ``Søren Olsen''.

\says{Revyt 2} Tror du vi kan få folk til at sige ``buuh'' hver gang vi
siger\ldots  ``Søren Olsen'', det kan da godt være, så skal de også sige
``Jubiii'' hver gang vi siger ``Charlotte'' eller
``1. delsadministrationen''.

\says{Revyt 1} Hvad tror du de vil sige til det nede i \em{(Pause\ldots )}
1. delsadministrationen?

\says{Revyboss/2 2} Kan du ikke lave en computergrafik intro ligesom sidste
år, det kunne folk godt lide, det var et hit!

\says{Revyboss/2 1} Det er altså rimelig hårdt at lave sådan en intro, det
tager så lang tid og folk er som allerede sagt, vildt fulde, jeg tror ikke
det er besværet værd.

\says{Revyt 1} Hvad med noget mystisk, f.eks. et dræberegern, der slår alt
og alle ihjel, ku' det ikke være fedt?

\says{Revyt 3} '94

\says{Revyt 1} Nå, hvad så med at lave en parodi på Doom, det er
måske\ldots

\says{Revyt 3} '95

\says{Revyt 1} Hvad med en parodi på lykkehjulet, vi kunne kalde det
``Løkkehjulet'', det\ldots

\says{Revyt 3} '93

\says{Revyt 1} Hvad med så med\ldots

\says{Revyt 3} '91!!

\says{Revyt 1}[Opgivende] Nå, OK\ldots Ja, så har jeg ingen id\'eer!

\says{Revyt 3} Fysikrevyen '96!


\says{Revyt 1} Jeg fatter ikke en skid af det hele\ldots Nu går jeg ind i
rygerkantinen og sætter mig på mine briller, så kan jeg ikke se en skid. Er
det ikke fantastisk???

\says{Revyboss/2 1} [Rømmer sig, forsøger at få noget opmærksomhed] Vi
mangler lige at drøfte noget vigtigt\ldots Vi må folk folk til at fatte, at
det er forbudt at bruge ild i auditoriet, sidste år var der nogle der
tændte lightere i auditoriet, det er strengt forbudt!

\says{Revyboss/2 2} Og folk kylede ting og sager op på scenen, ølkapsler og
mønter, det er også vigtigt at pointere at det vil vi ikke acceptere i år.

\says{Revyboss/2 1} [Supplerer] \ldots men sedler er naturligvis OK.

\says{Revysboss/2 2} [Fortsætter] Er der nogen der har et forslag til hvordan
vi ungår det i år? 

\says{Revyboss/2 1} Jeg foreslår, at vi stopper revyen indtil at den eller
de personer der kaster ting op på scenen har forladt auditoriet. Hvad siger
resten af revygruppen?

\says{Revyttere} [I kor] Ja, ja, Mmmmm, jo, ja, det er rimeligt!

\says{Revybos/2 1} Det bliver en ordentlig omgang!!
\says{Revybos/2 2} Vi får travlt af helvede til!!!

\scene{Revybosserne forlader scenen}

\says{Revyt 4} [Kommer løbende ind på scenen] Hvad laver I dog her, har I
fået lavet den der revy eller hvad? Er I klar over hvad klokken
er? Nede i det store auditorium, der\ldots der\ldots der\ldots DIKU-REVY. {\em
(begynder desværre at synge.)}

\scene{(De resterende revyttere forlader scenen}

\scene{``Der er DIKU-REVY'' starter\ldots)}

\end{sketch}

\end{document}
% Local Variables: 
% mode: latex
% TeX-master: t
% End: 


