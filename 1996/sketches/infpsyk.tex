\documentclass[10pt]{article}
\usepackage[utf8]{inputenc}
\usepackage{revy}
\title{Informationspsykologen}
\author{Theo Engell}
\version{1.2}
\parindent0pt
\parskip 1ex minus 1ex
\flushsingsright

\begin{document}
\maketitle

\begin{sketch}

\begin{roles}
  \role{Eliza} Den labre Informationspsykolog.

  \role{IB} Søren, den usikre institutbestyrer. Er småparanoid, ser sig
  over skulderen, er meget sammenklemt, usikker. Iført Smoking.

 \role{Videnskabsmand} Høj, iført kittel, penne i brystlommen, store
 VR-briller, etc.

\end{roles}


\scene{På scenen er en "briks" med en computer på. Ved siden 
af den en stol. Evt.\ hænger der et doktor-certifikat et eller
 andet sted. Psykologen sidder i stolen. Det ringer på døren.}

\says{Eliza} Kom indenfor.

\scene{Ind kommer IB. Hun stikker ham på næven, han tager 
straks sin nøgle frem og forventer at hun skal se den (nøglen).}

\says{Eliza} Goddag, mit navn er Eliza Pandesvejser. Nej, den 
får du ikke brug for her. Kom lad mig tage den {\em (Rækker ud
 efter nøglen)}.

\says{IB}  Er du rigtig klog? Det er MIN nøgle!
\says{Eliza} Ja, det er det også\ldots Kom med herover, og lig 
ned på briksen\ldots Navnet var\ldots

\scene{ Eliza er dog nød til at fjerne en computer fra briksen 
først. James Bond Tema, IB retter på tøjet lægger sig tilbage.}

\says{IB} Olsen, Søren Olsen. 

\says{Eliza} Nåh, fortæl mig så hvad der er galt. Hver gang du er 
færdig med at fortælle mig noget så afslut med en tom sætning. Kan 
du det?

\scene{Hver gang IB har sagt noget, somd er skal svares på, skal han 
bevæge læberne uden at sige noget, en tom sætning, hmm. Det går ikke så 
godt første og anden gang og han skal hjælpes med mundfagter fra Eliza}

\says{IB} Ja, det tror jeg da. Kære doktor, MIT institut forstår mig ikke!!
-- Jeg tror, de er efter mig. 
\says{Eliza} Du siger, de er efter dig?
\says{IB} Ja, de studerende fra DIKU!
\says{Eliza} Aha, fra DIKU, siger du?
\says{IB} Ja, det er der hvor jeg arbejder.
\says{Eliza} Der hvor du arbejder?
\says{IB} [grinende] Ja det er jo så meget sagt, jeg er institutbestyrer.
Et latterligt job med magt. Faktisk en masse magt, ja magt, hmmhm, magt,
MAGT, MAGT!!
\says{Eliza} Du talte før om de studerende. Kan du elaborere på det?
\says{IB} Ja, det er de mennesker, som tager kurser på instituttet.
\says{Eliza} De mennesker, der tager kurser på instituttet?
\says{IB} Ja!
\says{Eliza} De mennesker, der tager kurser på instituttet?
\says{IB} Ja!
\says{Eliza} Der tager kurser på instituttet?
\says{IB} Ja, hvad med dem?
\says{Eliza} [Forvirret] Det ved jeg ikke.{\em (dårlig rplik -- bør laves bedre.}
\says{IB} Det ved du ikke?
\scene{James Bond tema igen}
\says{Eliza} Nej, jeg er kun psykolog, informationspsykolog. Desuden er
det mig, der skal stille spørgsmålene.
\says{IB}  Nå, men der også DIKU-REVYEN, de er godt nok slemme.
\says{Eliza} Men hvad er det, der er så slemt ved DIKU-REVYEN.
\says{IB} Jeg har stærkt på fornemmelsen, de vil lave en sketch om mig!
\says{Eliza} Hvad er det, der er så slemt ved fornemmelsen, de vil lave en
sketch om mig ved DIKU REVYEN. Kan du elaborere på det?
\says{IB} Hu?
\says{Eliza} Bekymrer det dig, at ``huh?''
\says{IB} Nej, det er semantik-gruppen, der bekymrer mig.
\says{Eliza} Er det på grund af dine hobbies, at du siger,
semantik-gruppen, der bekymrer mig?
\says{IB} Øh, jaehøhæøh\ldots Jeg samler på nøgler, har det noget med sagen
at gøre?
\says{Eliza} Du sagde før, at DIKU REVYEN vil lave en sketch om dig. Kan du
elaborere på det?
\says{IB} Jeg tror bare de vil lave en ond sketch om mig.
\says{Eliza} Hvorfor skulle de være onde om dig?
\says{IB} Ja, hvorfor skulle de det? Pyha, nu er jeg mere rolig. Nå, 
men\ldots GUD! Klokken er også mange. Jeg skal over på DIKU, klokken er
snart  16:00. Så er det nøgletid, {\em (Muaahahaha, bøhmansgrin)}.

\scene (IB rejser sig op og suser ud af døren. Eliza står og venter, ser
på uret. Ind kommer en standard videnskabsmand.)

\says{Eliza} Nå, hvad var resultatet.
\says{Videnskabsmand} Jeg er ked af det, han bestod altså ikke Turing
testen.

\scene (Lys ud, cut!)

\end{sketch}

\end{document}
% Local Variables: 
% mode: latex
% TeX-master: t
% End: 
