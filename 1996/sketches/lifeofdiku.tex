\documentclass[10pt]{article}
\usepackage[utf8]{inputenc}
\usepackage{revy}
\title{Life of DIKU}
\author{Peter W. Nielsen, Jesper Holm Olsen og Søren Trautner Madsen}
\version{1.2}
%Opdateret 13.9.96 - Theo Engell-Nielsen

\parindent0pt
\parskip 1ex minus 1ex
\flushsingsright

\begin{document}
\maketitle

\begin{sketch}

\begin{roles}
  \role{Speaker, Francis, Commando}~
  \role{Reg, Loretta, Vagt1, Vagt2}~
\end{roles}

\scene{Her en lille ide til en sketch, den er originalt fra Life of Brian, der hvor 
de diskuterer hvor meget Romerne har givet dem. Jeg har ikke ændret navene men 
de skulle vi nok.}

\scene{Speakeren kommer ind på scenen. Han er flad af grin.}

\says{Speaker} Nu kommer det... Den {\em rigtigt} fede sketch. {\em (er ved at flække
sammen af grin)}. Den har bare (ha ha) den fedeste punchline ... ja, på en skala fra 1 til
10, er den {\em virkelig} sjov. For dem blandt publikum, der kender deres Monty Python,
der må vi beklage, at sketchen i det store hele bare er en omskrivning af et af deres
gamle numre, men ... VI MÅTTE GØRE DET!!!! Punchlinen var simpelthen for oplagt ... den
(ha ha) Det er jo fordi (ha ha ha). Nej, I må hellere selv se det. Her kommer
vindersketchen "LIFE OF DIKU"


\scene{Tæppet går fra. De sidder om et bord med en (masse?) tegninger på.}

\says{Francis} Vi kommer ind gennem døren ud mod HCØ {\em (peger)} her, ned
igennem store auditorium {\em (peger)} her, ned gennem gangen, rundt om
hørnet og ind i Valhal {\em (peger)} her. Når vi så har taget {\sc embla} i
forvaring informerer vi EDB-Afdelingen og fortæller dem vores krav. {\em
  (ser op på hver af dem)} Spørgsmål??

\says{Xerxes} [ser fortabt ud] Hmm, hvad er vores krav??

\says{Reg} Vi giver dem 2 dage til at nedlægge hele EDB-afdelingen og hvis de ikke 
gør det, ødelægger vi Embla.

\says{Loretta} Ødelægge den?

\says{Francis} Vi skærer alle "bittene" af og sender dem til EDB-afdelingen hver time ---
klokken 15 minutter over. Bare så de ser, at vi ikke er til at spøge med.

\says{Reg} Og vi påpeger selvfølgelig, at de har det fulde ansvar for at vi ødelægger
 Embla og at vi selvfølgelig ikke giver efter for afpresning!

\says{Alle} Ingen afpresning! Ingen afpresning!!

\says{Reg} De har udsultet os, de svin. De har taget vores retigheder fra os, og 
ikke bare fra os også fra vores forgængere og vores forgængere's forgængere.

\says{Loretta} Og fra vores forgængere's forgængere's forgængere.

\says{Reg} Ja!

\says{Loretta} Og fra vores forgængere's forgængere's forgængere's forgængere.

\says{Reg} Ja!. OK, OK, så er det nok. Og hvad har de givet os tilgengæld?

\says{Xerxes} X-Terminaler? 

\says{Reg} HVAD?

\says{Xerxes} X-Terminaler!.

\says{Reg} Hmm.  Ja, ja. Det har de givet os. Hmm, ja, Det {\em ER} rigtigt.

\says{Alle} Og mere RAM

\says{Loretta} JA, mere RAM, Reg. Kan du huske hvordan terminalrummene var før det?

\says{Reg} Ja, ja OK.  Du har ret X-Terminaler og mere RAM, det er 2 ting EDB-afd. har 
givet os.

\says{Loretta} Og Internet!

\says{Reg} Hmm, ja. Selvfølgelig Internet. Jeg mener, det behøver vi ikke at nævne 
det ved alle da. Men Bortset fra X-Terminaler, mere RAM og Internet.

\says{Xerxes} M68000 maskiner.

\says{Loretta} Hjælp med programmer.

\says{Xerxes} Uddannelse.

\says{Reg} Ja, ja, OK. Det er fair nok.

\says{Xerxes} Og printerkvoter

\says{Alle} hmm, ja, jaa!

\says{Francis} Ja, det ville vi virkelig savne, hvis EDB-afdelingen forsvandt.

\says{Xerxes} Netscape2.

\says{Loretta} Og vi kan få lov at arbejde ved maskiner, der virker hele døgnet.

\says{Francis} Ja, de kan sandelig holde maskinerne kørende. Lad os se det i øjene, 
de er de eneste, der kunne gøre det på et sådant sted.

\says{Reg} OK, men hvis vi ser bort fra X-Terminaler, mere RAM, Internet, M68000 
maskiner,  hjælp med programmer, Uddannelse, printerkvoter, Netscape2 og en 
maskinpark der kører, hvad har EDB-afd. så gjort for os?

\scene{(En pause. Trommevirvel. Speakeren står ude i kanten af scenen og
  giver "thumbs up" og lignende. Alle skuespillerne kigger
  forventningsfuldt på Xerxes. Pludselig kommer to vagter gående ind på
  scenen).}

\says{Vagt1} Ok, drenge! Må jeg lige se jeres studiekort og nøgler!

\says{Reg} Øhhh, kan du ikke lige vente lidt? Vi er lige kommet til punchlinen!!!

\says{Vagt2} Ja, det siger de alle sammen. Hvis I ikke har studiekort OG
nøgler, så må vi lige bede jer om at følge med her!

\says{Reg} Du kan få lov at se RØV og nøgler, kan du. Det er altså en vildt fed
punchline...

\says{Vagt1} Kan du huske Benjamin?

\says{Reg} Ja! Det var da ham der, med det der...

\scene{(fanfare)}

\says{Vagt2} Okay, kan du så huske Ingemar Stenmark???

\says{Reg} Øhhh... nej.

\scene{(Tordenskrald/pruttelyd eller andet)}

\says{Vagt1} Før dem bort!!!!!

\scene{(vagt 2 genner skuespillerne væk)}

\says{Vagt1} Inderst inde har jeg aldrig haft lyst til at være vagtmand.
{\em (han tager jakken af. Inde under har han en skovmands-skjorte)} Jeg
har altid villet være\ldots

\says{Vagt2} [der lige er kommet tilbage] Øjeblik min ven! Må jeg se studiekort og
nøgler?

\says{Vagt1} Fordømt! Hvad skal jeg dog gøre nu? {\em folk råber selvfølgelig smid tøjet!}

\scene{Pludselig starter musikken. Skuespillerne fra før kommer hoppende ind og smider
deres kutter. De har cancantøj på indenunder. Der synges "Hvor er nøglerne mon henne"}

\end{sketch}

\end{document}

% Local Variables: 
% mode: latex
% TeX-master: t
% TeX-master: t
% End: 

