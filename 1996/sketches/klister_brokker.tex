\documentclass[10pt]{article}
\usepackage[utf8]{inputenc}
\usepackage{revy}
\usepackage[OT1]{fontenc}
\title{Bar røv, nøgler og studiekort}
\author{Theo Engell-Nielsen}
\version{2.05}
\parindent0pt
\parskip 1ex minus 1ex
\flushsingsright

\begin{document}
\maketitle

\begin{sketch}

\begin{roles}
  \role{Stud og Stud2} Studerende.
  \role{Vagt, Vagt2} Rigtig dåhmme vagter.
  \role{RV1 og RV2} Rigtige vagter, måske?
  \role{IB} Institutbestyreren, rem og støvler???
  \role{Annelise Axen} ~
  \role{En del meta-roller} ~
\end{roles}

\begin{props}
  \prop{Nøgler, studiekort og profilafprøvningsapperatur} ~
  \prop{Døre} Lette nok til at de kan flyttes i en håndeveding af en
  person.
\end{props}

\scene{Forslag til klister mellem de forskellige sketches i første akt:
  Ovre på HCØ sidder vagten (vagterne) og ser fjernsyn. De kunne sidde som
  Bitvis og Nothead (Id\'e fra 1994, >Bool, bool<, he said ``pattern'')
  blandt publikum og kommentere det hele.  Når de så er utilfredse kan de
  stoppe sketchen, som parallel til prikkeren fra 1993, og så skifte kanal
  til noget de istedet vil se.\\ 
  Det kunne være OK, hvis de kunne fjernestyre et par vagter til scenen,
  der skal bede om studekort og nøgle, så ofte som muligt.  }

\section*{Afslutningssketch}

\scene Stud vil gerne ind på DIKU gennem fortæppet, men fordi døren er i
stykker må han vinke efter nogen indefor der kommer og lukker ham ind\ldots
Han bliver lukker ind af en medstud, mens tæppet går fra.

\says{Stud} Tak skal du have, det var pænt af dig!
\says{Stud2}Det var så lidt, det kan jo være svært og komme ind selvom man
har både studiekort {\em og nøgle}.
\says{Stud}Ja, det er rigtig nok.
\scene{Stud2 går sin vej. Stud begynder at gå og skal igennem en dør, der
  står en vagt bag døren.}
\says{Vagt2}[behersket]Må jeg se din nøgle og studiekort.
\says{Stud}Jamen, er klokken så mange, nej den er jo 17:01.
\says{Vagt2} Ikke så meget pjat, hit med kort og nøgler. {\em (Holder kort
  op i mod lyset)} Hmmm\ldots Det ser fint ud\ldots!

\scene Stud vandrer videre. Kommer til en dør nr. 2. Det samme sker her.
  Imens bliver dør nr. 1 flyttet frem som dør nr. 3 og så fremdeles, mens
  hver visitering bliver mere og mere voldsom, indtil
  han når frem til dør nr. 5, Annelise Axens dør. (Vagterne kan eventuelt
  bide i kortet, nøglen, bede om blodprøve, afføringsprøve, DNA-profil,
  retina-scanne ham og lign.\ ) 

\says{Stud}Hej Annelise
\says{Annelise}Hej Peter, tillykke med din kandidateksamen, hvordan er det
at være færdig?

\says{Stud}It's life, Jim, but not as we know it\ldots{\em (rømmer sig)}
Jo, det er jo både godt og skidt. {\em (Musicalmusik i baggrunden??)} Man
skal væk og det er jo sjov med alt det nye og det spændende. Men så skal
man jo også væk fra DIKU og det er jo lidt trist. Men man kan jo ikke (!!!)
bliver her for evigt, heller!  Så derfor skal du jo have min nøgle, ellers
går der jo kuk i systemet.

\says{Annelise}Ja, det er også rigtigt. Hvis du vil skrive under her\ldots
ja, der.

\says{Stud}Kan du nu ha' det godt. Vi ses jo nok på et eller andet\ldots

\scene{På vej ud af kontoret bliver han stoppet af en vagt\ldots}

\says{Vagt} Må jeg se din nøgle?
\says{Stud} Jeg har ingen nøgle, jeg har\ldots

\says{Vagt} Jamen, hvad er det for noget ynk? Ingen nøgle, sådan en
forbryderspire. Det skal du ikke slippe så let for. Kan du lige komme med
mig!

\scene{Den stakkels studerende bliver slæbt voldsomt ned til institutbestyren.}
\says{IB}[myndigt,autoritativt] Hvad er det for noget? Vi kan ikke {\em ha'} at du går rundt uden
nøgle! Det er et brud mod institutets regler og dets fundats. Her er din
nøgle\ldots!!
\says{Stud} Jam\ldots
\says{IB} Tag nu din nøgle\ldots hmmm!!!
\scene{Stud tager tøvende og uforstående nøglen op, vender det hvide ud af øjnene, sukker
  og vender sig om\ldots}

\says{IB} [Råber]Hey! Så let slipper du altså ikke!! Som straf for ikke at
være i besiddelse af en nøgle får du frataget din nøgle i et helt semester
--- en meget mild straf, din udåd taget i betragtning! Og da klokken er
gået hen og blevet 17:02 må jeg lige bede en vagt eskortere dig uden for.
VAGT!

\scene{Vagt kommer indefor, uden at sige noget viser IB sin nøgle til
  vagten. Han ser på den studerende, der trækker på skuldrene\ldots}
\says{Vagt}[henvendt til stud] Må jeg se din nøgle?
\says{Stud}Jamen\ldots {\em (ser på IB)}
\says{Vagt}[Dumt]Hvad har du ingen nøgle!!!
\says{IB}[Tager sig til hovedet] Få ham nu bare ud.

\scene{Inspireret af Monty Python:}
\says{Vagt} [forundret] \ldots\ og hvorfor skal alle sketches slutte med at der kommer en vagt ind og smider en skuespiller ud bare fordi,
han ikke har nøgle og studiekort på sig?

\scene{Vagt2 kommer ind}

\says{Vagt2} [henvendt til vagt1]Du\ldots\ må jeg se din nøgle\ldots

\scene Den ene dør i auditoriet smækkes op, og en rigtig vagt kommer ind
med en stavlygte, idet al scenelys slukkes

\says{RV1}[i megafon] STUDIEKORT!

\scene Den anden dør i auditoriet smækkes op, og ind kommer en anden rigtig
vagt med lygte

\says{RV2}[i megafon] -- OG NØGLER!

\says{RV1 og RV2}[i kor, råbende] Alle bliver siddende og tager deres
studiekort og nøgler frem.

\scene De tjekker nogle studerende

\says{RV1}[Tager fat i kraven på en studerende] Hvad har du intet
studiekort, husker du Mikkel?

\says{RV2}[råbende] Det viser sig at en del mennesker ikke har studiekort
og nøgler med, I bliver alle sammen nødt til at forlade auditoriet. Vi
foretager en visitering af alle publikummer i løbet af 20 minutter,
og kun de der har studiekort og nøgler på sig kan se anden akt kl. XX.XX.

\scene slut på første akt
\end{sketch}
\end{document}
% Local Variables: 
% mode: latex
% TeX-master: t
% End: 
