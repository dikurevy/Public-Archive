\documentclass[10pt]{article}
\usepackage[utf8]{inputenc}
\usepackage{revy}
\title{Der er DIKU-REVY!}
\author{Theo Engell-Nielsen \& Peter Harry Eidorff}
\melody{``You oughta know'' -- Alanis Morisette}

\version{1.2}

\parindent0pt
\parskip 1ex minus 1ex
\flushsingsright

\begin{document}

\maketitle

\begin{song}
\sings{Sanger}Det er tid igen, salen er fyldt med folk
Nu skal den trykkes af, til DIKU REVY'n
For eksamen er slut
Også middagen er slut
så folk er nu i vort auditorium
Mon "de" spiller heavy i år?
og noget alle forstår?
Er der en der råber no'ed med "smid tøjet"?

Men så pludselig slår det klik, man har glemt sin replik,
og der mangler et bord her på scenen, hvor er det flovt -- øv!
Hver gang det sker må vi improvis\'er' 
men det sker tem'lig tit, at det går 
og så er det sjovt, er det sjovt
For vi spiller live

\sings{Kor}%
Vi er her for at spille
den bedste revy I endnu har set 
Det ik' fair at buh'e
men vi tror det sker, for det er jo nu
der', der' DIKU RE-E-VY

\sings{Sanger}%
Hvad skal der nu ske? Hvem skal hænges ud?
Det er hemmeligt, men du får at se:
Hvordan revyen bli'r til
og du vil hør' bandet spil'
Vi håber at du sidder rigtig behageligt?
For Shu-bi-dua bli'r ved
og kerne-sketchen er med
men vi har også masser' nye ideer

Men så pludselig slår det klik, man har glemt sin replik,
og der mangler et bord her på scenen, hvor er det flovt --- øv!
Hver gang det sker må vi improvis\'er' 
men det sker tem'lig tit, at det går 
og så er det sjovt, er det sjovt
For vi spiller live
\scene Kor
\scene mellemspil

\sings{Sanger}%
For de jokes vi har med de er platte og nye
men vi ved I vil buh'e 
så snart at I hører dem -- og I ved det!
Vi øver os på sang og kor og lyden skrues op
så I kan hør' os... nå, kan I hør' os?
\scene Kor gentages 2 gange.


\end{song}
\end{document}
% Local Variables: 
% mode: latex
% TeX-master: t
% End: 
