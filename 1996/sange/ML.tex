\documentclass[10pt]{article}
\usepackage[utf8]{inputenc}
\usepackage{revy}
\title{Det' meget pæner' i ML}
\author{Tore Green}
\melody{``Diggin' on James Brown'' -- Tower of Power}

\version{1.101} % HUSK AT AJOURFØRE VERSIONSNUMMER!!

\revyyear{1996}
\parindent0pt
\parskip 1ex minus 1ex
\flushsingsright

\begin{document}
\twocolumn[ % alt hvad der står imellem [] bliver skrevet i hele sidebredden
            % Hvis man ikke vil have to spalter, fjernes `\twocolumn[' og `]'
\maketitle
]
\begin{song}
\sings{1.}  Når man læser her på DIKU
            Så lær' man mange sprog
            Og hvis man vil lære flere
            Så ka' man køb' en kursusbog
            Der' et der er det bedste
            Og Mads har lavet det selv
            Det' lige meget hvad du koder
            -- Det' meget pæner' i ML

\sings{Omkvæd} For typer de skal være striks'
            Ellers dur det nul og niks
            Brug en højere funktion
            Til din liste-konstruktion
            Drop din pointer og skriv {\tt val }
            -- Det' meget pæner' i ML

\sings{2.}  Det lig' meg't hvad Eric siger
            Det ska' være funktionelt
            Og er der ingen penge i det
            Ja, så må jeg bo i telt
            Jeg fortæller gerne om det
            Båd' på news og Diatel
            For alle har fortjent at vid' det
            -- Det' meget pæner' i ML

            \scene{Omkvæd gentages }

\sings{3.}  Jeg har prøvet C og ++
            Og jeg har kodet Emerald
            Pascal og Lisp og Miranda
            Men det' bare noget skrald
            Fordi der mangler jo noget
            De andre sprog slår ikke til
            Typer, filer, konstruktører
            -- Det' meget pæner' i ML

\sings{Bro} Køb lidt mere RAM
            Mere power, mere disk
            Brug resurser, kod lidt mer'
            Køretiden den er fæl
            -- Men det er pæner' i ML

            \scene{Solo, derefter Omkvæd igen }

\sings{Outro} -- Det meget pæner' i ML \act{gentages ad lib\ldots }

\end{song}
\end{document}
% Local Variables: 
% mode: latex
% TeX-master: t
% End: 
