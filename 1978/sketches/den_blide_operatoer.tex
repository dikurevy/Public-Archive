\documentclass[a4paper,11pt]{article}

\usepackage{revy}
\usepackage[utf8]{inputenc}
\usepackage[T1]{fontenc}
\usepackage[danish]{babel}


\revyname{DIKUrevy}
\revyyear{1978}
% HUSK AT OPDATERE VERSIONSNUMMER
\version{0.1}
\eta{$n$ minutter}
\status{Ikke færdig}

\title{Den blide operatør}
\author{JCC}

\begin{document}
\maketitle

\begin{roles}
\role{D}[] Dat1er
\role{F}[] Fortæller
\role{I}[] Intercom
\role{L}[] Lærer
\role{H}[] Herold
\end{roles}

\begin{sketch}

\scene{Scene 1}

\says{F} I kender sikkert alle situationen ved
selvbetjeningsterminalen.  20 dat0ere venter på udskrifter fra
linieskriveren, 20 dat2ere venter på print fra printeren, og 2
andendelsstuderende venter på, at deres sædvanlige lort kommer ud på
den symbiont, som de har symmet de til.  En stakkels dat1er stiller
sig op og snakker til søjlen.

\says{D} Hvad er der i vejen med min kørsel?

\says{I}[Bander og svovler] ... og sluk for helvede efter dig!

\scene{D rammes af et æbleskrog.}

\says{F} Da dette tydeligvis ikke bærer frugt, vil vi introducere den
nye disciplin indenfor datalogien: Den bløde datalogi.  Eller på
engelsk: Soft software.  Fremover vil det lyde således:

\says{D} Hvad er der i vejen med min kørsel?

\says{I}[blid] Ja, jeg er ked af at måtte sige det, men selvom dit
password lignede meget godt, er din konto opbrugt.  Men nu skal jeg se
efter om du kan få en af de små betalingsfri.  Her har du et æble så
længe.

\scene{Scene 2}

\says{F} Også i undervisningen vil vi indføre den bløde datalogi.
Ikke den hårde, direkte metode.

\says{L}[fumler med mikrofon, hænger den om halsen, memer et stykke
tid, skruer på en knap] KAN I HØRE MIG? \act{Højt, meget højt.  Skruer
  ned, messer meget monotont.}  Vi skal i dag beskæftige os med min
seneste algoritme til division med nul.

\says{F}[afbryder] Stop, stop, stop.  Det er jo det rene ævl.  Nu skal
I bare høre: God morgen, allesammen.  Emnet for i dag er træet
\act{peger på et æbletræ på tavlen.  Ser ud i salen} Hør min ven, du
ser træt ud; vil du ikke have et æble?  \act{Plukker et æble og rækker
  det til en publikummer}.  Hvis der er noget, I vil spøge om, må I
endelig sige til.

\scene{Scene 3}

\says{F} Også lærer/TAP forholdet bliver bedre.  For fremtiden vil
situationen se således ud:

\says{L} Hvis du tilfældigvis skulle få lidt tid tilovers, vil du så
ulejlige dig med at skrive dette for mig?  Naturligvis ville jeg
hjertens gerne selv skrive det, men du ved, hvor tavlt jeg har med
undervisning og forskning.  Det haster ikke med at få det færdigt, og
du må endelig give dig god tid.  Og der er næsten ingen tegninger i.
Skulle der være et enkelt spørgsmål, kan du finde mig på mit kontor
her på DIKU, dor det er jo her, jeg er bestyrer.  Her har du et æble
til at samle kræfter på.

\scene{Scene 4}

\says{F} Selv hotnews bliver mere venlig og personlig:

\scene{Trommehvirvler hist og her.}

\says{H} Vi, Gunn I, de operatørers, programmørers og såkaldte
videnskabelige medarbejderes master, gør herved vitterligt: Eftersom
Vi i Vor ophøjede visdom har taget til efterretning, at der optræder
små gule diskædere i lageret, har Vi udlagt gift på tromlerne.  Næste hotnews kommer på indre by temrinalen ved sopopgang.

Under vor datamatiske tastur og Segl, Gunn Rex.

Sidste nyt: Som kompensation for de senere dages forringede service,
uddeles der æbler i direktionen.

\end{sketch}
\end{document}
