\documentclass[a4paper,11pt]{article}

\usepackage{revy}
\usepackage[utf8]{inputenc}
\usepackage[T1]{fontenc}
\usepackage[danish]{babel}


\revyname{DIKUrevy}
\revyyear{1978}
\version{0.1}
\eta{$n$ minutter}
\status{Færdig}

\title{Sangen om EDB-elementerne}
\author{EM, JBC, EM}
\melody{Generalmajorens vise}

\begin{document}
\maketitle

\begin{roles}
  \role{S}[HHK] Sanger
\end{roles}

\begin{song}
  \sings{S} Vi lærer dem om sumkontrol, totalkontrol, kopikontrol,
  og nonsens- og kontroltotal samt brug af kontrolkonsol,
  om semantik og pragmatik og dynamik og heuristik
  og spadestik og overstik og kobberstik og øvrige stik

  Heltal og tilfældige tal på hexadecimale tal,
  normale, kolossale, optimale, digitale tal
  og tidstro drift og tidsdelt drift og bundet drift og ud'af drift
  og cyklisk skrift og logisk skift og højreskift og studieskift.

  Hvordan man pakker, ordner, søger, gætter, sætter, sletter, fletter
  overføring, overhuldning, overløb og oversætter,
  indre lager, arbejdslager, ydre lager, hjælpelager,
  pladelager, hejste flag og skrivning af eksamensklager.

  \scene{Talt: Herefter går vi over til forårssemestrets pensum.}

  \sings{S} Om cyklustid og nyttetid og lukket tid og ventetid
  og fil/register, bånd og blok og bit og post og individ
  om kilokroner, kilobyte og kildetekst og kildelæsning
  ophold i et opkald koblet op til optisk skriftindlæsning.

  Kortforsiders kortkolonner tegnet i en kort rapport,
  om styrekort og spillekort og lidt om marmelade-kort,
  med højremargen, venstremargen, øvremargen, sidenummer
  skrevet på maskine i students søde natteslummer.

  Om bittranchering, træbeskæring, båndsalat og bonuskog,
  og Algol, Cobol, Snobol, Lisp, Pascal og simuleringssprog,
  et struktureret multi-mikro-makro-programmeringssprog,
  og rigtig brug af termer fra den danske EDB-ordbog.

  ... og det var så de emner, som vi lærer dem på DIKU,
  der findes også andre som vi ikke synes de sku' - ku'.
\end{song}

\end{document}

