\documentclass[a4paper,11pt]{article}

\usepackage{revy}
\usepackage[utf8]{inputenc}
\usepackage[T1]{fontenc}
\usepackage[danish]{babel}


\revyname{DIKUrevy}
\revyyear{1978}
\version{0.1}
\eta{$n$ minutter}
\status{Ikke faerdig}

\title{DIKU'istens vise}
\author{AMS}
\melody{Når jeg ser det hele lidt fra oven}

\begin{document}
\maketitle

\begin{roles}
\role{S}[AMS] Sanger
\end{roles}

\begin{song}
I flere år end jeg kan li' at mindes
har jeg læst det her datalogi.
Jeg syn's det er det dejligste der findes
skønt andre kalder det for idioti.
Men jeg kan li' at pusle med problemer
hvis løsning ikke rager nogen en fjer.
Og det kan så vidt jeg ved
ikke gøre nogen fortræd
at jeg la'r min verdens navle hedde DIKU.

Jeg læser ikke meget i aviser
jeg bladrer oftest på en anden led
  (bladrer i "`printerliste"')
Hvordan kan jeg studere andres kriser
hvis {\em mit} program til stadighed går ned?
I Argentina spiller alle fodbold
så deres liv må nærmest vær' en fest.
Det er ikke altid at
jeg ved helt hvo'n det er fat
for jeg lader verdens navle hedde DIKU.

I Danmark er der mange arbejdsløse
på DIKU ku' vi sagtens bruge nogen.
Hvis myndigheder blot var generøse
bevilged de en slave pr. person.
Tænk hvis man aldrig mere selv sku' hulle
og skrive på maskine "= hvilken drøm.
Og kantinevagter er
terapi, det ved enhver
der er med på verdens navle hedder DIKU.

Hvis det sku ske at jeg en dag bli'r færdig
"= det er nu ikke noget jeg stiler mod "=
så må jeg vise jeg er DIKU værdig
og stadig tåler stress og natterod.
Og syn's jeg ikke om den store verden
men finder den for rodet og for sær
er jeg godt nok teknokrat
til at få et lektorat
så min verdens navle atter hedder DIKU.
\end{song}

\end{document}

