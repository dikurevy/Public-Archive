\documentclass[a4paper,11pt]{article}

\usepackage{revy}
\usepackage[utf8]{inputenc}
\usepackage[T1]{fontenc}
\usepackage[danish]{babel}

\revyname{DIKUrevy}
\revyyear{2011}
% XXX: HUSK AT OPDATERE VERSIONSNUMMER
\version{1.3}
\eta{$6$ min}
\status{Mangler arbejde}

\title{Corporate Yes Culture}
\author{Rune Perstrup}

\begin{document}
 \maketitle

\begin{roles}
    \role{X}[X Phillip] Instruktør
    \role{C}[Mark]    Coach
    \role{F1}[Valiant]    Forsker1 (lalleglad forsker)
    \role{F2}[Klaes]    Forsker2 (kritisk forsker)
    \role{N}[Naja]    Naja
    \role{G}[Brainfuck] Gadaffi
\end{roles}

 \begin{props}
   \prop{Ja-hat}[Phillip]
   \prop{Nej-hat}[Phillip]
   \prop{lederhat}[Phillip]
   \prop{bøllehat}[Phillip]
   \prop{kasse til hatte}[]
   \prop{en krukke med ``hundekiks''}[]
 \end{props}

 \begin{sketch}

\says{C} Vi har lært hos Microsoft, at de bedste medarbejdere er dem,
der har en positiv tænkning. Medarbejderen, der siger altid "ja". Det
er denne "Coorporate Yes-culture", der er årsagen til deres
\emph{fremragende} resultater. Ingen stiller spørgsmål. Alle gør som der
bliver sagt. Vi har brug for sleske fedterøve.

I forbindelse med en ny rationaliseringsrunde på DIKU, har vi derfor
udviklet en Ja-hat og en Nej-hat. Den, der bærer Ja-hatten altid vil
svare "ja" til lederens henvendelser - og den, der bærer en Nej-hat
vil altid sige "nej". Følgende kontrollerende forsøg skal nu afgøre,
hvilke medarbejdere, der er værd at satse det fremtidige institut
på. \act{C sætter ja hatten på F1 og Nej-hatten på F2} Vi starter
med et par spørgsmål. \act{Henvendt til F1} Er deres navn F1?

\scene{(Hatte: C: N/A, F1: Ja, F2: Nej)}

\says{F1} Ja.

\says{C} Elsker de ikke denne fantastiske arbejdsplads?

\says{F1} \act{endnu mere glad} Jo!  

\says{C} Synes de ikke, at jeg er en fantastisk coach?

\says{F1} \act{ekstatisk} Jo, DET gør jeg!

\says{C} \act{klapper ham på hovedet, F1 bjæffer - får en godbid}
Her er altså et fremragende eksempel på Coorporate Yes Culture. Man
kommer nemlig længst med at være slesk og oportunistisk. Lad os nu
undersøge medarbejderen med nej-hatten. \act{henvender sig til F2}
Synes de ikke, at dette er en fantastisk arbejdsplads?

\says{F2} Nej...  

\says{C} Synes de ikke, at jeg er en fantastisk coach?

\says{F2} Nej... \act{Ryster på hovedet}

\says{C} \act{stikker F2 en baglæns dumskalle, vender sig mod
  publikum} Det gider man jo ikke at høre på. Nu tænker I måske, at
man blot skal sige ja. Men \textbf{nej}! Coorporate team spirit kommer
\textbf{indefra}. Lad os prøve at bytte om på hattene \act{han bytter
  om på de to hatte, så F1 har nej-hat på og F2 har ja-hat
  på}. 

\scene{(Hatte: C: N/A, F1: Nej, F2: Ja)}

\says{C} \act{mod F2} Synes de, det her er en dårlig arbejdsplads?

\says{F2} Ja!

\says{C} \act{mod publikum} Og hvad med dig, F1. Du synes ikke, at
denne arbejdsplads er skidt, vel?

\says{F1} Neeej! \act{Ryster lalleglad på hovedet}

\says{C} Døøøgtig forsker \act{klapper ham på hovedet, han bjæffer -
får en godbid. Stikker F2 en baglæns dumskalle og vender sig så mod
publikum}

Med lidt snilde kan man dog også bruge Ja-hatten til at udnytte den negative medarbejders betydelige samarbejdspotentialer. \act{Mod F2} Du vil gerne arbejde helt vild meget over den næste måned, ikke?

\says{F2} Jo...  

\says{C} - Og gå 20\% ned i løn?  

\says{F2} \act{sammenbidt} Jo...  Nej, det her vil jeg altså ikke være med til. Det er jo totalt
uretfærdigt!

\says{C} Det \textbf{skal} du! Du har en Ja-hat på.

\scene{Skynder sig at bytte om på F1's hat og sin egen}

\scene{(Hatte: C: N/A, F1: Ja, F2: Nej)}

\says{C} \act{Samtidig, ud mod publikum} Det er vigtigt at ...

\says{F2} \act{Har nu Nej-hatten på} NEJ!

\says{C} Jo!

\says{F2} \act{stikker fjæset lige op i C's ansigt} Nej!

\says{C} Jo!  

\says{F2} Nej!

%\says{C} \act{bytter det to hatte tilbage på plads} Jo
\says{C} Jo!

%\says{F2} \act{tvinger ja-hatten over på C og holder den fast på hans hovede} Nej!
\says{F2} Nej!

\says{C} Stop! Det her går ikke. Nu tager jeg en lederhat på. \act{Tager
en hat frem af kassen, hvorpå der står ``lederhat''} Det er den, der
får folk til blindt at gøre alt, hvad man beder dem om.

\scene{(Hatte: C: Leder, F1: Ja, F2: Nej)}

\says{C} Sig mig F1, hvad synes du om min storstilede plan om at
skære 20\% af forskningsstaben?

\says{F1} Ih, \emph{det} lyder som en god idé. Der er \textit{alt for mange}
forskere her. Vi skulle hellere flytte midlerne til fysik.

\says{C} Døøøgtig forsker \act{giver godbid}. Og hvad siger du, F2?

\says{F2} Det synes jeg altså ikke er en god...  

\says{C} \act{afbryder} Godt! så er vi enige. Og da jeg har
lederhatten på, vil du gøre alt hvad jeg beder dig om. Du F2. Jeg
synes, du skal på knæ og titulere mig med en passende titel - lad os
sige ``Root''.

\says{F2} \act{Falder på knæ} Pißß...

\says{C} Uh, det er skønt at være leder. Er du min lille
forsker-bitch? Hva'? Er du det? Er det det du er?

\says{F2} \act{Forvirret} Det ved jeg ikke... Er jeg det?

\says{C} Fortæl mig, hvor fantastisk jeg er!

\says{F2} \act{Prøver at modstå}.. Mmmm

\says{C} Nå, hvad bliver det til?

\says{F2} ... Måske, hvis du lige bukker dig lidt forover - så kan du
bedre høre.

\says{C} \act{Bukker sig forover mod F2} Ja?

\says{F2} \act{Snupper lederhatten fra C og rejser sig op}

\scene{(Hatte: C: N/A, F1: Ja, F2: Leder, Gulvet: Nej)}

\says{F2} Ha! Nu bestemmer jeg. Fald så på knæ.

\says{C} \act{Falder på knæ} Pißß...

\says{F2} Ja, det faktisk er skønt at være leder. Fortæl mig, hvor
fantastisk \emph{jeg} er.

\says{C} Mmm.

\says{F2} Kom nu...

\says{C} \act{tror han er snedig} Hvis du nu bukker dig ned, så kan du
bedre høre...

\says{F2} \act{Bukker sig ned, men holder demonstrativt fast på sin
  hat} Hehe. Ok, fortæl mig det så.

\says{C} \act{Indser han ikke kan nå hatten og tager Nej-hatten fra gulvet}

\scene{(Hatte: C: Nej, F1: Ja, F2: Leder)}

\says{C} Nej!

\says{F2} Jo!

\says{C} Nej!

\says{F2} Jo!

\says{C} Nej!

\says{F2} Jo! \act{Eventuelt flere gentagelser...}

\says{F1} Wow. En deadlock! Sejt!

\says{F2} Aaargh! \act{Tager C's nejhat af C og sætter den på F1 -
  oven på hans ja-hat - og fastholder både sin egen lederhat og F1s to
  hatte}

\scene{(Hatte: C: N/A, F1: Nej-Ja, F2: Leder)}

\says{F2} Så, nu kan du ikke stritte imod! Er det ikke rigtigt, F1?

\says{F1} \act{Som nu både har Ja- og Nej-hat på} Nej-ja...

\says{N} \act{Stikker hovedet ud gennem bagtæppet} Ja?

\says{F2} \act{Til Naja} Nej! \act{Til F1} Svar ordenligt!

\says{F1} ...Mmm-måske? \act{Ruller tvivlende med øjnene}

\says{C} \act{prøver at stjæle lederhatten fra F2. F2 undviger}

\says{F2} Så stopper festen. Du skal opføre dig ordenligt.

\says{C} Næææ \act{hiver en ny hat frem} ... For nu har jeg en
bøllehat på. 

\scene{(Hatte: C: Bølle, F1: Nej-Ja, F2: Leder)}

\scene{C og F2 kigger begge hinanden i øjnene som om, de var i gang
  med en cowboyduel}

\says{F2} Du skal gøre som jeg siger.

\says{C} Næ. Jeg gør hvad der passer mig. Ikke nok med, at det her er
en BØLLE-hat. Det er også en Jihad!

\scene{C og F2 holder op med at snakke, og står stille}

\says{F1} G-hat? Har Google lavet en hat? Det er da lidt \textbf{søgt}.

\scene{Alle står HELT stille}

\scene{Vindheks triller ind over scenen...}

\scene{G kommer ind på scenen, hilser på F2 (alle andre er stadig frosset). G bytter hat med F2, trykker hans 

hånd, og går igen. C og F1 aktiveres.}

\scene{(Hatte: C: Bølle, F1: Nej-Ja, F2: Gadaffi)}

\says{F2} \textbf{Knus} oprøret!

\says{C} Aargh! \act{tøvende.. tager bøllehatten af} 

\scene{(Hatte: C: N/A, F1: Nej-Ja, F2: Gadaffi, Gulvet: Bølle)}

\says{F2} \act{Tager Ja-hatten fra F1}

\says{F1} \act{Nu kun med Nej-hat, begynder at knurre truende...}

\scene{F2 forsøger også at tage Ja-hatten fra F1, men F1 snapper efter ham. F2 sætter Ja-hatten på C i stedet.}

\says{F2} \act{Henvendt til C} Skaf hat!

\scene{C forsøger at tage hatten fra F1, men bliver bidt. Til sidst får han fat i hatten.}

\scene{F2 jonglerer med hattene, mens de andre danser russer-dans...}

\says{Alle} Vista!

\scene{...}

\scene{(Hatte: C: Nej, F1: Ja, F2: Gadaffi, Gulvet: Bølle)}

\says{F2} For mange kokke fordærver maden! \act{Alle er nu blevet enige, nikker}

\scene{Simon kommer hoppende ind i kokkehue. Siger noget på Lambda-sprog.}

\says{X} Det skal være \emph{endnu} sortere!

\scene{To negere hopper ind på hver deres side af Simon. De slæber F1 og C hen til deres landsby.}

\says{C} Neeeeejj!!

\says{F1} Jaaa! Jaaaaaaa!!

\says{F2} \act{Er ved at blive fanget af en neger} Holdt! Stop! Vent! Jeg er lederen. Jeg bestemmer hvornår jeg går ud. \act{Lille pause...} \emph{Nu} går jeg ud! \act{Forlader scenen}

\scene{Simon overtager scenen...}


 \end{sketch}
\end{document}
