\documentclass[a4paper,11pt]{article}

\usepackage{revy}
\usepackage[utf8]{inputenc}
\usepackage[T1]{fontenc}
\usepackage[danish]{babel}

\revyname{DIKUrevy}
\revyyear{2011}
% XXX: HUSK AT OPDATERE VERSIONSNUMMER
\version{2.0}
\eta{$1$ min}
\status{Brokker}

\title{Manden der forklarer Store UP1}
\author{.* Productions}

\begin{document}
 \maketitle

 \begin{roles}
\role{M}[Brainfuck] Manden der forklarer humor.
\role{R}[Ronni] Ronni
\role{H1} Henchman 1
\role{H2} Henchman 2
\role{X}[X Phillip] Instruktør
 \end{roles}
 
 \begin{props}
\prop{Stok}[] Til at kunne folk ind fra siden når de er blevet irriterende på scenen.  Tænk Zoidberg!
\prop{Stok med blød ende}[] Til at kunne blive slået oven i hovedet med.
\prop{Sabel}[] Kavalerisabel!
 \end{props}

 \begin{sketch}
 
\scene{\textbf{EFTER} næste nummer er \textbf{færdigt}, så kommer M op mellem scenen og publikum.}

\says{M} Og således kan I komme tilbage igen... \act{opdager hvor han er} mand, det er godt nok en større labyrint end jeg husker den. \act{kravler op på scenen} Ligemeget, her ville I også være nået hvis I gik den vej jeg tog, så lad være med det.

\scene{En stok kommer ind fra højre side.}

\says{M}[flygter hen imod bandet] Nej, vent, jeg har endnu ikke fortalt dem om hvilke valutaer de må bruge i baren!

\scene{Bandet tager også en stok frem og slår M i hovedet.  M falder om.  R kommer ind med H1 og H2.  Der kommer kavaleri-fanfare.  R trækker en sabel frem.}

\says{R}[peger på M] Revytter!  Tag ham!

\scene{H1 og H2 løber over til M og hiver ham ud i armene.}

 \end{sketch}
\end{document}
