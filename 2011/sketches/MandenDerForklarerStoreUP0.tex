\documentclass[a4paper,11pt]{article}

\usepackage{revy}
\usepackage[utf8]{inputenc}
\usepackage[T1]{fontenc}
\usepackage[danish]{babel}

\revyname{DIKUrevy}
\revyyear{2011}
% XXX: HUSK AT OPDATERE VERSIONSNUMMER
\version{2.1}
\eta{$2$ min}
\status{Brokker}

\title{Manden der forklarer Store UP0}
\author{Brainfuck og Phillip}

\begin{document}
 \maketitle

 \begin{roles}
\role{M}[Brainfuck] Manden der forklarer humor
\role{X}[X Phillip] Instruktør
 \end{roles}
 
 \begin{props}
\prop{}[]
 \end{props}

 \begin{sketch}
 
\scene{Står tilbage på midtergangen, efter forrige nummer. Tilbagelænet, selvfed...}

\says{M} Åh hej!  Har I savnet mig?  Det er godt at høre, for jeg har nemlig meget at forklare jer i år.  Jeg er dog først blevet bedt om at byde jer velkommen til DIKUrevyen 2011.

\says{M} Og nu tænker I sikkert; ``har der virkeligt været 2011 DIKUrevyer?''  \act{Griner} Ak nej, DIKUrevyens versionsnummer er baseret på det pågældende år i den gregorianske kalender. Havde dette eksempelvis været det gamle Rom, havde denne revy heddet ``DIKUrevyen 2764''.

\says{M} Men nok om kalendersystemer. (Evt. bemærkning til publikumstilråb: ``Ja, det er trist. Men vi har simpethen ikke \emph{tid} nok'')

\says{M} Så for at opnå den bedst mulige oplevelse for jeres medpublikum, så skal vi bede jer om at slukke for jeres mobiltelefoner eller sætte dem på lydløs, da det ellers vil forstyrre roen i salen. Det er vigtigt at I er helt stille under numrene, så skuespillerne kan koncentrere som om deres kunst!

\says{M} Det er også et krav at I ikke bruger åben ild inde i salen - især bål og ligende - da det kan være svære at styre i denne tørre tid.  Ild har for eksempel en dårlig effekt på ens udseende og helbred, og kan lede til varige men, som vi ikke kan sagsøges for.

\says{M} Skulle der under revyen - mod forventning - gå noget galt, som for eksempel at en mikrofon går ud, der er statskup i landet, bandet løber tør for øl eller at grænsen til Tyskland bliver genåbnet, kan det betyde at I må forlade salen i hast. Til dette formål har vi i anledning af revyen installeret 6 nødudgange i salen.

\says{M} Disse nødudgange er placeret her til venstre - altså jeres højre - ved siden af scenen, her under bandscenen, navngivet således på grund af bandet der er ovenpå scenen.  Jeg vil dog understrege at det er stadig kaldes bandscenen, selv når bandet \emph{ikke} står på den.  Der er to midt i salen og to allerbagerst.

\says{M} Altså, allerbagerst. Bag jer.  Ikke bag mig.  \act{vender sig om}  Nu er de bag mig. \act{vender sig om igen} Men ikke nu.  Okay? \act{Spørgene med armene... pause}

\says{M} Hmm... jeg tror altså ikke helt I forstår det.  Prøv at forstille jer at salen er et billiardbord.  Og dørene er hullerne.  \act{Står og trækker tiden, klør sig i skægget...}

\says{M} Mmmm... Det virker som om at dette stadig er lidt for abstrakt for jer.. Men lad mig hjælpe \act{løber hen til midterudgangen (publikums venstre side)}

\says{M} Lad mig lige give jer et klassisk eksempel på en nødudgang! \act{Laver fagter med armene} I dette eksempel manifesterer nødudgangen som en dør. Det er derfor vigtigt at I tager i håndtaget, før I løber ud.

\says{M}[gående mod venste midterudgang] Nu skal det siges, at selvom vi kalder dem \emph{nød}udgange, så betyder det altså ikke at I ikke bruge dem til andre ting, så som når I skal ud og tisse, eller går til pause.  Dog bør I under normale omstændigheder, \emph{ikke} bruge dørene ved scenen eller under bandscenen.

\says{M} Og nu tænker I sikkert: Har jeg virkelig betalt 75 kr. for at sidde og høre på det her. Svaret er ja. ...medmindre, selvfølgelig, at I har fået en fribillet, stjålet den, eller på anden vis undgået at betale.

\says{M} Hvis I er i tvivl om hvilken nødudgang I skal vælge, bør I benytte en grådig algoritme. Hvis I sidder oppe bagved, bør I f.eks. benytte denne nødudgang \act{løber op til den bagerste udgang}.





\says{M}[går op ad trapperne] Dertil kan I tage denne nødudgang.  Hvis I er i tvivl om hvilken I skal vælge, så vær doven og vælg den tættest på jer.  Det er ikke meningen at I alle skal gå ud af én nødudgang.  Vi har hele seks af en grund.

\says{M}[fortsætter mod midtergangen og de bagerste udgange] Heroppe er de to bagerste nødudgange.  Altså når I kigger lige ud mod scenen er de i den del I ikke kan se af salen.  Og husk, at alle nødudgange er markeret med at lysende grønt skilt, så I også kan finde vej hvis lyset går.  Dog ikke hvis lyset også går i skiltene.  Så vær opmærksom på dette.


\says{M} Og herover er salens venstre midterdør, ligesom den anden \act{peger mod højre dør}, men bare ud i venstre side.  Hvis I går ud af en af disse døre, så er der et helt nyt lokale I skal begive jer ud i.

\scene{M går ud af Store UP1s venstre dør.}

\says{M}[udenfor] Når I er herude, så husk at gå til venstre.  Eller til højre hvis I bruger den anden dør.  Ellers når I bare den forkerte vej.  Så fortsætter I ned af gangen, indtil I når til trappen som kan lede jer til--

\scene{M kommer ind af døren bag i salen.}

\says{M} Øh, vent et øjeblik. \act{M rør ved noget ved siden af døren, lyset tændes og slukkes i salen.} Nej, uh... det er vist ikke helt rigtigt.  Lige et øjeblik, jeg er straks tilbage.

\scene{M forlader salen igen.  Næste nummer går i gang.}

 \end{sketch}
\end{document}
