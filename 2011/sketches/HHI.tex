\documentclass[a4paper,11pt]{article}

\usepackage{revy}
\usepackage[utf8]{inputenc}
\usepackage[T1]{fontenc}
\usepackage[danish]{babel}

\revyname{DIKUrevy}
\revyyear{2011}
% XXX: HUSK AT OPDATERE VERSIONSNUMMER
\version{1.2}
\eta{$3.5$ min}
\status{Færdig}

\title{Human-Human Interaction}
\author{Naja, Spectrum, Mathias} 

\begin{document}
  \maketitle

  \begin{roles}
    \role{F}[Naja] HHI-forelæser (Kasper Hornbæk-type)
    \role{S1}[Mathias] Almindelige datalogistuderende
    \role{S2}[Sune] Almindelige datalogistuderende
    \role{S3}[Mark] Almindelige datalogistuderende
    \role{X}[X Kirsten] Instruktør
  \end{roles}

  \begin{props}
    \prop{Slideshow incl. nogle videoklip}[Person, der skaffer]
    \prop{Et podie}[Det finder vi ud af]
  \end{props}

  \begin{sketch}
      
\says{F} Godaften og velkommen til første forelæsning i Human-Human interaction.

\says{F} Nu tænker I sikkert: Hvad er HHI? Det er et fag der er blevet oprettet fordi Datalogisk Institut har taget et kritisk blik på bacheloruddannelsen og tænkt: Det handler simpelthen for meget om at kode!

\says{F} Som instituttets egne videoer på KUs hjemmeside siger, så
handler datalogi jo slet ikke om teknologi, men om mennesker, om
\act{tøselæsper} samspillet imellem programmer og den rigtige verden.

\says{F} Instituttet har derfor hyret mig til at lære jer at tage kode-regne-tænke-hatten af, og tage føle-hatten på.

\scene F viser de to hatte

\says{F} Jeg kan godt se at I er en smule utrygge ved situationen, sådan har de fleste RUC-studerende det også når de starter, men se blot hvilke resultater de opnår!

\says{F} Jeg lover at holde niveauet så I alle sammen kan følge med. I hvert fald her i starten.

\says{F} Som I kan se af vores pensumliste er hårdt spændt for, idet vi både skal nå igennem uddrag af ``Why women can't read maps - why men can't ask for directions'' og samtidigt nå igennem hele Georg Strøms ``Den chipløse mand''.

\says{F} Hvis nogle føler for lidt ekstrakurikulær aktivitet, vil der være en læsegruppe dedikeret til en rigtig perle af kommunikationsteoretisk indsigt: filmen ``What women want''.

\says{S1} Undskyld, men der er jo ikke noget af det der er baseret på forskning!

\says{F} Det jeg hører dig sige, er at du er bekymret for om du nu også
kan forstå materialet fordi det ikke ligner det du er vant til!

\says{F} Bare rolig. Jeg ved godt at det ser svært ud, men hvis du bare har en positiv indstilling fra starten af, er jeg sikker på at du nok skal klare det.

\says{F} Men nok indledning! Vi skal jo også ligesom nå at få hænderne lidt ned i stoffet og blive lidt beskidte, ikke? Jeg har medbragt en lille video der illustrerer en hyppig situation på datalogi - her sidder
Anders og Sebastian og skriver noget kode. Nu ser vi lige den og
overvejer hvad vi kan lære af det de gør.

\scene FILM:
\scene To personer sidder ved en datamat og kigger på en editor.
\scene Lang pause.
\scene Anders: Parseren virker ikke.
\scene Lang pause - Anders og Sebastian kigger begge to tænksomt ind i skærmen
\scene Sebastian: Prøv at fjerne de to parenteser dér.
\scene FILM SLUT

\says{F} Er der nogen der kan fortælle mig hva Sebastian gjorde galt i situationen her?

\says{S2} Det kan vi da ikke rigtig vide? Vi kunne jo ikke se hele programteksten.

\says{F} Nej, nej, nej. Jeg kan godt se jeg måske fik startet med noget af det svære stof her.
 Det vi så her var en typisk fejl, som dataloger begår hver eneste dag.

\says{F} Det I må lære at forstå er at når folk kommer til jer med et problem - som vi nu så Anders gøre i videoen - så er det de vil have i virkeligheden forståelse, ikke? I ved, sympati, empati, indlevelse!

\says{S2} Men hvad så med den der parser? De sagde jo at den ikke virkede..

\says{F} Dét er nemlig rigtigt! Og så er det jo godt at vi alle sammen husker på, at datalogi i virkeligheden handler om mennesker, ikke sandt? Det er lige præcis den slags aha-oplevelser, I kommer til at få masser af på HHI. Så når folk kommer til jer med den slags spørgsmål er det vigtigt at svare dem så at de føler at I lytter til dem.

\says{F} Hvad Sebastian kunne have sagt var: ``Virker din kode ikke? Jeg har også
engang prøvet at min kode ikke virkede.''

\says{S2} Det er jo ubrugeligt!

\says{F} Det jeg hører dig sige er at du føler nok tryghed ved mig og min undervisning til at italesætte dine følelser. Når du udtrykker dig på den måde får det mig til at føle at jeg lærer jer noget.

\says{S2} Men Anders' problem blev jo ikke løst, jeg synes ikke jeg har lært noget som helst 
her.

\says{F} Kan du mærke lige nu at du måske har den forkerte indstilling? Du er ikke åben for nye ideer. Hvor du ser problemer burde du se udfordringer, og udfordringer kan altid løses.

\says{S1} Hvad nu hvis det er en NP-hård udfordring?

\says{F} Men den rette indstilling bliver NP-hårde udfordringer NP-bløde.

\says{F} Jeg kan mærke at I ikke rigtig er åbne for nye idéer lige nu. Jeg synes I skal tage og prøve at tage føle-hatten på lidt tid så du kan forstå dine medmenneskers behov for indlevelse.

\says{F} Lad mig vise jer et godt eksempel i den næste film...

\scene Væggene der blev projiceret op på trækkes til side. Bagved sidder Simon og er ked af det.

\says{F} Er din server gået ned? Jeg har også engang prøvet at min server gik ned. Jeg er virkelig ked af det på dine vegne.

\scene Flydende skift til Server'n er crashet

  \end{sketch}
\end{document}

