\documentclass[a4paper,11pt]{article}

\usepackage{revy}
\usepackage[utf8]{inputenc}
\usepackage[T1]{fontenc}
\usepackage[danish]{babel}

\revyname{DIKUrevy}
\revyyear{2020}
\version{0.1}
\eta{$3$ minutter}
\status{Done?}

\title{VIM er Sejt}
\author{A-Boy}
\melody{Bamse: Vimmersvej}

\begin{document}
\maketitle

\begin{roles}
\role{x}[Bjørn] Instruktør
\role{A}[ABoy] Forsagner
\role{K}[Sejer] Kor
\end{roles}

\begin{sketch}
\scene{Lys op.  A træder ind på scenen.}

\says{A}[tænksom] At være eller ikke være\ldots \act{laver fagter med et kranie}

\end{sketch}
\begin{song}


\sings{K}%
Vimersejt...

\sings{A}%
Mine rus siger at vim er sejt
Men de tager så gruligt fejl
Selv de gamle tror de ved bedre
Men hør nu lige på mig

\sings{A}%
Viiiiiiiiim er squ da skrotteklar, bevars

\sings{Kor}%
Vimerskrald..

\sings{A}%
Når jeg selv skal skrive kode
Brug' jeg ej emacs
For det er skam da endnu værre 
Og det vil jeg nu sige

\sings{A}%
Eeeeeeeeeeeeemacs er noget værre bras, bevars 

\sings{Kor}%
emacsnej...

\sings{A}%
Alle mine venner siger 
at jeg er et kvaj
Men jeg elsker bar' vscode
Det er squ for nice

\sings{A}%
Jeeeeeeeeeeg elsker vscode, hurra?

\sings{Kor}%
vscode...

\scene{Lys ned.}
\end{song}
\end{document}
