\documentclass[a4paper,11pt]{article}

\usepackage{revy}
\usepackage[utf8]{inputenc}
\usepackage[T1]{fontenc}
\usepackage[danish]{babel}

\revyname{DIKUrevy}
\revyyear{2022}
\version{1.0}
\eta{$3$ minutter}
\status{Færdig}

\title{Tro på mig}
\author{Eva}
\melody{Bet On It - High School Musical 2 Soundtrack}

\begin{document}
\maketitle

\begin{roles}
\role{X}[Instruktørnavn] Jonatan
\role{S1}[Eva] Sanger 
\role{S2}[Lise] Sanger
\role{MIB1} [Torben Mogensen]
\role{MIB2} [Ken Friis]
\end{roles}
\section*{Noter:}
Tænker at vi kører det i conspiracy mode. Eventuelt med en lille start sketch, hvor Lise prøver at overbevise folk om noget, tænker at den slutter med at MIB stikker sangere med kanyler og trækker dem ud.
Havde lidt overvejet om det kunne være sjovt hvis det var John-sporing. 

\begin{props}
\prop{Conspiracy-board, rødt garn og knappenåle}
\prop{kritiske dataloger - trøjer x2}
\prop{kritiske dataloger hatte x4}
\prop{pistoler x2}
\prop{neuralizer x1}
\end{props}


\begin{sketch}
\scene{Dansere sidder ved et bord}
\says{S1} Velkommen til kritiske datalogers første møde
\scene{musikken begynder}
\end{sketch}


\begin{song}
\sings{S2} Her på DIKU går der mange rygter
omkring hvordan ting “virk’lig er”
tankegangen er helt i stykker
lyt til mig og tag ved lær’
\sings{S1} (rettelse et)
\sings{S2} Din GUI skal ik’ være blå
\sings{S1} (rettelse to)
\sings{S2} Emacs skader ikke hænderne
\sings{S1} (rettelse tre)
\sings{S2} man er kun rus på første år
ik’ halvandet år
hvad er det i ik’ forstår?

\sings{S1} van Emde Boas træ’r, er brugbare
Emojis i min kode, er praktiske 
den bedste strategi 
er testdriven
tro på mig, tro på mig, tro på mig
(tro på det)
\sings{S2} jeg kommer til at vis’
P lig NP
VIM er intuitivt
det vil i se
LISP har en passende parentesmængde
tro på det, tro på det, tro på det, tro på det

\sings{S1} netværk er CompSys’ bedste anpart
UIS er det bedste fag
at kode javascript er meget rart
NBB er bygget færdig snart /(niels bohr den står færdig snart)
\sings{S2} (tro dog på det)
\sings{S1} word er li’så godt som latex
\sings{S2} (tro dog på det)
\sings{S1} not not q er ikke lig q
\sings{S2} (tro dog på det)
\sings{S1} Fysikerne er ik’ sær’
og de kan blive andet end gymnasielærer

\sings{S2} For mange øl det smadrer leveren
\sings{S1}(leveren)
\sings{S2}I revyen er der kun kjolemænd
\sings{S1}(kjolemænd)
\sings{S2}Din arbejdsposition 
skader ik’ din lænd
\sings{S1} (ik' din lænd)
\sings{S2} tro på mig, tro på mig, tro på mig, tro på mig
(tro på det)
\sings{S1}i kommer til at se
VT er svært
Luk VIM med Ctrl-X
få det dog lært
internetfænomenet er temporært
tro på mig, tro på mig, tro på mig, tro på mig

\sings{S2}Hov, nej vent
hvorfor hører i ik’
på hvad jeg si’r
de tror ik’ på mig
går bare deres vej

\sings{S1} jeg ved jo godt
de andre går
og griner ondt bag
min ryg
bare fordi
jeg sætter sandhed’ fri, 
\sings{Alle} uuh
\sings{S2} der kommer en dag 
hvor de vil se
alt det smarte ved
fibonacci heaps

\sings{S1} jeg vil ikke stop
kommer ik til at stop vil ikke drop’
\sings{S2} skriv i typesvag’
sprog hver dag
\sings{S1} stjæl en kantinekop, du kan
tro på mig, tro på mig, tro på mig, tro på mig, der er  
\sings{S2} Sindsrevolution på datalogi
\sings{S1}Tankeændringerne
de melder sig
\sings{S2}en dag så vil i se
den rette vej
Jeg taler sandt, i kan tro på mig
tro på mig, tro på mig,
tro på mig, tro på mig 
\sings{S1} (I kan tro på mig)


\end{song}


\begin{sketch}
\scene{Lyset går ikke ned, S1\&S2 står stille}
\scene{MIB1\&2 træder roligt ind på scenen op til S1\&S2 og MIB1 skyder dem}
\says{MIB2} Hey, hvad laver du, hvorfor skød du dem? 
\says{MIB1} De var for farlige til at vi kunne lade dem leve
\says{MIB1} \act{kigger ud på publikum} De ved for meget
\scene{MIB1\&2 tager solbriller på, MIB2 tager neuralizer frem}
\says{MIB2} Hvis i vil vide hvad en monade er, så kig her \act{peger på neuralizeren}
\scene{Neuralizeren bliver aktiveret}
\act{her kan Torben\&Ken være med til at foreslå hvad de synes kunne være sjovt at sige de skal huske i stedet}
\says{MIB1} I har lige set er flersproglig eksamen og i er på vej ud for at købe 3 gulddamer til det næste akt. 

\end{sketch}

\end{document}
