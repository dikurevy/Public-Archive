\documentclass[a4paper,11pt]{article}

\usepackage{revy}
\usepackage[utf8]{inputenc}
\usepackage[T1]{fontenc}
\usepackage[danish]{babel}

\revyname{DIKUrevy}
\revyyear{2022}
\version{1.0}
\eta{$4$ minutter}
\status{Færdig}

\title{Spred den gode Stemming}
\author{Jonatan}

\begin{document}
\maketitle

\begin{roles}
\role{x}[Jonatan] Instruktør
\role{E}[Sia] Eksaminant
\role{M}[Eva] Medstuderende
\end{roles}

\begin{props}
\end{props}

\section*{notes}


\begin{sketch}
\says{E} Og derved opnår vi en asymptotisk køretid på nlogn. Hvor lang tid tog det?
\says{M} \act{opgivende} Halvanden time...

\says{E} For helvede over tid.. igen

\says{M} \act{stadig opgivende} Altså, hvis det er kunne du måske nøjes med at bevise Kruskal’s algoritme eller en af hob- og fibonaccihob-versionerne af Prim’s algoritme i stedet for dem alle tre…

\says{E} \act{Tænksomt} ”Hmmm, ja…”
\act{kort pause}
Med entusiasme:
”ELLER… jeg kunne optimere og dermed minimere min taletid!”
\says{M} \act{med gru i stemmen} ”Du har vel ikke tænkt dig at..?”
\scene{dramatisk pause}
\says{E}
       \act{Rejser sig op}
               ”Jo, jeg vil bruge”\act{
               Dramatisk pause}   STEMMING
\says{M} \act{forfærdet} åh nej alt endet end det
\says{E} Jovist! Med stem kan jeg reducer min fremlæg med en betyd del! Vink farvel til al endels! Nu er jeg den skarp kniv i skuff!
\says{M}
      \act{Rejser sig vredt}
               ”Ej, hallå! Du kan da ikke bare fjerne endelser fra ord under din præsentation!”
              \act{pause}
               ”Ikke uden også at fjerne…”
               \act{Dramatisk pause}
               ”STOPORD!”

\says{E}      ”STOPORD!!, fjern stopord hurtig fremlæg”
\says{M} Fjern stopord, fremlæg 30 minut
\says{E} \act{iritteret} hmm
\says{M} id! Bag word!
\says{E} Genial: 1, Bag: 1, word: 1, fremlæg: 1, , max: 1, et (1): 1, minut: 1!
\says{M} Fed: 1, ses: 1, færdig: 1, eksam: 1
\scene{E går ud og kommer ind igen og ser trist ud (evt sceneskift for at markere det er efter eksamen – evt efter anden sketch?}
\says{M}    ”Når, hvordan gik det med din eksamen?”
\says{E}        ”Minus: 1, tre: 1…”
\end{sketch}

\end{document}
