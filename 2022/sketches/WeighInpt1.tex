\documentclass[a4paper,11pt]{article}

\usepackage{revy}
\usepackage[utf8]{inputenc}
\usepackage[T1]{fontenc}
\usepackage[danish]{babel}

\revyname{DIKUrevy}
\revyyear{2022}
\version{1.0}
\eta{$4$ minutter}
\status{Færdig}

\title{Professor Weigh-in}
\author{Eva, Bjørn}

\begin{document}
\maketitle

\begin{roles}
\role{X}[Sean] Instruktør
\role{PW}[Jonatan] Pawel
\role{CW}[Torben] Christian
\role{A}[Eva] Announcer 
\role{C}[Mange] Crowd på scenen?
\role{PWS}[NAVN] Pawels støtter
\role{CWS}[Sia] Christians støtter 
\end{roles}

\begin{props}
\end{props}
\section*{notes}
Det er nok lidt urealistisk at få nok skuespillere til at skabe en crowd, men man kan eventuelt få teknikken til at køre crowd-lyde. Den anden del af sketch skal nok være lidt længere 

\begin{sketch}
\scene{Lys op}
\says{Announcer} 2 professorer fra DIKU møder snart hinanden på en datalogisk konference, med hovedemnet: “Asymptotisk effektive køretidsalgoritmer på træstrukturer”. Fra DIKUs Algoritmesektion, han har 65 artikler bag sig, med en kampvægt på ca. 1200 citeringer, i kender ham fra jeres AD eksamen: CHRISTIAAAAAAN WUUULFF!

\scene{CW kommer ind, hujende, hoppende, hyper publikum, han har et crowd af støtter med ud på scenen, høj energi}

\says{Announcer} I det andet hjørne, geometrimesteren fra Image-gangen, med 77 artikler bag sig, og en kampvægt på CA 5000 CITERINGER, i kendte ham fra jeres AD reeksamen, nu kender i ham også fra jeres MASD reeksamen: PAWEEEEL WINTEEER! 

\scene{PW går køligt og roligt ind på scenen. Hans støtter er hype bag ved ham. På overtexen står der “(pawel) Winter is coming”}

\scene{Alle på scenen køler lidt ned}

\says{Announcer} Pawel, det her er jo ikke dit første rodeo, ser du frem til konferencen?

\says{PW} Jeg synes altid konferencer er hyggelige, man møder søde mennesker, lærer spændende nye ting…

\says{Announcer} Og hvad med dig og Christians snarlige debat omkring asymptotisk effektive køretidsalgoritmer på træstrukturer?

\says{PW} Ja, det er jeg nu egentlig ikke så bange for, fordi alt Christian nogensinde har bevist er jo sådan set trivielt

\says{CW} Hva'?! (\textit{forvirret over "tonen"})

\says{PW} Ja, altså du er en glimrende underviser... 

\scene{CW nikker lidt, og til sig selv: "Ja det er rigtigt"}

\says{PW}... men alt du nogensinde har bidraget med forskningsmæssigt er jo ligegyldigt

\says{CW} HVAD!!! HOLD MIG TILBAGE!!!

\scene{CWS holder ham tilbage}

\says{PW} Ja, i må hellere holde ham, FOR MUNDEN, så han ikke spilder vores allesammens tid med ting som selv programmeringssprogsforskerne kan forstå!!

\says{PWS} Uuuuuuh

e
\says{PW} Hvad siger du til mig din hvalp?

\says{CW} Jeg siger at du ikke er kommet videre i din forskning i 20 ÅR

\scene{Nu bliver PWS vredd, PW beroliger dem med håndbevægelser}

\says{PW} Så du gør det personligt, hva? Jeg har forsket siden du gik i vuggestuen, ti nu stille før du kløjes i modermælken, bette ven

\says{CW} Jaja, nemt at sige for professoren der har DET NEMMESTE fag på kandidaten - HVOR SVÆRT KAN DET VÆRE AT SKÆRE TING OP I TREKANTER?

\scene{PW springer op - PWS holder ham tilbage nu. CW også holdt tilbage af CWS}

\says{PW} DET ER NP-HÅRdt AT FINDE NOGET AF VÆRDI I DIN FORSKNING.

\says{CW} DIT LIVSVÆRK ER POLYNOMISK TID REDUCERBART TIL EN BUNKE LORT… i store O AF ÉN TID!

\scene{deres støtter skubber dem langsomt væk fra hinanden, ud af scenen}

\says{PW} DET VILLE VÆRE DEN FØRSTE TING DU NOGENSINDE HAR FÅET TIL AT KØRE I KONSTANT TID

\says{CW}: JEG HAR FÅET DIN DATTER OP AT KØRE I KONSTANT TID

\says{PW}: JEG HAR FÅET DIN MOR OP AT KØRE I KONSTANT TID. 

\says{CW} DER ER CYCLER I DIT STAMTRÆ DIN TØLPERAGTIGE KANALJE   <kan godt ære disset her skal ændres>

\says{PW}: KAN DU BEVISE DET? 

\says{CW}: <Række af censur beeps>, KAN DU BEVISE DÉT?

\says{PW} DET ER MIN LINJE! FUCK DIG! VI SES TIL REEKSAMEN

\scene{de er helt ude ved scenekanterne nu, råber gibberish}

\says{Announcer} Som I kan se kære publikum bliver det en spændende debat - hvem sagde algoritmikere var kedelige, vi vender tilbage med debatten - I 3. AKT
\scene{Lys ned}
%### Torben og Jona Tilføjelse ### 



\end{sketch}

\end{document}
