\documentclass[a4paper,11pt]{article}

\usepackage{revy}
\usepackage[utf8]{inputenc}
\usepackage[T1]{fontenc}
\usepackage[danish]{babel}

\revyname{DIKUrevy}
\revyyear{2022}
\version{0.1}
\eta{$?$ minutter}
\status{Done}

\title{Løvernes Hule}
\author{Rasmus}

\begin{document}
\maketitle


\begin{roles}
\role{X}[Sean] Instruktør
\role{R}[Lukas] Rolf Tytteberg (pithcer)
\role{B}[Albert] Brigitta Guldgård
\role{C}[Sejer] Crypto Carsten
\role{H}[Jonatan] Humaniora Hennning
\role{A} [Amira] Anouncer
\end{roles}
\begin{props}
\prop{Tøjhund(e)}
\prop{3 x stole}
\prop{notesblokke}
\prop{kostumer}
\end{props}



\begin{sketch}
\scene{Lyset er slukket}
\says{A} I dag i Løvernes Hule skal en håbefuld iværksætter præsentere sin virksomhed for de tre løver.

\scene{Spot på Brigitta}

\says{A} Brigitta Guldgård startede i 1996 InternetVirksomhed, og har drevet flere af landets største IT-virksomheder til success. Derudover er hun angel investor for over 5000 virksomheder og har engang lugtet til Steve Jobs' turtleneck.

\scene{spot på crypto carsten}

\says{A} Carsten Hansen aka Crypto-Carsten har været blandt front-løberne i danske tech startups. Han har lavet sådan nogle øhh block chains eller sådan noget. Han er sikkert vigtig

\scene{Carsten prøver at få et ord ind}

\says{C} Ahh det er jo ikke bare block-chains, det er en hel ny måde at tænke økonomi på **Mikrofonen bliver cuttet**. En decentralisering og demokratisk pyramidespil.

\scene{Spot på humaniore henning}

\says{A} Henning Martin Sebastian Ketil Østergaard er repræsentant for statens innovationsfond og fokuserer på bæredygtige virksomheder indenfor miljø, socialt udsatte og gode vibes.

\scene{lys op}

\says{A} Dagens første iværksætter er Rolf Tytteberg, som står bag hunde-dating-appen "Hinder".

\scene{Rolf går ind på scenen med en tøjhund under armen}

\says{R} Hej Løver.
Hundes ensomhed er et socialt problem der alt for længe er blevet ignoret af mennesker. 1 ud af 3 hunde har ikke en numse at lugte til, og 1 ud af 2 hunde har prøvet at kopullere en pude ud af ren ensomhed.
Det er her at "Hinder" kommer ind i billedet. "Hinder" står for "Hunde-Tinder", og lader dig finde den helt rigtige partner til din bedste ven.
Vi er i stadig i de tidlige stadier, men vi leder efter 3.2 milliarder kr. for 2.5\% af vores virksomhed.

\says{B} Hej Rolf. Tak for din præsentation. Nu beder I om 3.2 milliarder kr for 2.5\%, hvilket jo er værdiansættelse på  \act{Tæller på sine fingre} ……. \act{kigger på overtex, hvor prisen står} Så mange penge. Har i et produkt som er kommet på markedet? Hvilken værdi bringer i som kan forsvare så høj en værdiansættelse?

\says{R} Ja, øh, vi har ikke rigtigt kodet noget endnu, eller designet så meget, for det kan jeg ikke lige finde ud af. Men til gengæld så fik vi en rigtig god idé! Og så kan vi bare hyre en masse dataloger som ubetalte praktikanter til at kode den, de kan jo alligevel ikke få andet arbejde!

\says{C} Hvordan kommer I til at tjene penge?

\says{R} Jammen altså ifølge vores estimater rammer vi 500 millioner brugere i løbet af de første tre måneder, hvorfra vi sælger hundedataen til Pedigree.

\scene{løverne nikker og skriver ned}

\says{B} Ja. Jeg er fan af idéen, og jeg ser noget great potential i jeres lille øhh app. Men jeg ser bare ikke lige det der potentiale i... dig. Altså hvordan kan du lede et über eksponentielt mega unicorn hvis du ikke engang har en sort turtleneck på. Derfor er jeg ude.

\says{H} Hej Rolf. Det er godt nok en sød ven du har med der. Det gør mig helt varm indeni at der endelig er nogen som fokuserer på hundes ensomhed. 
Jeg kender det fra min egen lille dansk-svensk-slovensk markrottehund Hans-Berit. Min lille skattebasse er så ensom at han piver hver gang vi ser Lassie på TV2 Charlie, og vil ikke engang røre den lækre veganske linsepostej som jeg har lavet til min lille snut.
Derfor vil jeg gerne byde 472 kr og en lunken linsepostej for 42.43\% af din virksomhed.

\scene{Rolf er overrasket over det generøse bud}

\says{C}
Jaaaa, altså jeg kan godt følge dig i det der hunde noget. Det kan jeg virkelig. 
Mee<en.
Jeg synes slet ikke at det er ambitiøst nok. Hvad nu hvis vi gjorde jer til en GDPR-baseret, blockchain-complaint SEO-fabrik af hundebillede-tokens?  
Vi kan skabe et helt nyt univers af demokratiske kæledyrs-'smart contracts' mens søde hundehvalpe miner ICOs og vi airdropper 'pump & dump' Ponzi NFT'er direkte ned i decentraliserede pyramideøkonomier.
Forestil dig: Om 2 år kunne I være den nye Dogecoin.

Der er *potential*!
Derfor vil jeg byder 5 NFT'er som jeg selv har udstedt for 69%.

\says{R} Tak for jeres bud løver, jeg vil lige drøfte det med min partner.

\scene{ROLF GÅR OVER I HJØRNET MED SIN TØJHUND OG HVISKE-TISKER}

\scene{VENDER SIG OM IGEN}

\says{R} Tak for jeres bud. Det har været et svært valg, men jeg har valgt...

\scene{rolfs telefon ringer}

\says{R} Aha. Aha. Ja. 500 milliarder kroner? Aha. Aha. Ja, helt sikkert!
\scene{Rolf ligger på}

\says{R} Jeg har lige scoret en investering fra Jobindex, de køber hele lortet. Ses bitches!

\scene{lys ned}
\scene{AV VISER LOGO MED DOGINDEX}


\end{sketch}
\end{document}