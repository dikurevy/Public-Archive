\documentclass[a4paper,11pt]{article}

\usepackage{revy}
\usepackage[utf8]{inputenc}
\usepackage[T1]{fontenc}
\usepackage[danish]{babel}

\revyname{DIKUrevy}
\revyyear{2022}
\version{1.0}
\eta{$4$ minutter}
\status{Færdig}

\title{Flersproglig Eksamensoplæsning}
\author{Bjørn}

\begin{document}
\maketitle

\begin{roles}
\role{X}[Lui] Instruktør
\role{I}[Bjørn] Instruktor
\role{ER}[Albert] Engelsk Rus 
\role{FR}[Torur (i teknikken)] Færøisk Rus 
\role{CR}[Sofie] C# Rus 
\role{SR}[Amira] Scratch Rus 
\role{BR}[Albert] Blind Rus 
\end{roles}

\begin{props}
\prop{Udkledning der representerer hver rus, scratch kat, blinde-briller...}
\end{props}


\begin{sketch}
\scene{Lys op på I og ER. Når der skiftes fra én rus til en anden, sker kostumeskift}
\scene{Rus sidder alle på scenen}
\says{I} Jamen velkommen så ! I skal til eksamen, og det har i jo nok regnet ud siden i ser så spændte og ... høhø svedige ud . I skal sidde her i 2 timer plus minus … ææh ... nogle timer?
Det skal nok blive rigtigt hyggeligt! jamen så vender i bare siden om og så kører starter festen … ææh … ju huuu!
\scene{Rus vender deres papirer}  
\says{I} I peger bare lappen op i vejret, hvis i har brug for en hjælpende hånd fra undertegnede… ja det er et farligt ord, men det er altså bare mig.
\scene{ER har stærk, Ole Henriksen-agtig accent}
\scene{ER rækker hånden op}
\says{ER} Yes Jeg er fra GREAT BRITAIN, og Jeg would like to høre if jeg can have the eksamens tekst in English.
\says{I} Det var farligt mange flotte ord, æææh Men eksamen er jo gået igang…
\says{ER} Jeg was loved, that jeg kunne få teksten på English!
\says{I} æææh si si eller yes… hvad med at jeg læser teksen op for dig på engelsk
\says{ER} what ?
\says{I} ææh, What med, at I reader den Op for you.
\says{ER}: Okay sure.
\says{I}: Velkommen… ææh … nå nej… Welcome to the Computer SYstemer Exam. You have 4 hours… okay du har så kun 3 timer og 50 minutter øøøh… nææh You can use your Penci…
\scene{FR har stærk færøsk accent}
\says{FR} Undskyld mig, jeg har et spørgsmål
\says{I}: What … eller æhem.. hvad er der min ven?
\says{FR} Jeg er fra færøerne og jeg ville høre om jeg kunne få den på færøsk.
\says{I} Færøerne.. ææh det er det sted med gederne ikke … ahem okay hvor svært kan det være… ahem…. (siger en række lyder hvor halvdelen af dem er gedelyde)
\scene{CR lyder lidt som en narkoman}
\says{CR} Undskyld mig, jeg har et spørgsmål
\says{I} Hvad er der nu… (puster ud)... hvad kan jeg hjælpe dig med min fine ven? 
\says{CR} NU har jeg haft Software udvikling 5 gange og jeg forstår overhovedet ikke ting, som ikke er object baseret!!
\says{I} øøøh, hvis du forestiller dig, at der er en klasse, som er alle opgaver og den her opgave har så nedarvet dens evne… til at være en opgave… eller noget… \says{I} nej vent lidt…. altså den her ene opgave med en stak er enkapsuleret fra den anden fordi…. at den ikke er den anden æææh
\scene{SR råbende}
\says{SR} UNDSKYLD!!! HVAD HVIS MAN IKKE HAR BESTÅET POP!
\says{I} Gør pop dig sur… DNUR ( måske drop den her del)
\says{SR} HVAD BETYDER DNUR!!! KAN DU IKKE FORKLARE OPGAVEN SOM SCRATCH
\says{I} okay så katten drejer rundt … og siger Miaw! 
\scene{BR er blind} 
\says{BR} Undskyld! Jeg har et spørgsmål…
\says{I} HVAD NU !!! VIL DU HAVE DEN I BINÆR ELLER HVAD !!!
\says{BR}: Rolig nu.. jeg ville bare høre om du kunne hjælpe mig med at forstå en enkelt ting.
\says{I} *(puster ud)* okay, hvad kan jeg hjælpe dig ven ?
\says{BR} Jo ser du. jeg har været blind i 10 år men jeg har kun lært tegnsprog. Jeg tænkte på, om du kunne oversætte for mig.
\says{I} Okay forstil dig en hånd som ligner en tyr …
\says{BR} Det giver ikke mening…
\says{I} Okay så en hånd med en omvendt tyr
\says{BR} okay, så en stak altså.
\scene{Alle eleverne begynder at råbe hjælp på hvert deres svar. I begynder at græde} 
\says{I} Jeg vil ikke mere. I kan få svarene, hvis I lover ikke at fortælle de andre at jeg er begyndt at græde..

\scene{I løber ud} 
\says{BR} høh… nemt 12 tal. ( måske: de sagde Compsys var svært )
\scene{Lys ned.}
\end{sketch}

\end{document}
