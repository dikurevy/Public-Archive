\documentclass[a4paper,11pt]{article}

\usepackage{revy}
\usepackage[utf8]{inputenc}
\usepackage[T1]{fontenc}
\usepackage[danish]{babel}

\revyname{DIKUrevy}
\revyyear{2022}
\version{1.0}
\eta{$4$ minutter}
\status{Færdig}

\title{Internetprotokoller}
\author{Lui, Lukas, Bjørn}

\begin{document}
\maketitle

\begin{roles}
\role{X}[Sean] Instruktør
\role{S}[Lukas] Straightman
\role{F}[Lui] Forelæser
\end{roles}

\begin{props}
\prop{Intet}
\end{props}


\begin{sketch}
\says{Announcer - Wave file} Følgende indslag er basseret på en ægte forelæsning og alt indhold er mere eller mindre sandt.. (Sean skal speake dette og levere til teknikken).

\scene{Lys op. S og F kommer ind på scenen.}

\says{F} Vi har alle sammen prøvet det, man sidder derhjemme og pinger forskellige IP adresser og pludselig slår det dig… hvis der er IPv4 og IPv6, hvad så med 0,1,2,3 og 5? 

\says{S} (overrasket) Ja det har faktisk også undret mig

\says{F}Bare rolig, din undren er ovre!...
Allerførst besluttede man at der kun gik 4-bits til at signallere ip-versionen, da man umuligt ville nå gennem mere end 16 versioner nogensinde. Det var derfor essentielt at hvert nummer udnyttedes optimalt...
Det hele startede i 1973 med IPv1, som… (afbrudt)

\says{S} Hvad med IPv0? Man 0-indekserer vel?

\says{F} Selvfølgelig gør vi det. IPv0 er reserveret.

\says{S} Reserveret til hvad? 

\says{F} Glimrende spørgsmål… Det hele startede med IPv1. IPv1! var skrald… Så ehm… Den snakker vi ikke om. Hvilket naturligvis fører os til IPv2 og det er jo et flot rundt tal, hvilket vi jo godt kan lide. En skam, at den også var skrald… Så altså, IPv1 og 2 var skrald og IPv3 er ikke et rundt tal, så IPv4 blev introduceret i 1981. Endelig havde verden en fungerende internetprotokol og vi havde kun brugt 5 af de 16 versionsnumre...
Den næste protokol der blev introduceret var i 1988 med IPv7.

\says{S} Hva’? Har vi ikke sprunget nogle tal over her? 

\says{F} Jo, det er fordi dem der lavede IPv7 troede at IPv6 allerede var i brug, fordi dem der lavede 5 også lavede en anden version af 5, som dem der lavede IPv7 troede var IPv6, men som altså ikke var IPv6, men bare en anden version af 5…

\says{S} \textit{(Skal se lidt forvirret ud og lige tænke et øjeblik, derefter have et “vent hvad” ansigts udtryk)} Vent, 5? Altså IPv5? 

\says{F} Nej, bare 5

\says{S} Bare 5?

\says{F} Ja bare 5..

\says{S} Hva' !? 

\says{F} Ja, altså, internetprotokollens, altså IP,s version 5, altså v5, syntes ikke det gav mening at hedde IPv, så den hed bare 5.

\says{S} Okay…

\says{F} Anyway, 5 blev udviklet til streaming af video, men det ved vi alle, at der ikke var nogen fremtid i, så den blev aldrig udbredt.

\says{S} Ja, hvad så med IPv6?

\says{F} Hov hov, nu skal du ikke springe frem i forelæsningen. Først kom 9 i 1992, men de opdagede, at det 9 skulle løse fra IPv4 allerede var løst i IPv4, så de samme mennesker, forsøgte i stedet med 8 i 1994, men opdagede at det også var IPv4…

\says{S} Så dem der lavede 9, lavede bagefter 8, men både 8 og 9, var faktisk bare IPv4?

\says{F} Ja, lige præcis! 

\says{S} Det giver sikert mening..

\says{F} Samtidig med at 8 blev lavet, blev IPv9 også introduceret. 

\says{S} Hva’? Jeg troede IPv9 blev lavet før 8? 

\says{F} Nej nej nej, det var 9.

\says{S} Det er jo det jeg siger? 

\says{F} Nej, 9 blev lavet før 8, men IPv9 blev lavet samtidig med 8.

\says{S} Hvad er så forskellen på 9 og IPv9?

\says{F} 9 var bare IPv4 imens IPv9 i 1994 blev annonceret til håndtering af nanomaskiner der anvendte blodstrømmen som kommunikationskanal til at transmittere til og fra implantater i bl. a. hjertet, nyrer og hjernen.

\says{S} \textit{(Overrasket, og måske lidt stoked)} Seriøst? Det lyder sindsygt!

\says{F} Det viste sig at være en aprilsnar

\says{S} \textit{(Lidt skuffet)} Ahh..

\says{F} Nå, men det fører os altså til IPv6 i 1995, som egentlig bare var en væsentlig bedre protokol end alle de andre.

\says{S} Så må vi vel også snart være færdig?

\says{F} Ja vores historie slutter i 2004 med IPv9, som var en internetprotokol udviklet af kineserne, som konkurrent til den amerikanske IPv6.

\says{S} Var IPv9 ikke en aprilsnar?

\says{F} Det var den anden IPv9 

\says{S} Så der er 9, som blev lavet før 8, men som bare er IPv4, så er der IPv9, som er en aprilsnar og IPv9 som er en kinesisk protokol? 

\says{F} Ja!

\says{S} \textit{Et opgivende suk}

\says{F} Index 9 var allerede assignet til både IPv4 og en aprilsnar, men kineserne observerede at de var mange nok til alle andre var nødt til at rette ind, og lavede deres egen IPv6, altså IPv9.

\says{S} \textit{(Igen sådan lidt opgivende)} Det kan umuligt havde forvirret nogen...

\says{F} Den dag i dag er der stadig en diskussion om hvorvidt den bedste IPv9 er IPv4 eller IPv6. Til resten af internettets levetid, har vi stadig 5 af de 16 versionsnumre, nemlig IPv10-14.

\says{S} Hvad med IPv15?

\says{F} Den er reserveret.


\scene{Lys ned}

\end{sketch}
\end{document}