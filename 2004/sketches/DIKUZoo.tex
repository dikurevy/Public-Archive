\documentclass[a4paper,11pt]{article}

\usepackage{revy}
\usepackage[utf8]{inputenc}
\usepackage[T1]{fontenc}
\usepackage[danish]{babel}

\revyname{DIKUrevy}
\revyyear{2004}
% HUSK AT OPDATERE VERSIONSNUMMER
\version{1.1}
\eta{4-5 min}
\status{Færdig}

\title{Datalogisk have}
\author{Uhd \& nokando}

\begin{document}
\maketitle

\begin{roles}
\role{ID}[Adam] Videnskabsmand (Ilmer Datblaad)
\end{roles}

\begin{props}
\prop{planche}
\prop{1 stk. flipoverstativ med billeder af dyr og nørdet navn under}
\prop{1 stk. VM kittel}
\end{props}

\scene{Tom scene}

\begin{sketch}
  
  \scene{Lys... ID kommer MEGET energisk ind med et flipoverstativ under armen.
    Mens han stiller FOS op begynder han sit foredrag. Undervejs må han gerne
    blive mere og mere træt i stemmen, som om han godt selv kan se at det er
    noget rod de endte med at lave}
  
\says{ID} [russisk accent] God aften - min navn er Dr. Ilmer Datblaad. Jeg ska'
forsøge at forklare om et lille meget spændende projekt vi ha lavt på Universitet
Petropavlovsk. Med gartner har vi lavet en... \act{leder efter ord} ... fusion...
ja - altså af vores arbejdsområder... rent videnskabeligt.

Vi har kombineret Datalogi med Biologi. Mejt mejt spændende. Det er lykkeds os at
oplaste flere dyr til Cyperspace. Nu jeg forklare:


\says{ID} [russisk accent] Man startede selvfølgeligt med \act{vender
  første side af FO'eren - der er et billede af en mus} mus, og disse
  har vist sig at fortrække eksterne styreenheder som
  habitat. Uheldigvis misforstod en ung forskningsassistent en besked,
  og derved blev et større antal lus \act{vender næste billede på FO -
  lus} også oplastet, hvor de i mangel af naturlige fjender straks
  formerede sig i stort antal.
  
  \says{ID} [russisk accent] Nu ville forskerne prøve andre dyrearter, og valgte
  som næste testobjekt en lille, hurtigt fugl \act{vender FO - billede af en
    Tux/pingvin}. Denne fandt sig strakst tilrette i netværket, og fik derfor
  navnet ping-viner. Efterhånden var musene dog blevet et problem, idet de også
  manglede naturlige fjender. Man oplastede derfor et antal mindre rovdyr. Disse
  skulle være små, genkende byttets bevægelsesmønstre og fremvise resultater.
  Valget faldt naturligt nok på \act{vender FO - billede af kat} my-awk-cat'te.
  
  \says{ID} Disse resultater var opmunterende nok til, at man nu gik videre med
  større dyr. Man havde startet joint venture med den lokale zoologiske have, og
  de havde et afrikansk klovdyr til overs \act{vender FO - billede af Gnu}. Efter
  oplasting trivedes den fint, og blev hurtigt meget anvendelige, hvorfor vi vi
  har startet et sideprojekt der studerer applikationer af GNU.

\says{ID} Der opstod nu et problem med udsultning. Desværre var
teknikken ikke brugbar til små planter, men vi prøvede at oplaste en
gran og et enkelt elmetræ. \act{vender FO- billede af gran og elm, med
'pine' og 'elm' under}. Elm viste sig hurtigt at uddø, desværre.

\says{ID} Men vi havde stadig problem med alle de lus. I samarbejde med engelsk
kollega, prøvede vi at samle alle disse lus sammen, på det han kalder en mutt
\act{vender FO - billede af beskidt hund}. Køteren viste sig meget god til dette.

\says{ID} Sidenhen har vi perfektioneret en mass-oplast teknik,
således at vi har oplastet \act{vender en FO per dyr}

\says{ID} ... Python

\says{ID} ... PERLehøns

\says{ID} ...DIKUFant\act{altså en elefant, ik?}

\says{ID} ... man-driller \act{en abe}

\says{ID} og mange flere. Men nu vi havde et sidste stort problem. Alle disse dyr
var slet ikke til at styre, og det virtuelle økosystem var helt til hest
\act{overvejer kort at oplaste heste}. Vi måtte finde en måde at fjerne alle
disse dyr på. Og nu til sidst har vi fundet en løsning, og vi venter spændt på at
se hvad resultatet bliver af vores oplastning af mange tusinde \act{vender FO -
  billede af undolat} undo-later. Tak for i aften.


\scene{Tæppe}

\end{sketch}
\end{document}

%%% Local Variables: 
%%% mode: latex
%%% TeX-master: t
%%% End: 

