\documentclass[a4paper,11pt]{article}

\usepackage{revy}
\usepackage[utf8]{inputenc}
\usepackage[T1]{fontenc}
\usepackage[danish]{babel}

\revyname{DIKUrevy}
\revyyear{2004}
% HUSK AT OPDATERE VERSIONSNUMMER
\version{1.0}
\eta{ca. 10-11 minutter}
\status{Færdig}

\title{En naturlig forklaring}
\author{marvin, uffe, adhh, mad\ss , jakob, sandberg}

\begin{document}
\maketitle

\begin{roles}
\role{K}[UffeC] Konspirationsteoretiker
\role{I}[Adam] Interviewer
\end{roles}

\begin{props}
  \prop{Evt. lidt gyser-musik fra bandet}
  \prop{2xStol}
  \prop{T-Shirt, for lille (K)}
  \prop{Skovmands skjorte, størrelse pleslie (K)}
  \prop{Sandal (K)}
  \prop{Hvide Tennissokker (K)}
  \prop{Bæltetaske ()}
  \prop{Briller ()}
  \prop{Forvaskede sorte shorts ()}
\end{props}


\scene{I står foran tæppet på scenen.}  
  
\begin{sketch}

  
  \says{I}[Studieværtsagtigt] Med eneret for DIKU Revy kan vi nu bringe en
  afsløring. En fordækt plan har styret dette forårs begivenheder. Det mener i
  hvert fald vores næste gæst, Søren Kummel, som ønsker at være anonym.
  \act{Laver velkomst-fagter mod K}

\act{K kommer ind på scenen og hilser på I. K prøver at smile mod
  publikum, mens han går duknakket og taler sagte, som om han er bange for at
  blive opdaget af ``nogen''.}

\says{K}[Hemmelighedsfuld, bebrejdende] Shhhh! Jeg sagde jo, at jeg ville være
anonym! Man ved aldrig, hvem der lytter med. \act{Kigger mistænktsomt mod salen}

\says{I}[Uforstående] Ja, det sagde jeg da også til publikum? \act{Mod
  publikum} Men her er han så: Den anonyme Søren Kummel!

\act{Mod K} Men du er kommet her idag, fordi du vil afsløre et mistænkeligt
komplot. Hvad kan du fortælle os?

\says{K}[Stadig hemmelighedsfuld, men nu også lidt ivrig efter at fortælle]
Jo, gennem massiv forsking har jeg afdækket en skummel plan.

\act{Lægger an til stor, hemmelig historie} Ser du, det drejer som en
sammensværgelse med rødder tilbage til 1940'erne. Den gang var Danmark besat af
tyskerne, og modstandsbevægelsen var desperate efter at få fat i sprængstoffer.
En fiffig gut finder så ud af, at der ligger masser af udetonerede granater på
det militære øvelsesareal på Amager Fælled. Problemet er blot, hvordan man uset
transporterer sprængstofferne fra Amager Fælled til modstandsbevægelsens
hovedkvarter i Vanløse. \act{Laver punkt-A-til-B-fagter}

Man beslutter sig derfor for at grave en tunnel \act{Laver nedenunder-fagter.
  Generelt laver K så mange fagter som overhovedet muligt} nedenunder København.
Men krigen slutter jo i 1945.

I 1960'erne sker der så det, at en række gamle modstandsfolk sidder i
byudviklingsudvalget. For at ære deres faldne kammarater beslutter de at lave en
underjordisk togbane fra Amager til Vanløse -- den gamle tunnel blev jo aldrig
helt færdig\ldots

Imidlertid laver en emdbedsmand en fatal brøler: Ud for en af banens stationer på
Amager skriver han ved et uheld navnet ``Universitetet''. Fejlen bliver først
opdaget, \emph{efter} at planerne er sendt videre i systemet.

\says{I} Hvorfor laver de det ikke bare om? Dengang lå der jo ikke noget
universitet på Amager?

\says{K}[Nedladende] Du har vist aldrig været offentlig ansat? Næh, når først
den slags er nedfældet på papir, kan ingen jordisk magt ændre det. Hvis
virkeligheden ikke passer med planerne, er det langt lettere at ændre
virkeligheden.

Og det gør man så. Via politiske forbindelser i Folketinget får man hævet optaget
af især humanister på Københavns Universitet. Dermed mangler Københavns
Universitet pludselig plads og ansøger derfor om bevilling til nybyggeri. De får
pengene, men på betingelse af, at de nye bygninger rejses i umiddelbar nærhed af
den planlagte station.

\says{I} Men KUA ligger jo ikke ved Universitet station?

\says{K}[Triumferende] Netop! Byudviklingsplanerne var dybt hemmelige, og derfor
vidste Københavns Universitet ikke præcis, hvor de skulle bygge. Byggeriet var
allerede godt igang da man opdagede, at man havde ramt forkert. Til gengæld
kaldte man så de nye bygninger for KUMUA -- Københavns Universitets
\emph{Midlertidige} Udflytning til Amager. Allerede den gang lovede man at flytte
det senere til den rigtige position.

Man prøver flere gange at få placeret et universitet ved stationen.  Næste forsøg
er nok den største fadæse. En ellers uskyldig fortegnsfejl i et regnestykke
sender det næste projekt helt ud på herrens mark -- nærmere bestemt en pløjemark
ved Roskilde! Efter \emph{den} omgang skal de lige sunde sig nogle år.

Næste tiltag er at plante en horde af brokkehoveder på KUA. De konstruerer
massevis af klager over, at KUA's bygninger er usunde og skaber asociale
studerende. Under dette påskud besluttes det at bygge det nye KUA.

\act{Slår imaginært i bordet (I bliver forskrækket)} BUM! Endnu en brøler.
Humanister stinker til geografi. De flytter i den forkerte retning, og det ny KUA
kommer \emph{ikke} tættere på den planlagte station.

I mellemtiden nærmer metroens åbning sig, og offentligheden begynder at stille
spørgsmål til stationens navn. Man prøver forskellige undskyldninger, skyder
skylden på togenes software osv. -- men intet hjælper. Man bliver
desperate. Noget må gøres. Heldigvis står IT-Universitet og mangler lokaler,
og man får kringlet det sådan, at de skal bygge nyt ved siden af stationen.

\says{I}[Glad, lettet] Ah, jamen så er problemet jo løst. Så er der jo et
universitet ved Universitetet Station.

\says{K}[Højt, I bliver forskrækket] BIIIIP! Fejl! ITU er \emph{ikke} et
universitet! det er bare en højskole, man har omdøbt!  Ikke\ldots godt\ldots nok!

Nu er der panik på. Pressen har fået nys om sagen, og der må omgående findes
en løsning. Man vender sig derfor mod gamle frænder: Københavns Universitet
har to gange fået penge til at løse problemet, men det er stadig ikke
lykkedes. Ved en gallamiddag på Marienborg lader man rektor forstå, at hvis
ikke KU løser problemet effektivt, vil regeringen rejse krav om
tilbagebetaling af pengene.

Dette skaber krisestemning i rektoratet. I forvejen har KU dårlig økonomi
p.gr.a. en lallet, talblind dekan på naturvidenskab. Som straf tvinges han til
at afgive et institut til udflytning til Amager. Han vælger DIKU, fordi han
vil have nedlagt Cafeen? -- der jo mest drives af dataloger.

\says{I}[Trist] Men man kan da ikke sådan bare flytte DIKU til Amager?

\says{K} Nej. Og man ved, at den danske befolkning vil rejse sig i vrede, hvis
den opdager planerne. Man har derfor brug for en plan, der kan holde medierne
beskæftiget. Og her må man trække på eksterne ressourcer:

Via tidligere rektor Ove Nathans venner i Mosaisk Trossamfund tog man kontakt
til Arne Melchior. Hans bror, Michael Melchior, er minister i Israel og har
derigennem stærke kontakter i Mossad.

\says{I}[En anelse skeptisk] Jamen, hvordan fik man Mossad til at hjælpe?

\says{K}[Trækker på skuldrene] En simpel vennestjeneste. Anyway, Mossad er
ikke i tvivl om, hvem der er de rigtige til at skabe forvirring: NASA. De er
vant til den slags.

\says{I} NASA? Jeg troede, det var sådan noget med rumfart?

\says{K} Og hvem fandt på det? JFK! Han brugte hele rumprogrammet til at aflede
opmærksomheden fra interne, amerikanske problemer. Annonceringen om at USA ville
til månen var kun beregnet til at få folk til at glemme Svinebugten. Siden da har
NASA stået bag en masse afledningsmanøvrer: Roswell, Golfkrigen, Tjernobyl, World
Trade Center og Irak-krigen.

\says{I} Hvordan skulle NASA stå bag 9/11?

\says{K}[Bedrevidende] National \emph{Aeronautics} and Space Administration,
ikkesandt?

Nå. Mossad er jo en efterRETningstjeneste kan derfor tvinge NASA til samarbejde
ved at true med at offentliggøre beviser om NASAs allerdybeste hemmelighed,
nemlig at månelandingen var et falsum.

NASA invilliger, men skal bruge en ung, attraktiv kvinde til deres plan. De
kontakter en tidligere ansat, John Donaldson og rekvirerer hans datter. Han
nægter først, trods NASAs trusler mod hans familie. Da hans kone så pludselig
dør under\ldots mystiske omstændigheder, og da NASA genfremsætter truslerne,
bukker han under for presset, og sender sin datter, Mary Donaldson, fra
Tasmanien til Sydney.


Samtidig trækker Københavns Universitet i en række andre tråde. Som bekendt
blev Københavns Universitet grundlagt i 1479 af den danske konge \emph{med
tilladelse fra Paven}. Selv efter reformationen og op i nyere tid er KU
infiltreret af Vatikanet. Gennem Vatikanet lægges pres på tudeprins Henrik,
der som bekendt er efterkommer af gamle tempelriddere. Han skal stille med en
ugift søn.

\says{I}[Protesterende] Nej hør nu! Prins Henrik og tempelriddere?

\says{K} Jo, den er god nok. Der var faktisk en udsendelsesrække om det på TV2
og Discovery for nogle år siden. En fyr ved navn Erling Haagensen beviste sammenhængen
mellem den hellige gral, de bornholmske rundkirker, tempelridderne og prins
Henrik. Den er god nok. Der er også en webside om det.

% For dem, der ikke tror på det: Se
% http://www.thirax.dk/Erling_Haagensen/Erling_2.htm

Det virkelige problem var at få kronprins Frederik til Sydney. Heldigvis er
han sportsinteresseret.

\says{I} Hvad hjælper det?

\says{K} I 1996 skulle OL oprindeligt afholdes i Athen for at fejre 100-året
for de moderne olympiske lege. Imidlertid pressede NASA -- via CIA -- Coca
Cola til at betale for at få OL til Atlanta i 1996. Coca Cola har som bekendt
flere penge end Grækenland. Dermed var den olympiske tidsplan brudt, hvilket
åbnede for, at olympiaden i 2000 kunne afholdes i Sydney.

Og hvor OL er, er Frederik. Derefter var det en smal sag at føre ham og Mary
sammen, hvilket som bekendt førte til bryllup her i maj måned. Og det var dét,
der var planen!

\says{I}[Undrende] Jamen? Var det ikke bare et bryllup? Arrangeret af ugebladene
og sådan?

\says{K}[Tager sig til hovedet] Nej, nej, nej! Det er jo det, de vil have os
til at tro! 

Hele bryllupsidéen gik jo ud på at aflede befolkningens opmærksomhed fra DIKUs
tvungne flytning til Amager. Og ikke mindst håbede man, at DIKU Revyen ville
bruge tid på bryllupet i stedet for flytningen!

\says{I}Jamen, DIKU skal jo ikke flytte før om tidligst en tre--fire år?

\says{K}Neeej, men med {\em så} mange variable i spil, er det altså {\em
lidt} svært at ramme helt præcist...

\says{I} Nåh, sådan. Du har jo ret. Tusind tak, fordi du ville dele dette med
os. Og tak fordi du kom!

\scene{K rejser sig og går ud}

\says{I}[Mod publikum] Og på denne måde fik vi så afsløret den egentlige grund
til DIKUs nye placering -- og al den ballade om brylluppet. Mine damer og
herrer, giv et stort bifald for den anonyme Søren Kummel!

\scene{Tæppe}


\says{I}[Eftertænksomt] Hmmm, egentlig har denne sketch jo handlet om
brylluppet i stedet for DIKUs flytning. Gad vide, hvem der står bag dét?\ldots

\scene{Lys ud. Gerne lidt uhygge-musik fra bandet}

\end{sketch}
\end{document}


%%% Local Variables: 
%%% mode: latex
%%% TeX-master: t
%%% End: 
\typeout{CIA ser dig!}

