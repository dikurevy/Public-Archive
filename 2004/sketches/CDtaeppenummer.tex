\documentclass[a4paper,11pt]{article}

\usepackage{revy}
\usepackage[utf8]{inputenc}
\usepackage[T1]{fontenc}
\usepackage[danish]{babel}

\revyname{DIKUrevy}
\revyyear{2004}
% HUSK AT OPDATERE VERSIONSNUMMER
\version{1.0}
\eta{1 min}
\status{Færdig}

\title{CD tæppenummer}
\author{Marvin, severin}

\begin{document}
\maketitle

\begin{roles}
\role{S} Speaker
\end{roles}

\begin{props}
\prop{oplæsningspapir}
\end{props}

\scene{\#include stdrevyhederne.h foran tæppet.}
  
\begin{sketch}

\scene{Band spiller Jingle}

\says{S} Her er Revyhederne med en ekstraudsendelse.

\says{S}Endnu en CD-ROM er nu faldet for regeringens smagsdommeri.

\says{S}Denne gang er det en oplysnings-CD om binære tal, der er blevet stoppet 
af Ondhedsminister For-Next-Last Løkke Rasmussen.

\says{S}CD'en, der er udarbejdet af foreningen Hex og Samfund, skulle ellers 
have været uddelt i matematiske gymnasieklasser over hele landet. Men et 
opslag om 2er-potenser faldt ministeren så meget for brystet, at han 
egenhændigt stoppede udgivelsen.

\says{S}SF'eren Kamel-bog Karachi har svaret igen igen ved at lægge CD'en ud til 
download på sin hjemmeside.

\says{S}Det har fået Centrumdemokraternes Mini Indstilling til at protestere: 
``Hvis nogen skal lægge CD'er ud, må det være os. Vi er CD og vi er 
ude!'', udtaler hun til revyhederne, og fortsætter\ldots

\says{S}``Rygterne om at CD har nogensomhelst potens er aldeles usande! Problemet er, at
ingen kan tilslutte sig partiets meninger, da Antipiratgruppen har optrappet
kampen mod CD-kopiering!''

\says{S}Godaften

\end{sketch}
\end{document}

%%% Local Variables: 
%%% mode: latex
%%% TeX-master: t
%%% End: 
