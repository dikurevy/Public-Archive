\documentclass[a4paper,11pt]{article}

\usepackage{revy}
\usepackage[utf8]{inputenc}
\usepackage[T1]{fontenc}
\usepackage[danish]{babel}

\revyname{DIKUrevy}
\revyyear{2004}
% HUSK AT OPDATERE VERSIONSNUMMER
\version{1.0}
\eta{4 min}
\status{Færdig}

\title{Sælg DIKU}
\author{Uhd}

\begin{document}
\maketitle

\begin{roles}
\role{S}[Fluff] DIKUfant/sælger
\role{K1}[Heidi] Køber 1
\role{K2}[AllanS] Køber 2
\end{roles}

\begin{props}
\prop{Lyd: Alarm}
\end{props}

\scene{S står på scenen og modtager K1 og K2} 
  
\begin{sketch}

\says{S}[holder døren for K1 og K2] Goddag. Ja kom indenfor, kom
indenfor...\act{vinker dem ind. Alarmen går i gang} Eh, hov. Ja, undskyld, nu
skal jeg... \act{illuderer at slå alarmen fra og går indenfor}. Sådan. Ja det var
blot vores alarm. Den går i gang fra tid til anden uden at der rigtigt er nogen
grund til det. Men, velkommen, d'Herrer til Datalogisk institut, eller DIKU som
det så kækt hedder blandt de studerende. Det er min fornøjelse at fortælle dem
lidt om stedet, som de ønskede før de vil købe det.

\says{K1} Ja, vi vil jo nødigt lave en fejl.

\says{S} Ja, de er også svære at behandle, det er rigtigt. Men lad mig så
forsikrer dem med at vi netop har udskiftet hele vores maskinpark, så der
nu kun er solstråler i terminalrummene.

\says{K2} Hvad? Er de tomme?

\says{S} Nej, nej. Der skam lidt i dem..et skærmkort, et netkort og en smule
elektronik.

\says{K1}[skeptisk] Hvor kan kan det fungere?

\says{S} Ser de, alt det der virker har vi sat ned i kælderen.

\says{K1}[usikker] Øh..aha. Fortsæt endeligt.

\says{S} Jo, nu kan vores studerende langt nemmere arbejde på deres studie, de
få der altså gennemfører.

\says{K2} Uddanner de ikke en masse? Vores tal viser...

\says{S} Ja, men de snyder en smule. De fleste af vores studerende tager et
arbejde istedet, da de kan tjene langt mere der. Og da de allerede har de
evner som erhvervslivet kræver...\act{trækker på skuldrene} Ikke meget vi kan
gøre der, vel?

\says{K1} Hvorfor uddanner de så overhovedet nogen?

\says{S} Jamen, vi skal jo forske! Og vi behøver også instruktorer. Hvem skulle
eller undervise de studerende? Desuden er det meste af vores system
administret af studerende.

\says{K2} Det system som kun består af solstråler..

\says{S} [glad for at der ud til at forstå] Ja, lige netop! Lad mig fortælle om
vores andre faciliteter...

\says{K1}[afbryder] Ja, nu de nævner det. De ser mig noget nedslidt ud.

\says{S} Ja, jeg har just har afleveret en rapport.

\says{K1} om faciliteterne tilstand?

\says{S}[forvirret] Ehh..nej. Om simulering af superskalare arkitekture. Det
tog mig kun tretten år. 

\says{K2}[prøver at fange tråden igen] Hvad så med faciliteterne tilstand?

\says{S} Jo...\act{siger hastigt} det er fordi vi skylder 40 millioner væk. Vi kommer her
til...

\says{K1} HVAD! 40 millioner?!?

\says{S} Eh, ja. Men det er nu ikke udelukkende vores gæld. Vi deler økonomi
med resten af naturvidenskab.

\says{K1}[vred] Jamen, hvorfor i himlens navn dog det?

\says{S}[som om det forklarer alt] Jo, det er af principielle årsager.

\says{K2}[prøver at udrede situationen] Altså en slags kollektivt ansvar?

\says{S} Nej, det er skam samfundets skyld. \act{går videre} Vi kommer nu til
DIKUs kantine, hvor studerende og medarbejdere har mulighed for at købe og lave
mad. Den fungerer efter et frivilligheds princip, hvor enhver er ansvarlig for at
sørge for at betale for sin mad.

\says{K2}[triumferende] Aha! Så er det altså et kollektiv.

\says{S} Ehh..nej. Det er en forening. Som det sidste kommer vi til kælderen. Der
findes mange sjove anekdoter om de rum vi har hernede. Da bygningen engang var
anatomisk institut, har vi f.eks. stadigt både kisterum og dyrestaldene hernede!

\says{K1} Hvadbeha'r! Opbevarer de døde mennesker og opstalder dyr?

\says{S}[forsikrer grinende] Nej, nej. Det er skam lang tid siden. Dyrestaldene
ligger nu og forfalder og kisterummet bruges til deponering af en række
mere eller mindre farlige kemikalier. Vær de blot helt roligt.

\says{K2}[smiler nervøst]

\says{S} Ja mine herrer, det var så slut på turen. Hvad synes de om det?

\says{K2}[hvisker lidt sammen med K1. Siger derefter] Ja, i det mindste er det
langt bedre end fysik! 

\scene{Badum...tschi. Tæppe}


\scene{Tæppe}

\end{sketch}
\end{document}

%%% Local Variables: 
%%% mode: latex
%%% TeX-master: t
%%% End: 
