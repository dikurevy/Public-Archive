\documentclass[a4paper,11pt]{article}

\usepackage{revy}
\usepackage[utf8]{inputenc}
\usepackage[T1]{fontenc}
\usepackage[danish]{babel}


\revyname{DIKUrevy}
\revyyear{1982}
\version{1.0}
\eta{$n$ minutter}
\status{Færdig}

\title{Lidt kærlighed}
\author{MMJ}
\melody{``Ein bisschen Liebe''}

\begin{document}
\maketitle

\begin{roles}
\role{S}[] Sanger 
\end{roles}

\begin{song}
  \sings{S} Fremtiden tegnes i sort perspektiv,
            problemerne truer med skændsel og kiv,
            mistro og had overalt, hvor vi går,
            trængsel og ringe kår.

  \sings{S} Kamp mod hinanden, i slag efter slag,
            ingen vil opgi' en eneste dag.
            Frihed og lighed er floskler i flæng,
            venskab et groft refræn.

  \sings{S} Lidt kærlighed og lidt tolerance.
            Et institut, vi vil gi' en chance.
            Lidt kærlighed og en smule tillid
            og at ku' vedgå hinandens krav.

  \sings{S} Lidt kærlighed og lidt hjertevarme
            i følsomt samspil med dikucharme.
            Lidt kærlighed og en fælles fremtid,
            hvor vi ta'r hensyn til andres tarv.

  \sings{S} Ku' vi mod værne om DIKU i flok?
            Ville gensidig accept være nok,
            at respektere hinandens behov?
            Jamen, så gør det dog!

  \sings{S} Lidt kærlighed og lidt tolerance.
            Et institut, vi vil gi' en chance.
            Lidt kærlighed og en smule tillid
            og at ku' vedgå hinandens krav.

  \sings{S} \act{hæves $\frac{1}{2}$ tone} Lidt kærlighed og lidt hjertevarme
            i følsomt samspil med dikucharme.
            Lidt kærlighed og en fælles fremtid,
            hvor vi ta'r hensyn til andres tarv.

  \sings{S} /: Syng i kor
            en lille sang:
            DIKU tror
            på fred engang! :/

\end{song}

\end{document}

