\documentclass[a4paper,11pt]{article}

\usepackage{revy}
\usepackage[utf8]{inputenc}
\usepackage[T1]{fontenc}
\usepackage[danish]{babel}


\revyname{DIKUrevy}
\revyyear{1982}
% HUSK AT OPDATERE VERSIONSNUMMER
\version{1.0}
\eta{$n$ minutter}
\status{Ikke færdig}

\title{Vejledersketch}
\author{Vilmar, Berit m.m.fl.}

\begin{document}
\maketitle

\begin{roles}
\role{O}[] Opråber
\role{I}[] Institutbestyrer
\role{S1}[] Student \#1 (pige)
\role{S2}[] Student \#2 (ny)
\role{S3}[] Student \#3 (gammel)
\end{roles}

\begin{props}
\prop{Bord}[]
\prop{Talerstol}[]
\prop{DIKU-blad over vejledere}[]
\end{props}


\begin{sketch}

\scene{Et bord i den ene side. S1, S2, S3 sidder ved bordet.
       I den anden side af scenen står en "talerstol".
       Bagtæppet er et DIKU-blad over vejledere.}

\says{S1} Hva er klokken?

\says{S2} \act{kigger på uret} 10 minutter over et, der er 5 minutter endnu til vejlederne blir præsenteret.

\says{S3} Ja, det er også pokkers at man skal gennem det her gedemarked igen.

\says{S1} Har du prøvet at vælge vejleder før?

\says{S3} Ja, ja for 4 år siden, men han har forladt instituttet. Det' skide strengt, man ved ikke hvor man har folk.
          Jeg har kun arbejdet i $3.5$ år på mit speciale: "Tölfram-Böffelholdts sætning om forekomsten af $N$-fasanpunkter
          i uendelige Flåbert-rum", så han kunne da godt være blevet et par år mere, så jeg kunne blive færdig.

\says{S2} Ja, jeg kommer lige fra 1.del, så \ldots

\says{S3} \act{afbryder} Nå, men så bliver det spændende for dig, det her. Der er mange af lærerne
          der aldrig viser sig på 1.del, tag nu \ldots

\says{O} \act{afbryder S3, højttalende og deklamerende fra talerstolen. For hver lærer, der præsenteres skal dennes silhouet
         op på bagtæppet} Velkommen, vekommen til vejlederdagen. Allerførst vil jeg gerne vide \act{henvendt til salen}
         hvor mange vil gerne have en specialevejleder? \act{kunstpause, tæller} Det var temmelig mange, det ser ikke
         godt ud. Vi kan nu og her kun tilbyde $x$ specialevejledere, så I der sidder længere tilbage i salen kan I
         ikke læse bifag eller blive exam.scient. eller sådan noget. \act{vent til roen er faldet over folk} Det var det.
         Så kan vi vist gå over til præsentationen af vejlederne. Først en lassisk model \act{BP}. Ansat her i mange år uden dog
         at have fornyet sig noget videre. Motte: Jeg higer og søger og sorterer og søger.

\says{S3} Ham er der ikke meget grin ved. Jeg havde et kursus hos ham for 2 år siden: Uanvendelige algoritmer fra min litteratur.
          \act{banker sig på tindingen}

\says{O} Hvis der er nogen af jer der bor i Nordsjælland passer denne lærer måske noget bedre \act{JK}. Det er i hvertfald nemmere
         at finde ham deroppe end herinde. Han kan også tilbyde tværfaglig undervisning med et vist medicinsk skær, idet
         han både er doktor og beskæftiger sig med operationsanalyse \ldots

\says{S1} Han er ellers meget god. Jeg fik engang 50 mundtlige timer for at danse med ham til en julefrokost.

\says{S2} \act{Hivende sig i skægget} Hmm.

\says{O} Og nu stedets stærke kort på spindeside, den eneste lektor der har været 1000 gange over Atlanten. \act{ES, ridende på kosteskaft}.
         
\says{S2} \act{undrende} Skal man over Atlanten for at komme til RUC?

\says{O} Hun har just beskæftiget sig med 1.dels-undervisning så det er ikke rigtig til at vide, hvad hun vil lave, men spændende bli'r det.

\says{O} Ja, så har vi Torben Zahle, han er lige vendt tilbage fra det pulvariserende erhvervsliv. Han ligger sikkert inde med mange friske ideer!

\says{S2} Hvem er det?

\says{S3} Tja, jeg så ham for et par år siden. Han ligner en af Dalton-brødrene.

\says{O} Ja og Zahle har lige haft orlov i 2 år, det er jo en stor fordel, da det vil vare mange år inden han kan få det igen.
         Og for udenlandske studerende kan vi tilbyde vejledning i både den danske retskrivning og øh, øh, øh
         \act{vildt rodende i sine papirer} Nå, jeg husker ikke lige hvad det er han plejer at nusse med, men det er
         da sikkert meget spændende. Ja, og videnskabeligt - meget videnskabeligt.

\says{S2} Nej føj. Han tapper alt hvad der bliver sagt på kassette.

\says{S3} Jeg er også dødtræt af at få rettet stavefejl i mine rapporter.

\says{O} Den næste i rækken af traditionelle specialevejledere er Steensgaard-Madsen. Han har jo virkelig meget at tilbyde.
         Det kan jo tit være et problem at finde på et speciale-emne. Noget med gods i. Men her er Steensgaard uovertruffen,
         han kan få et problem ud af hvadsomhelst. Det er ikke alle der kan det.

\says{S1} Jamen kan han også løse dem?

\says{S2} Nej da, det er da derfor han skal ha' studerende.

\says{O} Det var det hele kære venner, instituttet kan desværre ikke tilbyde flere vejledere i år, men prøv igen til næste år
         hvis I skulle trække et nittelod i specialelotteriet.

\says{S2} Nittelod?

\says{S3} Om lidt trækker de lod om vejlederne, og den, der får en vejleder, der har forladt instituttet, har trukket et nittelod.

\says{S1} Hvad er forskellen - med \emph{DET} udvalg i vejledere?

\end{sketch}
\end{document}
