\documentclass[a4paper,11pt]{article}

\usepackage{revy}
\usepackage[utf8]{inputenc}
\usepackage[T1]{fontenc}
\usepackage[danish]{babel}

\revyname{DIKUrevy}
\revyyear{2012}
\version{1.0}
\eta{$2$ minutter}
\status{Færdig}

\title{Fakultetsfunktionen}
\author{Troels, Mikkel, Phillip}
\melody{Animaniacs: ``Nations of the World''}

\begin{document}
\maketitle

\begin{roles}
  \role{S}[Nana] Sanger
  \role{X}[Phillip] Instruktør
\end{roles}

EN GUIDE TIL LÆSEREN:

Sangens rytme baserer sig på en figur, der bliver gentaget igen og igen.

\begin{itemize}
\item 1-2-3  4-5-6
\item 1-2-3  4-5-6
\item 1-2-3  4-5-6-1(-2)
\end{itemize}

Nogle gange synges der på optakten inden 1-slaget. Det er noteret med
kantede parenteser.

[6]-1-2-3  4-5-6

Det hele går pænt stærkt, så det er let at miste overblikket. Husk at
1-slaget altid falder til højre for de kantede parenteser i teksten.

\begin{itemize}
\item "1337" udtales "leet".
\item "HaXe" udtales "hex".
\item "Malbolge" udtales på engelsk, dvs. med 2 stavelser
\item ".NET" udtales "dot net"
\end{itemize}

Følgende ord udtales som alm. ord, og ikke som forkortelser:
ALGOL, AMOS, BASIC, BETA, BLISS, COBOL, COMAL, COMPASS, FLOW-MATIC,
INTERCAL, LOLCODE, MATLAB, MOO, MUMPS, NATURAL, POP-2, REBOL, REFAL, REXX,
SAIL, SNOBOL, UNITY

Ordene AIMMS og MEL skal udtales "A-I-M-M-S" og "M-E-L".


Troels har lovet at skrive fakultets-funktionen i samtlige sprog, og jeg
skal nok smide det sammen som slideshow med musik i en video (ligesom
Lambda-kalkylen sidste år).

Jeg tror at den er umulig at synge live på scenen, men det må selvfølgelig
være op til sangerene/sanginstruktører. Den kan med fordel laves som
playback-sang, eller til nøds som 100\% videomateriale.

\begin{sketch}
  \scene{En underviser (U) kommer ind på
    scenen. Scene-elementer/lærred står klar til fremvisning af
    "slides".}

  \says{U} De studerende klager over at de ikke får erfaring med nok
  forskellige programmeringssprog i løbet af deres tid her på DIKU.

  \says{U} Da revyen sidste år havde success med at gennemgå
  lambdakalkyen, har vi fået til opgave også at rette op på dette hul
  i jeres uddannelse.

  \says{U} Vi gennemgår derfor nu et repræsentativt stykke programmel
  - nemlig fakultetsfunktionen - i et lille udvalg af
  programmeringssprog. Så følg godt med.
\end{sketch}

\begin{song}
  \scene{Musik starter}

  \sings{U}

  [RE]FAL, Lua, Simula,
  Oberon, Modula,
  Processing, Alef, Pascal,
  [Ob]jective-C, CSP,
  Dart, BitC, R, make, D
  CoffeeScript, Ada, COMAL,

  [Mathe]matica, Maxima,
  Eiffel, Go, Rapira,
  Limbo, REBOL, Delphi, Self,
  [B]CPL, Tcl,
  Perl, Erlang, OCaml,
  HyperTalk, APL, Twelf,

  [ID]L, occam, INTERCAL,
  JavaScript, JCL,
  Maple, Lisp, PL/I, sed,
  Prolog, Coq, Logo, Batch,
  Scala, Bash, Fortress, Scratch
  Visual Basic .NET,

  \scene{Vildere musik!}

  \sings{U}
  QtScript, Factor, MOO,
  Dylan, dBase, FoxPro,
  BETA, BLISS, AMOS, MATLAB,
  NXC, Fancy, Squeak,
  XMLmosaic,
  MDL, Malbolge, F\#,

  S, Oriel, Excel,
  FLOW-MATIC, SQL,
  Pico, Prograph, Averest,
  [A]MPL, DataFlex,
  AIMMS, REXX,
  LaTeX, X10, ALGOL 60,

  [J]Script, Janus, Miranda,
  Plankalkül, Unlambda,
  GHC Core, NATURAL,
  [Charm], Clojure, Glagol, Curry,
  Cayenne, Amiga E,
  M4, HTML,

  \scene{Vildere musik!}

  \sings{U}
  Mercury, PostScript, 1337,
  IRP, LOLCODE, Piet,
  Kinesisk BASIC, Haskell,
  [Windows] PowerShell, J, Java,
  Oz, Zeno, Pawn, Pizza,
  Z notation, ZPL,

  [Scheme], Julia, Shen, Homespring
  ATS, Boomerang,
  OmniMark, UNITY, AWK,
  [M]EL, Spice Lisp, Whitespace,
  Vala, Troll, Hope, FailSafe,
  Q, Tea, Joy, Mirah, Smalltalk,

  [Kreds], Epigram, Gödel, K,
  TXL, ksh,
  Cyclone, B, VBA, Python,
  MultiLisp, Cobra, SAIL,
  UnrealScript, CPL,
  ColdFusion, Arc, Agda, Fortran,

  \scene{Vildere musik!}

  \sings{U}

  [Forth], AutoIt, PHP,
  BASIC, Expect, bc,
  STG-kode, SNOBOL,
  ABC, ActionScript,
  Lava, Lisaac, Lucid,
  LilyPond, Octave, COBOL,

  [Brainfuck], Emerald, Unix Shell,
  Babbage, F, Mouse, Squirrel,
  Oak, RPL, Legoscript,
  [Go]diva, Rust, E, Shakespeare,
  Racket, POP-2, Fjölnir,
  AutoHotKey, Emacs Lisp,

  [Sindre], System F, Verilog,
  Alma-0, Datalog,
  Promela, Ratfor, C Shell,
  [d]c, COMPASS, L, Groovy,
  zsh, XQuery,
  Onyx, HaXe, Newspeak, Qi,
  Escher, Aldor, Ruby,
  VBScript, RPG,
  C++, ANSI C,
  rc, C\#, SML!

  \scene{Musik slut.}
\end{song}
\begin{sketch}
  \says{U} Og for at drage nytte af at den nye regering igen har tænkt sig at
  tillade gruppe-eksamination, synes jeg at vi skal tage eksamen med det
  samme. Syng med!

  \scene{Musik start - sprogene gennemgås igen - musik slut.}

  \says{U} [ser skuffet ud] Aah... Vi ses næste år.

  \scene{Lys ud.}
\end{sketch}

\end{document}

