\documentclass[a4paper,11pt]{article}
\usepackage{revy}
\usepackage[utf8]{inputenc}
\usepackage[T1]{fontenc}
\usepackage[danish]{babel}
\revyname{DIKUrevy}
\revyyear{2012}
% HUSK AT OPDATERE VERSIONSNUMMER
\version{1.0}
\eta{$4$ minutter}
\status{Færdig}
\title{Specialeangrebet}
\author{Mikkel, Troels, Sune, Nana}
\begin{document}
\maketitle
\begin{roles}
  \role{S}[Jenny] Studerende
  \role{C}[Ejnar] Censor
  \role{V}[Morten] Vejleder
  \role{X}[Troels] Instruktør
\end{roles}
\begin{props}
  \prop{Diaspræsentation}[Troels]
  \prop{Bord}
  \prop{To stole}
  \prop{Pistol}
  \prop{Papirrod}
\end{props}
\begin{sketch}

\scene{Almindelig eksamensscene, alle tre står på scenen.  På en tavle
  er et dias med teksten ``DIKUrevyen præsenterer:
  Specialeangrebet!''.  Der lydes en fanfare.}

\says{V} Ja, goddag og velkommen til specialeforsvar på DIKU.  Der er
jo dukket flere tilhørere op end sædvanligt, men mon ikke vi kan finde
plads.  Jeg gør opmærksom på at dette er en eksamenssituation, og at I
derfor skal være helt stille undervejs.

\act{Alle hilser på hinanden}

\says{C} Og det er så mig der er censor i dag.  Værsgo, Jenny, du går
bare i gang.  Hvad er det dit speciale handler om?

\says{S} \act{Fnyser hånligt} Det burde du da allerede vide.  Men
nuvel: titlen på mit speciale er ``Almene Teoridannelser om generisk
programmering af numeriske løsninger til sædvanlige \textit{og}
partielle differentialligninger, med specielt henblik på anvendelse af
Standard Fortran 76, og dennes henvisninger til Turings artikel, "Über
die Wesen des Primtalalgorimus des kvadratischer Bubblesort anno 1943"
- bilag F.  Ført, nu og i fremtiden!''.  Jeg har i løbet af mit
arbejde naturligvis også måtte løse en række andre småproblemer.

\says{S} Jeg måtte rette en fejl i et af mine værktøjer...

\scene{Dias: ``Fjerne Python's GIL''}

\says{S} Tilpasse platformene til mit arbejdsmiljø...

\scene{Dias: ``Porte Linux til en Etch-a-Sketch''}

\says{S} Samt undersøge kompleksiteten af mine algoritmer.

\scene{Dias: ``Afgøre om $P=NP$''}

\says{C} \act{Afbryder} Og hvad har du så fundet ud af?

\says{S} \act{Klagende} Sig mig, har du ikke engang læst mit speciale? jeg
har brugt måneder på det!

\act{V ser lidt skræmt ud}

\says{C} Jo selvfølgelig, lad mig lige... \act{C samler specialet op fra
bordet og bladrer i det}

\says{S} Du skulle bare læse et sølle abstrakt!  Har du slet ingen
respekt for den forskning jeg har lavet?!

\says{C} Jo, men vær nu rimelig, jeg synes bare ikke du præsenterer
nok data til at underbygge de konklusioner, som du...

\says{S} \act{Afbryder} Og hvad baserer du så dét på?

\says{C} \act{Forvirret} Dit speciale..?

\says{S} \act{Smiler nedladende} Altså ét datapunkt?  Så synes jeg
snarere, at det er dig, der overfortolker her.

\says{V} Rolig, jeg tror måske vi kommer lidt ud af en tangent,
selvfølgelig er din censor godt inde i dit speciale.

\says{S}[Skipper hurtigt en masse dias] Det kan vi hurtigt få
afgjort!

\act{Dias: En stor tabel med en masse tal}

\says{S}[Peger på et af tallene] Hvad viser dét?

\says{C}[Bladrer i specialet] Jeg erindrer ikke lige den figur...

\says{S} \act{Himler med øjnene og stønner, er meget ophidset} Tallet
på position $i,j$ er LIX-tallet for sætning $i$ på side $j$.  Sig mig,
hvordan kan du have læst mit speciale, og ikke øjeblikkelig genkende
dets struktur på letlæselig form?

\says{C} Jamen...

\says{S} Fra bunden af: Denne matriks tager...\act{afventende}

\says{C} ...tager et kvantitivt syn på...

\says{S}[afbryder] Nej!  Den tager lang tid at skrive i \LaTeX{}!  Har
du slet ingen forståelse for datalogi?

\says{V} \act{Rejser sig og adskiller de to} Jenny, kan du ikke
fortælle os lidt om din arbejdsmetodik?

\says{S}[Samler censorens rettenoter op fra bordet] Faktisk har jeg
selv nogle spørgsmål.  Jeg ser at du ikke har noteret så meget omkring
mit valg af Hackson's algoritme.  Hvordan kan det egentlig være?

\says{C}[Står nu hen mod præsentationen, mens S sidder ved bordet] Den
ser ud da ud som om...

\says{S} Men du tænker ikke på hvorfor jeg ikke henviser til
Peterson's algoritme?

\says{C} Altså, den er jeg sikker på også ville være et fint valg...

\says{S} Den er jo lort!  Peterson er forældet og har dokumenterede
fejl!  Har du slet ikke sat dig ind i litteraturen?  Et bedre valg
ville derimod være...?

\says{C} Markus's reverse?

\says{S} For bagvendt.

\says{C} Schneier's march?

\says{S} For frembrusende.

\says{C} Jyrki's datorkorv \act{bevidst uforståeligt}?

\says{S}[Kigger vantro] For Jyrki.

\says{V} Jeg tror det er bedst at vi går videre, hvad med den konkrete
programmering...

\scene{C gør tegn på at sætte sig tilbage ved bordet, men bliver
  afbrudt af S.}

\says{S} Jeg ser at du har bidt særligt mærke i min brug af SML; kan
du redegøre for hvorfor du finder dét interessant?

\says{C} Altså, det er jo et dødt sprog...

\says{S} FORKERT! ``På DIKU opstanden fra de døde'' - kan du ikke de
Hellige Skrifter?

\scene{V nikker indforstået.}

\says{C} Hvad snakker du om?  Matlab er da noget mere...

\says{S} Hvad fuck er et det for et fucking uciviliseret sprogbrug?

\says{C} Jamen jeg forstår overhovedet ikke hvad der foregår!

\act{S kommer med en lang rant, hvor under C langsomt kryber sammen på gulvet}

\says{S} Og du skulle være min censor?  Din mide, din orm, din
søpølse, du er ikke værdig til at undervise i en børnehave, nej ikke
engang på RUC! Min lillefinger ved mere om datalogi end du nogen sinde
har lært, hvad har du overhovedet af kvalifikationer?!

\says{C} Jeg har været professor i 10 år

\says{S} Jeg vil tørre min røv med dit professorat!  Du kan jo ikke en
gang skrive Hello World i Python!

\says{V} Jeg tror vi har set nok.  Lad os votere.

\act{Alle står og venter i nogle sekunder, hvorefter V vifter C ud,
  der forbløffet forlader scenen.  V og S vender sig mod hinanden og
  nikker indforstået.}

\says{V} Ja, så må du godt komme ind igen.

\act{C kommer ind og ser lidt forvirret ud, S trækker en pistol og
  skyder ham.}

\scene{Nu bliver det besværligt: Lyset dæmpes og scenen rekonfigureres
  (hurtigt!) på en sådan måde at V og C forsvinder, og S sidder og
  sover ved et rodet bord.}

\says{S}[mumler i søvne] 13-skalaen bare for min skyld... jo, jeg
tager bare PhD-graden med... Turing-award...

\says{V}[kommer ind] Så Jenny, kommer du ned til dit specialeforsvar?

\scene{S vågner med et sæt og ser meget forskræmt ud i et par
  sekunder.  Hun Rækker derefter ind under bordet, trækker en pistol
  ud, tager ladegreb, og kigger med beslutsom mine på publikum.}

\scene{Lys ned, tæppe for}

\end{sketch}
\end{document}
