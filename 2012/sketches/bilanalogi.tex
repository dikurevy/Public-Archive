\documentclass[a4paper,11pt]{article}
\usepackage{revy}
\usepackage[utf8]{inputenc}
\usepackage[T1]{fontenc}
\usepackage[danish]{babel}
\revyname{DIKUrevy}\revyyear{2012}
% HUSK AT OPDATERE VERSIONSNUMMER
\version{1.3}
\eta{$3.5$ minutter}
\status{Færdig}
\title{Bilanalogi}
\author{Troels, Phillip}
\begin{document}
\maketitle
\begin{roles}
  \role{D1}[Sune] Datalog 1
  \role{D2}[Nana] Datalog 2
  \role{U1}[Mikkel] Underviser 1
  \role{U2}[Johan] Underviser 2
  \role{X}[Spectrum] Instruktør
\end{roles}

\begin{sketch}
  \scene{To dataloger med /.-t-shirts sidder og snakker.}

  \says{D1} ...men du forstår det jo ikke: Proprietær software er som
   en bil med kølerhjelmen svejset fast! Med fri software kan du kigge
   ned til motoren og lave de ændringer du har lyst til.

  \says{D2} Men så er det jo fuldstændig forrykt, at man bliver tvunget
  til at betale for en Windows-licens, når man køber sin datamat. Det
  svarer til at du køber en Mercedes, og så er nødt til at betale
  overpris for sorte affaldssække som sæde-betræk.

  \scene{FRYS SCENE!}

  \scene{Der kommer 2 smarte undervisere i jakkesæt ind på scenen. Lys går ned på D1 og D2 som kort 
tid efter forlader scenen diskret}

  \says{U1} Ja, Bilanalogien har, som i ser, for længst vist sit værd som pedagoisk
  virkemiddel. Derfor har vi hos Instituttet for Naturfagenes Didaktik
  udviklet en komplet og konsistent bilanalogi for hele datalogien.

  \says{U2} Ser I, en datamat er lidt som en bil. Og når man som
  bruger benytter sig af en datamat, er det ligesom at sætte sig ind i
  en taxa, og bede chaufføren om blive kørt et sted hen.

  \says{U1} Men her uddanner vi jo dataloger. IT branchens jægersoldater \scene{U1 mimer meget fesent en pistol og ``skyder'' ud mod publikum for derefter at puste den imaginære røg væk fra hans imaginære pistol}, så det er jer, der skal sidde
  bag rettet, hvor I kan køre helt nye steder hen - steder der ikke
  nødvendigvis står på ruteplanen.

  \says{U2} For at få lidt respons på vores formidling, vil vi bede
  jer om at give tegn med horn og lyfter, hvis I skulle "stå af"
  (høhø).

  \says{U1}[Ligesom Georg Strøm, men mimer at han trykker på et horn]
  Båt-båt.

%   \says{U2} I har måske hørt, at der findes forskellige
%   programmeringssprog. Det svarer til forskellige typer af
%   styremekanismer i bilen. F.eks. findes der det der hedder
%   "funktionelle" programmeringssprog - det svarer til at forskellige
%   knapper og stænger ikke påvirker hinanden. Du vil således aldrig have
%   en kobling i en funktionel bil, da koblingen påvirker gearstangen.
% 
%   \says{U1} I et imperativt programmeringssprog derimod, kan de
%   forskellige funktioner godt påvirke hinanden. Til gengæld må du kun
%   bruge én af funktionerne ad gangen, ellers bliver det meget svært at
%   gøre korrekt - det er det man kalder parallelprogrammering.

  \says{U2} Når du så tanker din datamat.. øeh, bil, svarer til at du
  sætter strøm til datamaten.

  \says{U1} Og en dieselbil svarer til en datamat, der kun kører på
  jævnstrøm!

  \says{U2} I har måske hørt at der findes forskellige slags
  datamat-arkitekturer, det svarer til forskellige biltyper, såsom
  pickups...

  \says{U1} Som svarer til Itanium!

  \says{U2} ...Sedan'er...

  \says{U1} x86!

  \says{U2} ...station cars...

  \says{U1} amd64!

  \says{U2} ...og ellerter.

  \says{U1} ARM! \scene{Bevæger sin højre arm så hurtigt at han hopper}

  \says{U2} Men med datamater er der faktisk endnu større
  mangfoldighed. Inden for datalogien har vi
  f.eks. klyngedatamater. Det svarer til at man spænder 4 biler
  foran...

  \says{U1} Nej, det er en bil med 4 motorer...

  \says{U2} Nej, det er multikerne - du blander tingene sammen! Det
  svarer til at du skal køre en strækning på 100 km, og så starter du
  med 10 forskellige biler, placeret på strækningen med 10 km
  mellemrum, som så kører 10km hver samtidig!  Og det er jo hurtigere
  end hvis Brian Vinter skulle køre de 100 km i kun én bil!

  \says{U1} Ja, du har ret. Det er faktisk en ganske pædagogisk model, \scene{læner sig ned og prøver at være hyggelig} nu kan vi vist alle sammen være med --- for bilanalogien er jo både komplet og konsistent!

  \says{U2} CPUen er ligesom en cylinder - der kommer strøm, altså
  benzin, ind i den, og tændingsrøret, som er er clocken, det udfører
  så én operation hver gang stemplet bevæger sig... øhe...

  \says{U1} Og akslen er så bussen!

  \says{U2} Ikke at forveksle med almindelige busser, som svarer til
  et flerbrugersystem!

  \says{U1} Nej et flerbrugersystem er car-pooling.

  \says{U2} Nej, car-pooling er multiplayer hotseat.

  \says{U1} Hvem er så ham der den uforståelige taxa-chauffør, som jeg
  altid møder når jeg er fuld midt om natten?

  \says{U2} Nåeh... Det er bare Jyrki.

  \scene{Lys ned.}

\end{sketch}
\end{document}
