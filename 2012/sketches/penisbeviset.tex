\documentclass[a4paper,11pt]{article}
\usepackage{revy}
\usepackage[utf8]{inputenc}
\usepackage[T1]{fontenc}
\usepackage[danish]{babel}

\revyname{DIKUrevy}
\revyyear{2012}
% HUSK AT OPDATERE VERSIONSNUMMER
\version{1.0}
\eta{$0.5$ minut}
% Her skrives et estimat af sangens/sketchens varighed
\status{Færdig}

\title{Penisbeviset}
\author{Søren, Sander, Klaes \& Jenny}

\begin{document}
\maketitle

\begin{roles}
	\role{S}[Taus] Studerende
	\role{E}[Julie] Eksaminator
	\role{X}[NB] Instruktør
\end{roles}

\begin{props}
	\prop{Whiteboard}[Person, der skaffer]
\end{props}

\begin{sketch}

\scene Lys op. 2 personer står på scenen. S står ved whiteboard og har lige færdigskrevet sit bevis. På tavlen står der:\\\\
Sætning:
$$ P(penis) = 1 + 2 + \cdots + penis = \frac{penis(penis + 1)}{2} $$
Bevis:\\
\indent $P(1)$ er sandt, da $1 = \frac{1 \cdot 2}{2}$\\
\indent Antag, at $P(penis)$ er sandt. Så gælder:
$$
\begin{array}{ll}
	1 + 2 + \cdots + penis + penis + 1 & = \frac{penis(penis + 1)}{2} + penis + 1 \\
	& = \frac{(penis + 1)(penis + 2)}{2}
\end{array}
$$
hvilket beviser at $P(penis + 1)$ er sandt - Paa den side der vender vaek, og det samme med $n$ paa den side publikum kan se.

\scene S læser beviset op. Efter et stykke tid afbryder E.
\says{E} Du må sgu undskylde jeg afbryder dig.
\says{S} Det gør ingenting.
\says{E} For det første, det er skide godt. Det er helt rigtigt.
\says{S} Tak.
\says{E} Men jeg tænker, er det ikke lidt underligt at du skriver $n$?
\says{S} Er det mærkeligt jeg skriver n?
\says{E} Ja det synes jeg. $n$ er for upræcist.
\says{S} Det kan jeg godt se. Det har jeg ikke tænkt over.
\says{E} Øhhh, nu får jeg en ide. I stedet for at skrive $n$, kunne du saa skrive penis?
\scene Han vender tavlen og siger
\says{S} Skrive penis i stedet for n?
\says{E} Ja, tag den med penis!
\says{S} Det var ogsaa saadan, at jeg blev bachelor.
\scene Lys ned.

\end{sketch}
\end{document}
