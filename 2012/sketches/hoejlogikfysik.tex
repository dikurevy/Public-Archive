\documentclass[a4paper,11pt]{article}
\usepackage{revy}
\usepackage[utf8]{inputenc}
\usepackage[T1]{fontenc}
\usepackage[danish]{babel}
\revyname{DIKUrevy}
\revyyear{2012}
% HUSK AT OPDATERE VERSIONSNUMMER

\version{0.31}
\eta{$6$ minutter}
\status{Færdig}
\title{Højlogikfysik}
\author{Phillip, Troels, Brainfuck, Nana}

\begin{document}
\maketitle
\begin{roles}
  \role{A}[Johan] Agitør
  \role{X}[Mark] Instruktør
\end{roles}

\begin{props}
  \prop{2 Flûte}
  \prop{Papkasse}
  \prop{Død kat (i papkassen)}
\end{props}

\begin{sketch}

\scene{A kommer ind på scenen med en papkasse og to baguettes.}

\scene{Her begynder rantet. Det er vigtigt at A konstant er vred og
  forarget, og taler HURTIGT men bestemt. Han er virkelig skuffet over
  at folk der kalder sig videnskabsfolk kan være så dumme.}

\says{A} Datalogien er en fornuftig videnskab! Vi løser vores
problemer med grundlæggende snusfornuft. Og helt ærligt, nogle af de
ting de andre videnskaber går og arbejder på er direkte pinlige!

\says{A} Tag for eksempel Teologi. Hvis formålet med Jesus var at
sprede hans budskab over hele verden, hvorfor sendte Gud ham så ikke
på et tidspunkt, hvor de sociale medier var opfundet?  Der er fuldt ud
plads til et bibelvers i en Twitterbesked!

\says{A} Økonomerne ved sgu heller ikke hvad de laver. Hver gang man
ser i medierne at aktierne er gået op, bliver alle så
glade. Men aktiekurserne er jo styret af mennesker: Hvorfor aftaler man
ikke bare at alle aktier er ti gange mere værd end før, så alle bliver
10 gange rigere? Så kan man give nogle penge til de fattige.  Fattigdom
løst. Nobelpris, tak! \act{Rækker hånden ud mod publikum.}

\says{A} Og hvordan vil I bruge jeres elskede ``udbud-og-efterspørgsel''
til at forklare at hylderne i supermarkedet nogle gange er tomme?  Jeg
efterspørger den vare der skulle have stået på den tomme hylde.  HVOR
ER MIN VARE?

\says{A} Så er der jura\ldots Hvis jura-studiet havde noget som helst
substans, hvorfor er der så altid uenighed i retssalen?  Advokater
burde være enige - de har jo læst den samme bog! Du hører da aldrig to
dataloger skændes om hvordan noget skal kodes.

\says{A} Og hvorfor skal det tage medicinerne 10 år at lære hvorfor
folk er syge?  Jeg går bare ind på Netdoktor, så har jeg svaret på 5
minutter. Hvis man googler sine symptomer, og det minder om skørbug,
SÅ ER DET NOK SKØRBUG!  Næste sygdom!

\says{A} Apropos biologi: Hvis evolutionsteorien er sand, hvorfor er der så
stadig aber?  Og hvorfor er der stadig insekter? De er jo så lette at
trampe på. Hvorfor er vi ikke alle meget større og stærkere end vi er?

\says{A} Rollespillere er kendt for at være meget overtroiske, men der
er faktisk noget om snakken: Vi ved jo alle at en terning gennem sit
liv vil slå et bestemt antal 6'ere, så hvis man slår en 6'er med den,
vil den derfor have én færre sekser tilbage, og derved er terningens
kvalitet blevet reduceret.  Ja, selv noget så simpel, er det tydeligt,
at matematikerne er for dumme til at forstå.

\says{A} De siger bl.a. også at $\pi$ ikke kan beskrives med en
brøk. Vås! Vi ved alle sammen at $\pi$ er en cirkels omkreds divideret
med dens diameter. Så du lægger bare en snor med en kendt længde i en
cirkel, og måler dens diameter. Så har du en brøk der præcist
beskriver $\pi$. Værsgo', matematikere. Det var så lidt.

% FYSIK

\says{A} \act{Peger hen på fysikerne} Og fysikerne! Hvis I virkelig
har så svært ved at finde de der partikler, hvorfor laver I så ikke
partikel-acceleratorerne \textit{mindre}? Det er jo umuligt at finde
noget i CERN\ldots

\says{A} Og hvis den endelig skal være så stor, hvorfor så lave den
rund og uden hjørner som man ville kunne jage partiklerne ind i?

\says{A} Fysikerne påstår endvidere, at mængden af energi i universet
er konstant. \act{Suk.} 0) Sæt en lille vandmølle på vandhanen, 1)
Tænd for vandhanen, 2) uendelig energi!  Fysikerne påstår også at
varme er energi, men jeg tænder for en lighter eller skruer op for min
radiator, så bliver der varmere. Ergo har jeg igen tilføjet energi til
universet!

\says{A} Fysikerne er også enormt stolte over deres atom-model, men
den er jo helt ude i skoven\ldots Elektroner, der kredser i perfekte
baner rundt om en positivt ladet kerne? Helt ærligt\ldots Hvordan kan
elektronerne blive ved med at kredse rundt? På et eller andet
tidspunkt går de vel i stå pga. luftmodstanden.

\scene{Fysikerne brokker sig.  A afbryder.}

\says{A} Der er ALTID luft, nogle gange er der bare lidt mindre luft.

\says{A} Og sig mig engang: Når nu universet har 3 dimensioner, hvor
ligger origo så? Eller den har I måske ikke taget jer sammen til at
finde endnu?

\says{A} Og hvorfor har I så svært ved neutrinoer?  Hvis problemet er,
at de er for små, hvorfor sætter I så ikke flere af dem sammen, så de
er lettere at studere? Det virker med mel, som du kan sætte sammen til
et brød. Mel er pænt småt! Er neutrinoer måske mindre end mel?
Neutrinoer + vand og varme, så har man sig et neutrino-flûte.

\scene{Publikum buer.}

\says{A} Ja, man kan ikke lave højenergifysik, uden at der kommer
til at lugte lidt brændt.

\says{A}[A samler de to baguetter op, og vifter med dem] Fuglen der
tabte en baguette i CERN prøvede at give jer et hint!

\says{A} Fæstning af neutrinoerne i en brødmatriks gør også
eksperimenterne meget lettere at udføre! \act{A smadrer de to brød
  sammen, så der kommer neutrinoerne-krummer ud over det hele} Sådan!
Og hvor lang tid var det så lige det tog jer at bygge CERN?

\says{A} Så er der Schrödingers kat. Det er bare en kat i en kasse som
fysikernes ikke kan se. Hvorfor ikke bare banke på kassen og høre
efter om katten er hjemme? Eller lade den stå i en måned. Så er man sgu sikker på at
katten er død. \act{A sparker til papkassen på scenen} Nå, den sagde
altså noget til generalprøven!

\says{A} Og de der neutrinoer der fløj hurtigere end lysets hastighed,
hvad fanden er problemet?  Hvis du flyver med lysets hastighed, så kan
du nemt accelerere dit neutrinoflûte yderligere, sådan her.

\scene{A kaster neutronflûtet.  Lys ned.}

\end{sketch}

\end{document}
