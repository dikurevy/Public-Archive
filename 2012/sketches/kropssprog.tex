\documentclass[a4paper,11pt]{article}
\usepackage{revy}
\usepackage[utf8]{inputenc}
\usepackage[T1]{fontenc}
\usepackage[danish]{babel}

\revyname{DIKUrevy}
\revyyear{2012}
% HUSK AT OPDATERE VERSIONSNUMMER
\version{2.0}
\eta{$3.5$ minutter}
\status{Færdig}
\title{Kropssprog}
\author{Troels, Phillip, Nana}
\begin{document}
\maketitle
\begin{roles}
  \role{F}[Mark] Forsker
  \role{P}[Amanda] Pige
  \role{X}[Spectrum] Instruktør
\end{roles}
\begin{sketch}

\scene{En forsker (F) træder ind på scenen.}

\says{F}[stammer en smule] Vi kender alle folk der dyrker "The Game",
de såkaldte "pickup artists". De er da godt nok klamme og ulækre, og
de fatter det ikke selv! Meeeen: De har faktisk fat i noget af det
rigtige.

\says{F} Kropssprog kan og bør formaliseres, på samme måde som et
programmeringssprog!

\says{F} Ser I, i den menneskelige arkitektur har hvert led en
instruktion, og instruktionens parametre angiver hvordan ledet kan
bøje. F.eks. kunne vi have denne konfiguration: \act{holder armen
  vandret ud med flad hånd}. Så afvikler jeg en instruktion der hedder
"bøj højre arm 45 grader", hvorefter følgende sker:

\scene{F heiler lystigt.  Der går et par sekunder, hvorefter han ser
  overrasket på armen og indsér hvad han har gjort.  Han trækker den
  hutigt til sig.}

\says{F}[fåret] Ja, man skal naturligvis passe på med afvikling af
vilkårlig kode.

\says{F} Men nu opstår så det første problem for en typisk datalog:
For hvis man afvikler instruktionerne sekventielt, så virker det meget
stift og unaturligt.

\says{F} Tag f.eks. et fiktivt eksempel - man skal kysse en pige på
Caféen?: Hvis man først begynder at trutte med munden, \textit{efter}
at man har skubbet den helt op i ansigtet på hende, virker det meget
forceret og forvirrende.

\says{F} Det handler altså om at \textit{parallisere} visse af
kropssprogets instruktioner. Det er ikke blot et spørgsmål om ydelse -
i sidste ende handler det også om semantik.

\says{F} Det kan ikke understreges nok, at vi \textit{ikke} henvender
os til halv-psykopatiske gamere på pigerov. Vi udfører reel,
tværfaglig datalogi!

\says{F} Det sømmer sig for enhver datalog, at konstruere et bibliotek
af software-routiner, for helt ærligt, venner: De fleste dataloger
\textit{har kun} software at arbejde med.

\says{F} En sådan routine kan f.eks. være en fast sekvens af
instruktioner for at række ud efter ens cola, og drikke den. Eller -
for at tage et hypotetisk eksempel - liste en ampul rohypnol ned i en
sød piges drink. \act{F giver et eksempel på sidstnævnte bevægelse.}

\says{F}[kigger sultent på en pige i publikum] I skal huske at have en
objektiv tilgang til de dam-, øh... data I manipulerer med. Kropssprog
handler ligesom programmeringssprog \textit{kun} om datamanipulation,
og om at opnå en succesfuld afvikling \act{støder underlivet frem}.

\says{F}[Opdager hvad han laver, genvinder fatningen, rømmer sig, og
  siger behersket]: Ahøm, grundforskning!

\says{F}[nu mere manisk] Og lad i den sammenhæng være med at fokusere
på noget som helst andet end returværdien! Når du sender beskeder til
en serielport, bekymrer du dig vel heller ikke om hvad den føler?

\says{F} Jeg stod f.eks. engang og udførte subroutinen GNUB på en
piges skulder, mens jeg forsøgte at brute-force mig adgang. Ret
hurtigt så kom systemet dog, og smed en undtagelse - dvs. mig - ud af
Caféen?, selvom at jeg ikke optog væsentligt flere systemressourcer
end andre pers-, øh... processer.

\says{F} Det til trods for, at jeg fulgte den puplicerede API for
piger! Det nævnes ingen steder i The Mystery Method, at man højest kan
knal-, øh... kalde en bestemt metode et endeligt antal gange.

\says{F} Så derfor tilgår jeg nu \textit{kun} implementeringer, som
strengt følger den foreskrevne API.

\scene{En pige (P) kommer ind på scenen. Hun bevæger sig meget stift
  og udfører kun 1 bevægelse ad gangen.}

\says{P}[taler langsomt og robotagtigt] Hej. Må jeg få en gratis
drink?

\says{F} (taler ligeså): Ja. Det ville jeg da være glad for... hvis du
fik.

\scene{P og F gnubber mekanisk og akavet på hinanden.  Lys ud.}

\end{sketch}
\end{document}
