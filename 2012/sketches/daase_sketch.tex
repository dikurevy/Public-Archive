\documentclass[a4paper,11pt]{article}

\usepackage{revy}
\usepackage[utf8]{inputenc}
\usepackage[T1]{fontenc}
\usepackage[danish]{babel}

\revyname{DIKUevy}
\revyyear{2012}
% HUSK AT OPDATERE VERSIONSNUMMER
\version{1.0}
\eta{$1$ minut}
% Her skrives et estimat af sangens/sketchens varighed
\status{Færdig}
% skriv \status{færdig} hvis sketchen er færdig, ellers \status{ideer}

\title{Dåsesketch}
\author{Nana}

\begin{document}
\maketitle

\begin{roles}
\role{P1}[Mikkel] 1. person
\role{P2}[Sune] Rolle 2
\role{P3}[Taus] Rolle 3
\role{P4}[Arinbjörn] Rolle 4
\role{P5}[Caro] Rolle 5
\role{P6}[Morten] Rolle 6
\role{X}[Troels] Instruktør
\end{roles}
%Liste over roller og deres indehavere.

\begin{props}
\prop{Kaffekop}[Troels]
\prop{Pruttelyd}[Troels]
\end{props}
%Liste over rekvisitter. Behold teksten [Person, der skaffer],
%indtil det er sikkert, hvem der skal have ansvaret for rekvisitten

\begin{sketch}

\scene Lys op.
%Brug \scene inden alt scenespil inkl. lys og lyd fra TeXnikken.
Første person kommer ind på scenen, stiller en stol, lader som om han
skal sige noget, går igen.

2. person kommer ind og sætter en kaffetop på stolen. Går igen.

3. person sætter sig på stolen, drikker fra koppen, og går igen.

4. person kommer ind med en gaffel, tager koppen og går mens han
interesseret roder i den.

5. kommer ind og tager stolen med ud.

Alle skuespillere marcherer ind på række og kigger mod publikum.
6. person tager et skridt frem, imiterer at han slår en stor flot bøvs
(men det er faktisk en pruttelyd), og træder tilbage i rækken.  Alle
bukker.  Skuespillerne drejer og marcherer ud igen.  \scene Overtekst:
Husk hvad fortegn kan gøre.  Lys ned

\end{sketch}
\end{document}
