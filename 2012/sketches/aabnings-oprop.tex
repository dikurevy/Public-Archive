\documentclass[a4paper,11pt]{article}

\usepackage{revy}
\usepackage[utf8]{inputenc}
\usepackage[T1]{fontenc}
\usepackage[danish]{babel}

\revyname{DIKUrevy}
\revyyear{2012}
% HUSK AT OPDATERE VERSIONSNUMMER
\version{0.1}
\eta{$2$ minutter}
\status{Færdig}

\title{Åbnings-OPROP}
\author{Troels, Nana, Phillip, Brainfuck, \hspace{3.5cm}Toke}

\begin{document}
\maketitle

\begin{roles}
\role{A}[Simon] Revyt
\role{B}[Brainfuck] Revyboss
\role{X}[Guldfisk] Instruktør
\end{roles}

\begin{props}
\prop{Revymanualen}[]
\prop{Notater til tale}[]
\end{props}

\begin{sketch}

\says{A} Hej. Og velkommen til DIKUrevy 2012. Den \textit{sidste}
revy.

\says{A} I år har vi 8 akter, og pauserne varer 2 minutter.

\says{A} Der sælges øl i alle de pauser der repræsenteres ved et primtal. Da
vi er dataloger, vil vi naturligvis nul-indeksere.

\says{A} Hvis I nu skulle blive sultne undervejs, sidder der gammelt
tyggegummi under de fleste borde og sæder. Husk at dele med
sidemanden.

\says{A} Hvis man vil snakke i telefon, skal I huske at råbe højt nok til at
larmen fra salen bliver overdøvet.

\says{A} I modsætning til tidligere, behøver man ikke frygte at misse
punchlines, da skuespillerne er instrueret til at gentage replikker
der ikke får den ventede respons.

\scene{Venter lidt, og kigger forventende på publikum og gestikulerer
  med armene.}

\says{A}[Langsommere] I modsætning til tidligere, behøver man ikke
frygte at misse punchlines, da skuespillerne er instrueret til at
gentage replikker der ikke får den ventede respons.

\says{A} Bemærk at hvis man synger med på en sang, men synger falsk, vil
bandet stoppe og starte forfra.

\says{A} Og til de af jer, der har medtaget et skydevåben - I må
\textit{ikke} åbne ild under forestillingen.

\says{A} Skulle der alligevel opstå brand, er man selv ansvarlig for at
finde ud af hvor vi i år har placeret nødudgangene, da vi kun således
kan være sikre på at det er de stærkeste og mest snarrådige, der vil
overleve, og videreføre vore stolte studie-traditioner.

\scene{B kommer ind på scenen, med et ringbind under armen.}

\says{B} Hvad fanden er det du laver?

\says{A} Nogen skal da holde en åbningstale!

\says{B} Er du sikker?

\says{A} Jaøh....

\scene{B åbner Den Store RevyManual.}

\scene{Det er et tomt ringbind, med en postit i.}

\scene{B tager postit'en op i hånden og smider ringbindet til side.}

\scene{B studerer begge sider af postit-noten grundigt.}

\says{B} Nope. Man altså ikke have en åbningstale. Det var jo det vi
prøvede at sige til sidste års revy.

\says{A} Men nødudgange og åben ild, og æeh.....

\says{B} Hør nu her. De er alligevel for fulde til at kunne huske det.

\says{B} Smut nu ned og klæd om, du skal på scenen lige om lidt.

\scene{A går ud.}

\scene{B vender sig om til publikum, strækker armene ud, tager nogle
  papirer op af baglommen og rømmer sig.}

\says{B} [trækker vejret ind] ...ja, vi kører!

\scene{B går ud.}

\scene{Lys ud.}

\end{sketch}
\end{document}
