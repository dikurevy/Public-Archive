\documentclass[a4paper,11pt]{article}

\usepackage{revy}
\usepackage[utf8]{inputenc}
\usepackage[T1]{fontenc}
\usepackage[danish]{babel}

\revyname{DIKUrevy}
\revyyear{2012}
\version{0.1}
\eta{$5$ min}
\status{Skal nazificeres (evt. tilføjet fjollet scenespil)}

\title{Vølvens Spådom}
\author{Troels, Phillip, Nana, Andreas, Brainfuck, Ejnar}

\begin{document}
\maketitle

\begin{roles}
\role{O0}[Johan] Oldermand
\role{O1}[Amanda] Oldermand
\role{O2}[Mikkel] Oldermand
\role{S0}[Caro] Skjald/Tingsvidne
\role{S1}[Andreas] Skjald/Tingsvidne
\role{S2}[Morten] Skjald/Tingsvidne
\role{S3}[Taus] Skjald/Tingsvidne
\role{X}[NB] Instruktør
\end{roles}

\begin{props}
  \prop{Økse}[Taus]
  \prop{Ringbrynje}[Taus]
  \prop{3 hvide skæg}[Daisy]
  \prop{3 munkekutter}[Amanda]
  \prop{Kapper/kofter}[Caro]
  \prop{Hjelme}[NB/Caro]
\end{props}

\begin{sketch}

\scene{Året er 1012. Der afholdes domsting. Samtlige medvirkende er
  klædt som blodtørstige vikinger, med våben og ar i ansigtet, og evt.
  trælle ved deres side.}

\says{O0} Vi er samlet til lovdag her på tinge, grundet en stævning
fra Skjaldenes Laug. Den stævnede, munken Snorre Sturlison, har
\textit{krænket} ophavsretten og distribueret mine klienters eddaer
uden gyldig licens!

\says{O1} Det kundgøres hermed, at Snorre Sturlison stævnes in
absentia, da han i øjeblikket opererer fra sit offshore-kloster på
Island.

\says{O0} Indkald tingsvidnerne!

\scene{Sn kommer ind på scenen.}

\says{O0} Tal.

\says{S0} Ja, vi har kigget i vølvernes logfil, og set at denne
``Snorre'' har foretaget uautoriseret ristning af digtet Gylfaginning,
og videredistriburet det vha. P2P-teknologi.

\says{S1} P2P?

\says{O1} Ja, Pen-til-Pergament-teknologi.

\says{O0} Enhver der besidder pen og pergament er under mistanke! Hvad
skal de dog med det, ikke for at kopiere runesten?  Vi må bevare det
pergamentløse samfund!

\says{S2} Undskyld mig, men jeg synes ikke at Skjaldelauget længere
tilgodeser den menige skjalds interesser. Hvorfor insisterer I på
disse rets-togter mod nye distributionsmetoder? Der er jo tydeligvis
et markedsønske!

\says{O1} Vi ser udelukkende de lærde benytte pergament-mediet til
ulovlig videregivelse af kvad og sagn.

\says{O2} Og tænk på alle de arbejdspladser der går tabt, når der ikke
længere skal rejses runesten på hver en bakketop. Mange danere er uden
for plyndrings-sæsonen beskæftigede i disse erhverv, og munkenes
\emph{pirateri} er en trussel mod velfærdsdømmet!

\says{O1} Ja, hvis det bliver ved på denne måde, er vi nødt til at
afskaffe efterrovsordningen.

\says{S0} Men I brokkede jer også dengang folk begyndte at genfortælle
sagaer i langhusene.  Det har da ikke ødelagt edda-industrien?

\says{O0} Efter et langvarigt rets-togt, lykkedes det Skjaldelauget at
vinde hævd på tabsløs gengivelse, og sikre en pligtskyldig forringelse
af kvaliteten ved mundtlig overlevering. Den såkaldte DRM-beskyttelse.

\says{S1} DRM?

\says{O1} Distributions-Restriktion via Mjød. Den påbudte
alkohol-påvirkning under en sagafortælling er nu på mindst 1,5‰, samme
grænse som for at føre langskib. Sagaen forringes, når man
genfortæller den. Folket vil så opsøge de originale runesten.

\says{O2} Ja, disse uautoriserede pergamentkopier, kan kopieres igen
og igen, uden degradering i kvaliteten. På få måner er en munk jo i
stand til at foretage en komplet afskrivning af en krønike.

\says{O0} Og når det bliver \textit{så} let at foretage uautoriseret
distribution, er der jo ingen der gider at digte nye kvad!

\says{O1} Ja, myterne taber værdi, når de bare bliver udbredt til alle
og enhver. Om 500 år vil de være en saga blot!

\says{S1} Men hvad gør vi så?

\says{O2} Lad os stramme op på kongsloven, så sikrer vi os mod
yderligere lovbrud! Strengere dødsstraffe!

\says{O0} Tillige bør vi kun tillade jernbyrd som forsvar for de
pirat-anklagede! For kun det at berøre rødglødende jern, har samme
juridiske relevans, som vort bevisgrundlag.

\says{S2} Vølverne har allerede for vidtrækkende beføjelser til at spå
om almuens gøren og laden. Det er en nidingsdåd mod privatlivets fred!

\says{O1} Tværtimod - Jeg frembyder at vølverne skal koge sejd noget
oftere, og lagre deres spådomme i 2 år, selv uden retslig
kendelse. Lad os beskytte den immaterielle ejendomsret, om det være
sig skjaldekvad, hemmeligheder udi skibsbygning, eller navnene på de
vigtigste ånder.

\says{S3} Hvad?! Denne lovdag hviler på en marsk! Man kan ikke begå
ran mod immateriel ejendom!

\says{O2} Visse-vasse, vi må værne om kulturelle særejer. Vi bør
indføre skarpere kontrol med import af sydlandske illiader og
oddyseer, og der skal pålægges en kopi-afgift på blanke
pergament-medier!

\says{O0} Ja, mere regulering! Vi genbemander grænsekontrollen ved
Dannevirke!

\says{S0} Men hvad med de udenlandske klostre? Det er jo dem der
foretager den uautoriserede kopiering af de danske produkter?

\says{O1} Vi indkalder ledingsflåden, så den kan drage mod klostrene
på antipirat-togt!

\says{O2} Slip patent-troldene løs!!

\scene{O0 trækker sin økse.}

\says{O0} Er der nogen der stemmer imod?

\scene{Ingen siger noget.}

\says{O0} Godt så. Tilbage til lovdagen.

\says{O0} Snorre Sturlison, thi kendes for ting! Du tilkendes afhug af
følgende lemmer: ... [bliver afbrudt]

\says{O1} Holdt!

\says{O2} Stop!

\says{O0} Vent?

\says{O1} I henhold til den nystrammede kongslov om ophavsret, har vi ikke
tilladelse til at gentage ordlyden af en tidligere afsagt dom, uden at
betale en bondegård i afgift.

\says{O2} En stor bondegård! Vi skulle nærmest afstå Skåne.

\says{O1} Vi må finde på noget originalt!

\scene{Oldermænd samles og snakker.}

\says{O0} Snorre, din skæbne følger! Lige indtil Ragnarok skal dine værker
blive afskrevet og lukreeret på. Dit navn vil blive synonym med
kedelige sagaer og ligegyldige kvad. Og hver en lærd yngling skal for
evigt forbande dit navn!

\says{O2} Thi, så kan eder lære det!

\scene{Oldermænd rækker tunge mod publikum.}

\scene{Lys ud}

\end{sketch}
\end{document}
