\documentclass[a4paper,11pt]{article}

\usepackage{revy}
\usepackage[utf8]{inputenc}
\usepackage[T1]{fontenc}
\usepackage[danish]{babel}


\revyname{DIKUrevy}
\revyyear{1974}
% HUSK AT OPDATERE VERSIONSNUMMER
\version{0.1}
\eta{$n$ minutter}
\status{Ikke færdig}

\title{Huset}
\author{?}

\begin{document}
\maketitle

\begin{roles}
\role{S1}[] Studerende 1
\role{S2}[] Studerende 2
\role{O}[] Omstillingen
\role{St1}[] Stemme 1
\role{St1}[] Stemme 2
\role{St1}[] Stemme 3
\role{St1}[] Stemme 4
\role{F}[] Fremmedarbejder
\end{roles}

\begin{sketch}

\scene{Skilt: Sigurdsgade 41}

\scene{Skilt: Omstillingen}

\says{S1} Jeg skulle finde Datalogisk Institut.

\says{O} Det er her.

\says{S1} Hvor finder jeg lokale 00-01-02-007-1315?

\says{O} Det er der ikke noget hedder.  Mener du ikke
00-01-03-007-1315?

\says{S1} Det er muligt.  Jeg skulle tale med Peter Naur.

\says{O}[bedrevidende] Nå, så er det 00-01-02-007-1316.

\says{S1} Hvor er det så?

\says{O} 1. sal.  Ind ad døren til venstre.  Lige igennem biblioteket.
Gennem døren til venstre.  Lige ved siden af lokummet.

\scene{S1 begynder sin vandring.  Ind i {\bf SKILT} elevator, hvor
  han/hun står mens næste scene foregår.}

\says{F} De ha' arbejde jeg få?

\says{O} Nej desværre -- vi laver ikke noget her.

\says{S1}[ud af elevator] Hvor finder jeg nu biblioteket?  Ind ad
døren til venstre.  Her er en jeg kan spørge.  Hvor ligger
biblioteket?

\says{O} Oppe på 1. sal.

\says{S1} 1. sal?  Jamen, er det her ikke 1. sal?

\says{O} Nej, det er stuen.  Du skal op ad trappen.

\scene{S1 går.}

\scene{Skilt: 1. sal}

\says{S1} Hvor ligger nu biblioteket?  Til venstre.  Der er nogle
hylder med hæfter.  Det er nok biblioteket.  Her er et kontor
\act{Banker på}.  Jeg skulle tale med Peter Naur.

\says{St1} Så skal du gennem biblioteket.

\says{S1} Hvor er det?

\says{St1} Lige ud.

\says{S1}[igen på vandring] Nå, {\em her} er biblioteket.  Har de SÅ
mange bøger?  Gennem biblioteket.  Til højre.  Nej --
jo. \act{overrasket} Det er en trappe.  Hvordan var det nu?  Op ad
trappen til venstre.

\scene{Skilt: 2. sal -- kontorlandskab}

\says{S1}[frygtsomt, alene i verden] Er her nogen?

\says{St2}[fra baggrunden -- evt. gennem råber] Hvad leder du efter?

\says{S1} Peter Naur.

\says{St2} Det er nedenunder.  Gennem biblioteket.

\says{S1}[undrende] Foregår al trafik her i huset gennem biblioteket?

\says{St2} Ja.

\says{S1}[ud i den ene side - {\em Skilt: 1 sal} ind i den anden] Her
er biblioteket.  Der har jeg været.  Så må jeg prøve til højre.

\scene{Skilt: Kantinen}

\says{St3}[råbende] Hvem vil være med til en makro?

\says{St4} Det kan du ikke mene.  Vi skal da se, når RECKU's bygning
vælter!

\says{St3} Vi kan jo alligevel ikke se noget allesammen, når der ikke
er panoramavindue.

\says{S1} Det er vist ikke her.

\says{St4} Hvad er klokken?

\says{St3} 2 min. i 5.

\says{St4} Så må vi skynde os.

\scene{Alt disponibelt personale styrter over scenen og er ved at løbe
  den stakkels S1 over ende.}

\says{S1} Hvor mon de skulle hen?  Nå, jeg må finde Peter Naur.
\act{overrasket} Døren er låst. \act{Går forvildet rundt på scenen,
  hvor han møder S2.}

\says{S2} Er du også blevet låst inde?

\says{S1} Ja, åbenbart, hvordan kommer vi ud?

\says{S2} Vi må planke den.  Følg mig.

\scene{Begge ud, mens en wienerstige anbringes over bagtæppet.}

\scene{Begge ind, S2 medbringende en cykel.}

\says{S2} Hjælper du mig lige med min cykel?

\says{S1} Hvordan?

\says{S2} Over porten.

\scene{Begge forsvinder sammen med cyklen over bagtæppet.}

\end{sketch}
\end{document}
