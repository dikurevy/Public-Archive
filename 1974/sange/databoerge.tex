\documentclass[a4paper,11pt]{article}

\usepackage{revy}
\usepackage[utf8]{inputenc}
\usepackage[T1]{fontenc}
\usepackage[danish]{babel}


\revyname{DIKUrevy}
\revyyear{1974}
\version{0.1}
\eta{$n$ minutter}
\status{Ikke faerdig}

\title{Databørge}
\author{?}
\melody{Slikke-Hans}

\begin{document}
\maketitle

\begin{roles}
\role{S}[] Sanger
\end{roles}

\begin{song}
  Har I hørt om Data-Børge?
  Han var fra Kerise
  En af dem man uden a' spørge
  Kalder datagrise
  Databørge, Databørge
  Han skal ha' en vise

  Ned og hulle sit program
  hutig må han være
  vise at han kan sit kram
  så han sig kan blære
  Databørge, Databørge
  Han har nok at lære.

  Børge har det rigtig nød
  glemte alt om orden
  Hulkort over bordet flød
  bag ham står der fjorten
  Databørge, Databørge
  bli' nu ikke knorten.

  Datamaten lød forpint
  da børges kort den æder
  outputtet er ikke fint
  så Børge på det træder
  Databørge, Databørge
  bli'r det aldrig 'bedder'

  Rette fejl, men ej fuldtud
  nærmest som et lotto
  hulkort ind og output ud
  det er Børges motto
  Databørge, Databørge
  Hvem ligner mon hans foto.
\end{song}

\end{document}

