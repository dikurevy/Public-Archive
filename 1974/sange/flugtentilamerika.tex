\documentclass[a4paper,11pt]{article}

\usepackage{revy}
\usepackage[utf8]{inputenc}
\usepackage[T1]{fontenc}
\usepackage[danish]{babel}


\revyname{DIKUrevy}
\revyyear{1974}
\version{0.1}
\eta{$n$ minutter}
\status{Ikke faerdig}

\title{Flugten til Amerika}
\author{pjo}
\melody{Christian Winther: ``Flugten til Amerika''}

\begin{document}
\maketitle

\begin{roles}
\role{S}[] Sanger
\end{roles}

\begin{song}
Dengang da jeg var en ung student
og efter kundskaber tørsted'
min skolegang var netop end
så kom jeg på HCØrsted

Da var jeg ikke så from som nu
da havde jeg ben i panden
mit blod var hedt, og ilter min hu
jeg forstod mig kun lidt på forstanden

En middagstid mellem tolv og et
engang var jeg rigtig i vinden
med næven knyttet og hjertet hedt
jeg stod og med tårer på kinden

Den første prøve i datalogi
gav et nul, da maskinen var nede
det gik ligeså dårligt i geometri
også dette vakte min vrede

Og sekretæren som før var så sød
og lo når jeg gik og hulled'
hun så nu ud som det intet betød
og kiggede blot ned i gulvet

Nej, tænkte jeg da, det er for galt
langt mer' end jeg kan udholde
det skal alle få dyrt betalt
som mig denne tort forvolde

Nu flygter jeg bort til Amerika
og når man så er forlegen
hvor atter man får en Erik fra
Ja - så er Erik af vejen

Forbavset solvejg så op fra sin stil
så hægterne sprang i en kæde
på ansigtet stod et usikkert smil
omtrent mellem le og græde

Ja, hør du Solvejh, du skal gå med
som venner vi bør holde sammen
her er ej længer' vort blivende sted
hist finde vi trøst og gammen

Solvejg så på de nye sko
og glatted' sin bluses folder
hvor lang er vel vejen, skal jeg tro
og tror du skoene holder?

Ja, længer der er til Amerika
end hen til RECKU blandt andet
men for at komme didhen herfra
man sejle må over vandet

Men hvis man så farten overstår
fortryder man aldrig løbet
man får til foræring en herregård
og penge oven i købet

Maskinerne der, de er som en drøm
de er vældigt godt organiseret
tryk på en knap og pist som et søm
det hele er programmeret

Som programmør i USA
man lever som prinser og konger
og hvis systemet går ned en dag
kan man altid - sælge balloner

Enhver kan lave programmer, min ven
så det giver slet ingen penge
men for hver fejl får man penge igen
så man bliver rig inden længe

Og frihed man har endnu dertil
fra morgen til aftens ende
man spiller agurk så længe man vil
og lader cigarerne brænde

Ja så, så Solvejh, så er det et ord
nu er da min rejse besluttet
min appetit er just ikke stor
til at blive på instituttet

En lille stak hulkort jeg henter endnu
de skal gøre godt på vejen
og også dat0-bogen, du
den, ved du jo, er lidt egen!

Med lærebogen i favn hun kom
men kortene de far forsvunden
så stod vi lidt stille og tænkte os om
thi bitter var afskedsstunden

Vort kære rolige datahjem
vi skulle for evigt savne
da åbned Nils Andersen vindu'et på klem
og råte på vores navne

Erik og Solvejh, hvor skal I hen?
Hvad laver I nede i parken
skynd jer nu straks herind igen
hvortil dog den megen snakken?

Betuttede begge to vi stod
vor rejse gik rent i glemme
og uvilkårligt lystred' vor fod
Nils Andersens myndige stemme

Vor vrede nedslog hans faste røst
som jernkuglen Hartmann-fabrikken
Vi glemte vor sorg og fandt vor trøst:
(vi) sov ind under datalogikken
\end{song}

\end{document}

