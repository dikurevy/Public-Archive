\documentclass[a4paper,11pt]{article}
\usepackage{revy}
\usepackage[utf8]{inputenc}

\title{Løkkehjulet}

\author{Mia og Harry}

\version{2} % HUSK AT AJOURFORE UDGAVE-NUMMERET!

\begin{document}

\maketitle

\begin{roles}

\role{vært}
\role{bogstavvender, Miranda} iført Mias lange, 
               lyse paryk og strømpebukser.
\role{3 quiz-deltagere (benævnt Steen, Pia og Kurt)}
               
\role{1 speaker, Dennis}

\end{roles}

\begin{sketch}




\paragraph{Rekvisitter}\strut

\begin{tabular}{ll}
evt.\ 1 & løkkehjul\\
             1 & tavle til ordene der skal gættes\\
               & (kan evt. laves med papskilte, som kan vendes)\\
             3 & hovedpræmierepræsentanter (de runde paphjul monteret\\
               & på en ståltrådsstang, der kan åbnes så man kan se,\\
               & hvilken af hovedpræmierne, der vindes til sidst),\\
             3 & skilte der for hver quiz-deltager viser, hvor mange\\
               & K de har vundet.\\
                &  keyboardet skal have indprogrammeret div. lyde.\\
\end{tabular}

\scene

De 3 quiz-deltagere står side om side i højre side af scenen før
lyset tændes. Når lyset tændes spiller musikken temaet til det
rigtige lykkehjul, og Miranda og værten kommer storsmilende gående
ind på scenen. Miranda stiller sig ved ordtavlen (som bør være pla-
ceret til venstre på scenen), og værten går hen til quiz-deltagerne
og stiller sig lidt til venstre for dem.

Værten henvendt til Miranda:

\says{Værten}      Hvad har du når du står med to grønne kugler i hånden?

\says{Miranda}     Det ved jeg ikke.

\says{Værten}      Kermits fulde opmærksomhed!

\scene Værten henvendt til publikum:

\says{Værten}      Velkommen til løkkehjulet. I aften har vi besøg af
             lærere og studerende fra Datalogisk Institut på
             Københavns Universitet.

\scene Værten præsenterer publikum med en stor gestus.


\says{Værten}      Vi skal også præsentere de 3 quiz-deltagere: Steen,
             Pia og Kurt.

\says{Værten}[Henvendt til Steen]      Steen hvor kommer du fra?

\says{Steen}        Jeg kommer fra København.

\says{Værten}      Hvad beskæftiger du dig så med?

\says{Steen}        Jeg er læser fysik med datalogi som bifag. 
              Det har jeg gjort i snart 8 år.

\says{Værten}      Så må du også have meget travlt. Har du været til
             nogle eksamener i år?

\says{Steen}        Ja, jeg har lige været til eksamen i Fysik 2.

\says{Værten}      Det lyder spændende. Hvad får du ellers tiden til at
             gå med -- altså når du ikke læser til eksamen?

\says{Steen}       Jeg har frivilligt taget et kursus i varmelære; det kan jo 
             være meget nyttigt. 

\scene Værten henvender sig så til Pia, og spørger:

\says{Værten}      Hvad med dig Pia, hvor kommer du fra?

\says{Pia}         Jeg kommer også fra København.

\says{Værten}      Læser du også?

\says{Pia}         Ja, jeg læser datalogi og matematik. I dette
             semester læser jeg 1P, 1M og Mat-2. Til næste semester 
             vil jeg læse Dat-2, Mat-3 og nogle mundtlige andendelskurser
             på datalogi.

\says{Værten}      Hvad får du så resten af tiden til at gå med?

\says{Pia}         Jeg har en del forskellige fritidsbeskæftigelser bl.a. heste,
             og desuden har jeg en del erhvervsarbejde. Endelig har
             jeg tænkt mig at stille op til studienævnsvalget
             næste gang.

\scene Værten henvender sig til den sidste quiz-deltager, Kurt.

\says{Værten}      Kurt, hvor kommer du fra?

\says{Kurt}        Jeg kommer fra København. Jeg har brugt de sidste 13
             år på at læse datalogi. Jeg har siddet i studienævnet
             i 6 år, og har været instruktor på flere af første-
             delskurserne. Jeg er netop blevet kandidat, og er ved
             at søge arbejde.

\says{Værten}      Det lyder spændende, men nu må vi igang med selve
             spillet.

\scene På tavlen markeres et ord på 8 pladser.

\says{Værten}      Vi skal have en kategori. Miranda?

\says{Miranda}      Kategorien er "signed char". Kategorien er "signed
             char".

\scene Første quiz-deltager drejer løkkehjulet. Det lander på 64K (hvor-
dan dette sikres må været et teknisk anliggende).

\says{Steen}        Jeg gætter på 2.

\scene Musikken laver en lyd, der markerer, at gættet var forkert. Turen
går videre til Pia. Pia drejer løkkehjulet. Det lander på 16K.

\says{Pia}         Jeg gætter på 1.

\scene Miranda vender 1-tallerne, så der kommer til at stå 1x1x1x1x. De
ubekendte er vist med x. Pia drejer løkkehjulet igen. Det lander
på 512K.

\says{Pia}         Jeg gætter på 0.

\scene Musikken laver en lyd der markerer, at der ikke er flere cifre at
gætte (en klokke der ringer), og Miranda vender de sidste cifre
så der kommer til at stå 10101010.

\says{Pia}         Så vil jeg gerne have lov til at gætte. Jeg gætter
             på 170.

\scene Musikken laver en lyd, der markerer, at gættet var forkert. Turen
går videre til Kurt.

\says{Kurt}        Så gætter jeg på -86.

\scene Musikken spiller en fanfare, der markerer, at gættet var rigtigt.

\says{Værten}      Nu har du jo ikke fået nogen K, så du får 1K i
             trøstepræmie. Dennis vil fortælle, hvad du kan
             købe.

\says{Dennis}      Du kan vælge mellem et toolkit, der kan genskabe
             slettede filer og kataloger under UNIX, en bog med
             256 muligheder for at få forlænget sin S.U. samt
             en CD med 8 autistiske børn, der reciterer egne
             digte og synger egne sange. Toolkit'et koster 8K, 
             bogen 4K og CD'en 1K.

\says{Kurt}        Da jeg kun har råd til CD'en, køber jeg den.

\says{Værten}      Stillingen er nu: 0 - 0 - 1. Kurt går videre til
             bonusrunden.

\scene Værten giver de to andre quiz-deltagere et ur og giver dem hånden
som symbol på, at de nu er ude af spillet. Derefter træder værten
og Kurt et skridt frem.

\scene På ordtavlen markeres et ord på 8 pladser.

\says{Værten}      Vi skal nu se på aftnens tre hovedgevinster.

             A: Du vinder en SUN Sparc-station med 16Mb RAM,
                1Gb harddisk, farveskærm, mus og styresystem.
                Værdi: 250.000 kr.
             B: Du vinder to års gæstestudier på MIT med rejse
                og ophold betalt. Værdi 370.000 kr.
             C: Du kan få opsat nye gardiner i dit hjem fra
                Gardinland til en værdi af 500.000 kr.

             Du skal nu vælge en af gevinsterne.

\scene Kurt vælger en af papløkkehjulene.

\says{Værten}      Vi skal nu have en kategori.

\says{Miranda}      Kategorien er et ASCII-tegn. Kategorien er et
             ASCII-tegn.

\scene Knud spiller en trommehvirvel. Til at starte med ret lavt.

\says{Værten}      Nu skal du vælge 1 ciffer.

\says{Kurt}        Jeg vælger et 0.

\scene Miranda vender 0'erne på ordtavlen, så der kommer til at stå
0x00000x. Trommehvirvlen tager til i styrke.

\says{Værten}      Nu har du 15 sekunder til at gætte ASCII-tegnet.

\scene Man kan høre tikkelyde, mens Kurt ser ud til at tænke, så det knager.
Publikum vil givetvis forsøge at gætte med. Når tiden er gået
siger værten:

\says{Værten}      Tiden er gået. Kom så med et gæt.

\says{Kurt}        Store-A.

Stor fanfare.

\says{Værten}      Det var rigtigt! Lad os se hvad du har vundet.

\scene Kurt åbner papløkkehjulet og fremviser et C til publikum.

\says{Værten}      Du har vundet gardiner fra Gardinland efter eget
             valg for en værdi af 500.000 kr.

\scene Kurt ser lidt desorienteret ud, mens Miranda kommer hen og giver
ham en stor buket blomster.

\says{Kurt} Det vil søreme pynte på mit ti-kvadratmeters
kollegieværelse!

\scene Lyset slukkes.

\end{sketch}

\end{document}
