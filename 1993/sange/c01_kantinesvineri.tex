\documentclass[a4paper,11pt]{article}
\usepackage{revy}
\usepackage[utf8]{inputenc}

\title{Der er svinet i kantinen}

\author{Bo Arlif}

\version{1.0} % HUSK AT AJOURFORE UDGAVE-NUMMERET!

\begin{document}

\maketitle

\begin{roles}
\role{3 sangere}\strut
\end{roles}


Ole, Robert og mig (Bo) har lavet følgende forslag til et
ekstranummer, da vi som lysfolk ikke ellers kunne få mulighed til at
komme på scenen. Nummeret er selvfølgelig frit, hvis det bedre kan
bruges i anden sammenhæng, eller med andre aktører. Vi ville bare
blive rigtig ultra meget super duper glade, hvis vi måtte få lov til
at fremføre det selv. Det gør også prøver, instruktion og lign. meget
nemmere :-)

\begin{sketch}

\scene
Efter en lang pause efter sidste nummer eller ekstranummer, hører man
følgende samtale over lydanlægget:

\says{Lys} Hej, hva' sker der nu? Ska vi gå hjem eller hva?

\scene
(hviske tiske)

\says{Lys} Okay så prøver jeg intercommen: Scene, Scene, hvad fanden laver I?
     Publikum venter da på ekstra numre.

\scene
(hviske tiske)

\says{Lys} Halllåååååå \dots Nå der er I, hvad sker der egentlig? kommer I snart
     ud? Nej??, jamen publikum råber efter ekstra numre\dots

\scene
(hviske tiske)

\says{Lys} Hvad mener I med ``rend og hop, vi står lige og drikker'', I må sku
     da have forberedt nogle ekstranumre??

\scene
(hviske tiske)

\says{Lys} Vi kan da ikke bare lade Steen sætte nogle plader på, det
er sku da
     for tyndt!

\scene
(hviske tiske)

\says{Lys} Det kan I sgu ikke være bekendt! Jamen - folk er helt vilde! Vi
     bliver flået i småstykker\dots --- Bare fordi I gemmer jer nede bag
     scenen. Vi får da ikke lov til bare at gå.

\scene
(hviske tiske)

\says{Lys} Min hvadfornoget?!

\scene
(hviske tiske)

\says{Lys} Det er vist godt, din mor ikke kan høre dig sige sådan noget!
   Nå, fanden heller, vi kan vel godt selv. Steen, kan vi få lidt lyd?


\scene
De tre lysfolk: Ole, Robert og Bo blænder op med to store spots på
scenen, tager intercom'erne af og går ned på scenen og synger følgende
sang a capella:


\begin{center}
{\large mel: ``Somebodys calling my name''}

Stemmer: Lys, Normal og Bas\\
Rytme: knips, evt. enkelt instrument til
at bakke op: Akustisk guitar, keyboard eller bas.
\end{center}


\begin{verbatim}
L: Det er sørme' noget: Prut-,   , Prut-,
N:                      Prut-,   , Prut-,
B:                           , Bæ,      , Bæ.

L: Kantinen, sviner igen.   -   Sviiiiiner.
N: Kantinen, sviner igen.   -   Sviiiiiner.
B: Kantinen, sviner igen.   -   Sviiiiiner.

L: Kantinen, -den er ret         , AWDR,     ,     , FøJ!
N: Kantinen, -den er ret         ,     , BADR,     , FøJ!
B: Kantinen, -den er ret sle-e-em,     ,     , YRRK, FøJ!

L: Du burde ta' og gøre rent.
N: Du burde ta' og gøre rent. 
B: Du burde ta' og gøre rent.

L: Samle flasker og papir,  -ja vi sir
N: Samle flasker og papir,  -ja vi sir.
B: Samle flasker og papir,

L: Du burde rydde op,
N: Du burde rydde op,
B: Du burde rydde op, - og gøre rent

L: Og duu kan: kalde på din moder, MOAR!, din mor arbejder ikke her
N:             kalde på din moder,      , din mor arbejder ikke her
B:             kalde på din moder,   MEN, din mor arbejder ikke her

L: kaaaaaald på din moooaaaar,MOAR!, din mor_er ikke ansat 
N: kaaaaaald på din moooaaaar,     , din mor_er ikke ansat 
B: kaaaaaald på din moooaaaar, MEN , din mor_er ikke ansat heeeeeer
\end{verbatim}
\pagebreak
\begin{verbatim}
L: (hvisk) kalde på din moder, 
N: (hvisk) kalde på din moder, din mor_er ikke ansat her
B: (hvisk) kalde på din moder, din mor_er ikke ansat her

L: Du ska' rydde af, vaske op
N:                 , vaske op
B:                 , vaske op

L: Du burde rydde op,           reeent.
N: Du burde rydde op,           reeent.
B: Du burde rydde op, - og gøre reeent.
\end{verbatim}

\scene
Derefter gør vi op, blænder ned og blænder op til næste ekstramunner.

\end{sketch}

\end{document}
