\documentclass[a4paper,11pt]{article}
\usepackage{revy}
\usepackage[utf8]{inputenc}

\title{{\sc Pia}}

\author{Henrik Damborg}

\version{1.0} % HUSK AT AJOURFORE UDGAVE-NUMMERET!

\begin{document}

\maketitle


\begin{sketch}

\begin{verse}
Hun kom med et firtog li' fra provinsen\\
Tog man et foto duggede linsen\\
En masse drenge sad i kantinen\\
Blot for at se på {\sc Pia}-studinen.\\
\end{verse}

\begin{verse}
Men {\sc Pia} kom aldrig på {\sc diku} til gilde\\
Hun vidste jo godt hvad drengene ville\\
Noget med {\em top down}\/ og noget med {\em push-pop}\\
Noget med {\em date}\/ og noget med {\em nop-nop}.\\
\end{verse}

\begin{verse}
{\sc Pia} var flittig og kodede livligt\\
Hendes struktur var helt ubeskriv'lig\\
Koden var sexet, ja temmelig lækker\\
For {\sc Pia} var nemlig kun til Rick Decker!\\
\end{verse}

\begin{verse}
{\sc Pia} hun havde en plan der var særlig\\
Havde en grund til at vær' utilnærm'lig\\
``Først vi jeg læse {\sc dat nul} og et bifag\\
og arbejde fuldtid med heste ved sid'n a'\dots\\
\end{verse}

\begin{verse}
Så ta'r jeg {\sc dat et} og {\sc dat to} paå kort tid\\
Og mat'matik, ja det bli'r nok hårdt slid!''\\
(Så) {\sc Pia} smed alle fyre på {\sc porten}\\
Så hun ku' få fred og ro til rapporten.\\
\end{verse}

\begin{verse}
En masse drenge stod der og sukked'\\
De tigged' og bad, men døren var lukket.\\
Det mærk'lige er med han-dataloger\\
De læser en masse men bliver ej kloger'!\\
\end{verse}
\end{sketch}

\end{document}




