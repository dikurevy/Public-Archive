\documentclass[a4paper,11pt]{article}
\usepackage{revy}
\usepackage[utf8]{inputenc}

\begin{document}

\title{Stig's sang}

\author{Null \& Jacob Lorentsen}

\version{1.0} % HUSK AT AJOURFORE UDGAVE-NUMMERET!

\maketitle

\begin{roles}

\role{Skelboe}

\role{Engel}
Engledragt
\role{Djævel}
Djævlekostume samt jakkesæt og Rolex
\role{Sekretær}
\end{roles}

\begin{sketch}

\scene
Stig sidder midt på scenen bag sit bord. Djævel og Engel på
hver sin side.

\begin{center}
{\bf Melodi : Krig og Fred (Shu Bi Dua)}
\end{center}

Engel synger :

Djævel synger =

$<$regibemærkninger, mest til Stig$>$

{\rm 
\begin{verse}
: Det' ikke all' og enhver\\
: der ka' bli' professor her, $<$sekretær ind med kaffe$>$\\
: men Stig han sidder på tronen $<$Klap stolen$>$\\
: Hvis han ka la vær $<$Ræk ud efter sekretærs numse$>$\\
: at voldta' sekretær'n $<$ Grib hånden (a la Dr. Strangelove) $>$\\
: har han sikret pensionen $<$ Klap tegnebog?$>$\\
= Skelbo, Skelbo $<$Djævel: henvendt til Stig, "hviske i øret"$>$\\
= vil du vel bo\\
= i Ishøj eller Albertslund\\
= kom til klatten $<$"Kom her" finger$>$\\
= snyd i skatten\\
= køb et hus ved øresund\\
= Kom ud til os i det private\\
= Jobbet er dit, hvis du vil ha' det\\
\end{verse}

\begin{verse}
= Stort kontor\\
= skrivebord\\
= frokostpause og egen bil\\
= Jakkesæt $<$Djevel: Tag i jakkesæt$>$\\
= Pærelet\\
= Et Rolex i den dyre stil $<$Djævel: Vis ur frem$>$\\
\end{verse}

\begin{verse}
: Da næste forår kom\\
: var Stigs kontorstol tom.\\
: Han var stået på ræset.\\
: Og nu sku' han til\\
: At styre UNI C\\
: Stirre Mammon i fjæset\\
= Hr professor\\
= 68'er\\
= Vil du være millionær\\
= Tag nu chancen\\
= Scor avancen\\
= Vis at du er lønnen værd\\
= Men hvis du ikke laver penge\\
= Bliver du nok ikke længe\\
\end{verse}

\begin{verse}
= Stort kontor\\
= skrivebord\\
= frokostpause og egen bil\\
= Jakkesæt $<$Djævel: Tag i jakkesæt$>$\\
= Pærelet\\
= Et Rolex i den dyre stil $<$Djævel: Vis ur frem$>$\\
\end{verse}

\begin{verse}
: Men på et rigtigt job,\\
: der kan man siges op,\\
: sparkes og blive fyret.\\
: Din orlov fra DIKU\\
: Er tidsbestemt, ved du\\
: Nye kræfter bli'r hyret\\
: Tænk på Dorte\\
: Hun er borte\\
: snydt for sit professorat\\
: Pas på jobbet\\
: Bli'r du droppet?\\
: Vores dør den smækker snart.\\
: Men hvis du skynder dig tilbage\\
: bli'r det lissom gamle dage\\
\end{verse}

\begin{verse}
: Tavlelak\\
: hyggesnak\\
: Kaffepauser og kryds-og-tværs\\
: Ro og fred\\
: Sikkerhed\\
: Pensionen, når du bli'r halvfjerds.\\
\end{verse}
}
\end{sketch}

\end{document}
