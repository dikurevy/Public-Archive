\documentclass[a4paper,11pt]{article}

\usepackage{revy}
\usepackage[utf8]{inputenc}
\usepackage[T1]{fontenc}
\usepackage[danish]{babel}


\revyname{DIKUrevy}
\revyyear{1985}
\version{1.0}
\eta{$n$ minutter}
\status{Færdig}

\title{Kurts genvordigheder}
\author{JTP}

\begin{document}
\maketitle

\begin{roles}
\role{D}[KB eller KM] Dvärg
\role{L}[JTP] Lærer
\role{S}[LOL] Søster
\role{K}[KHH] Kurt
\role{M}[TL] Mor
\end{roles}

\begin{props}
\prop{Terminal til VAXen}[]
\prop{Potteplante}[]
\prop{Et par hundrede farvestrålende brochurer fra banker og sparekasser, der
  lover guld og grønne skove}[]
\end{props}


\begin{sketch}

\says{L} Ja, Kurt. Her er den så. Den er din for i aften og i nat. Gå nu til
den,; knus den, kram den, brug den og -- hvad var det jeg ville sige
... \act{tager en edb-bruchur frem}

\says{K} \act{går imens hen til planten, stikker en finger i jorden og kommer
  tilbage} Ja, den er da meget pæn, men tror du den kan klare den behandling?

\says{L} Ja, konstruktøren har skam tænkt på den side af sagen.

\says{K} Skægt, Jeg vidste ikke I var religiøse herinde. Nå, men det er da
rigtig nok, at sådan nogle kan tåle lidt af hvert, men den trænger nu alligevel
til noget vand.

\says{L} Vand! Jeg tror nu ikke sådan nogle skal have vand Kurt. \act{bladrer
  febrilsk i sin brochure} De unge ved da også alt. Hvor har de det fra? Det er
snart ikke til at følge med ... næh, det ar vist ikke det. ...

\says{K} \act{har i mellemtiden hentet planten} Du an selve se her. Jorden er
helt tør.

\says{L} Hvad i alverden laver du med den plante \act{tager den ud af hånden på
  Kurt og smider den på gulvet}. Naturen, det billige skidt. \act{går hen til
  terminalen og peger} Her, Kurt!

\says{K} Nåh, er det den du mener. Ja, jeg ved ikke rigtig. Det er vist ikke
lige mig.

\scene{L's sang * (forspil sammen med K's sidste replik. Kor = M + S)}

\says{L} Nå, Kurt. Er den feset ind på lystavlen. Lad mig nu høre med dine egne
ord.

\scene{K's sang * (forspil videre fra L's sang)}

\says{L} Udemærket Kurt. Du skal hurtigte komme efter det. Men du får lige
denneher påmindelse:

\scene{L's omkvæd * (K's sang direkte over i dette)}

\says{L} Og med disse ord vil jeg ønske er en god nat.

\scene{L, M og S går ud}

\says{K} \act{tager en af brochurerne; læser lidt} Ja, ja. Tryk på knappen, så
er man næsten på vej sydpå. \act{sætter sig hen ved terminalen} Men der jo flere
knapper! \act{lille pause} Så er det nok den, der står S på, man skal bruge,
hvis man vil sydpå. Og den med J er nok, hvis man vil til
Jurabjergene. \act{rejser sig igen. kigger i andre brochurer} Her kan man få et
nyt hus. Men der var vist også en knap med H. \act{sætter sig hen igen} Ih, hvor
er der altså mange muligheder. Det er jo slet ikke til at vælge.

\scene{L's omkvæd * (hvor publikum måske synger)}

\says{K} Ja, ja. Jeg skal nok. Nu lukker jeg øjnene, tæller til tre, og så
trykker jeg. \act{som sagt, så gjort. Bagtæppe viser blinkende lamper, så man
  kan se, at maskinen kører. Kurt tager langsomt hånden fra øjnene og kigger sig
  forsigtigt omkring}

Hvad er der sket? Hvor er jeg? \act{lille pause} Hm. Der jo ikke sket en skid,
og jeg er her, hvor jeg hele tiden har været. \act{Så pludeselig! Lamperne
  holder op med at blinke, et klimpt fra musikken, D kommer til syne ovenpå
  terminalen}

Hov! Hvor kommer du fra. Du ser ellers noget medtaget ud. \act{i hvert fald når
  stykket har været øvet et par gange} Hvad vil du her? Hvem er du?

\scene{D's sang *}

\says{K} \act{giver D hånden} Det skal jeg nok.

\scene{K's anden sang *}

\scene{L's omkvæd * (i direkte fortsættelse fra K's)}

\says{K} Hør. Der kommer nogen. Du må gemme dig. \act{gemmer D bag nogle af
  brochurerne}

\scene{L ind}

\says{L} Hvad er den af Kurt. Står du der endnu? Har du slet ikke prøvet
maskinen?

\says{K} Jo.

\says{L} Nå, hvordan var det så? Var det godt? Var der gang i den? -- Du vil de
vel ikke fortælle mig, at du ikke ku' få den op at stå?

\says{K (eller publikum)} For meget.

\says{K} Det er da slet ikke det, det handler om. Jeg bliver helt isoleret af at
være her.

\says{L} Det er heldigt for dig, for jeg bliver helt elektrisk af at se dig stå
og hænge der. Ærligt talt Kurt, se dog at komme lidt op i omdrejninger. Man
skulle tro, du ikke havde spillet et ærligt computerspil en hel uge.

\says{K} Nej! Jeg må have tid til at tænke mig om.

\says{L} Tænke! Nu vil han min sandten til at tænke selv. Jamen, det findes der
da også programmer, der kan -- Godt og vel endda.

\says{K} Så hjælp mig dog!

\scene{* D's omkvæd * (Kurt komer til at skubbe til brochurerne, så D kommer
  frem igen)}

\says{L} Pas på Kurt! Lige bag ved dig. \act{slår ud efter D med hånden} Flyt
dig lidt! Den kan være farlig. Kom så væk med dig \act{tager sin sko af og
  pander D en}

\scene{musikken har været crescendo, men standser tumultisk idet D slås ihjel}

\says{L} Gisp, støn. Det var for meget Kurt. Åh! \act{tager sig til hjertet,
  falder om på gulvet} Hjælp Kurt, hjælp!

\says{K} DU kan få maskinen til at hjælpe dig.

\says{L} Det er jo dig og maskinen sammen. Vær nu rar Kurt. Brug din fantasi.

\says{K} Den har du jo lige slået ihjel.

\says{L} Sådan en helt fri fantasi er også alt for farlig. Men de findes da i
mere begrænsede udgaver.

\says{K} NU må jeg først have begravet min ven \act{tager planten fra gulvet og
  stiller ovenpå D}. Men jeg så for øvrigt før, at der vare et
lægeprogram. \act{sætter sig ved terminalen} Det må være, når man trykker på L.

\says{L} Skynd dig Kurt.

\says{K} Her har vi det. Indtast patientens navn. Hvad hedder du?

\says{L} \act{sætter sig op} Skulle det kunne hjælpe noget.

\says{K} Jeg ved ikke, hvad du synes, men jeg kan ikke bruge den vaccine til
noget. Den er jo det rene gift.

\says{S} \act{ind} Gift! Er I blevet gift? Kan man det?

\says{L} Det må du da bedst vide. Var det ikke datajura, du ville læse?

\says{S} Nej, det var arkæologi.

\says{L} Nå, så kan du måske grave terminalen frem fra Kurts fantasi.

\says{S} Nej, det må være en gartner, vi skal have fat i. Moar!

\says{M} \act{ind} Jeg har jo sagt, at det der med maskiner, har eg aldrig
kunnet finde ud af.

\says{K} Kan I da slet ikke tage nogen ting alvorligt?!

\says{L} \act{rejser sig} Det er dig, der er helt gal på den. Alle er med på
edb, men du \underline{vil} jo ikke høre.

\says{S} Vi kunne da lige prøve én gang til.

\scene{* L's omkvæd * (L, M og S synger. K protesterer: Hej stop. Vent.)}

\scene{afhængig af publikums reaktion, er der nu to fortsættelser: if ... then
  ... else you know}

\scene{1. Kurts protester overhøres.}

\scene{Kurt går hovedrystende ud}

\says{L} Så slap vi da af med den humanistspire.

\says{M} Ja, jeg beklager meget på familiens vegne, men nu skal I til gengæld få
en rigtig datalog at høre.

\scene{2. Kurts protester vinder gehør hos publikum.}

\scene{som altså ikke synger med på omkvædet}

\says{L} Hold kæft, hvor er I hellige. Tror I ikke hellere I skulle holde op med
at læse på DIKU. \act{publikum: neej.} Nå, ikke. Så tag lige og tænk over, hvad
I faktisk er ved at uddanne jer til. Nu skal I få en rigtig datalog at høre.

\end{sketch}

\begin{song}
\scene{Lærerens sang:}

\scene{Forspil: g a h c d e f e d c h a g e g a Am}

\sings{L}%
Ka' det virk'lig vær' så svært,
hvad har du egentlig lært,
\sings{K} hvad har du egentlig lært
\sings{L}%
i skolen, sku' jeg mene,
man lærer om edbe.
Har du glemt, hvad du sku' ku'?
så får du chancen nu,
\sings{K} så får du chancen nu
\sings{L}%
til at få det lært igen,
så bare sæt dig hen,
og tag så og tryk på knappen Kurt!

\sings{L}[omkvæd] (gentages)%
TRYK, TRYK, på knappen Kurt,
jah, tryk på knappen Kurt.

\sings{L}[mellemspil]%
Det virker faktisk nærmest helt utroligt,
at man kan leve uden edb.
Men selv om det engang har været muligt,
er det for altid slut,
nu skal du bare få at se.

\sings{L}%
Det er slet ikke svært,
du får det hurtigt lært;
\sings{K} du får det hurtigt lært
\sings{L}%
i dag der er da næsten
ethvert barn født med edb.
Her er alt, hvad du skal bru',
her er din fremtid nu.
\sings{K} her er din fremtid nu.
\sings{L}%
Det er ingen sag min ven,
så bare sæt dig hen,
og tag så og tryk på knappen KUrt!
\sings{L}[omkvæd] (gentages)%
TRYK, TRYK, på knappen Kurt,
jah, tryk på knappen Kurt.
\sings{L}[mellemspil]%
At klare sig igennem dagligdagen
helt uden edb må være surt.
Men VAX'en er din redning, den er sagen,
så bare sæt dig hen,
og tag så og tryk på knappen Kurt.
\sings{L}[omkvæd] (gentages)%
TRYK, TRYK, på knappen Kurt,
jah, tryk på knappen Kurt.
\end{song}

\begin{song}
\scene{Kurts sang:}

\scene{Forspil: a h C D E F G Am}

\sings{K}%
Den er go', den er go' min maskine,
den er go', den er go', ja den er go'
Jeg vil tro, jeg bli'r glad for min maskine,
den kan planlægge en fremtid for os to.

\sings{K}[omkvæd]%
Den kan ta' mig vidt omkring,
lære mig en masse ting.
Den er go', den er go' min maskine,
den er go', den er go', jah den er go'.

\sings{K}%
Den kan gi' mig mere fritid, min maskine,
den kan regne alle opgaver for mig.
Så vi hygger sammen mig og min maskine
med copmuterspil og anden munter leg.

\sings{K}%
Jeg er blevet ganske bit af min maskine,
uden hende bliver dagen sort som kul.
Jeg kan ikke leve uden min VAXine,
nu bli'r vi 1 -- før da var jeg kun et 0.
\end{song}

\begin{song}
\scene{Dvärgens sang:}

\scene{Forspil: e-f trille}

\sings{D}%
Hør nu Kurt, jeg er kommet langvejs fra for at møde dig;
først med databussen, det var ingen leg.
Så fanget ind af skærmens lys, men nu er jeg her;
og nu vil du vide, hvem jeg er.

\sings{D}[omkvæd]%
Jeg er din drøm, jeg er din drøm om frihed nu,
og jeg håber du vil ta' mig til dig.
Jeg er din drøm, jeg er din drøm om frihed nu;
men husk Kurt -- ingenting kommer af sig selv.

\sings{D}%
Hør nu Kurt, du bli'r isoleret her ved dit tastatur;
drages du ej læng're mod den fri natur?
Tænk dig om; du bli'r svinebundet til dit maskineri.
Riv dig løs -- brug din fantasi.
\end{song}

\begin{song}
\scene{Kurts 2. sang:}

\sings{K}%
Er den go', hvad skal jeg tro om min maskine,
er det virk'ligt eller er det fantasi?
For det lille under her på min maskine,
har da fået mig til at tvivle indeni.

\sings{K}[omkvæd]%
Kan den ta' mig vidt omkring,
lære mig en masse ting?
Er den go', er den go' min maskine,
er den go', er den go', åh er den go'?

\sings{K}%
Bli'r jeg isoleret her ved min maskine,
har jeg ændret mig fra dengang da jeg kom?
Kan jeg overleve uden min vaccine;
ja, det er vist bedre, jeg får tænkt mig om.
\end{song}
\end{document}
