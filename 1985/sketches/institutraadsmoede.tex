\documentclass[a4paper,11pt]{article}

\usepackage{revy}
\usepackage[utf8]{inputenc}
\usepackage[T1]{fontenc}
\usepackage[danish]{babel}


\revyname{DIKUrevy}
\revyyear{1985}
\version{1.0}
\eta{$n$ minutter}
\status{Færdig}

\title{Institutrådsmøde}
\author{JC}

\begin{document}
\maketitle

\begin{roles}
\role{O}[JC] Oplæser

% Er med i oversigt over numre men ikke i beskrivelsen af nummeret.
%\role{A}[HVO] Andet
%\role{F}[LGJ] Far
\end{roles}


\begin{sketch}

\says{O} På lokalnettets underholdningskanal bringer vi nu en direkte
transmission fra den store månedlige mediebegivenhed på DIKU --
institutrådsmødet. Vi stiller om til vores udsendte medarbejder.

\says{Nogen} Ja, velkommen til seerne her fra UP1's store auditorium, hvor
underholdningen for ganske få år siden bød på anderledes blodige begivenheder --
ak ja, nu må vi nøjes med sparekniven. Med mig har jeg en af DIKU's tidligere
medarbejdere, der har lovet at bistå med sin ekspertise.

Tilskuerpladserne er tæt pakket her lige før mødets start, de enkelte gruppers
tilhængere har placeret sig, så chancen for kedelige voldsepisoder synes mindst
mulig -- det tavse flertal har f.eks. slet ikke placeret sig.

Og lad mig kort resumere reglerne, hvis nogen skulle have glemt dem: Man skal
trække vejret mindst een gang i minuttet, brug af tale og våben er kun tilladt
efter håndsoprækning -- og nu kommer rådets medlemmer ind med bestyreren i
spidsen til tonerne af ``semaforernes indtogsmarch''.

-- og klokken ringer -- første halvdel af mødet r i gang. Ordet gives op af
bestyreren, der spiller bolden over til studienævnsformanden, der er foreslået
som dirigent, uha uha, er der allerede optræk til aftalt spil ??? Dirigenten
spiller ordet tilbage til bestyreren, der holder på det mens han trækker
beretnngen i langdrag -- urimeligt -- men med en kvik afslutning lægger han godt
op til efterfølgende spørgsmål. Der kommer et spørgsmål til bestyrelsen fra den
billige langside -- jeg kan ikke se om det er VIP eller stud.  Næstformanden
snor sig elegant ud af off-side-fælden men nedlægges ureglementeret af
studentergruppens sweeper, som afbryder debatten og snakker langt uden om sagen
-- men dirigenten og situationen er reddet -- beretningen er godkendt.

Mødet går nu ind i en lidt sløv fase, hvor der berettes fra diverse organer og
udvalg -- ordet går til lokalefordelingsudvalget, der lynhurtigt starter på
langsomt at gennemgå det sidste referat fra udvalgets forrige møde -- og der er
dømt ævl, -- dirigenten har dømt ævl 2--3 minutter inde i indlægget, helt
korrekt dømt. Dirigenten ser på uret -- han har lagt tid til pga næstformandens
uheld -- der er tid til et enkelt dagsordenspunkt inden pausen.

Og det bliver punktet om arbejdsfordeling -- sikken en afslutning på
halvlegen. Peter tager ordet -- et fint oplæg hen over midten, der er fint
fodslag -- men tøver Peter -- altså ikke den Peter -- han får lang line at løbe
på... løber han den helt ud -- nej nej nej, der er obstruktion, kedeligt at se
den slags midler taget i brug, men dirigenten har grebet ind og idømt Peter 3 år
på Dat 0, helt rigtigt set og dømt, instituttet kan ikke være tjent med den
slags -- og nu ringer klokken -- der er 12,75 minutters pause.

Og lektor Ørjan Mensgaard Stadsen -- genkender du noget fra din egen tid i
rådet? ``ja*'' synes du ikke dirigenten klarer hvervet udmærket? ``nej*'' --
Føler du ikke trang til at vende tilbage: ``Nej*'' -- Du har jo mange gode år
tilbage i dig endnu: ``Ja, netop derfor''. -- Ja, tak fordi du kom, Ørjan.

-- og klokken ringer nu for en spændene anden mødehalvdel, der traditionelt i
tid kun er to tredjedel af første halvdel, men hvor begivenhederne til gengæld
er så meget mere dramatiske. Studenterfløjen starter med et godt indlæg fra
venstrefløjen til punktet om VIP-stillingsopslag, men bestyrelsen har i pausen
fået foretaget en udskiftning så punktet glider ud og erstattes af pnutkte om
visitkort til VIP-staben -- skal det være guld -- eller sølvtryk ?? Men
studentergruppen protesterer -- dirigenten løber ud til referenten, der har
markeret ude ved sidilinien -- oj, oj det ser dramatisk ud -- og ja --
bestyrelsens udskiftinng underkendes -- og der er dømt afstemning om
stillingsopslaget -- alle de stemmeberettigede styrter frem med ændringsforslag
til opslaget -- der opstår klumpspil, de enkelet grupper er tilsyneladende ikke
helt klar over, hvor deres mål og modstanderen er*** Og nu, nu stemmes der -- en
skov af arme; dirigenten tæller stemmerne: 13 for, 13 imod og 13, der undlader
at stemme -- men nej, nej, nej, rådets 38 medlemmer kan ikke have 39
stemmer. Tilsukerne raser, der kastes disketter, print og ILWO-snask ind på
arenaen.

Dirigenten ser på sit ur -- og nu, nu lyder klokken, punktets afgørelse ved
afstemnig må udskydes til et nyt ekstraordinært møde om 14 dage.

Og med disse billeder af stærkt utilfredse tilskuere, der forsøger at nå frem
til deres idoler, slutter vi transmissionen -- vi bringer naturligvis en
reportage om det ekstraordinære møde om 2 uger.

\end{sketch}
\end{document}
