\documentclass[a4paper,11pt]{article}

\usepackage{revy}
\usepackage[utf8]{inputenc}
\usepackage[T1]{fontenc}
\usepackage[danish]{babel}


\revyname{DIKUrevy}
\revyyear{1985}
\version{1.0}
\eta{$n$ minutter}
\status{Færdig}

\title{Velkommen til revyen}
\author{KM, HVO, KB}
\melody{Erik Grip: ``Velkommen i det grønne''}

\begin{document}
\maketitle

\begin{roles}
\role{S0}[KB] Sanger
\role{S1}[JTP] Sanger

% Er med i oversigt over numre men ikke i beskrivelsen af nummeret.
%\role{A}[KHH] Andet
\end{roles}


\begin{song}
\sings{Nogen}%
Velkommen her i ABC hvor vores glade sange
sku' gerne få jer til at le og klappe mange gange
Nu spiller vi igen revy idenne gamle ramme
og selvom melodierne er ny er vitserne de samme

Nu får I både dans og sang succes'en den er sikker
vi synger falsk hver anden gang og glemmer vor's replikker
og vitserne er flade emn det må I acceptere
for dem Vi ikke har ta'et med er trods alt meget værre

De ting vi nu fortæller jer dem har vi alle digtet
vi viser ingen mensker her, der også fimdes rigtig
Hvis nogen føler sig trådt nær så er det højst uheldigt
og vi beklager meget at det sker det er jo helt tilfældigt

I denne grå eksamenstid får vi for mange grå hår
vi sidder hjemme med vort slid og glemmer alt om forår
men drop det blot en enkelt gang gå ud og se på byen
for der skal også være tid til sang velkommen til reyen

hver fugl må synge med sit næb og livet her i byen
var uden sang kun slid og slæb velkommen til revyen
\end{song}

\end{document}
