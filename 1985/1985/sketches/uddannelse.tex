\documentclass[a4paper,11pt]{article}

\usepackage{revy}
\usepackage[utf8]{inputenc}
\usepackage[T1]{fontenc}
\usepackage[danish]{babel}


\revyname{DIKUrevy}
\revyyear{1985}
\version{3.1}
\eta{$n$ minutter}
\status{Færdig}

\title{Uddannelsessketch}
\author{DIKUrevy m.fl.}

\begin{document}
\maketitle

\begin{roles}
\role{F}[LGJ] fader, Edward D. Berg; så tæt op ad hattemageren fr. Alice i 
eventyrland som muligt
\role{M}[TL] moder, Edit H. Berg; relativt almindeligt tøj
\role{D}[LOL] datter, Lis P. Berg; relativt almindeligt tøj
\role{S}[KHH] søn, Kurt Berg; relativt almindeligt tøj
\role{J}[CE] dommer; fodboldstøvler, korte sorte benklæder, sort kappe,
dommerparyk, ugle, fløjet hammer
\role{O}[HVO] Oplæser
\end{roles}


\begin{sketch}

\scene{Et teselskab (som I Alice i eventyrland). På væggen (= bagtæppet) hænger
  et billede fra teselskabet i A. i E. En telefon skal være til rådighed.}

\scene{Familien Berg sidder omkring bordet fastfrosset i akavede stillinger mens
  følgende læses op i højttalerne:}

\says{O} I dette femhundredeogtolvte afsnit af Matador følger vi familien Berg
igennem de svære år, hvor børnene er blevet så store, at de flyver fra
reden. Edward har problemer på arbejdet, hvor maskinerne ikke helt vil makke ret
efter at hans gamle ven og kompagnon Tage har forladt firmaet fordi hans morbror
Mortensen er blevet lærer på DIKU efter anbefaling fra familiens ven
studehandleren. Edward er derfor lidt konfus -- der er vrøvl med semantikken --
og under indflydelse af den frygtelige strejke på de forenede bryggerier har han
lidt svært ved at huske, om hans søn -- Kurt Berg -- hedder Kurt Berg eller Carl
S. Berg. Edwards kone Edit -- som har jord i hovedet efter et slemt styrt på sin
vandcykel -- bliver holdt nede af sin mand, men støttes i sin traditionelle
kvinderolle af datteren Lis P. Berg, som efter en lang med kedelig skolegang nu
vil til at studere eller rejse eller noget andet. Edward har store ambitioner på
børnenes vegne, specielt Carl -- øhh Kurt -- som han håber vil føre
familiefirmaet Berg og Dværg videre. Men Kurt -- som også har jord i hovedet
efter at Edit's vandcykel styrtede ned over ham -- er konceptuelt mere
velfunderet end sin familie med begge ben solidt plantet i den blå luft, og han
har ganske andre planer for fremtiden.

\says{F} Nå, Lis. Har du fundet ud af, hvad du vil læse ?

\says{D} Ja!

\says{F} Nå, hvad er du så kommet frem til? Sygeplejerske måske?

\says{D} \act{bestemt} Nej! Jeg skal læse Jura!

\says{Alle} Juuuuura???

\says{D} Ja! Jeg skal nemlig gøre karriere!

\says{F} Nåeh, en længerevarende uddannelse? Jamen der er da også mange andre
muligheder: Eksimologi, gastronomi og sådan noget.

\says{D} Nej! Jeg skal læse jura!

\says{F} Jamen var det ikke en bedre ide med datalogi f.eks. Det er der jo
fremtid i.

\says{D} Jaeh. Jeg havde oprindeligt tænkt på datalogi kombineret med HD men der
er for meget lal på datalogi. De spiller revy og sådan noget.

\scene{Tæppe for, lyd af lydbånd der spoles tilbage (cueing). Alle rykker en
  plads mod venstre. Figurerne på billedet på bagtæppet ligeså. Tæppe fra,
  musikken spiller titeltemaet fra matador. Hvert familiemedlem sidder med to
  STORE hundredekronesedler i hånden og styrrer mystificeret på dem.}

\says{S} Nå, mor. Har du fundet ud af hvad du vil læse?

\says{M} JA!

\says{S} Nå, hvad er du så kommet frem til? Var det noget med jordemoder?

\says{M} Tænker du på gartner, min dreng? Nej, jeg har aldrig kunnet finde ud af
det med maskiner. \act{trækker et lommeur frem, ser på det} Hvad dato er det i
dag? \act{ryster uret, holder det op til øret}

\says{F} \act{trækker et kort fra bunken å bordet, læser op} Tillykke, det er i
dag den 15. juni. Der er revy!

\says{M} \act{sukker} Det går to dage forkert. \act{til F, halvvredt} Jeg sagde
jo til dig, at det ikke kunne nytte at komme smør i værket!!

\says{F} \act{spagfærdigt} Men det var det allerbedste smør.

\says{M} Så må der være kommet nogle krummer i! Du skulle ikke have smurt det
med brødkniven.

\says{D} Lad mig se! \act{tager uret, dypper det i sin the, hiver det op og ser
  på det, banker det nogle gange ned i bordet og slår på det med flad hånd}

\says{F} \act{imens D roder med uret} Det var \underline{virkelig} godt
smør. Tror du ikke du skulle tage på ferie, min ven?

\says{D} Ja, du kunne tage til jurabjergene.

\scene{Enter dommer! Kommer drønende ind, fløjter og pander uret en med
  hammeren, tager et rødt kort fra bunken på bordet og læser op}

\says{J} I dømmes til at gå tilbage til start, thi kendes for ret: Her er for
meget lal -- og denne gang får I kke 200 kroner!!
\end{sketch}
\end{document}
