\documentclass[a4paper,11pt]{article}

\usepackage{revy}
\usepackage[utf8]{inputenc}
\usepackage[T1]{fontenc}
\usepackage[danish]{babel}


\revyname{DIKUrevy}
\revyyear{1985}
\version{1.0}
\eta{$n$ minutter}
\status{Færdig}

\title{Manden pa måtten}
\author{KHH m.fl., KB}
\melody{Står ikke}

\begin{document}
\maketitle

\begin{roles}
\role{S}[KHH] Sanger

% Er med i oversigt over numre men ikke i beskrivelsen af nummeret.
%\role{L}[JTP] Lærer
\end{roles}


\begin{song}
\sings{S}%
Her står jeg hver morgen på måtten
på måtten ved DIKU's dør
og nyder lærdommen sådden
så jeg bliver ganske ør
studerende er jeg næsten
hver dag til Dat-0 jeg går
og lytter spændt til Kaj Madsen
jeg klogskab og kunnen får
jeg har med rapporter og bøger
ej vrøvl som de andre har her
forstår ganske alt hvad jeg hører
og lidt til og meget er'
Jeg elsker at regne flydende
og pascal er mit yndlingssprog
men jeg får aldrig andet end lydende
men af dem kan man godt blivk klog
mens de derinde har øv'lser i yordon og bobs
bliver jeg datalogisk af ekkoet fra Reckus jobs

Jeg elsker at afprøve eksternt
og internt og bruger især,
desværre ligger det lidtfjernt
når man ikke har øvelser her.

Men allermest elsker jeg kerner,
i assembler laver jeg dem.
Men Motorola'n de fjerner
når man kun har 7 , 5.

For søvnløse næter jeg slippe
for skrivekrame og stress.
Jeg har al min tid til at flippe
og leve og ryge græs.

Min auditoriedør
jeg har den i smug på klem.
Tilfreds med bare at høre.
Med mit snit på 7 , 5.

Nu gad jeg vide, om lyden fra englenes kor,
an være bedre, end lyden fra datalo'r!
\end{song}

\end{document}
