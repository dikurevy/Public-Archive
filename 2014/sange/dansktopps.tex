\documentclass[a4paper,11pt]{article}

\usepackage{revy}
\usepackage[utf8]{inputenc}
\usepackage[T1]{fontenc}
\usepackage[danish]{babel}
\usepackage{hyperref}


\revyname{DIKUrevy}
\revyyear{2014}
\version{1.0}
\eta{$3.5$ minutter}
\status{Færdig}

\title{Dansk-TOPPSen}
\author{Phillip, Brainfuck, Nana, Troels, Bitre-Mikkel}
%\melody{Kunstner: ``Originaltitel''}

\begin{document}
\maketitle

\begin{roles}
\role{S1}[Spectrum] Sanger
\role{S2}[Mia] Sanger
\role{S3}[Bette-Mikkel] Sanger
\role{X}[Troels] Instruktør
\end{roles}

\begin{props}
\prop{Guitar - børnestørrelse?}[]
\end{props}

\url{http://harlem.dikurevy.dk/~roschnowski/dansktopps.mp3}

Kun S1 på scenen.

\textbf{Sådan noget som DOS}

Melodi: Poul Krebs "= ``Sådan nogen som os''

\begin{song}

\sings{S1} Engang var brugerne tilfredse
Nu' Ap(ple) det som alle ønsker sig
Men heldigvis er vi parate
Ja, Win-dows 1.0 den er på vej

\sings{S1} Sådan noget som DOS
Har jo brug for en GUI
Men den GUI er nog't værre hurlumhej
Så vent nogl' år
Vi' jo lig' ved at lær' det
Windows stifinder (er) på rette vej

\end{song}

(Afbrydes af S2, der kommer ind på scenen og hyggesnakker lidt til
publikum "= se \url{www.youtube.com/watch?v=BvXqFJPl80Q} "= det er
\textit{super} fesent.)

\begin{sketch}
  \says{S2} Nej, nej, nej... for lang tid siden, da sad jeg og kaldte
  en funktion med tre argumenter.  Og som funktioner nu er, så havde
  den en krop og et kald...

  \says{S2} Og kaldet... det gik sådan her.
\end{sketch}

\textbf{Kald den der funktion}

Melodi: Lars Lilholt "= ``Kald det kærlighed''

\begin{song}
\sings{S2} Kald den der funktion
Kald den lig' der i kod'n
Åh-uh-øv
Den findes i libc
Men mit kald det går itu
For den er ik' linket endnu

\sings{S2} Fortæl mig hvor jeg finder den funktion
Befri mig nu fra denne frustration
Er den i /lib eller i en anden lokation? (udtales: slash lib)
Befinder den sig mon i /usr/lib? (udtales: slash user slash lib)
Den svarer ik' når jeg prøver med en grep

\sings{S2} Kald den der funktion
Kald den lig' der i kod'n
Åh-uh-øv
Den findes i libc
Men mit kald det går itu
For den er ik' linket endnu
\end{song}

S3 kommer ind.

\textbf{8 bit}

Melodi: Shubidua "= ``Stærk Tobak''

\begin{song}
\sings{S3} Nu skal I høre hvad vi godt ka' li'
\sings{S1+S2} 8 bit!
\sings{S3} Det kan man gemme mange farver i
\sings{S1+S2} 8 bit!

\sings{S3} Bit 5 til 7 er rød
\sings{S1+S2} (Og) 0 og 1?
\sings{S3} Dem bruger vi til blå
\sings{S1+S2} (De) sidst' er grøn'
\sings{S3} Så har vi to-hundrede-seks-og-(halv)treds
farver at kigge på

\sings{S1+S2} 7 bit - uuu - aaa...

\sings{S3} Nej nej nej nej jeg sagde 8 bit!
\sings{S1+S2}[forvirrede] 8 bit?
\sings{S3} Ja da, din ASCII ska ha' paritet
\sings{S1+S2} 8 bit!
\end{song}


\end{document}

