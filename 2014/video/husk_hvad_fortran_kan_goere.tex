\documentclass[a4paper,11pt]{article}

\usepackage{revy}
\usepackage[utf8]{inputenc}
\usepackage[T1]{fontenc}
\usepackage[danish]{babel}


\revyname{DIKUrevy}
\revyyear{2014}
% HUSK AT OPDATERE VERSIONSNUMMER
\version{1.0}
\eta{$0.5$ minut}
\status{Manus klar, skal filmes}

\title{Husk hvad FORTRAN kan gøre}
\author{Phillip, Troels, Camilla}

\begin{document}
\maketitle

Marks følelsesnote: Denne videogram-produktion har en følelse af en
munter, uærbødig omgang med vores fælles faglighed, men efterlader
også en bitter note som tankestreg, der får os til at stoppe op, og
spørge os selv, om vores kortsigtede retning og motivation i
virkeligheden gør os lykkelige og fuldendte som mennesker.

\begin{sketch}
\scene{F sidder ved sin datamat i den fjerne ende af kantinen. Han er den eneste ved bordet. Billedet er gråt og støvet eller sepia-farvet.}

\scene{Klip til den anden ende af kantinen. Liv og glade dage. Masser af farver. Diskokuglen i hyggehjørnet drejer rundt. En flok dataloger sidder og koder. Vild musik. Go-go-piger danser på bordene. Alle smiler og griner.}

\scene{Klip tilbage til den modsatte ende af kantinen. En D0 kommer hen til fysikeren.}

\says{D0} Hey, vi er ved at kode et højereordens bevissystem i Haskell. Kom og vær med.

\says{F}[trist] Naah... Jeg skal lige invertere denne matrice.

\scene{Klip til den sjove ende af kantinen. Voldsom dubstep-musik i baggrunden. C bruger sit studiekort til at hakke en mælkesnitte i stykker på et spejl. Hun sniffer en bane.}

\scene{Mere fest.}

\scene{D1 kommer hen til fysikeren.}

\scene{C tømmer automaten for mælkesnitter i baggrunden.}

\says{D1} Hey, kommer du ikke ned til os andre? Vi er ved at porte Linux til Rust. Det bliver pissefedt.

\says{F} Naah.. Jeg skal lige gange nogle matricer sammen...

\scene{Scene: Klip til den sjove ende af kantinen. C sidder og koder med et dødt blik i øjnene og mælkesnitte-flødeskum i ansigtet.}

\scene{En datalog tager en cola-bong, mens han taster febrilsk på sit tastatur. Der står en flok rundt om ham og råber ``TeX! TeX! TeX! TeX! TeX! TeX! ...''}

\scene{D2 kommer hen til fysikeren.}

\says{D2} Hey, vi er ved at kode Absalon om i SML.

\says{F} Naah... Jeg skal lige finde en eigenvektor. Det er egentlig meget sjovt.
\scene{Datalogerne går. Fysikeren stønner opgivende.}

\scene{Fade til sort: "`Husk hvad FORTRAN kan gøre"'}

\end{sketch}
\end{document}
