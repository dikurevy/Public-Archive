\documentclass[a4paper,11pt]{article}

\usepackage{revy}
\usepackage[utf8]{inputenc}
\usepackage[T1]{fontenc}
\usepackage[danish]{babel}
\usepackage{hyperref}


\revyname{DIKUrevy}
\revyyear{2014}
% HUSK AT OPDATERE VERSIONSNUMMER
\version{1.0}
\eta{$4$ minutter}
\status{Færdig}

\title{Bitcoin-sketchen}
\author{Phillip, Troels, Brøns, Bette-Mikkel}

\begin{document}
\maketitle

\begin{roles}
\role{F1}[Daniel] Forsker
\role{F2}[Sebastian] Forsker
\role{B}[Arinbjörn] Bartender
\role{G1}[Ejnar] Caféen?-gæst
\role{G2}[Simon] Caféen?-gæst
\role{G3}[Brainfuck] Some guy
\role{G4}[Nana] Caféen?-gæst
\role{G5}[Nanna] Caféen?-gæst
\role{G6}[Peter] Caféen?-gæst
\role{G7}[Troels] Caféen?-gæst
\role{M1}[Kasper] Matematiker
\role{M2}[Niels] Matematiker
\role{M3}[Bitre-Mikkel] Matematiker
\role{K}[Caro] Prebens kone
\role{PS}[Person] Publikumssufflør
\role{H}[Amanda] Helle-Thorning Schmidt (fra politisk permutation)
\role{X}[Troels] Instruktør
\end{roles}

\begin{props}
\prop{Læssevis af ringbind}[]
\prop{Kagerulle}[]
\prop{Flødeskumskage}[]
\end{props}


\begin{sketch}

\scene{Helt mørkt. Spot på F1.}

\says{F1} Godaften. Jeg hedder Preben.

\scene{F1 peger til den ene side.}

\says{F1} Og dette er min kollega, Preben.

\scene{Spot på F2.}

\says{F2} I kender jo alle til bitcoin.

\says{F1} Men da vi er timelønnede, så lad os lige opsummere:

\says{F2} Bitcoin er en såkaldt krypto-valuta.

\says{F1} Det lyder måske lidt kryptisk, så lad os lige opsummere:

\says{F2} Dette bliver en meget lang sketch...

\says{F1} Så lad os lige opsummere:

\says{F2} Bitcoin er en valuta, der ikke er styret af nogen centralbank, men
derimod et distribueret peer-to-peer-netværk.

\says{F1} Nu hvor I forstår alle detaljerne, kan vi undersøge brugernes praktiske
interaktion med systemet.

\says{F2} Vi er på Caféen?.

\scene{Caféen?-lys, bar-disk i den ene side af scenen. Der er også
  nogle gæster (G1-G2). Mos Eisley Cantina Theme spilles i
  baggrunden:
\url{http://youtu.be/stbYF6XpTYE?t=6s}}

\scene{F1 går op til baren.}

\says{F1} Undskyld, hvor meget koster en GT?

\scene{Over\TeX: Bitcoin-kursen fluktuerer helt vildt.}

\says{B} Hvornår vil du købe den?

\scene{F1 og B kigger op på bitcoin-kursen. Der er monteret et
  enarmet-tyveknægt-agtigt håndtag i baren.}

\scene{F1 trækker i håndtaget.}

\scene{Over\TeX: Kursen fryser.}

\says{B} Det bliver 8 bitcoins.

\says{F1}[til publikum] Ser I, bitcoins er baseret på kryptografi. Jeg kan
bevise at jeg er møntens ejer, fordi jeg har dens private nøgle. For at
overføre mine bitcoins til sælgeren...

\scene{B bukker for publikum.}

\says{F1} ... er jeg nødt til at signere hans offentlige nøgle ved hjælp af
simpel elliptisk-kurve-kryptografi. Det er ganske enkelt.

\says{F1}[til bartenderen] Har du en blyant jeg må låne?

\scene{F1 kigger grundigt på B's navneskilt, og begynder at skrible på
  et stykke papir.}

\says{F2} Ja, vi har jo alle med stor bekymring set, hvordan NSA er
begyndt at overvåge al elektronisk kommunikation. Derfor er alle
\textit{fornuftige} mennesker efterhånden gået over til
offline-bitcoins. Man bør således foretage alle de nødvendige
beregninger i hånden.

\scene{F1 står stadig i baren, og beregner hashen i hånden. Køen bag
  ham vokser og vokser.}

\says{F1}[vender sig mod publikum] Er dette bare ikke meget smartere
end tradionelle papir-penge?

\says{F2}[til publikum] Nu skal bartenderen så verificere at de
overførte bitcoins faktisk eksisterer. Til det formål bruger vi den
såkaldte blockchain. En distribueret liste over \textit{alle}
bitcoin-transaktioner nogensinde.

\scene{B tager papiret fra F1. B bukker sig ned bag baren, tager et
  KÆMPE ringbind op, bladrer lidt i det, og sætter et kryds.}

\says{B} Jep, det er noteret.

\says{B}[til de andre gæster på Caféen?] Kan I andre ikke også lige så
skrevet det ned?

\scene{Alle på Caféen? stopper hvad de er i gang med, og finder KÆMPE
  ringbind frem fra den blå luft. De begynder at bladre, finder den
  rigtige side, og skribler lidt med en blyant eller lign.}

\says{G1} Det er noteret!

\says{G2} Jeg er også færdig!

\says{B}[til F1] Fint, her er din øl.

\says{B} Næste!

\scene{Næste kunde betjenes. Der kommer 5 über-nørdede matematikere
  ind på Caféen?. De har alle ukæmmet hår og det samme tøj på: Hvide
  polo-trøjer (som er for lange), kuglepenne og lommeregnere i
  brystlommen.}

\says{F1} Åh nej! Matematikere!

\says{F2}[til publikum] Siden bitcoin-netværket er konsensus-baseret,
kan man påvirke beslutningerne hvis man har størsteparten af
netværkets regnekraft.

\scene{B forlader scenen for at hente noget ude på lageret(og klæde om til næste sketch)}

\says{M1} Sikke en flot bitcoin du har der.

\scene{M2 og M3 begynder at omringe F1.}

\says{M1} Det ville jo være en skam, hvis jeg kunne bevise at det var min. Tag
den fra ham drenge!

\scene{M2 og M3 har nu helt omringet F1, og begynder at taste febrilsk
  på deres lommeregnere.}

\says{F1} Stop dem, stop dem! De tager mine penge!

\scene{Alle matematikerne står omkring F1 og regner. På et tidspunkt
  bliver de færdige, løber over til de andre gæster på Caféen?, og
  viser stolt resultatet af deres beregninger frem.}

\says{G1} Ja, det kan jeg godt se.

\says{F1} Jeg er ruineret!

\scene{Matematikerne forlader Caféen?.}

\says{F2} I skal ikke være bekymrede. Det er ikke første gang vi mister alle
vores bitcoins. En gang hver anden måned er der en stor børs, der bliver
hacket.

\says{F1} Ja, det er en herlig følelse at være fri for statens formynderiske
indflydelse!

\scene{G3 kommer løbende ind på scenen.}

\says{G3} Hey, gutter! Preben har givet 0,04392 bitcoin til en stripper! Siger I
det lige videre?

\says{G2}[pifter efter publikum] HEY! I DER! PREBEN HAR BETALT EN
STRIPPER!

\scene{Prebens kone (K) kommer løbende ind på scenen med en kagerulle
  og en flødeskumskage i hænderne.}

\says{K} \textit{Min} Preben?! Nej, det passer altså ikke! Han skrev i
\textit{min} blockchain at han brugte dem på at købe en buket
blomster.

\says{F2} Som I kan se, opstår der nogle gange en kortvarig uenighed på grund af
blockchainens distribuerede natur. Men konsensus genvindes altid hurtigt.

\scene{K smasker en kage i hovedet på G3 og slår alle gæsterne på
  Caféen? bevistløse med sin kagerulle.}

\says{K}[vender sig mod publikum] Hvad købte Preben?

\scene{Publikumssuffløren holder et skilt op, hvorpå der står "BLOMSTER".}

\scene{F1 forsøger at genvinde ro og orden i salen.}

\says{F1} Det er muligt at I endnu ikke er helt overbeviste. Men vi har i hvert
fald fået overbevist Helle Thorning. Faktisk har regeringen besluttet at
binde kronen op på bitcoin-kursen. Lad os se hvordan det går med
finanslovsforhandlingerne:

\scene{Over\TeX: Bitcoin-kursen svinger stadig. Nedenunder ses et
  regneark og nogle grønne/røde søjler der svinger i takt til kursen.}

\scene{H kommer løbende ind på Caféen? og tager fat
  i håndtaget.}

\says{H} Finansloven er vedtaget .....nu!

\scene{Over\TeX: H trækker i håndtaget. Bitcoin-kursen og finansloven
  fryser med røde tal over det hele.}

\scene{Lys ud.}

\says{H} Pißß...

\end{sketch}
\end{document}
