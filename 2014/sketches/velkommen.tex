\documentclass[a4paper,11pt]{article}

\usepackage{revy}
\usepackage[utf8]{inputenc}
\usepackage[T1]{fontenc}
\usepackage[danish]{babel}


\revyname{DIKUrevy}
\revyyear{2014}
% HUSK AT OPDATERE VERSIONSNUMMER
\version{1.1}
\eta{$0.5$ minutter}
\status{Færdig}

\title{Velkommen}
\author{Julie}

\begin{document}
\maketitle

\begin{roles}
\role{V0}[Kasper] Revyt
\role{V1}[Jonas] Revyt
\role{X}[NB] Instruktør
\end{roles}

\begin{sketch}

\scene{Bandet åbner revyen med temaet fra Compilerjammer. Musikken dæmpes, og Velkommen går i gang.}

\scene{Lys op. V0 og V1 står på scenen med ryggen til publikum, og kigger ind gennem revnen i bagtæppet (altså væk fra publikum). Det skal forestille at de kigger ind på ``scenen'', mens de snakker, så publikum får fornemmelsen af at være backstage. Man kan evt. hænge en lampe op bag bagtæppet, der lyser igennem sprækken, så det ligner scenelys, eller lave noget der ligner ``indersiden'' af en scene på Høj- og Over\TeX.}

\says{V0} Jeg er så nervøs. Hvad hvis jeg fucker det op?
\says{V1} Du har én replik. Du kan umuligt sige ``Velkommen'' forkert.
\says{V0} Velbekommen.
\says{V1} Et trick er at få publikum til at virke mindre skræmmende ved at forestille sig at de ikke har nogen bukser på.
\says{V0} Ad! Det synes jeg er vildt skræmmende. Nu har jeg glemt min replik.
\says{V1} Velkommen.
\says{V0} Har du også glemt din replik?

\scene{Lys ned. Musik op.}

\end{sketch}
\end{document}
