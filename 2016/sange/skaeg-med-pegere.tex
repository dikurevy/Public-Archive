\documentclass[a4paper,11pt]{article}

\usepackage{revy}
\usepackage[utf8]{inputenc}
\usepackage[T1]{fontenc}
\usepackage[danish]{babel}


\revyname{DIKUrevy}
\revyyear{2016}
\version{1.0}
\eta{$3.35$ minutter}
\status{Derefereret}

\title{Skæg med pegere}
\author{Sebastian Paaske Tørholm}
\melody{Hr. Skæg: ``Skæg med bogstaver''}
% http://open.spotify.com/track/0zQ3QyoHe17yzic2mtsWjd

\begin{document}
\maketitle

\begin{roles}
\role{S}[Niels] Hr. Skæg
\role{K0}[Bette-Mikkel] Kor
\role{K1}[Nicklas] Kor
\role{K2}[Theis] Kor
\role{X}[Nathalia] Instruktør
\end{roles}

\begin{song}
\sings{S}%
Kom med, min ven, vi skal lege med C
Her er en peger, den kalder vi p
Men lige nu er der garbage i den
Hvor skal den dog pege hen?

Vi sætter den, så den nu peger på 'k'
Den peger gern' på både store og små
Vi kan jo også pege den på din mor
Hun er på hoben -- hun er jo ret stor

Vi ska' ha' skæg med pegere
Vi peger fingre og så følger vi dem
Peger' er ikke så slemm'

Vi ska' ha' skæg med pegere
Vi hilser gerne, ja på må og på få
C gør det let at forstå:

char stjerne p (tekst: '\texttt{char *p}')
lig malloc tre (tekst: '\texttt{= malloc(3)}') \act{p bliver allokeret 3 pladser før "din mor"}
stri'n-copy p (tekst: '\texttt{strncpy(p}')
kom-ma streng hej og tre (tekst: '\texttt{, "hej", 3)}') \act{ups, ingen plads til strengterminator}
put-s af p (tekst: '\texttt{puts(p)}') \act{der udskrives "hejdin mor"}
Hvor er det dejligt!
Din mor er -- jo på besøg igen!

Lad os sig' farvel
stri'n-copy p (tekst: '\texttt{strncpy(p}')
komma farvel, og den er længde seks (tekst: '\texttt{, "farvel", 6)}')
put-s af p (tekst: '\texttt{puts(p)}') \act{der udskrives "farvel mor"}
Nu skal vi huske
Vi skal altid rydde op igen
Vi kalder free på p (tekst: '\texttt{free(p)}')

\sings{S}%
Ja, det' skægt med pegere
Vi peger fingre og så følger vi dem
Peger' er ikke så slemm'

Vi ska' ha' skæg med pegere
Derefererer, siger hej til din mor
Pegeren ved hvor hun bor

\sings{S}%
Vi ska' ha' skæg med pegere
Og når vi' færdig rydder vi op igen
Peger' er ikke så slemm'

Vi ska' ha' skæg med pegere
Vi peger fingre og så følger vi dem
Peger' er ikke så slemm'

\scene{Lys ned.}
\end{song}

\end{document}
