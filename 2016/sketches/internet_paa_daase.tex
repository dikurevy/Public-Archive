\documentclass[a4paper,11pt]{article}

\usepackage{revy}
\usepackage[utf8]{inputenc}
\usepackage[T1]{fontenc}
\usepackage[danish]{babel}


\revyname{DIKUrevy}
\revyyear{2016}
\version{0.1}
\eta{$0.5$ minutter}
\status{Færdig}

\title{Internet på Dåse}
\author{Troels}

\begin{document}
\maketitle

\begin{roles}
\role{A}[Kasper] En DIKUfant
\role{X}[Brainfuck] Instruktør
\end{roles}

\begin{props}
\prop{Hjørneskabet}[]
\end{props}

\begin{sketch}

\scene{Hjørneskabet.}

\says{A}[roder i hjørneskabet] Næh!  Hvad er dog det!  Nogen har puttet noget særligt i hjørneskabet.

\scene{A tager en dåse fra hjørneskabet.}

\says{A}[tager produktet] Næ!  Internet på dåse!

\scene{A åbner dåsen.  Der kommer et væld af gamle internetlyde over A/V, som
  f.eks. All Your Base, The Hamster Dance, Nyancat, Gangnam Style, Numa Numa
  Song og Peanut Butter Jelly Time, og en mand der råber "Det er jo episk!".}

\says{A} Hov - den er vist blevet for gammel..
\scene{A tænker kort, og sætter dåsen tilbage i hjørneskabet igen}

\scene{Lys ned.}

\end{sketch}
\end{document}
