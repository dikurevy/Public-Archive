\documentclass[a4paper,11pt]{article}

\usepackage{revy}
\usepackage[utf8]{inputenc}
\usepackage[T1]{fontenc}
\usepackage[danish]{babel}

\revyname{DIKUrevy}
\revyyear{2016}
\version{1.0}
\eta{$2.5$ minutter}
\status{Læs ikke!  Ellers forsvinder det jo!}

\title{Engangsbogspressemødet}
\author{Sebastian, Niels, Simon, Bitre-Mikkel, Brainfuck}

\begin{document}
\maketitle

\begin{roles}
\role{U}[Jenny] Ulla Tørnæs
\role{J0}[Ejnar] Journalist
\role{J1}[Mads] Journalist
\role{X}[Brainfuck] Instruktør
\end{roles}

\begin{props}
\prop{Bog}[]
\prop{Podie}[]
\end{props}

\begin{sketch}

\scene{Ulla Tørnæs er lige blevet minister, og skal holde sit første pressemøde som uddannelsesministeren.}
\says{U} Jeg er Ulla Tørnæs.  Netop udnævnt uddannelsesminister.
\says{J1}[ruller med øjnene] Det ved vi godt, Ulla.
\says{U} Nu skal I høre godt efter, for jeg siger det kun én gang.
\says{U} Jeg ved ikke om I kender dette koncept, men dette er en bog! \act{U vifter med en bog.}
\says{U} Men det er ikke bare en bog, det er en \emph{e-bog}. Dvs, en \emph{engangs-bog}
\says{J0} En engangs-bog?
\says{U} Jo, ser I, når man læser en side, så forsvinder den. \act{U bladrer i bogen og river nogle sider ud.}
\says{U} Kender I ikke det når man læser en bog, og så kommer man til at lukke den, og så har man glemt hvor langt man er kommet? \act{U gestikulerer til journalisterne.}
\scene{U viser bogen frem som mangler en masse sider.}
\says{U} Tænk på hvor mange problemer det vil løse! Tænk på alle dem der godt kan lide at læse bøger flere gange! Det vil øge salget! Og skabe jobs!
\says{J1} Men er det ikke hovedsageligt maskiner der trykker bøger?
\says{U} Jo, jo. Men det gavner jo også børnene. Vi ønsker jo alle, at vores børn kender til alle de store mesterværker, og derfor skal læse dem i skolen. Men år for år kommer der flere og flere gode bøger. Ja, mængden af læsestof går mod uendelig; og det kan vi da ikke byde vores børn.
\says{J0} Hvad så med alle de bøger vi har derhjemme?
\says{U} Bare rolig, dem skal vi nok få brændt.
\says{J1} Men vi elsker jo vores bøger!
\says{U} Ja, og Tyskland elskede også Hitler. I vil ikke være ligesom Tyskland, vel? Så skal I altså tage og brænde jeres bøger.
\says{U} Vi lever trods alt i et videnssamfund. Og et videnssamfund er jo drevet af viden. Det er brændstoffet, så selvfølgelig skal det brændes! Jeg troede I var kloge!
\says{U} Når bøger kun kan læses en gang, så vil det også styrke hukommelsen!
\says{J0} Men Fru minister, vil du så gerne at man afskaffer alle former for bøger, og er dette blot et kompromis du er gået med til?
\says{U} Ja.
\says{J0} Men læser du overhovedet selv?
\says{U} Nej da, det er jo derfor jeg er minister!
\says{J1} Hvornår slutter høringsrunden om lovforslaget?
\says{U} I morgen. Men vi modtager kun høringssvar der er direkte relevante i forhold til teksten i lovforslaget.
\says{J1} Hvornår kan vi så få lovforslaget at læse?
\says{U} Aldrig! Så forsvinder det jo!

\scene{Mens musikken til næste sang (``e-bog'') går i gang, og sangeren kommer ind, siger U følgende:}

\says{U} Ah!  Her kommer uddannelsesministeriets departmentchef for at fortælle jer lidt mere om engangsbøger.

\scene{U og J0/1 tager podiet ud.}

\end{sketch}
\end{document}
