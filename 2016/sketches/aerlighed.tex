\documentclass[a4paper,11pt]{article}

\usepackage{revy}
\usepackage[utf8]{inputenc}
\usepackage[T1]{fontenc}
\usepackage[danish]{babel}


\revyname{DIKUrevy}
\revyyear{2016}
\version{1.1}
\eta{$1.5$ minutter}
\status{Ikke færdig overhovedet}

\title{Helt ærligt}
\author{Bette-Mikkel}

\begin{document}
\maketitle

\begin{roles}
\role{P}[Bitre-Mikkel] Forsker
\role{C1}[Kim] Caféen?-gænger
\role{C2}[Sebbe] Caféen?-gænger
\role{C3}[Mads] Caféen?-gænger
\role{C4}[Mia] Caféen?-gænger
\role{F}[Rune] Frivillig Caféen?-bartender
\role{X}[Simba] Instruktør
\end{roles}

\begin{props}
    \prop{C?-disk}[]
    \prop{Bord}[]
    \prop{To raflebægre}[]
\end{props}


\begin{sketch}
    \scene{Lys op}
    \says{P} Tak for at I er kommet til demonstration af min, Dr. Forsker, Ph.D.s, \emph{prisvindende} simulation.
    Som I alle ved, så handler den om sandfærdig interpersonel interaktion. Lidt om metodik: Jeg startede
    med et cloud-baseret neuralt netværk, hvor jeg havde sat sandhedsværdien til 1. Men i praksis viste det sig at være
    lettere at bruge russer! Igennem empriske studier har den bedste kontekst for simulationen vist sig at være Caféen?
    Det er her den største sandhedskoefficient har vist sig! Med disse ord, læn jer tilbage og nyd frugterne af mit hårde
    arbejde! \act{grand gesture}

    \scene{lys ned, lys op. C2-C3 sidder ved bordet; C1, C4 og F står i baren.}
    \says{C1} Jeg vil gerne have en Grøn Tuborg, så jeg bestiller en almindelig øl.
    \says{F} Han får en Gylden Dame, for det er en almindelig øl her på Caféen?.
    \scene{F langer C1 en GT; C1 sætter sig ned ved bordet.}

    \says{C2} Jeg vil gerne drikke jer fulde, så lad os spille Meyer.
    \says{C1} Jeg gider egentlig ikke, men har ikke noget bedre at lave. \act{rafler og kigger} 64.
    \says{C2}[C2 rafler bægret og kigger] 43. Jeg retter lige på terningen. Sådan, par 4.
    \says{C3}[C3 rafler bægret og kigger] 63. Det her er et værre lortespil. \act{tager en tår af sin øl} Okay, en gang mere\ldots
        \act{rafler igen} Par et!

    \says{C1}[C1 rafler bægret og giver det videre uden at kigge] Jeg er ligeglad med spillet og tror ikke at C2 løfter. Par to.
    \says{C2}[C2 rafler bægret og kigger] Det slog han ikke, men jeg er ligeglad fordi jeg snyder. Lille-Meyer.
    \says{C3}[C3 rafler bægret og kigger] 52. For helvede, jeg taber hele tiden!! \act{tager en stor tår af sin øl}

    \says{C2} Det er fordi jeg pakker terningerne hver gang du kigger væk.
    \says{C3} ... Hvor er I nederen! Bare jeg havde nogen andre venner.

    \says{C2} Nå, jeg er gået tør.
    \scene{C2 går op i baren til F}
    \says{C2}[sætter sin plastickop i disken] Jeg er utroligt kræsen.
    \says{F}[rækker C2 en frisk mad på fad] Den her koster maks. fem kroner at lave.
    
    \says{C2}[mens han går tilbage til bordet] Jeg keder mig, og jeg er liderlig.
    \says{C4}[til C2] Hej C2, jeg er fuld og liderlig.
    \says{C2} Hej C4, jeg var ikke i bad i morges eller i går.
    \says{C4} Det er okay, jeg har også bræk i mit hår.
    \says{C2} Så tager vi bare hjem til mig.

    \scene{C2 og C4 forlader scenen hånd i hånd}
    \says{C4}[VO] Forresten bliver jeg altid alt for vild med mine one night stands.
    %\says{C2} Det gør jeg også.

    \scene{C1 og C3 sidder stadig tilbage}
    \says{C3} Jeg scorer aldrig... 53?
    \says{C1} Det er ikke over par 3.
    \says{C3} Fuck!
\scene{Lys ned.}

\end{sketch}
\end{document}
