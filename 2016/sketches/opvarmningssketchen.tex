\documentclass[a4paper,11pt]{article}

\usepackage{revy}
\usepackage[utf8]{inputenc}
\usepackage[T1]{fontenc}
\usepackage[danish]{babel}


\revyname{DIKUrevy}
\revyyear{2016}
\version{1.0}
\eta{$1.25$ minutter}
\status{Kender I det med at man har skrevet en sketch, og så vil man TeX'e den, men så skal man finde på en sjov status?}

\title{Forelæsnings-opvarmeren}
\author{Sebastian, Brainfuck, Niels, Kasper Elbo}

\begin{document}
\maketitle

\begin{roles}
\role{K}[Bette-Mikkel] Keld, dårlig publikums-opvarmer/entertainer
\role{X}[Brainfuck] Instruktør
\end{roles}

\begin{props}
  \prop{Høj skammel}
  \prop{Mikrofonstativ}
  \prop{Enten stort glas vand, eller en flaske kildevand}
\end{props}

\begin{sketch}

\scene{Spots viser frem og tilbage på en tom scene.}

\says{K}[VO] Velkommen til forelæsning i Advanced Language Processing!  Nu kommer jeres opvarmer, KEEEELD!

\scene{K kommer hoppende ind.}

\says{K} Hvad sker der?!  Er I klar?!  Jeg kan se at I sidder helt lamme i sædderne, fordi I skal til \emph{for}elæsning!

\scene{Publikum griner.}

\says{K} Nå... det \emph{ged}er I ikke høre på.

\scene{Det er sjovt, K går videre til næste }

\says{K} I skal jo til at høre om oversættere, og københavnske cykelister er de eneste der oversætter grønt lys til ``jeg må hellere sætte farten \emph{helt ned}''.

\says{K} Bilisterne i Jylland sætter bare sømmet i bund indtil de indhenter en københavner.

\says{K} Hvis jeg nu havde haft en kæreste, og jeg spurgte om vi skulle have sex, så ved jeg bare at hvis hun sagde måske -- så ville jeg jo vide at det betød nej!

\says{K} Oversætter... det sådan noget med træer.  Og træer er der i skoven, og jeg var en gang i skoven.  Og ude i den skov, ku' jeg så se jeg var på Eduroam, fordi der ikke var noget internet.

\scene{K kigger op på bandet.}

\scene{Bandet ryster på hovedet.}

\says{K} Nå, øøøh, men så er det tid til Tooooor-beeeen Moooo-geeeen-seee...
         Nej, hov.  Det er sgu da Cosmin.

\scene{K forlader scenen.  Lys ned.}

\end{sketch}
\end{document}
