\documentclass[a4paper,11pt]{article}

\usepackage{revy}
\usepackage[utf8]{inputenc}
\usepackage[T1]{fontenc}
\usepackage[danish]{babel}
\usepackage{hyperref}


\revyname{DIKUrevy}
\revyyear{2019}
\version{0.1}
\eta{$2.5$ minutter}
\status{Færdig}

\title{Forkert kanal på revy-TV}
\author{Niels}

\begin{document}
\maketitle

\begin{roles}
\role{VO}[Kasper] Voice over
\role{R0}[Æmilie] Revyt
\role{R1}[Romeo] Revyt
\role{R2}[Brandt] Revyt
\role{X}[Niels] Instruktør, videoansvarlig
\end{roles}

\begin{sketch}

\scene{Note: Skal hedde noget andet i programmet.}

\scene{Lys ned.  Helt mørk sal.  Trommehvirvel.}

\says{VO} LIVE FRA STORE UP1, VELKOMMEN TIL... CIRKUSREVYEN!

\scene{Festlig baggrundsmusik.  I det følgende tages der klip fra fx
\url{https://www.youtube.com/watch?v=8mQostkfVt0} og andre lignende ting.}

\says{VO} I aften skal I grine sammen med ULF PILGAARD \act{der vises sjove klip
med Ulf Pilgaard på OverTeX}, LISBETH DAHL \act{der vises sjove klip med Lisbeth
Dahl}, HENRIK LYKKEGAARD \act{der vises sjove klip med Henrik Lykkegaard} og med
kyndig musikalsk vejledning af JAMES PRICE \act{der vises kyndige klip med James
Price}, så glæd jer--\act{der kommer lys på scenen, lyden til videoen skrues
ned, men der fortsættes med at være sjove klip på OverTeX.}

\scene{R0 går ind på scenen med et stort telegram i hånden og et kostume halvt
på.}

\says{R0} Jeg skal undskylde for den tekniske fejl.  Det er selvfølgelig
\emph{ikke} Cirkusrevyen der er på programmet i aften.

\scene{R0 tager en fjernbetjening frem og peger den mod OverTeX og trykker på
nogle knapper.  Det skal ikke gå hurtigt.  Der kommer en
``revy-TV''-kanaloversigt frem hvor der står at CIRKUSREVYEN er på revykanal 1,
DIKUREVYEN er på revykanal 0, og de andre studenterrevyer er på kanaler med
meget høje tal.  Der vises at R0 klikker hen på kanal 0.  Der er DIKUrevyens
logo samt teksten ``Venter på live feed''.}

\says{R0} Nåja -- revy-TV er nulindekseret.  Så, publikum: Nu har jeg æren af at
præsentere \act{kigger hurtigt ned i telegrammet} DIKUREVYEN \act{kigger ned i
telegrammet i længere tid, skal lige tænke} 2020!

\scene{R1 går ind på scenen med et større telegram i et kostume endnu mere halvt
på.}

\says{R1} Det lyder altså lidt forkert.

\says{R0} Men der står her \act{peger, R1 kigger med}: ``DIKUREVYEN, komma, det
totusindeogtyvende år''.

\says{R1} Ja -- nulindekseret!

\scene{R0 slår sig selv i panden og går ud af scenen mens R1 overtager
snakke-teten.}

\says{R1} Ærede publikum, I ser foran jer \act{kigger hurtigt ned i telegrammet}
2019-REVYEN i \act{kigger ned i telegrammet i længere tid} FYSIK!

\scene{R2 går ind på scenen med et endnu større telegram i et kostume endnu mere
endnu mere halvt på.}

\says{R2} Det lyder altså lidt forkert.

\says{R1} Men der står her \act{peger, R2 kigger med}: ``2019-revyen, komma,
revyen der ligger den sidste weekend i måned nummer fire''.

\says{R2} Ja -- nulindekseret!  Og du kan jo også se lige der \act{peger på
OverTeX} at der står DIKUREVY!

\says{R1} Nåå, \act{slår sig selv i panden} jeg troede at det var en
trickopgave!

\scene{R1 går ud af scenen mens R2 overtager snakke-teten.}

\says{R2} Så!  Velkommen til--\act{kigger ned i telegrammet og bliver
forvirret}.

\says{R2}[henvendt til backstage] Kom lige herover!

\scene{R0 og R1 kommer ude fra bag tæppet og hen til R2.}

\says{R2} Denne opgave giver slet ingen mening.  Altså, der står bare
ordret...

\scene{R2 vender telegrammet, så publikum kan se at der står ``DIKUREVY 2019''
på det i stor tekst.}

\says{R1+R2+R3} ``DIKUREVY 2019''!

\scene{Idet replikken er sagt begynder startsangen, og videoen stoppes.}

\end{sketch}
\end{document}
