\documentclass[a4paper,11pt]{article}

\usepackage{revy}
\usepackage[utf8]{inputenc}
\usepackage[T1]{fontenc}
\usepackage[danish]{babel}


\revyname{DIKUrevy}
\revyyear{2019}
\version{1.0}
\eta{$1.15$ minutter}
\status{Færdig}

\title{Shirley tager på Caféen?}
\author{Niels, Simon, Søren P., Markus og Caroline}

\begin{document}
\maketitle

\begin{roles}
  \role{S}[Amalie] Shirley
  \role{B}[Theis] Bartender
  \role{F1}[Brandt] Statist
  \role{F2}[Romeo] Statist
  \role{F3}[Vivien] Statist
  \role{F4}[Æmilie] Statist
  \role{X}[Caro] Instruktør
\end{roles}

\begin{sketch}
  \scene{Shirley kommer tudende ind på Caféen?}
  \scene{Bartenderen står og pudser et lexanglas (det gør de hele tiden på Caféen?)}
  \says{B} Er du dumpet, eller er du blevet dumpet?
  \says{S} \emph{Hulk} \emph{hulk} Der er ingen der forstår mig, jeg troede at jeg havde venner
  \says{B} (henvendt til publikum) Det tror alle dataloger
  \says{B} Men så er du kommet til det rette sted
  \says{S} Er der da venner her?
  \says{B} På Caféen? kan en sød pige altid finde nogen, der vil snakke til hende, længe... og insisterende... (kigger bebrejdene på publikum)
  \says{B} Nu skal jeg lære dig spillereglerne: For det første skal du aldrig lytte til nogen der tilbyder dig noget at drikke, eller insisterer på at skåle med dig (kigger på publikum)
  \says{B} Eller spørger dig om du vil spille et spil
  \says{S} Ihh tak, du er så hjælpsom, du virker som en sød fyr.
  \says{B} Helt sikkert, værsgo, en lille fysiker - uden is, \act{giver hende en drink} skål.
  \scene{De drikker, hun går kold}
  \says{B} (kigger ud på publikum) Lad os så spille et spil

\end{sketch}



\end{document}
