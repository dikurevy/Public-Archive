\documentclass[danish]{article}
\usepackage{revy}
\usepackage[utf8]{inputenc}
\usepackage{babel}
\usepackage{a4wide}

\title{Den Store Dekansketch}
\author{dunkel}

\version{2 -- færdig} % HUSK AT AJOURFØRE VERSIONSNUMMER!!

\revyyear{1999}

\begin{document}
\maketitle

\begin{roles}
  \role{A} Interviewer
  \role{B} Dekanen
\end{roles}

\begin{sketch}

\scene Interviewsituation

\says{A} Vi har i aften besøg af dekanen for at høre, hvad han har gået og
bedrevet det sidste stykke tid.

\scene Dekanen kommer ind og sætter sig

\says{A} Nå, sidst du var her snakkede vi jo om, at det ikke var så godt med
den måde, du havde forvaltet fakultets penge på, ikke? Der var vist
nogen biologer, der ikke var så glade.

\says{B}[formælt] Jo.

\says{A} Og så er du kommet her i aften for at fortælle, at du har forbedret
dig, ikke?

\says{B} Jo, det var et sidespor, jeg var kommer ud på det, men det har jeg
ligesom lagt bag mig nu.

\says{A} Ok, ikke mere at det nu?

\says{B} Nej. Istedet for har jeg jo gjort det, at jeg har kastet mig over
SAFIR og nedlagt det.

\says{A} HVAD?

\says{B} Jo altså, for at prøve at forhindre rusturene.

\says{A} Nej. Prøv at hør, det må du da ikke! Det bliver alle de frivillige
idealistiske vejledere da kede af!

\says{B}[forundret] OK?!? OK?!? Ja, der var du hurtig! Når du fremlægger
det fra den vinkel, så kan jeg godt se, det... ja....

\says{A} Ja, man skal ligesom se det fra flere vinkler. \act{gør fakter med
armene} Så det skal du altså ikke gøre!

\says{B} Ja ok, jeg forstår godt hvor du vil hen. Men jeg har dag også gjort
andre ting. F.eks. har jeg sørget for, at rusturen kommer til at ligge
i den første uge af semesteret

\says{A} NEJ!! Det er da ikke godt!

\says{B} Jo, fordi så skal folk jo ikke bruge deres sommerferie på at være
rusvejlere.

\says{A} Det kan du da ikke! Der er da ingen seriøse studerende, der vil
bruge tid i deres eget semester på at lave rustur, det kan du da nok se!

\says{B}[pause] OK!, ja, ja.. OK! ja, på den måde, ja.. Det havde jeg ikke
liiiige tænkt over, men når du siger det på den måde, så ja.. point
taken, point taken.

\says{A} Ja, man skal ligesom se tingene hele vejen rundt. \act{gør fakter} DU
må altså passe lidt på, ik'?

\says{B} Ja. Det kan du jo have ret i. Men rusturene har jo altid været noget
forfærdelig useriøst noget. \act{kunstpause}. Det er jo et kendt faktum, at der
er blevet drukket ALT for meget øl \act{længere pause}, så det har jeg gjort
noget ved! Jeg har sørget for, at overskudet fra ølsalget IKKE må gå til
medfinanciering af turen. På den måde kan der ikke blive opfordret til unøden
druk!

\says{A} Ok?!? Men på den måde kommer en øl jo til at koste noget nær
indkøbspris, hvilket jo nok næppe for folk til at drikke færre øl!

\says{B} JA, ok!! Der var du liiiige hurtig der! ok ok.

\says{B} Men det er jo også kommet nogle andre gode ting ud af det, ikke? Jeg
har f.eks. også sørget for, at der er undervisning LIGE fra starten af!

\says{A} Undervisning? Fra starten? Jamen de nye studerende kender jo slet
ingen endnu og har ingen ide om, hvordan universitetet fungerer? De vil
da hellere have svar på alle deres spørgsmål om studiet, end have
forelæsning i fysik!

\says{B} OK?!? OK, OK!??!?!? Ja, DEN måde havde jeg IKKE liiige set det på. OK!?

\says{A} Har du slet ikke gjort noget godt?

\says{B} Jo da. Jeg har jo hygget mig med at ændre min politik hele tiden så
der ikke har været sikkerhed omkring rusturenes fremtid, således at
der ikke kunne bestilles hytter i god tid.

\says{A} Det er da ikke godt

\says{B} Jo, for så var den eneste mulighed, at bestille hytte på Bornholm
og så skal rusturen jo bruge 7 timer på transport hver vej!

\says{A} Det er da ikke fedt!

\says{B} Hvorfor ikke?

\says{A} Fordi så bliver turen jo meget kortere og de bliver jo nødt til at
tage afsted, så de ikke kan komme til imatrikulation!

\says{B} Og?

\says{A} Ja, så bliver de nye studerende da kede af det!

\says{B} Kede af det? Nå....., ok, ok! På den måde, ja det havde jeg ikke liiiige
tænkt over. Der var du lige hurtig der. point taken. ok!  Nå, men det
skal da have konsekvenser det her! Mon ikke bare jeg skal afskaffe alt
hvad der hedder rustur, så er der ingen problemer!

\says{A} NEJ!!! Det er jo ikke det, du skal! Tænk på de nye studerende

\says{B} OK! OK! OK!, ok ok ok! Nå, så det er heller ikke måden. Jamen så
tror jeg bare, at jeg går op på mit kontor og dyrker min
batiksmølfefetish mens tingene sejler deres egen sø.

\says{A} Ja gør du det!

\scene tæppe for


\end{sketch}

\end{document}
% Local Variables: 
% mode: latex
% TeX-master: t
% End: 
