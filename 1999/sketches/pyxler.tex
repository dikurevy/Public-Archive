\documentclass[danish]{article}
\usepackage{revy}
\usepackage[utf8]{inputenc}
\usepackage{babel}
\usepackage{a4wide}

\title{Pyxler}
\author{marvin, panic, beyond, myth, uffefl, agnete, runem, andresen, guldfisk, gimse, hynne}

\version{3 -- helt fægrip} % HUSK AT AJOURFØRE VERSIONSNUMMER!!

\revyyear{1999}

\begin{document}
\maketitle

\begin{roles}
  \role{SS} Stig Skelboe
  \role{JS} Jørgen Sand
  \role{T} Tekniker
\end{roles}

\begin{props}
  \prop{L} Lysbilledeapparat/overheadprojektor
  \prop{overheads/lysbilleder}  til "Informationspsykolog"
\end{props}
                
\scene
De mange pyxler (er det med x eller ks?) på DIKU er til stor moro for alle
(bortset fra de stakkels medarbejdere (hæ)). Her er pyxlingsteorien
bag det hele\ldots
                
\begin{sketch}

\scene SS \& JS kommer ind foran tæppet mens musikken spiller dæmpet
Kaj\&Andrea i baggrunden

\says{SS}[peger på JS] Tillad mig at præsentere: nørdeSJang !

\says{JS}[peger på SS] Og dette er belgiSk oSte !

\says{begge}[i kor] Og vi er \emph{hobby-}informationspsykologer! \act{nikker stolte}

\says{SS}[mens han kigger på publikum] Ja, i \emph{virkeligheden}
hedder du jo ikke nørdeSJang.

\says{JS}[mens han kigger på publikum] Ja, og i \emph{virkeligheden}
  hedder du ikke belgiSk oSte.

\says{begge}[i kor] for vi er blevet \emph{pyxlet!}

\scene musikken stopper (som i den rigtige Kaj \& Andrea-sang)

\verb"\begin{Kaj & Andrea}"
\says{SS} Så hvis du siger nørdeSJang -- bliver det til Jørgen Sand

\says{JS} Og hvis du siger belgiSk oSte -- bliver det til
\act{kunstpause} Stig Skelboe

\verb"\end{Kaj & Andrea}"

\says{SS}[gestikulerer med armene] \emph{Pyxling} består i at man
bytter godt og grundigt rundt på bogstaverne\ldots

\says{JS}[bryder hurtigt ind] så "Fotokopiering" bliver til "Fogedkontor!"

\says{SS} og ``Print'' bliver til ``Prut!'' og ``Køkken'' bliver til
``Kukkeø''

\says{JS} Lad os se et eksempel på Min
Yndlingsalgoritme: PYXELSORT! \act{peger ind på scenen mens tæppet går 
  fra}

\scene Lys væk fra SS og JS, og ind på \fbox{\large Pyxelsort}

\says{JS} UPS! Det sidste var vist en pyxlingsprogramfejl\ldots

\says{SS} Som I kan se, kan man også \emph{vende} bogstaverne om

\says{JS} så "d" kan blive til "p", "z" kan blive til "N", og med lidt god
vilje kan "e" blive til "ø"

\says{SS} Og da ``i'' kan blive til udråbstegn kan Print altså blive til
Prut! Dette underviser vi i på basal pyxlingspsykologi.

\says{JS} Dér lærer man om \emph{pyxlingstypologi}: pyxler hvor ``e'' bliver
til ``ø'' er \act{siges med let afsky} \emph{lavere} ordens
pyxler, mens de pæne pyxler -- som skabes ved \emph{pretty-pyxling} -- er
1. ordens pyxler. De allerfineste har store bogstaver som
begyndelsesbogstaver, de såkaldte \emph{højere ordens polymorfe pyxler}.

\says{SS} Arne Glenstrup er perfekt til højere ordens polymorf
pyxling \ldots  Navnet kan oplagt pyxles til "Grens
Alenprut", mens 1. ordens pyxler er "rAul neGerstud" og "ernA
prutsneGl"\ldots{}

\says{JS} På kurset i pre-pyxlingspsykologi lærer man om
\emph{pyxlingspotentialer}

\says{SS} Tag for eksempel forskningsgrupperne på DIKU.

Det meget teoretiske Topps kan højst blive til "sTop d" eller "p
posT", og har derfor et lavt pyxlingspotentiale, i modsætning til
 Distlab, der kan blive til alt fra
"liDt bas" til "tab SilD".  Og musiklaboratoriet kan blive til
urinakrobatik i mosML\ldots{}

\says{JS} Ikke mindst medarbejderne har et stort
personlighedspyxlingspotentiale, og det siger noget om deres
personlighed.

\says{SS} Ja, Torben U. Zahle bliver for eksempel til "NyT
oracle ben", Pawel Winter bliver til "Mandetimer!", og  "Martin
Elsmann" bliver til ``Ml nar i mastEn''%"MuntrE salami"

\says{JS} De laverer personlighedspyxlingspotentialer finder vi "Gregers Koch"
der kun bliver til "Gog herresoK", Mads Nielsen der bliver til
"PalMeNissen" og Mai-britt Pedersen der bliver til "ib - Pres Mine
patter"\ldots{}

\says{SS} Og nogle er endda direkte kontrapyxliske. Jeg kan i hvert fald
ikke tro, at Lazlo Kovacs driver coLa-iNKasso.

\says{JS} Enkelte medarbejdere var pre-pyxlede da de blev ansat på
DIKU. F.eks. har frontpyxelforskningen de-pyxlet Jyrki Katajainen til
Jan Kaj kanetyr!!, Jan j tyrenaKke, eller Jay Karatejunk!!

\says{SS} Forskningen har også afsløret såkaldt \emph{forceret de-pyxling}.
MAN har truet folk til at ophøre med at pyxle.  MAN har fortalt, at en
unavngiven afdelingsleder blev træt af at hedde "go nat i hAreM" og "g
AnticharMe" \act{kunstpause så publikum kan de-pyxle navnet}

\says{JS}[begejstret] Men det giver anledning til ny pyxlotologisk
forskning! \emph{Re-pyxling!}

Nogle pyxler kan ikke dø. Det eviggyldige "Kukkeø" er et glimrende
eksempel. Det er blevet pyxlet, de-pyxlet og re-pyxlet så ofte, at man
kan tale om pyxlisk cykling.

\says{SS}[bedrevidende, ser overlegent ud på publikum] Jaja, du mener
jo cyklisk pyxling.

\says{JS} Nejnej, pyxlisk cykling.

\says{SS} Cyklisk pyxling, \act{bliver gradvist ophidset} din, din,
\ldots \emph{informationspsykolog!}

\says{JS}[slået ud af det stygge skældsord] Hvad, skal du kalde mig
informationspsykolog? Din lille lede\ldots
fysikgoon I normal post

\says{SS} Du kan selv være en \ldots fysikmatros og nul polo

\says{JS} NÅRH, skal du fornærme mig, du er s'gu da en \ldots fiks
astronomIpolygon

\says{SS} Det er s'gu da ikke noget i forhold til dig, du er 
ramt af
 \ldots gratis fynsk mono polio

\says{JS}[jagter JS ud af scenen]  Du ligner en fra \ldots ``symfonisk
antropologi'', \ldots{} du er et  ``most girafkoloniopsyn''

\scene Tekniker kommer ind og skal lige til at fjerne transparenten
fra projektoren, da han stopper op og studser.

\says{T}[undrende] symfonisk antropologi??? Hmmm\ldots
\act{han retter det til} ``mystik foran spionlogo''
Hmmm\ldots{} \act{Slår sig ``åh-nej, selfølgelig!''-agtigt på panden,
  og retter det til}
 ``tokyo spIonforsamling''

\scene Fin.

\end{sketch}

\end{document}

% Local Variables: 
% mode: latex
% TeX-master: t
% End: 
