\documentclass[10pt]{article}
\usepackage{revy}
\usepackage[utf8]{inputenc}
\usepackage[danish]{babel}
\textwidth 160mm
\evensidemargin 0pt
\oddsidemargin 0pt
\title{90 års 1M revisionen}
\author{Uffe, Søren, Rasmus S., Lars}
\frenchspacing
\version{1.0} % HUSK AT AJOURFØRE VERSIONSNUMMER!!

\revyyear{1997}
\parindent0pt
\parskip 1ex minus 1ex
\flushsingsright
\begin{document}
\maketitle

\begin{sketch}

\begin{roles}
  \role{F} Finn Schiermer Andersen
  \role{I} En anden 1M instruktor med propel-hat
  \role{S1,S2} To 2. årsstuderende (til tæppenummer)
  \role{O} Voice-over (til tæppenummer)
\end{roles}

\begin{props}
  \role{} Propelhat, C++-bog, Ugeseddel fra 84, Slides, Overhead
    projektor, død motorola maskine, kerne-rapport 76
\end{props}

\scene{Indledende tæppenummer}

\says{S1} [overstadig, barnagtig] Finn, Finn! Vi vil have en opgave om mikroarkitektur.
\says{S2} [værre] Og om operativsystemer.
\says{S1} [endnu værre] Og det skal være noget med en oversætter!
\says{F} [forælder-agtig] Jamen kære små, det er jo hele 3 ting. Det
         går virkelig ikke.
\scene{Pause. F bevæger sig væk fra S1 og S2. S1 og S2 ud.}
\says{F} [bekymret] Det \emph{går} virkelig ikke.
\says{O} [reklamemand] Og dog! Med det nye Kinder Dat1 får du alle 3
         ting i \'et!
\scene{F smiler og bliver glad.}
\says{O} [slogan] Kinder Dat1. Finn's på køl, lige ved siden af
         førstedelen.

\scene{Tæppe fra. F ind på plads.}

\scene{F sidder for enden af et imaginært (?) bord. I
  bevæger sig rundt om bordet i takt med konversationen \--- ligesom i
  den oprindelige 90 års fødselsdag. Knud og Jyrki ``sidder'' med
  ryggen til publikum og Niels og Møffer med fronten til publikum. Imellem Knud og Jyrki er placeret
  en død Motorola maskine. I taler skiftevis med sin egen stemme,
  Knuds stemme (fynsk?), Jyrkis stemme (dyb, mumlende, finsk-svensk) og
  Niels' stemme (lidt høj, lidt ligesom \texttt{marvin}). Finn taler
  skiftevis med sin egen stemme og med Møffers stemme (falset,
  dukkefører-stemme). Transparenterne er udført i bedste Knud-stil,
  med store tydelige bogstaver og bløde rammer. 
  Bemærk: selvom der
  nedenfor står at der skal stå en ting på transparenten og han siger
  noget andet, så skal det hele med!
}

\says{F} Er vi her alle sammen?
\says{I} Ja Finn, de er alle kommet til \emph{din} revision af 1M.
\says{F} Knud Henriksen?
\says{I} [peger på første plads på det imaginære bord] Ja, Knud sidder
her i år.
\says{F} Jyrki Kata\ldots{}hata\ldots{}hata\ldots{}
\says{I} Prosit!
\says{F} Tak.
\says{I} [peger på den anden plads] Ja, Jyrki sidder her.
\says{F} Niels Elgård?
\says{I} [peger på den tredje plads] Ja, Niels har jeg placeret her.
\says{F} Og Møffer?
\says{I} [peger på den fjerde plads] Til højre for dig, som du bad om.
\says{F} Tak. Lad os først diskutere C++-bogen.
\says{I} [henter C++-bogen] Ja, C++-bogen.
\says{F} Jeg foreslår at vi dropper undervisningen i C++. Hvad mener
         du Knud?
\scene{I lægger transparenten på. Det er dette manuskript.}
\says{I} [Knud] Udemærket. C++ er alligevel så let at folk kan lære
         pensum uden forelæsningerne! Er det klart som en umudret skovsø, som
         sommerens azurblå himmel?
\says{F} Jyrki?
\says{I} [tøvende] Årh, skal jeg, Finn?
\says{F} [bestemt] Kun for at gøre mig glad.
\says{I} [Jyrki] Ol\"aka ol\"aka tal.
\says{F} Godt. Niels?
\says{I} [Niels] Øh, er C++ pensum?
\says{F} Og Møffer?
\scene{F strækker sin hånd ud og former den som en mund ved Møffers
  plads. I holder sin mikrofon hen til hånden.}
\says{F} [Møffer] Du har ret som altid, Finn!
\says{F} Godt. Ud med C++ og ind med noget nyt! Hvem vil lave det?
\scene{Pause. I kigger på alt andet end F.}
\says{I} Dvs. samme C++-bog som sidste år?
\says{F} [sukker] Samme C++-bog som \emph{hvert} år.
\scene{Pause.}
\says{F} Okay. Hvad med ugesedlerne?
\says{I} [henter ugesedlen] Ja, ugesedlerne.
\says{F} Jeg synes at ugesedlerne skal opdateres \emph{hver} uge.
         Knud?
\scene{I lægger transparenten på. Det er dette manuskript.}
\says{I} [Knud] Da jeg var ung havde vi ikke sedler, endsige uger! Er det
         klart som en umudret skovsø, som sommerens azurblå himmel?
\says{F} Jyrki?
\scene{På vej over til Jyrkis plads snubler I over Motorolaen. Kigger
  lidt befippet på den.}
\says{I} [ked] Er det nu nødvendigt?
\says{F} [mere bestemt] Vær så venlig.
\says{I} [Jyrki] Concurrent programmering af datorer.
\says{F} Tak. Niels?
\says{I} [Niels] Øh, ugesedler. Er det pensum?
\says{F} Møffer?
\scene{F strækker sin hånd ud og former den som en mund ved Møffers
  plads. I holder sin mikrofon hen til hånden.}
\says{F} [Møffer] Du har ret som altid, Finn!
\says{F} Godt. Vi skal altså have en til at opdatere ugesedler hver
         uge. Hvem gør det?
\scene{Pause. I kigger på alt andet end F.}
\says{I} Dvs. samme ugeseddel som sidste uge?
\says{F} [suk] Samme ugeseddel som \emph{hver} uge.
\scene{Pause.}
\says{F} Næste punkt. Rapportopgaverne.
\says{I} [henter rapportopgaven] Ja, rapporterne.
\says{F} [selvglad] Simulatoropgaven er der jo ikke noget i vejen med.
\says{F} Men til kernen synes jeg vi skal opdatere maskinparken, så vi
         får en mere tidssvarende opgave. Knud?
\scene{I lægger transparenten på. Det er dette manuskript.}
\says{I} [Knud] Vi kunne jo købe en akustisk fotokopimaskine. Så kunne
         vi også trykke slides på den. \emph{På manuskriptet skal stå:}
         Er det klart som en umudret skovsø, som sommerens azurblå
         himmel? \emph{Men han siger:} Er det kvalmt som et umodent skovbær, som blommernes nazigrå
         skimmel?
\says{F} Jyrki?
\scene{På vej over til Jyrkis plads snuble I over Motorolaen. Noget
  mere voldsomt denne gang. Kigger vredt på den.}
\says{I} [opgivende] Nu igen?
\says{F} [meget bestemt] Ja tak!
\says{I} [Jyrki] Rigor Motorolis. Hemmupgifter!
\says{F} Og Niels?
\says{I} [Niels] Øh, kernen. Er det pensum?
\says{F} Møffer?
\scene{F strækker sin hånd ud og former den som en mund ved Møffers
  plads. I holder sin mikrofon hen til hånden.}
\says{F} [Møffer] Du har ret som altid, Finn!
\says{F} Godt. Vi skal have udviklet en ny kerneopgave. Hvem står for
         det?
\scene{Pause. I kigger på alt andet end F.}
\says{I} Dvs. samme kerne som sidste år?
\says{F} [suk] Samme kerne som \emph{hvert} år.
\scene{Pause.}
\says{F} Sidste punkt er eksamen.
\scene{Pause. I kigger på alt andet end F.}
\says{I} Dvs. samme studerende som sidste år?
\says{F} [stort suk] Samme studerende som \emph{hvert} år!

\scene{Tæppe for.}

\end{sketch}

\end{document}
% Local Variables: 
% mode: latex
% TeX-master: t
% End: 
