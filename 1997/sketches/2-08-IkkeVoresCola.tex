\documentclass[11pt]{article}
\usepackage{revy}
\usepackage[danish]{babel}
\textwidth 160mm
\evensidemargin 0pt
\oddsidemargin 0pt
\title{Ikke vores Cola}
\author{Skrevet af Ida Gimsing, Katrine Hommelhof, Jonas Ussing, Lars WM}

\version{1.0} % HUSK AT AJOURFØRE VERSIONSNUMMER!!

\revyyear{1997}
\parindent0pt
\parskip 1ex minus 1ex
\flushsingsright
\begin{document}
\maketitle

\begin{sketch}

\begin{roles}
  \role{Pigen} ~
  \role{Nørden} ~
  \role{Kor} tre (?) piger off-stage, som udgør et "zum-zum"-kor
\end{roles}

\begin{props}
  
  \role{En masse:} bord, 2 stole, 1 balje, 2 glas, 1 vase m. blomster, 1
  "køleskab", 2 stearinlys (optional), 1 tom flaske Coca-Cola, 1 fuld
  flaske Pepsi, 1 Coca-Cola etiket (med klister på), 1 taske med vådt
  EDB-grej, 2 burgers, 1 flaske ketchup. Desuden 1 morgenkåbe, 1 kjole, og
  1 skjorte der ikke skal bruges til noget nyttigt EFTER revyen.

\end{props}

\scene{Tæppet går fra, og bandet begynder at spille "zum-zum". På scenen
  står to stole (på den enestår en balje), et spisebord med dug og
  stearinlys, og et "køleskab" som står i en vinkel så publikum ikke kan se
  ind i det\ldots

Pigen, iført en morgenkåbe, står med et glas i hver hånd, og valser meget
overspillet hen imod bordet i takt "zum-zum", mens hun sender smilende
blikke ud imod publikum. Hun sætter glassene på bordet, og retter på dugen.

Pigen stiller sig overfor et "spejl" og netter sig - læbestift, øjenvipper,
øreringe, pudder, og måske får håret en tur med børsten.

Efter makeup'en sender pigen et frækt blik ud mod publikum (løfter på
øjenbrynene et par gange), og smider morgenkåben - surprise, hun har en
kjole på indenunder.

Hun går derefter hen til "køleskabet", tager en $1,5$ liters Coca-Cola flaske
ud, og holder den op så pulikum kan se at - ÅH NEJ - den er tom!  Men frygt
ikke, pigen smiler overlegent og rækker igen ind i køleskabet for at
frembringe en fuld 1,5 liters PEPSI-MAX-LIGHT flaske! Yrgh! Buh!

Pigen lægger Coca-Cola flasken ned i baljen (som står på stolen) og smøjer
ærmene op (på sin selskabs-kjole?), og roder rundt i baljen med hænderne.
Efter et par sekunder hiver hun stolt en Coca-Cola etiket op (vi havde lige
snydt lidt og lavet en i forvejen), og sætter den på Pepsi-flasken! -BUH,
ØV!

Op ad scenetrappen kommer nu nørden, iført sit scoredress med en dusk
blomster i den ene hånd (komplet med vase og det hele), og en taske (med en
bærbar PC) i den anden, og sender charme-selvtillid-og-lækkert-hår-blikke
ud mod publikum. Han ringer markerende på en virtuel dørklokke (som er
placeret i en vinkel osv.)

Pigen, som netop er færdig med Cola-trikket, smiler glædestrålende over
lyden af den lydløse, virtuelle dørklokke, og valser ud for at åbne.

De to hilser på hinanden med knus og smil, og pigen modtager med stor glæde
vasen med blomster, og sætter den på bordet.

Nørden får derefter udleveret sit glas, og pigen skænker "Cola" op -- hun
gør meget ud af at vise ham, at det skam er Coca-Cola, og han er tydeligvis
imponeret over hendes gode smag. Hun sætter flasken på spisebordet, med
stor forsigtighed, så etikken ikke hopper af.

Da hun sætter flasken på bordet, kommer hun midlertid til at vælte
blomstervasen ned i fyrens taske! De farer begge hen til tasken, og
begynder at hive dryppende disketter, printkort, en bærbar mm. op af
tasken. Fyren holder sin molestrerede bærbare op, men ryster smilende på
hovedet og trækker på skuldrene, altimens han hælder vandet tilbage i
vasen. For det gjorde jo ikke noget.

Men nu skal de spise, så pigen placerer fyren på en stol, og henter maden:
burgers.  For at kompensere for bommerten med blomsterne, insisterer pigen
på at komme ketchup på fyrens burger for ham, men "den smutter i hænderne
på hende", og der kommer en ordentlig klat lige på fyrens skjorte. Hun flår
ham op af stolen for at hjælpe ham med at få det af, og hun er selvfølgelig
enormt flov, og uha-nej, hvad skal hun dog gøre, men fyren tværer det ud
med en ligegyldig mine, og trækker smilende på skuldrene. For det gjorde jo
ikke noget.

Og så skal der skåles\ldots Fyren nipper optimistisk til sin "Cola", men
spytter hurtigt ud igen! Fy fa'n, nu er han sur! Pigen står med et
spørgende og uskyldigt udtryk og trækker på skuldrene, for hun ved jo ikke
hvad der foregår... Med et ordenligt "DING- DONG!" sparker fyren hende i
klokkeværket, giver hende sit glas, drejer om på hælen og går ud af scenen.
Tæppet går for, og til de sidste toner af "zum-zum" projiceres teksten
"ikke Vores Cola" op på et lærred el.\ lign.}


\end{sketch}

\end{document}
% Local Variables: 
% mode: latex
% TeX-master: t
% End: 
