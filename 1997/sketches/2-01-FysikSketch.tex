\documentclass[11pt]{article}
\usepackage{revy}
\usepackage[utf8]{inputenc}
\usepackage[danish]{babel}
\textwidth 160mm
\evensidemargin 0pt
\oddsidemargin 0pt
\title{Fysiksketch}
\author{Theo, Søren, Peter, Rune, \ldots}

\version{--0.1}

\revyyear{1997}
\parindent0pt
\parskip 1ex minus 1ex
\flushsingsright
\begin{document}
\maketitle

\begin{sketch}

\begin{roles}
  \role{fysikstud} som de nu engang ser ud, men med frakke på
  \role{doktor} med jakke. 
  \role{speaker} klædt i smoking, men han står bag scenen
       i en vinkel som publikum ikke kan se
\end{roles}

\begin{props}
  \role{bøjle} der skal stå foran tæppet på scenen
  \role{gral} 
  \role{integraltegn}
\end{props}
\newcommand{\skrot}[1]{}
\skrot{
\scene{Tæppet er trukket for. Det er sidst på aftenen. Pausen er slut. 
Publikum har indfundet sig på sine pladser, og stilheden sænker sig som en 
dyne over det store auditorium. Folk er spændte. Bagest i salen hører man 
lyden af auditoriets trædøre, der forsigtigt lukkes af to revyttere. 
Steen ser nervøst på sit ur. Henning konfererer med lysmændene om hvem der 
skal give signal. De bliver enige. Loftlyset slukkes. Mørket indtræffer ---
i mere end \'en forstand, og man hører speakerens klare stemme helt oppe 
på de bagerste rækker. Anden akt er nu i gang.}

\says{speaker} [med klar stemme] Vigtig meddelelse! Ny, SENERE dato for
2.akt: lørdag d. 30. august. Revygruppen har desværre ikke kunnet overkomme at
sende breve ud til alle der har købt billet. For ikke at skuffe evt.
fremmødte i dag, har vi derfor valgt at opføre en {\sc Beta}-udgave af
et kommende fysikrevynummer.
}

\scene{I denne sketch afspilles en kort sekvens med dåselatter for hver
punchline, samtidig med at de medvirkende peger ud mod publikum. Sketchen
starter med at doktoren kommer noncharlant ind på scenen foran tæppet. 
Pludselig overraskes han glædeligt af publikums tilstedeværelse}

\says{doktor} God{\sc Gauss}! Min navn er Dr. {\sc Ion} {\sc Ångstrøm}.

\says{doktor} [lidt overstadigt] Sikke en {\sc masse} mennesker. 
Det er jo {\sc strålende}. Se her har jeg en gral\ldots og jeg har 
en-til-gral ({\sc integral}). 

\scene{Den fysikstuderende kommer en på scenen}

\says{fysikstud} Sikke noget fis {\sc Ion} ({\sc fission}). 

\scene{De gi'r hinanden hånden}

\says{fysikstud} Hej, jeg hedder M.C. {\sc Einstein} og jeg {\sc laser}
fysik.  Jeg er {\sc hook}'ed.

\says{doktor} Det er jo {\sc strålende}, men hør hvorfor kommer du så sent?
Vi aftalte tid kl. 8.

\says{fysikstud} {\sc Centripedalen} faldt af min {\sc cyklotoron}. 
Så bli'r man sku {\sc centrifugal}!

\says{doktor} Værs'god at tage {\sc La Place} henne ved mit {\sc Bohr}, 
så kommer jeg {\sc flux}.

\says{fysikstud} Okey, jeg hænger lige min frakke på denne {\sc Boyle}.

\scene{Den fysikstuderende peger pludseligt ud mod publikum}

\says{fysikstud} Hov! Der er en der {\sc volt}ager en {\sc pi}!

\says{doktor} Se hun gør {\sc modstand}. Hvilken {\sc effekt}!

\says{fysikstud} Der er godt nok {\sc tryk} på. {\sc Spændingen} stiger! 

\says{doktor} Tage er en gæv gut, mand!

\says{fysikstud} Øv. Den var {\sc Watt}'et.

\scene{doktoren betragter den fysikstuderende}

\says{doktor} Du har noget {\sc Avogadro} siddende mellem tænderne.
Må jeg anbefale at du bruger en kold {\sc gate} med 
{\sc Planck} kontrol.

\says{fysikstud} Jeg {\sc Tycho Brahe} tyggegummi.

\says{doktor} Jeg må vist hellere tage din puls.

\says{fysikstud} En puls? Bevares!

\says{doktor} Tag tøjet af.

\scene{Den fysikstuderende trækker bukserne ned.}

\says{fysikstud} Se mine {\sc Kelvin} Klein underbukser.
Er jeg ikke kold i {\sc Røntgen}.

\says{doktor} En {\sc Curie}øsitet! Hvor meget vejer du egentlig?

\says{fysikstud} Ikke mere end 1000 {\sc hologram}.

\scene{Bukserne trækkes op igen.}

\says{doktor} Hvor skal du fejre {\sc Joule} i år?

\scene{Sekvensen med dåselatter afspilles nu to gange}

\says{fysikstud} [henvendt til publikum] Det må være {\sc Dobblereffekten}.
I må gerne lave {\sc bølge}.

\says{fysikstud} [henvendt til doktoren] Hvornår {\sc Pascal} jeg komme igen?

\says{doktor} Hvad med på {\sc Ohm}sdag?

\says{fysikstud} Det passer mig meget dårligt, for jeg skal høre
{\sc Isotop}-20. Jeg er helt vild med den der ``{\sc Kvant}es lykkelige dag''.

\scene{den fysikstuderende smutter ud fra scenen.}

\says{doktor} Vent! Du glemte at {\sc Beta}le!

\scene{Doktoren følger efter ham, og første egentlige nummer i anden akt
kan nu begynde. Det er selvfølgelig ``Se mit ML-program\ldots ''.}

\end{sketch}

\end{document}
% Local Variables: 
% mode: latex
% TeX-master: t
% End: 



















