\documentclass[a4paper,11pt]{article}

\usepackage{revy}
\usepackage[utf8]{inputenc}
\usepackage[T1]{fontenc}
\usepackage[danish]{babel}


\revyname{DIKUrevy}
\revyyear{1987}
\version{1.0}
\eta{$n$ minutter}
\status{Færdig}

\title{TAP-pavillion-sang}
\author{Står ikke}
\melody{Står ikke}

\begin{document}
\maketitle

\begin{roles}
\role{S}[Står ikke] Sanger(e)
\end{roles}


\begin{song}
\sings{S}
Vi er to af DIKUS små veninder
de fromme kvinder
der nu forsvinder
Aldrig, aldrig vil vi glemme
vores fars stemme
når han sir' nej.
Vi er kun en brik i det store spil,
men nu ved vi hvad vi vil.

Vi tænker på en fremtid
hvor vi er bænkede på rad
ved institut og fakultet
og hvor vi tæver i vor macintoch
der er fanme ingen pardon
vi blir' ved indtil den skri'r
så kommer de løbende fra ale sider
for at finde ud af hvad
vi nu gør galt
jo vi skal si' jer det er nemlig tider
i den nye pavillion.

Til vores lille student
er din vej lidt sej
så vil tapperen altid hjælpe dig
og til det pusle-brug
ja der har vi fået bord
der blir' ingen kære mor
vi er ej gået ind til den evige hvile
hvis du tror det, ja så har du taet fejl
for der er nemlig rigeligt at passe
i den nye pavillion

Oh kære, kære vipper
vi er bedre und du tror
men du må gi' os ekstra snor
det her med teknologien
der er fanme ingen pardon
vi blir' indtil I skri'r
vi slider som små heste på vores kurser
vores service den skal være helt i top
for vi ska' si' jer der ska' være klasse
i den nye pavillion.
\end{song}

\end{document}
