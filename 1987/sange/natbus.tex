\documentclass[a4paper,11pt]{article}

\usepackage{revy}
\usepackage[utf8]{inputenc}
\usepackage[T1]{fontenc}
\usepackage[danish]{babel}


\revyname{DIKUrevy}
\revyyear{1987}
\version{2.0}
\eta{$n$ minutter}
\status{Færdig}

\title{Natbus til Rådhusplads'n}
\author{KM, LGJ}
\melody{I \& T Turner: ``Nutbush City Limits''}

\begin{document}
\maketitle

\begin{roles}
\role{S}[Står ikke] Sanger(e)
\end{roles}


\begin{song}
\sings{S}%
Rapportskrivning natten lang.
Sommerfest, dans og sang.
Problemer med hjemtransport
Svaret er ganske kort

Så ta' en natbus -- ååh natbus.
Så ta' en natbus te' Rådhusplads'n.

Den hedder ni-otte-fem.
Fra to til fem ta'r den dig hjem.
Nul-otte er minutterne,
så du må rubbe futterne.

Du ta'r en natbus -- ååh natbus.
Du ta'r en natbus te' Rådhusplads'n.

Hvis du skal den anden vej,
har vi også råd til dig.
Fra et til fire som et lyn,
ka' du komme ud af by'n.

Du ta'r en natbus -- ååh natbus.
Du ta'r en tabsu FRA Rådhusplads'n.

Hillerød det ka' du nå.
Ved Fælledparken står du på.
Ni-femogfirser'n ta'r afsted
tre-og-tred've -- ve' du med..

en natbus -- ååh natbus.
Så ta' en natbus fra Rådhusplads'n.

Til Tagensvej der ka' du gå
og klokken otte'r-tred-ve få
en bus te' Farum uden stress,
der hedder ni-hundred'hal'treds.

Du ta'r en natbus -- ååh natbus.
Du ta'r en natbus fra Rådhusplads'n.

Otte'r-tred've -- Farum.
Tre-og-tred've -- Hillerød.
Nul-otte -- Rådhusplads'n.
En gang i timen får den gassen.

Du ta'r en natbus -- ååh natbus.
Du ta'r en natbus te' Rådhusplads'n.
\end{song}

\end{document}
