\documentclass[a4paper,11pt]{article}

\usepackage{revy}
\usepackage[utf8]{inputenc}
\usepackage[T1]{fontenc}
\usepackage[danish]{babel}


\revyname{DIKUrevy}
\revyyear{1987}
\version{1.0}
\eta{$n$ minutter}
\status{Færdig}

\title{Banansangen}
\author{MJ}
\melody{``I niggerland bananen gror''}

\begin{document}
\maketitle

\begin{roles}
\role{S}[Står ikke] Sanger(e)
\end{roles}


\begin{song}
\sings{S}%
I USA BRinch Hansen bor
og dovne 'stud'er flygter.
Men nogle de har altid travlt,
de Brinch'ens øjne frygter.
De har slet ingen fritid der,
nej, de får lov at slæbe
sig til en figtig karakter,
hvad nytter det at flæbe.

De kender ikke Diku der,
ta'r ræset helt for givet.
Og mange af dem glemmer helt,
at de sku' leve livet.
Bag skærmens ånd man gemmer sig, åh hør, pc'en suser,
den holder mest af kodedej,
den grimme dataknuser.

Hver morgen står man tdiligt op,
pc'erne skal 'strømmes',
og fodres skal en læser her,
og priteren skal tømmes.
I 'lab'et møder Billy op,
der møder han den større
Brinch Hansen, se han ligner jo
en forsker på det tørre.

Et stort program skal skrives ud,
Brinch Hansen på det venter.
To tusind kasser printpapir
de små studenten henter.
De har så travlt, de sølle små,
men vil så gerne fremad,
de slæber hele dagen lang,
ak, når skal Brinch'en hjemad.

De sover i studenterby'r,
har ikke egen bolig,
og selv de er trætte, bliver søvnen aldrig rolig.
Du danske 'stud', du har det godt,
sig tak for studiets goder,
og sende en kærlig hilsen til
Brinch Hansens travle poder.
\end{song}

\end{document}
