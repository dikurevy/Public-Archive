\documentclass[a4paper,11pt]{article}

\usepackage{revy}
\usepackage[utf8]{inputenc}
\usepackage[T1]{fontenc}
\usepackage[danish]{babel}


\revyname{DIKUrevy}
\revyyear{1987}
\version{1.0}
\eta{$n$ minutter}
\status{Færdig}

\title{Sænke slagskibe}
\author{KDJ, NCKH, CKG}

\begin{document}
\maketitle

\begin{roles}
\role{A}[KDJ] Person
\role{B}[NCKH] Person
\role{C}[CKG] Person
\end{roles}

\begin{props}
\prop{Et bord}[]
\prop{En skærm}[]
\prop{Et antal papirer med tegninger over instituttet}[]
\prop{Misselhyletone (syntheziser) og knaldeffekt (trommer)}[]
\end{props}


\begin{sketch}

\says{B} Jåhå, før vi bejynna.

\says{A} Ja lad os nu se, jeg tror, jeg prøver et skud i tågen. N205.

\scene{-bang-}

\says{C} Ikke dårligt, ikke dårligt. Den sad lige midt i et lektorkontor. Hvad
er det dog der står på døren... Turban O Zehle.

\says{A} Ha, fik jeg ham?

\says{C} NÆSE! Kontoret var tomt.

\says{A} Øv, det siger du hver gang.

\says{B} A propos Zahle, har I hørt han smed Grue ud fra forelæsningen.

\says{A, C} ??

\says{B} Ja han tog fejl af auditorierne, og så smed han Grue ud fra dat-2
forelæsningen, og begyndte på sine MIME-numre og sexede normal-former. Det er
vist første gang han er blevet klappet ud fra en forelæsning.

\says{A} Ind i mellem får man det indtryk, at visse af lærerne er lige så dumme
som skydedørene i kælderen.

\says{B} Hvad er der nu med dem?

\says{A} Nå... De kan bare ikke rigtig bestemme sig for om de il være åbne eller
lukkede. Så er det da noget andet med vinduerne. Der kan man alligevel ikke
mærke forskel.

\says{B} Ok, så er det far. Jeg prøver SP12.

\says{C} ÅÅÅÅhhhh nej, missilt har kurs mod maskinstuen.

\says{B} Scchhllææ, scchhllææ.

\says{C} Det har kurs lige mod vaxen, det kommer nærmere...

\scene{-bang-}

\says{C} Den sad lige i nettet - Puh, hvor er her i øvrigt varmt.

\says{B} Ja, det er køleanlægget.

\says{A} Det forbandede køleanlæg er da også altid i stykker.

\says{B} Hvad mener du? Det varmer da ganske fortrinligt.

\says{A} Jamen, hvad skal Jørn Bo med varme pilsnere.

\says{C} Nå, din tur \act{peger på A}

\says{A} Okay, nu skal jeg nok bøffe dig \act{peger på B}, N225.

\says{C} Nå da, nu rammer du en balsal. Næ det var sgu et TAP-kontor.

\scene{-bang-}

\says{A} Ha, fik jeg nogle TAP'er med i købet?

\says{C} Næ, de sad i den anden ende ... men nu er der kaffe ud over det hele.

\says{A}[bedende] Ikke så meget som et eneste lille bitte lig.

\says{C} Lille bitte -- scchhllææ, men a propos lig, har I hørt, at der var
nogen, der kom med et lig til Jens Damgård?

\says{B} Og hvad så? Han er jo også mediciner.

\says{A} Havde det ikke været mere passende, hvis Krarup havde fået liget.

\says{B} Øh, jeg er vist ikke helt med.

\says{A} Jo, han er jo professor i operatinsanalyse.

\says{B} Min tur! S225, tag den, Fritz!

\says{C} Nå, nå, der sigtes på et professorkontor. Ralle, missilet styrer lige
mod Krølles seng. Neeeeej.

\scene{-bang-}

\scene{-tavshed-}

\says{B}[spændt] Nå fik jeg fyren?

\says{C} Jaaah -- og hende med!

\says{B} Øh, var der nogen der døde?

\says{C}[med et sjofelt grin] Næh, -- tværtimod ville jeg sige!

\says{A} A propos Krarup, har I tænkt på, at DIKU nu åbenbart arbejder efter
devisen: Fyr en fed og ansæt to tynde.

\says{B} Nå, du tænker på at Brincken blev erstattet af Krarup og den gamle gris
til Neil.

\says{C} Gries? Hvad har Divid Gries med det her at gøre?

\says{B} Ikke \texttt{G r i \underline{\textbf{e}} s}, men 
\texttt{g r \underline{\textbf{i}} s}. Hvad andet kan man kalde en lærer der går
i seng med en førstedelsstuderende?

\says{A} Førstedelsstuderende??!! Neil??

\says{B} Ja, scchhllææ.

\says{C} Ja, men han har nu også en meget tiltrækkende kone.

\says{A} Hvad har det med sagen at gøre?

\says{C} Jo ser du, hun startede på dat-0 for to år siden.

\says{B} Ja, det vil jo sige, at hun har haft \texttt{A I D S} i år.

\says{C} Nå ja, hun har været udsat for stoffet, men det er nu hos de færreste
det har sat sig på rygmarven.

\says{A} Kunne hun ikke bare have beskyttet sig med kondom? Det gør de jo så
meget ud af på busserne.

\says{C} Desværre ikke. Zahle udspreder denne form for AIDS ved verbal kontakt.

\says{B} Yes, yes det ER far, M203.

\says{C} Gisp, det er jo kantinen; du skyder på DIKU's kantine!

\scene{-bang-}

\says{C} Det er fyrgteligt. Den ligner fødeafdelingen i Calcutta, ja endda et
lokum fra Hiroshima!

\says{B} Fy føj, rate jeg skraldespanden?

\says{C} Næ sådan ser der altid ud.

\says{B} Nå ja, men var der nogen omkomne, nogle lemlæstede?

\says{C} Ja, du fik faktisk ram på en stakkels fyr, som : vaskede op, sorterede
bestik, fyldte automater, skar ost, købte øl, skældte ud, købte leverpostej,
lavede kaffe og ryddede op!

\says{B} Åh ja, stakkels Frank. Og det lige nu, hvor han havde fået en kontrakt
med Ajax.

\says{A} Ajax? Altså rengøringsmiddel til kantinen?

\says{B} Nej, han skulle være med i deres reklamefilm ... han skulle spille
tornado.

\says{C} Suk ja disse reklamer. Tag nu fx de her busreklamer.

\says{B} Du mener kondomerne? De er da ganske udmærkede.

\says{C} Oh jeah. Hvornår har du sidst haft brug for et seks meter langt kondom
...

\says{B} Ja nu fx...

\says{C} med svingdøre på midten??

\says{A} Det er da meget rart at man kan komme ind.

\says{B} Det skulle have været en frask bus....Franskmændene stiger på både
foran og bagi.

\says{A} Nå, hævnen er sød ! N204.

\says{C} Din nulhjerne, det er der da ikke noget der hedder!

\says{A} Nå nej, N209 mente jeg.

\says{C} Uha, du har sikker kurs mod et lektorkontor. Endda et med en lektor i.

\says{A} Scchhllææ, scchhllææ. Får jeg ham?

\says{C} Missilet har kurs mod hans hoved, åh nej enden er nær, ralle !

\scene{-bang-} 

\says{A}[spændt] Nå fik jeg ham??

\says{C} Det var satans, han redded livet. Missilet røg lige ind af det ene øre
og ud af det andet.

\says{A} \act{Taber underkæben}

\says{C} Men du var lige ved at få ham.

\says{A} Nå så lad mig prøve igen. Informationen.

\says{C} Informationen, hvor fandn får man information her på stedet?

\says{A} Nå nej, omstillingen mente jeg.

\scene{-fald- \act{op ned}}
\end{sketch}
\end{document}
