\documentclass[a4paper,11pt]{article}

\usepackage{revy}
\usepackage[utf8]{inputenc}
\usepackage[T1]{fontenc}
\usepackage[danish]{babel}


\revyname{DIKUrevy}
\revyyear{1987}
\version{1.0}
\eta{$n$ minutter}
\status{Færdig}

\title{Indledning}
\author{KM}

\begin{document}
\maketitle

\begin{roles}
\role{S}[Stig] Stig
\role{T}[Står ikke] Tale- og syngekor
\role{S}[Står ikke] To eller tre stemmer
\end{roles}


\begin{sketch}

  \scene{Når folk er kommet ind og har sat sig, slukkes alt lyset i salen og
    eftersom der ikke er noget lys på scenen er der nu helt mørkt. (Dette kræver
    formedentlig, at vi husker at blænde ovenlysvinduerne før publikum kommer
    ind.)}

\says{En stemme} Nej, hvor er mørkt.

\says{En anden stemme} Ja, her må have været slukket meget længe.

\scene{Nu begynder talekoret:}

\says{T} De't godt nok mørkt, man kan ingenting se, de't godt nok
mørkt. \act{gentages}

\scene{Efter et passende stykke tid høres en stemme: ``Stig, for fanden, gør dog
noget, det er jo dit institut!'' Hvortil Stig replicerer: ``Ja, ja da.  Der
blive lys!'', hvilket der selvfølgelig straks bliver. (Dette kan implementeres
på to måder, enten tænder vi alt lys på scenen, eller også afblænder vi
ovenlysvinduerne -- langsommere, men mere effektfuldt, hvis der er lys nok.) Kan
evt. laves i to tempi}

\scene{Når lyset tændes står Stig midt på scenen og ser sig omkring: ``Neej,
  sikke et dejlig stort kontor, der kunne blive her. Det vil jeg have, når jeg
  bliver professor.''}

\scene{Derefter opdager han publikum og går igang med en præsentation: ``Ja, De
  må virkelig meget undskylde, at der sådan er et par detaljer, der er lidt
  ufærdige, men der jo meget at se til, både med køleanlæg og nyansættelser og
  opsigelser og nu vil vi jo også have løntillæg fra fastholdelsespulkjen, fordi
vi er så populære ude i det private erhvervsliv og der er sandelig meget at se
til som bestyrer, og undervise skal jeg også og man kan jo ikke både blæse og
godaften mine damer ogherrer, og velkommen til DIKU's åbent hus-aften, som vi
afholder for at præsentere vor smukkke nye bygning for den undr.. jeg mener
interesserede offentlighed.'' osv, afsluttes med et populært indslag, nemlig
afsyngelsen af ef sang (gerne med opfordringer til fællessang.}

\end{sketch}

\begin{song}
\scene{Bestyrerens sang. Melodi: Kim Larsen: ``Vi er dem de andre ikke må lege
  med.''}

Vi er dem de andre ikke vil lege med,
vi er det dårlige selskab.
Fakultetet ville ikke gi' os det vi skulle bru' --
de't da godt at vi har fået os en særbevilling nu!

Lige siden DIKU's start
syn's studenter her var rart
fremfor alle andre fag.
Men fordi de skulle spare,
sagde alle andre bare:
``De't da deres egen sag,
sådan er det jo idag.''

Vi er dem de andre ikke vil lege med,
vi er det dårlige selskab.
Men en dag tog vi os sammen og gik ind til DVU,
det er sådan vi har fået os en særbevilling nu!

Vi var ganske desperate,
vi ku' næsten ikke klar' det,
her var ikke til at vær'
og så rejste vores lærer'.
Men så fik vi særbevilling,
så nu er de andre sure
de går rundt med narrehuer

Vi er dem de andre ikke vil lege med,
vi er det dårlige selskab.
Fakultetet ville ikke gi' os det vi skulle bru' --
de't da godt at vi har fået os en særbevilling nu!
ouh jeea
\end{song}
\end{document}
