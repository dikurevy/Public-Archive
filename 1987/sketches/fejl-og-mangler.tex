\documentclass[a4paper,11pt]{article}

\usepackage{revy}
\usepackage[utf8]{inputenc}
\usepackage[T1]{fontenc}
\usepackage[danish]{babel}


\revyname{DIKUrevy}
\revyyear{1987}
\version{1.0}
\eta{$n$ minutter}
\status{Færdig}

\title{Fejl og mangler}
\author{JC efter idé af S(usanne) S(jölin)}

\begin{document}
\maketitle

\begin{roles}
\role{Sek}[Står ikke] Sekretær
\role{Best}[Står ikke] Bestyrer
\role{Ark}[Står ikke] Arkitekt
\end{roles}


\begin{sketch}

\scene{Bestyrerens kontor, hvor bestyrer og sekretær sidder og snakker. Desuden
  medvirker arkitekt og sekretær fra arkitekt-firmaet.}

\says{Sek} Det var da godt din tur til Malmø i denne uge ku' byttes med en tur
til Hawaii om 14 dage, så du ka' være med i dag.

\says{Best} Hva' søren -- man må jo ofre noget, når man sidder på bestyrerposten
-- og med alle de skavanker det her byggeri har, er det jo vigtigt at der er
myndighed bag ved kravene om reparation, når arkitekterne Snøbel kommer til
drøftelse.

\says{Sek} De hedder nu Koppel, men det er da godt de vil se på sagen.

\says{Best} Med den regning ku' det lige se godt ud andet -- man ku' jo ha'
bygget to nye institutter med både vinduer og døre, og så havde der alligevel
været penge tilovers.

\scene{det banker på døren -- sekr. rejser}

\says{Sek} Nu må du prøve at styre din sarkasme \act{Hun åbner døren -- ind
  træder en typisk arkitekt og en sekretær (der ser vigtig ud)}.

\says{Ark} Goddag, goddag -- Ja vi skal se lidt på byggeriets små mangler inden
den endelig aflevering -- men De er vel godt tilfredse ?

\says{Best} Jaa... -- Men der er nu enkelte punkter ...

\says{Ark} Ja er der ikke noget med vinduerne i Nordfløjen .. ?

\says{Best} Jo de er ikke helt tætte -- faktisk sner det ind og temperaturen når
højst op på 5 grader i løbet af dagtimerne.

\says{Ark} Men dog -- men dog. Men sig mig -- åbner De vinduerne ?

\says{Best} Om vi åbner vinduerne ? -- Jah .. -- for at få lidt af Nørre Alle's
friske luft ind -- det sker da sommetider ...

\says{Ark} Ja -- det er jo totalt forbudt -- det kan de sandelig ikke tåle -- så
kan de ikke lukkes igen.  Vinduerne er kun til at se ud af -- det kan De da nok
forstå . Men lad os nu gå videre .. der var noget med toiletterne i samme fløj?

\says{Sek} Ja -- de er næsten konstant forstoppede så vi må rende flere hundrede
meter hver gang vi ska'...

\says{Ark} Sig mig ... -- forretter De Deres nødtørft her ?

\says{Sek}[let forbløffet] Om vi går på toilettet her ? Hvor skulle vi ellers
... ?

\says{Ark} Hvor de ellers skulle forrette Deres nødtørft er da sandelig ikke mit
problem -- det skal bare ikke være her -- afløbet er ført ud i nedløbsrøret fra
taget. Det er jo det rene hærværk at -- øh -- ... -- bruge det.

\says{Best} Jamen....

\says{Ark} Begynd nu ikke på alt det jamen -- her i rummet kan De kun bruge
varmluftsblæseren til hænderne -- vi har netop installeret den med tanke på
temperaturproblemerne i denne fløj. Skal vi fortsætte med edb-kælderen ?

\says{Best} Ja nu er det vist på tide jeg kommer til ! -- Det er jo en skandale
med det køleanlæg -- se denne temperaturkurve -- OP-NED-OP-NED -- som en
parisisk horebule  i højsæsonen -- det er simpelt hen for galt -- maskinerne må
slukkes ...

\says{Ark} Sig mig -- udvikler disse maskiner megen varme når de er i drit ?

\says{Best} eget og meget ... de ku' vel i en snæver vending opvarme en sauno
(hø-hø).

\says{Ark} Så er det jo en indendørs isvinter, De har brug for - dette køleanlæg
er dimensioneret til at kunne sænke rumtemperaturen i et kosteskab 1 grad, hvis
denne i forvejen er under 15 grader. Det virker fortrinligt, når det bruges
efter forskrifterne.

\says{Sek} Jamen...

\says{Ark} Begynder De nu også ... De har jo stadig papir, blyant, og regne
maskine, ikke sandt ? -- Skal vi gå over i trykkeriet ?

\says{Sek} Ja her og i kopirummene har vi problemer med udsugningen -- den
nærmest puster udstødningsgas ind i stedet for at suge dampe ud...

\says{Ark} Selvfølgelig Hvorfor tror de f.eks. at punktudsugningen er sluttet
til en luftkanal, der munder ud i frokoststuen i stedet for i det fri ?

\says{Sek} Ja -- det synes vi jo også er lidt sært ...

\says{Ark} Selvfølgelig fordi den ikke skal bruges -- den er til pynt !

\says{Sek} Jamen så kan vi jo hverken trykke eller kopiere...?

\says{Ark} Jeg har jo allerede en gang påpeget, at blyant og papir er opfundet
og har været brugt til mangfoldiggørelse i århundreder -- og hvis der ikke er
mere ?.... \act{arkitekten er på vej ud, godt stram i masken}

\says{Best}[lidt opgivende] Jo -- så er der de her yderdøre.De har ikke kunnet
låses i flere måneder, og vi venter egentlig blot på at nogen i nattens mulm og
mørke tømmer hele bygningen for inventar.

\says{Ark} Ja -- sådan som De har vanrøgtet Deres rare nye bygning er inventaret
vel næppe værd at rende med, så den risiko er nok minimal. Men vi skal nok sørge
for at få monteret en solid lås, så De og Deres medarbejdere kan forhindres
adgang til bygningen, når de hærværksskader, De allerede har forvoldt, er blevet
udbedret, FARVEL!
\end{sketch}
\end{document}
