\documentclass[a4paper,11pt]{article}

\usepackage{revy}
\usepackage[utf8]{inputenc}
\usepackage[T1]{fontenc}
\usepackage[danish]{babel}


\revyname{DIKUrevy}
\revyyear{1987}
\version{1.0}
\eta{$n$ minutter}
\status{Færdig}

\title{Revynummeret 5xKurt med Brødrene Lisp}
\author{Står ikke}

\begin{document}
\maketitle

\begin{roles}
\role{OK}[Står ikke] Over Kurt. Leder af firmaet 5xKurt. En Forretningsmand med
stort F.
\role{KK}[Står ikke] Kode Kurt. Firmaets hacker. En \emph{rigtig} programmør med
fedtet hår og slidt tøj.
\role{Ku}[Står ikke] Kunden. En kunde fra Sydafrika. Er lidt forsagt.
\role{BrL}[] Brødrene Lisp. Agenter fra Dansk Sprognævn. Er klædt i hvide
bispe-kostumer med $\lambda$-er på hattene og store (hhv.) malet på mave og ryg.
\role{FL}[Står ikke] Franz Lisp (en bror)
\role{KL}[Står ikke] K. Lisp (en bror)
\end{roles}


\begin{sketch}

  \scene{I den ene side sidder KK ved sin terminal. Den er skruet op til højeste
    blus, og blinker ind i mellem vildt. OK står midt på scenen og præseneter
    firmaet.}

\says{OK} Goddag, mit navn er Over Kurt. Som I måske har bemærket hænger der i
kantinen et utraditionelt stillingsopslag fra konsulentfirmaet 5xKurt: Et
dynamisk team af kreative unge medarbejdere. Jeg vil på opfordring af
institutbestyreren fortælle lidt mere om denne virksomhed som er en spændene
jobmulighed for unge nyuddannede dataloger. Jeg vil starte med at præsentere jer
for en af mine trofaste og værdsatte medarbejdere: Kode Kurt.

\scene{Kode Kurt kommer ind med en stor udskrift på armen. OK blader lidt i
  udskriften.}

\says{OK} Det dufter effektivt.

\says{KK} \act{Trækker udskriften til sig}. Hov pas på boss! Det er stærke
sager. Jeg kom til at lade det ligge fremme så en bruger kiggede i det. Han
ligger stadig i respirator på intensive afdelingen. Politimesteren har lige
ringet og sagt, at vi skal huske at få sat de mørklægningsgardiner op i
kodekælderen, og han har givet påbud om at jeg ikke må skrive ud uden at der er
lagt sort klæde over printeren. Det er kode, der virkelig kan flytte noget boss.

\says{OK} Ja, ja, det er godt Kode Kurt.

\scene{KK går over til sin terminal og koder videre. Kunden kommer ind.}

\says{Ku} Jeg vil gerne se på et EDB system.

\says{OK} De er kommet til det rette sted. Hvad drejer det sig om?

\says{Ku} Ja, det er en lidt delikat affære.

\says{OK} Betro dem kun trygt til mig. Konsulentfirmaet 5xKurt behandler alle
sager med største diskretion.

\says{Ku} Jo, jeg skal have ændret lidt på et personregistreringssystem
\act{pause} til Sydafrika.

\says{OK} Aha! Sort arbejde. Den klarer vi. Kom med over til Kode Kurt, så kan
vi snakke om det.

\scene{De går over til KK. Kunden roder i sin taske og finder nogle papirer og
  to store disketter frem. frem.}

\says{Ku}[til KK] Her har De specifikationerne på ændringerne. Her har De
systemet, og her er kildeteksterne.

\says{KK} Kildetekster! Hvad skal jeg med kildetekster!? Giv mig en
karaterorienteret editor og en binær fil og den er fixet om en time.

\says{Ku} Husk nu at vi har apartheid, så oplysningerne om de sorte og de hvide
må ikke blandes.

\says{KK} Den klarer jeg! Vi lægger bare de hvide på den ene side af disketten
og de sorte på den anden.

\says{Ku}[Forsigtigt] Nåh ja.

\says{KK} Det er da skidesmart brormand. Hvis der kommer revolution kan I bare
vende disketten.

\says{OK} Nu er sagen i gode hænder, så nu kan vi gå over og snakke om de
forretningsmæssige sider af sagen.

\scene{OK og Ku går. KK sidder ved sin terminal og kigger på papirerne.}

\says{KK}[flere gange] Det kan jeg da godt kode, ja, ja. Det kan jeg da godt.

\scene{Brdr. Lisp kommer ind og stiller sig bag KK som ) ( . De kigger på
  hinandens paranteser, og skynder sig at bytte plads. KK opdager dem først nu.}

\says{KK} Gisp! Brødrene Lisp!

\says{KL} \act{Tager en sort fan-fold liste op af lommen}. Du er blevet
sortlistet af sprognævnet.

\says{KK}[nervøst] H-hvorfor?

\says{FL} OG det spør' han om? Her tar' vi ham på fersk gerning mens han sidder
og masserer stak-pointeren. (Uden at lukke af for afbrydelser, i parantes
bemærket).

\scene{KL er i mellemtiden gået hen og kigger på skårmen.}

\says{KL} Du, Franz. At se på den kode er som at kigge i en gryde med
spaghetti. Det er bare Vanvittig Vax.

\says{FL} Kurt, du har vist siddet for længe foran din skærm og vendt bit, så du
har helt glemt hvad der sker i den \emph{virkelige} verden. Du har vist glemt at
det er Torsdag i dag. Og hvad sker der hver Torsdag?

\says{BrL}[I kor] Garbage Collection!

\says{KL} Franz, går du ud og henter vores Car?

\scene{FL går ud.} 

\says{KK} Åh nej! Hav medlidenhed. Over Kurt! Hjæælp!

\scene{OK kommer ind}

\says{OK} Hvem er du.

\says{KL} Mit navn er Lisp. K-Lisp. Jeg kommer fra sprognævnet. Du må vist
hellere finde en ny programmør.

\says{OK} Jeg havde ikke ventet at høre fra sprognævnet.

\says{KL} Ingen venter det Danske Sprognævn.

\scene{FL kommer ind med en affaldsvogn.}

\says{FL} Nåh, hopper du selv op, eller skal vi bruge rekursion.

\scene{KK stritter imod, men BRL hægter op i ham og sætter ham op i vognen.}

\says{FL} Det er bare Vanvittig Vax!

\says{KL}[til OK] Nu har vi befriet dig for denne kodende kræftsvults. Find dig
en datalog i stedet.

\scene{BrL kører væk med KK. Ok står alene tilbage midt på scenen.}

\says{OK}[til publikum] Jamen, han \emph{var} jo datalog!

\end{sketch}


\scene{Vanvittig VAX. Indspilles på bånd og spilles under brd. Lisp.
  Forfattere: KL, CDJ. Melodi: Står ikke}
\begin{song}
2000 Mega -- hvad mener du.
Vi ta'r derudaf med UNIX -- nu.
Her p åDIKU, hvor maskinen kø'r.
Sig bare til, hvis du har hørt den før.

Vanvittig VAX!
Vild når disken den er varm.
Vanvittig VAX!
Kernen er grisset og fuld af slam.

Batch-systemer er helt passe.
Vi gør det bedre end UNI*C
og før du når at vende dig om,
har den spist al din kode og filen er tom.

Vanvittig VAX!
Vild når disken den er varm.
Vanvittig VAX!
Kernen er grisset og fuld af slam.
\end{song}
\end{document}
