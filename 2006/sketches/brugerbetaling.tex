\documentclass[a4paper,11pt]{article}

\usepackage{revy}
\usepackage[utf8]{inputenc}
\usepackage[T1]{fontenc}
\usepackage[danish]{babel}

\revyname{DIKUrevy}
\revyyear{2006}
% HUSK AT OPDATERE VERSIONSNUMMER
\version{0.2}
\eta{ca. $6$ minutter}
\status{Næsten færdig}

\title{DIKU som pengeløst institut}
\author{Munter og Madssß}

\begin{document}
\maketitle

\begin{roles}
\role{SB}[Steffen] Institutbestyrer
\role{VIP}[Madss] VIP
\role{S}[Guldfisk] Studerende
\end{roles}

\begin{props}
\prop{Kæmpe Studiekort}[]
\end{props}

  
\begin{sketch}

\scene{Institutbestyrelsesmøde} 

\says{SB}  Velkommen til dette Institutbestyrelsesmøde. Første punkt på
           dagsordenen er det ``pengeløse institut''. Studiekortet
           \act{Hiver et studiekort frem} er fremtiden. Mine herrer,
           vi er håbløst bagefter.
\says{VIP} Ja, men er det ikke et problem at instituttet ikke tjener
           penge på det - jeg mener, de studerende har trods alt
           rigtig mange penge. Det mindes jeg ihvertfald fra min
           studietid.
\says{SB}  Ja, det husker jeg også. Det må kunne udnyttes!
           Det er jo et udbredt problem at de studerende ødelægger
           vores tastaturer med spildt cola. Hvad med brugerbetaling
           på brug af tastaturet?
\says{VIP} Ja! Hvis de studerende nu indsatte et depositum i
           tastaturet før brug, og derefter ødelagde det - ja, så
           ville tastaturet ikke kunne returnere depositummet, og
           dermed er udgiften lagt over på de studerende.
\says{SB}  Fremragende ide, men hvorfor skal de studerende kun betale
           for tastaturet. Det er jo {\em deres} slid der får skærmene
           til at stå af. 
\says{VIP} Man kunne jo lave en tilsvarende løsning på skærmen.
\says{S}   Men så kan man jo ikke skrive og se resultatet på skærmen
           samtidig!
\says{VIP} Visse vasse. I kan lære blindskrift, kan i. Da jeg var ond
           havde vi kun papir og blyant, så det er vel ikke noget
           problem at programmere med kun en af de to tilgængelige.
           Lidt praktisk erfaring med binære semaforer, fint.
\says{SB}  Ja, og ellers må de jo bare danne grupper, og ad den vej få
           adgang til flere studiekort.
\says{S}   Jamen, hvis nu gruppen ikke har mulighed for at mødes så
           ofte...
\says{SB}  Jamen, så er det jo vedtaget.
           Næste punkt drejer sig om vores linjeskrivere. Tidligere i dag
           fandt jeg et magisk +5 sværd i den ene linjeskriver. Teknikeren
           antager et det vil koste et sted mellem mange og rigtig
           mange penge at få den repareret.
\says{VIP} Jeg ser det som endnu en oplagt mulighed for genbrug af
           systemet med terminalerne.
\says{S}   Men hvordan skal man kunne have studiekortet siddende i
           både printeren og det tastatur man udskriver med.
\says{SB}  Den har jo ret. De studerende skal jo have mulighed for at
           udskrive deres rapporter.
\says{VIP} Okay. Så lader vi de studerende betale for de næste N
           udskrifter på forhånd til den enkelte linjeskriver...
\says{SB}  Hvis vi så samtidig reducerer antallet af linjeskrivere, lad os
           bare kalde det ``centralisering'', så er der chance for at
           den indsatte pengemængde på den enkelte linjeskriver overstiger
           værdien af linjeskriveren selv.
\says{VIP} Så et linjeskrivernedbrud ville faktisk give os overskud.
\says{SB}  Med lidt held, eller tilstrækkeligt få af dem, ja.
\says{S}   Helt ærligt. Hvad bliver det næste? Skal vi til at betale
           for at sidde i Store UP1?
\says{VIP+SB} \act{Kigger på hinanden og lyser op} JA!!!
\says{VIP} Selvfølgelig!
           Hvis man ikke deltager i undervisningen, dumper man
           eksamen. En studerende lægger et års depositum, og for hver
           forelæsning man deltager i, for man en del retur.
           Dermed vil de studerende, selv betale for den manglende STÅ.
\says{S}   Du mener vel SID?
\says{VIP+SB} !?
\says{S}   Glem det. Kan i slet ikke se at det er et problem. Hvad nu
           hvis man f.eks. dropper ud?
\says{VIP} \act{Tænker og ser mere og mere glad ud} Det ville jo give
           overskud!
\says{SB}  Jamen, så er det vedtaget. \act{SB og VIP giver hinanden hånden}
\says{S}   Men bliver det ikke dyrt for de studerende med alt den
           brugerbetaling?
\says{VIP} Intet problem. Vi overfører blot deres SU til studiekortet.
           Så undgår vi også at de bruger dem på druk, ovre på Cafeen?
\says{S}   Nej, det kan simpelt hen ikke passe. Jeg skal jo betale
           husleje og alt muligt... hvis det skal være sådan så finder
           jeg et andet institut. \act{Stormer ud af scenen}
           \act{VIP og SB kigger på hinanden}
\says{VIP} Så kan vi jo passende inddrage det overskydene beløb fra
           studiekortet.
\says{SB}  Vedtaget. Næste punkt: Det studenterløse
           institut. Jeg har hørt de havde gode erfaringer med det på
           dansens æstetik og historie.



\scene{Tæppe}

\end{sketch}
\end{document}

% Noter
% inbrudstyv
% printkøer
% Købe ects i automaten/købe for ects

