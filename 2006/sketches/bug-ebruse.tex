\documentclass[a4paper,11pt]{article}

\usepackage{revy}
\usepackage[utf8]{inputenc}
\usepackage[T1]{fontenc}
\usepackage[danish]{babel}

\revyname{DIKUrevy}
\revyyear{2006}
% HUSK AT OPDATERE VERSIONSNUMMER
\version{1.5}
\eta{$n$ minutter}
\status{Færdig}

\title{De 3 bug-ebruse}
\author{hustler, strammet af guldfisk og sandfeld}

\begin{document}
\maketitle

\begin{roles}
\role{K}[Allan] Kaj
\role{A}[Jakob] Andre
\role{L}[Guldfisk] Pawel Köller
\end{roles}

\begin{props}
\prop{Et bord}[]
\prop{Tre stole}
\prop{En børnetime jingle}
\prop{Plastik prinsessekrone}
\prop{kaj-dukke}
\prop{andrea-dukke}
\prop{vand + bro}
\end{props}

  
\begin{sketch}

  \scene{L sidder ved et bord, med K og A på hver sin side. Børnetimejigle
    spiller} <børnetime-jingle>
% 
%   \says{L} Nå, hvad skal vi så lave nu?  Skal vi synge en sang sammen?
% 
%   \says{K+A}[i kor] Neej!
% 
%   \says{K} Det er pissekedeligt at synge, det er hvad det er!
% 
%   \says{A} Og jeg har altså også lidt ondt i halsen, det har jeg altså.

  \says{L} Nå, hvad skal vi så lave nu?

  \says{K} Jeg vil ha' et eventyr

  \says{L} Et eventyr? Hva' med de tre bug-eBruse?

  \says{A} Jaaa! Og så, og så vil jeg være prinseeeesse, det vil jeg altså!

%   \says{K} Narj! Jar ve' ha' den om de tre bug-ebruse, år da'r altså ingen
%   prinsesse i, at du bare ved det!
% 
%   \says{L} Rolig nu, rolig nu. Andreaa kan da godt være prinsesse, selv om i
%   spiller de 3 bukkebruse.
% 
%   \says{A} Ja, der kan du selv se, Kaaaj

  \says{K} Narj, hvis Andrea er prinsesse, så gider jeg ikke være med!
  Der er ingen prinsesse i de tre bug-ebruse.

  \says{A} Arh kom nu Kaaaj, så må du være trolden, så er jeg de tre bug-eBruse.

  \says{K} Ok så, men ikke for meget lyserødt brugervenligt snask, så står jeg
  altså af!

  \says{L} Godt så, hvis i er parate, så begynder jeg at fortælle.  \act{med
    oplæsnings-fortælle-stemme} Der var engang 3 bug-eBruse der skulle ud på
  over-sæteren for at memory-leake sig tykke og fede...

%   \says{A} Jaa, det er mig, det er det altså! Og så skal jeg lige have denneher
%   prinsessekrone på så jeg ligner en riiigtig prinseeese.  \act{fumler med en
%     plastikkrone}
% 
%   \says{K} Ti nu stille til det bliver din tur, man kan jo slet ikke høre
%   historien!
% 
%   \says{L} Ja, få nu fumlet færdig med det der tingel-tangel, så jeg kan komme
%   videre med historien. Hvis vores seer-rating falder, så er det din skyld.  Hvor
%   var jeg nu... Nåh jo: \act{fortsætter med lalleglad fortælle-stemme} Det var en
%   dejlig dag, og de tre bug-eBruse spadserede lystigt afsted mens de nød vejret

  Pludselig kom den første bug-eBruse til en I/O-stream som den var nødt til at passere for at komme til over-sæteren.
   
  \says{A} Naarj altså, jeg vil altså ikke have våde fødder. Så bliver jeg nemlig
  forkøøølet, jeg gør.

  \says{K} Ih altså hvor er du dum, Andrea, der er da en Bridge over
  I/O-streamen.
   
  \says{L} Det er nemlig rigtigt, Kaj.  \act{lalleglad fortællerstemmeleje} Over I/O streamen var der en smal ponton-bro bygget af flydende tals-operatorer.

  \says{A} Jaaa, jaaa, nu er jeg den første lille bug-ebruse, ikke, og jeg går
  over broen \act{imiterer lyd af trampende hest} flip-flop, flip-flop, flip-flop
  \act{snakker videre} og så er du trolden, Kaj, og så...

  \says{K} Jaja, jeg ved godt hvad jeg er.  \act{dyb troldestemme} HVEM ER DET
  DER STEPPER OVER MIT BREAKPOINT?

  \says{A} Breakpoint??? Arj altså Kaj, det siger en trold altså ikke, den gør ej!

  \says{K} Jeg er heller ikke nogen trold, jeg er en debugger at du ved det.

  \says{A} En debugger? Arj Kaj, der er altså ikke nogen debugger i eventyret, der er ej.

  \says{K} Hvis du kan være en prinsesse, så kan jeg altså også være en debugger
!
  \says{L} Ja Andrea, når du er en prinsesse, så kan Kaj også godt være en
  debugger - det er kun fair.

%   \says{A} Men der er altså ingen debugger i eventyret, der er ej.
% 
%   \says{L} Og der er heller ingen prinsesse, så ti nu stille og kom videre i
%   eventyret.
% 
  \says{A} Hmmmmm. Nå ok.  \act{mindste bukkebruse stemme} Det er bare mig, den
  aller mindste bug-eBruse

  \says{K}[troldestemme] HA! Jeg har indkredset dig, og nu RETTER jeg dig!

  \says{A} [mindste bukkebruse stemme] Nej nej nej, det er spild af tid, at du
  bare ved det.  I stedet for at trace mig sku' du meeeget hellere rette min
  bror, der kommer lidt længere nede i koden.  Han er en meeeget større buk...

%   \says{K} Hø hø hø. Ligesom Pawel.

  \says{A} ... som du vil få meeeget mere ros for at rette. \act{andrea-stemme}
%   Hvorfor griner du, Kaj?

%   \says{K} Nååårh, ik' for noget.
% 
%   \says{A} Kaj altså, du'e underlig at du bare ved det.

%   \says{L} Kaj, efter udsendelsen tror jeg liige vi skal have en lille snak om at
%   det faktisk er et BØRNEprogram du optræder i....

  \says{K} [troldestemme] Nåmen du er jo ikke meget mere end en tastefejl, dit
  lille skravl. Skrid med dig, så jeg kan fange en stor buk (hø hø)

  \says{A} [bukkestemme] Ok, så smutter jeg på over-sæteren.  Klip-klop-klip-klop

  \says{L} Og nu kommer den mellemste bug-eBruse

  \says{K} [kajstemme] høhø hø \act{Troldestemme} HA HAAAA, du røg i mit
  breakpoint, og nu kommer jeg og RETTER dig!

  \says{A} Neeej nej, det kan virkelig ikke betale sig for dig at du ved det. Du
  skal meeeget hellere kaste dig over min storebror, for han er nemlig en
  yyyderst kritisk bug, ja han er så, neeemlig, mhmmm.

%   \says{K} Arj altså veddu nu hvad? Helt ærlig, Andrea, hvis du skal være kritisk
%   så gider jeg altså ikke være med længere.
% 
%   \says{A} Narj altså fjolle, det var buggen der var kritisk, ik'
% 
%   \says{K} Nå okæj...  
  \says{K}\act{troldestemme} så skrid da med dig din mølædte bunke liniestøj

  \says{A} Klippetiklopagtig nu slap jeg væk 
  
  \says{L} Og nu kommer den største buk, han tramper og tramper, for han han har så stort et footprint at IO-brigden bliver helt overbelasted.

  \says{K} Hvem er det der edder-hammer-vold-tramper på mit breakpoint???

%   \says{A} Arj altså Kaj, så meget tramper jeg altså heller ikke!
% 
%   \says{K} Med det memory-footprint gør du eddersateme.
% 
%   \says{A} Øh.. ah! Hi hi, hvor er du sjov, Kaj.
% 
%   \says{K} Det' nemmerlig det jeg er
% 
  \says{A}
% Hi hi, ok: 
    \act{BigBB stemme} Det er MIG, den ALLER største bug du nogensinde har set.

  \says{K} HA HAAAA! Nu RETTER jeg dig!

  \says{A} Narj, for jeg er så grundlæggende en designfejl, at du ikke kan nå at
  rette mig inden rapporten skal afleveres.

  \says{K} Ihhh altså, Andrea, det er snyd, det er!  Så gider jeg altså slet ikke
  være med mere.

  \says{A} Arh lad nu vææære, Kaj, du surmuler altså osse bare altid!

  \says{L} Og så levede de tre bug-eBruser lykkeligt til maskinen blev rebootet.

  \scene{Tæppe}

\end{sketch}
\end{document}

%%% Local Variables: 
%%% mode: latex
%%% TeX-master: t
%%% End: 


