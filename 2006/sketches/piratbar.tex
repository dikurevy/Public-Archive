\documentclass[a4paper,11pt]{article}

\usepackage{revy}
\usepackage[utf8]{inputenc}
\usepackage[T1]{fontenc}
\usepackage[danish]{babel}

\revyname{DIKUrevy}
\revyyear{2006}
% HUSK AT OPDATERE VERSIONSNUMMER
\version{1.1}
\eta{$1$ minut}
\status{Færdig}

\title{Pirat bar}
\author{Munter}

\begin{document}
\maketitle

\begin{roles}
\role{F}[Munter] Festboss
\role{A}[Uhd] Ansi-Pirat
\role{P}[Steffen] Pirat
\end{roles}

\begin{props}
\prop{Pirat Kostume}[Person, der skaffer]
\prop{Ansi-Pirat Kostume}[Person, der skaffer]
\prop{klap for øje (samme til bar/køkkenhold)}
\prop{2 x T-shirt (samme til bar/køkkenhold)}
\end{props}

  
\begin{sketch}

\scene{Beskrivelse} 

\says{F} Godaften igen mine ærede publikummer.

\says{F} I kan sikkert huske at jeg for ikke længe siden forklarede konceptet bar vores ølsalg for jer.
         Der er mange der har taget konceptet til sig, og det er vi glade for.
         Imidlertid er vi blevet gjort opmærksomme på at der har været lidt problemer med enkelte individer
         som er begyndt at benytte sig af beer-to-beer.

\says{F} Vi er naturligvis ikke interesseret i at være på kant med loven. Derfor har vi inviteret Ansi-Piratgruppen
         til at holde øje med at alting foregår inden for lovens og licensernes rammer.

\says{A} \act{Kommer ind på scenen} Godaften.

\says{F} Godaften og velkommen til.

\says{A} Jo tak. Hvis jeg lige må sige et par ord. \act{henvender sig til publikum}
         Det er kommet os for øre, både at der benyttes beer-to-beer og at der er forkommet salg
         af piratlicenser på åbning af øl!

\says{A} Dette kan naturligvis på ingen måde tolereres! Personer som forefindes med øl åbnet på
         piratlicens vil blive sagsøgt og eksekveret!

\says{F} Hov hov, det går vist lige lidt ud over de aftalte detaljer.
         Hvis Ansi-Pirat gruppen forefinder folk med øl åbnet på piratlicens, har vi givet
         dem retten til at lukke øllen igen. Og som vi har forklaret tidligere i aften
         skal man op i baren og betale en licens for at få lov at åbne sin øl.

\says{A} Akkurat som jeg sagde.

\says{F} Øh, nej. Men nok om det. Jeg skulle forresten huske at sige at alle øl med patentlåg har forudbetalt
         licens til et vilkårligt antal åbninger.
         Vi håber I har en fortsat god revy. \act{forlader scenen med Ansi-Piraten}

\says{P} \act{Kommer ind på scenen} Er de gået?

\says{P} Godt, så har jeg lige lidt interessant information til jer.

\says{P} Jeg kommer fra Piratgruppen. Vi har besluttet af løsrive os fra patentdiktaturet og levere vaskeægte
         Free Beer!

\says{P} Det betyder at vi åbner jeres øl uden ekstra omkostning, og kan på denne måde frigøre endnu en del af
         de studernedes dagligdag.

         \says{P} Desværre har det været dyrt at fragte alle de øl til
         DIKU. Det er jo ikke småting de drikker her.  Derfor må vi
         tage lidt penge til at dække transportomkostninger, kautioner
         til pirater der ender i fængsel og andre småudgifter.

\says{P} Størrelsen på tilskuddet ses på listerne i baren. Men husk! Det vigtige ved denne operation er:
         FREE BEER! \act{forlader scenen}

\says{F} \act{Kommer ind på scenen. Opdager publikum} Nå så I sidder og glor?
         Hvad venter I på? Der er pause i 30 minutter.

\scene{Tæppe}

\end{sketch}
\end{document}

%%% Local Variables: 
%%% mode: latex
%%% TeX-master: t
%%% End: 

