\documentclass[a4paper,11pt]{article}

\usepackage{revy}
\usepackage[utf8]{inputenc}
\usepackage[T1]{fontenc}
\usepackage[danish]{babel}

\revyname{DIKUrevy}
\revyyear{2005}
% HUSK AT OPDATERE VERSIONSNUMMER
\version{1.næb-r1}
\eta{5 minutter}
\status{Stort set (uret)færdig, $n+3$, kunne godt bruge en punchlime }

\title{Publikumsaktiverende\ldots Sketch}
\author{uffe \&  nokando \&  marvin}


\begin{document}
\maketitle
\begin{roles}
\role{UB}[Allan] Urimelig Revyboss
\role{A}[Madss] Ny Revyt Nummer 1: Alice
\role{B}[Steffen] Ny Revyt Nummer 2: Bob
\role{C}[Uhd] Ny Revyt Nummer 3: Charlie
\role{D}[Jakob] Ny Revyt Nummer 4: Dennis
\end{roles}

\begin{props}
\prop{4 ``rollehæfter'', dvs. nogle ark papir}
\prop{A: Paryk}
\end{props}


\begin{sketch}

\scene{En komplet ufuld scene. UB kommer ind}


\says{UB}[Forklarende] Godaften allesammen. Vi er samlet her for at
finde RevyStjerne\texttrademark for en aften. Vi har ude i kulissen
fire forhåbningsfulde nye revytter, der gerne vil være stjerner. Til
det formål har vi besluttet at underkaste dem en lille test for at se,
hvem der modtager størst hyldest fra publikum. Lad os hilse på
deltagerne. \act{Klapper mod ``indgangen''}.

\scene{A-D kommer ind}

\says{UB}[Vendt mod deltagerne] aftenens opgave er at opføre et
ekstemporalt stykke. I får hver en rolle, og hvert jeres
publikumssegment. Hvert publikumssegment får tildelt en unik
hyldestmetode, så man kan afgøre, hvem de hylder.

Det er naturligvis vigtigt, at de 4 publikumssegmenter er vægtet
retfærdigt, Vi må derfor kompensere for forskelle i \act{remser op}
\begin{itemize}
  \item massefylde
  \item vægt
  \item psykoakustisk potentiale
  \item alder
  \item \ldots og mental resonansprofil.
\end{itemize} 


\says{UB} \act{Kigger overskuende ud over publikum} Lagmignuse\ldots
Gruppe 1 udgøres af personer under 23 år, som vejer under 92 kg.
Gruppe 2 er personer på 23 år eller derover, som vejer under 92 kg.
Gruppe 3 er personer under 23 år, som vejer 92 kg eller derover.
Gruppe 4 er dermed personer på 23 år eller derover, som vejer mindst
92 kg. Forstået?

Godt så! Gruppe 1 skal hylde Alice ved at råbe hurra. Gruppe 2 hepper
på Bob ved at råbe ``skål''. Gruppe 3 stemmer på Charlie ved at synge
``tjikkerlikkertjikkerlikker'', og gruppe 4 støtter Dennis ved at sige
som et næbdyr.

Lad os prøve engang: Alice\ldots Bob\ldots Charlie\ldots og Dennis
\act{Peger ned i salen} Det var ikke et næbdyr!

Men ellers lyder det godt. Aftenens tekst er ``Guldlok og de 3
bjørne''. Alice, du spiller Guldlok \act{uddeler papirer}, Bob spiller
bjørnefar, Charlie: bjørnemor, og Dennis: Lillebjørn. Sæt i værk!

\says{A} \act{Skal til at læse op} Næh, sikke et pænt hus\ldots

\says{D} \act{D taler generelt med en ufattelig belastende
  fistelstemme.} Hov, det er uretfærdigt, Guldlok har jo
uforholdsmæssigt mange replikker i begyndelsen.

\says{UB}[irriteret] OK! Så spiller du Guldlok, Alice spiller
bjørnefar, Bob spiller bjørnemor og Charlie spiller lillebjørn. Og vi
kommer ind i stykket der, hvor Guldlok lige har lagt sig til at sove.
Værsgo!

\says{A} \act{Skal til at læse op} Der er nogen, der har spist af min
grød\ldots

\says{D}[afbryder] Hov, det er uretfærdigt, jeg vil ikke have alle de
fede, gamle dataloger i mit segment. De er allesammen allerede helt
bevidstløse af druk. Desuden tager fordelingen ikke hensyn til køn.

\says{UB} OK, så flytter vi alle kvinder over 25 år under 70 kg fra
gruppe 1 til gruppe 4. Desuden får du mænd mellem 24 og 26 år fra
gruppe 2 og 3. Til gengæld -- for at kompensere -- flytter vi alle
mænd over 25 år fra gruppe 4 til gruppe 1. Så er du fri for de gamle.

Lad os få en lydtest: Alice\ldots Bob\ldots Charlie\ldots og Dennis.

Fint. Fortsæt.

\says{A} \act{Skal til at læse op} Der er nogen, der har siddet på min
stol\ldots

\says{D}[afbryder] Hov, det er uretfærdigt! Du har heller ikke taget
forbehold for hverken uddannelsesmæssige forløb eller økonomisk
status. Det er helt tydeligt, at fattige, uuddannede revygængere ikke
ved hvordan et næbdyr siger.

\says{UB} Det må jeg jo indrømme. \act{vender sig mod A}. Hvad vil du
foreslå?

\says{A}[opofrende] Men jeg kan da godt afstå alle biologistuderende
på 4. og 5. år fra mit segment -- hvis jeg til gengæld får alle
fysikere i salen.

\says{D} Jamen så vil jeg også have alle dataloger med studiejob.

\says{UB} OK. Men kun hvis det ikke er fagrelevant studiejob og hvis
de ikke ender med at droppe ud på grund af for meget erhversarbejde.
Du må også afgive personer, der har opbrugt deres SU til gruppe 2 og
alle, der har bestået 1. del til gruppe 3. Til gengæld får du alle,
der har gået på Nørre Støvring Gymnasium mellem 1994 og '95.
\act{Vendt mod publikum} Fik I det? Lad os høre engang: Alice\ldots
Bob\ldots Charlie\ldots og Dennis.

Det lyder rigtigt. \act{giver tegn til A}

\says{A} \act{Skal til at læse op} Der er nogen, der har ligget i min
seng\ldots

\says{D}[afbryder] Hov, det er uretfærdigt! Der er heller ikke taget
højde for etnisk arv og religiøse tilbøjeligheder. Og iøvrigt kan mit
segment stadig ikke sige som et næbdyr. Og desuden\ldots

\says{B}[afbryder vredt] Nej, nu kan det fand'me være nok. Jeg gider
ikke være med til det her mere. Nu går jeg op i rygerkantinen og
sætter mig på mine briller, så kan jeg ikke se en skid af det hele!
Kom! \act{Tager C med sig ud af scenen}

\says{UB}[befippet] Nå, øh\ldots så må vi jo dele yderligere op.
\act{vendt mod A} Hvad gør vi nu?

\says{A}[Løsningsvillig, konstruktiv, hurtigt] Ja, hvis jeg nu
får alle i gruppe 2, og Dennis får alle i gruppe 3. Så kan vi
kompensere for etnisk baggrund og reliøsitet ved at flytte alle
Jehovas Vidner i salen over til gruppe 4, mens ateister, eskimoer samt
folk af delvis skotsk eller østeuropæisk afstamning kommer i gruppe 1
sammen med katolske sydeuropæere og folk fra Jylland. 

Buddhister og islamister flyttes over i gruppe 4, medmindre de er af
asiatisk eller arabisk oprindelse i 1. til fjerde led. Gruppe 1 kan så
fremover klappe og råbe ``bagermester'', mens gruppe 4 buh'er og siger
``ekiekieki\-tapangsufoiptulr''.

\says{D}[lidt barnlig] Så vil jeg i hvert fald også have alle, der er bosat
på Øster- eller Vesterbro, borset fra dem, som bor i lejligheder til venstre
og på noget tidspunkt har boet sammen med en kat. Du få døtre af
tidligere eller nuværende borgmestre, men så vil jeg have alle, der indenfor
de sidste 3 år har sunget i et kor med over 10 medlemmer eller har gået til
svømning på Amager.

\says{UB}[konkluderende] Altså, for at opsummere: Gruppe 1 skal
generelt bare klappe, råbe og larme så meget som muligt, og gruppe 4
skal hviske Shakespeares samlede værker -- uden fejl, ellers tæller
det ikke. Gruppe 1 udgøres af alle i salen foran eller bagved
midtergangen og Gruppe 4 består så af Morten Wolf -- undtagen når han
er vågen.

%% Erstat gerne nedenstående med en punchlime
Det lader til, at Alice vandt. Farvel \act{Haster ud}

\scene{Tæppe}

\says{D} Hov! Det er uretfærdigt!

\end{sketch}
\end{document}

%%% Local Variables:
%%% mode: latex
%%% TeX-master: t
%%% End:
