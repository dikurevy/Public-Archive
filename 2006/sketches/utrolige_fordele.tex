\documentclass[a4paper,11pt]{article}

\usepackage{revy}
\usepackage[utf8]{inputenc}
\usepackage[T1]{fontenc}
\usepackage[danish]{babel}

\revyname{DIKUrevy}
\revyyear{2006}
% HUSK AT OPDATERE VERSIONSNUMMER
\version{0.1}
\eta{$n$ minutter}
\status{Færdig}

\title{Utrolige fordele}
\author{Allan Sandfeld Jensen}

\begin{document}
\maketitle

\begin{roles}
\role{O}[Munter] Voice-over/Que-man
\role{A}[Uffe] Fransk profezor
\role{B}[Guldfisk] Fransk profezor
\end{roles}

\begin{props}
\prop{Diagram med låger, se sketch}[Skal laves]
\prop{1. Franskmand kostume, med alpehue og stribet trøje}
\prop{2. Franskmand kostume, med alpehue og stribet trøje}
\prop{Raket i skumgummi}
\prop{papirbygning der skal trampes på}
\end{props}

\begin{sketch}

\says{O} Som vi alle ved har rektor i sin uendelige visdom besluttet
at udflytte DIKU til amager for derved at skabe det såkaldte
IT-kraftcenter. Blandt de studerende er dette naturligt nok blevet
kendt som DIKUA og er blevet til genstand for megen polemik. Fra DIKU
revyens side spørger vi blot hvad \'er egentligt fordelene ved udflytningen til DIKUA?

\scene{Tæppet går fra: To tydelige franske professorer stå ved en
overdægget "flip-over". A+B: Gennem hele skecten: Taler meget hurtigt
og fuldstændig menigsløst fransk (for det meste af publikum anyway).
Tempoet øges som sketchen bliver mere fjollet.}

\says{A} Bonsoir - ici nous avons les diagramm precentation d'un di'KUA         
danoi-français. Bernard Trubshawe                                            
\says{B} \ldots
\act{B: Trækker på skuldrene og ser forbløffet ud}
\says{A} D'accord, d'accord. Maintenant, je nous présente min collègue, le      
fattiuag célèbre, Jean-Brian Telepathique.
\act{A: Overfører sit overskæg til B, meget overdrevet}
\says{B} Maintenant, di'KUA \ldots Zee fordelle unh.. Le view.. Voila!          
\act{B: Fjerner tæpper og man kan se diagrammet er en blond storbarmet          
pap-blondine}
\act{A+B: Løfter kokket bennene mens de præsentere diagrammet}
\act{B: Overfører sit overskæg til A}
\says{A} Bon! \ldots Ou sont le terrrminal-rrrum?  ou le interrrnet?                
\act{A+B: kigger på hinanden, løfter skuldre og laver fjollet jeg ved det godt  
men lader som om jeg ikke gør det mimik.}
\says{A} C\'e La!                                                               
\act{A: Afslører triumferende et terminal-rum i blondines barm}
\says{A} Sont le \ldots et la \act{A+B: laver taste bevægelser og
  bevæger hovederne frem og tilbage.}
\act{A: Overfører overskæg til B}
\says{B} Et for ci spise-spise et zommmmerrrfææzt. \act{B: laver sommerfest bevægelser}. \says{B} Ce ici!                                
\act{B: Afslører triumferende en kantine i blondines mave}
\says{B} C'est formidable, n'est ce pas ...                                     
\act{A: nikker}
\act{B: overfører overskæg til A}
\says{A} Ahh. Que il sont ne ce la!                                             
\says{B} C\'e ne la?                                                            
\says{A} Pas le last de' sont: comme de Fysik?
\says{B} comme de Fysik?
\says{A} comme de Fysik?
\says{B} comme de Fysik?
\says{A} comme de Fysik! C'est la Grande Rocketzzz!
\says{B} la Rocketzz!
\act{A+B: Laver affyring af raketlyde og ligeledes raketter der
  destruerer noget. På et tidspunkt trækker B et hus der står HCØ på
  frem og tramper på det mens raketten kastes mod huset.}
\scene{Tæppe}

\end{sketch}
\end{document}
%%% Local Variables: 
%%% mode: latex
%%% TeX-master: t
%%% End: 

