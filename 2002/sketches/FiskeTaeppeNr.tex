\documentclass[a4paper,11pt]{article}

\usepackage{revy}
\usepackage[utf8]{inputenc}
\usepackage[T1]{fontenc}
\usepackage[danish]{babel}

\revyname{DIKUrevy}
\revyyear{2002}
\version{1.0}
\eta{2 min.}
\status{Færdig}

\title{Ørred (tæppenummer)}
\author{Morten Siebuhr}

\begin{document}
\maketitle

\begin{roles}
\role{P1}[Jakob] Person 1
\role{P2}[Niels C] Person 2
\role{VO}[Adam] Voiceover
\end{roles}

\begin{sketch}

\scene Foran tæppet, begge på scenen fra start

\says{P1}[Kigger ud over publikum, så P2, og tilbage på publikum. Siger
med stor overbevisning] Ørred!
\says{P2}[Kigger sig lidt undrende omkring] Ørred???? \act{meget forundret}
\says{P1}[Igen med stor overbevisning] Jaaa, Ørred!
\says{P2} Ørred? \act{Er nu kun vagt forundret}
\says{P1} Ørred!
\says{P2}[Nu er det gået op for ham] Nåå - ØRRED!
\says{P1}[bekræftende] Jaa, Ørred!
\says{P2}[glad] Ørred!
\says{P1}[også glad] Ja. Ørred!

\scene Enten slutter den sort her, ellers kunne man slutte med

\says{VO} DIKUrevyen præsenterer.... en sketch der gik i fisk!

\scene Lys ned, folk ud

\end{sketch}
\end{document}

%%% Local Variables: 
%%% mode: latex
%%% TeX-master: t
%%% End: 
