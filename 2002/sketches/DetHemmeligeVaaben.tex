\documentclass[a4paper,11pt]{article}

\usepackage{revy}
\usepackage[utf8]{inputenc}
\usepackage[T1]{fontenc}
\usepackage[danish]{babel}

\revyname{DIKUrevy}
\revyyear{2002}
\version{1.01}
\eta{7 min.}
\status{Færdig}

\title{Det Hemmelige Våben}
\author{Anders Sewerin Johansen, Niels H. Christensen og Jørgen
  Elgaard Larsen}

\begin{document}
\maketitle

\begin{roles}
  \role{F0}[Sky] Fysiger 0
  \role{F1}[Carsten] Fysiger 1
  \role{F2}[Søren H] Fysiger 2
  \role{SSD}[Niels C] En supersej datalog
  \role{LP}[Maja] En Lækker Pige
  \role{SH}[Per] Steffen Hawking
  \role{HBJ}[Anders] En Holger Bech Jensen
  \role{JK}[Christoffer] En Jyrki
\end{roles}

\begin{props}
  \prop{Armbåndsur}
  \prop{Bæltetaske}
  \prop{2 * Champagneglas}
  \prop{3 * Cowboy-bukser}
  \prop{Cykelhjelm, grim}
  \prop{3 * Fysikker T-shirts}
  \prop{Holger-tøj}
  \prop{2 * Jakkesæt}
  \prop{Jyrkitøj}
  \prop{3 * Hvide kitler}
  \prop{Lækker-pige-tøj}
  \prop{Nørdbriller}
  \prop{3 * Billige øl (ens)}
  \prop{Ølkasse, billig}
  \prop{Palleløfter}
  \prop{3 * Stole}
\end{props}

\scene{Cafeen, tirsdag eftermiddag. Tre fysikere (med fysiknørdede
  t-shirts) er ved at drikke sig ned i øl fra egen kasse.  (Foran
  tæppet).}

\scene{\textbf{teknik}: Når foregående nummer er slut vil de tre
  fysikere flytte ud foran tæppet og tæppet vil gå i efter dem.}
  
\begin{sketch}
  
  \says{F*} [halvtrætte] Skål! \act{De drikker}
  
  \says{F0} Puh, jeg er træt af de her sure discountøl!
  
  \says{F2} Jamen oppe i baren koster de altså mindst... SYV kroner.
  
  \says{F1} Det har jeg altså ikke råd til med mit instruktorjob på
  Fysik-1.
  
  \says{F*} [Drikker igen og sukker]
  
  \says{F0} Man skulle være en af de der dataloger, der sidder oppe i
  baren og kyler Månedens Øl i sig.
  
  \says{F1} Ja, og byder pigerne på dem også!
  
  \says{F2} [Skumler] De kan vel nok...\act{leder efter ord}..
  sagtens.
  
  \says{F*} [Drikker igen og sukker]

  \says{SSD + LP} [Træder ind foran tæppet]
  
  \says{SSD} Hey, var det ikke jer, jeg havde MAT-1 med sidste år?
  Var det fysik i læste? Jeg er ham datalogen, der ikke hoppede over
  på Mat A.
  
  \scene{SSD går hen og hilser. LP hvisker SSD en undskyldning, giver
    ham et kindkys og går igen...sensuelt vrikkende. Lange blikke fra
    F*.}
  
  \says{F0} Var det ikke hende silden fra TV?
  
  \says{SSD} Jo, jeg er lige blevet interviewet af hende. Og så fik
  jeg lidt ``andre ydelser''\ldots
  
  \says{F1} Er det champagne, I drikker?
  
  \says{SSD} Jah, det er jo tirsdag.
  
  \says{F2} [peger] Er det ikke sådan et Rolex?
  
  \says{SSD} [pinligt berørt] Jo, der fik du mig, men det skal jo også
  bare vise, hvad klokken er...
  
  \says{F1} [Surt] Det er Planck'me uretfærdigt, at alle I dataloger
  er så heldige. Det er os fysikere, der er de seje. Det er os, der
  har bygget verden.
  
  \says{F2} Ja, og atomer og fotoner og plutonium og.. og.... sådan
  noget!
  
  \says{F*} [Ser meget selvtilfredse ud, nikker på en "tag den"-agtig
  måde]
  
  \says{SSD} [trækker på skuldrene] Tja, hvis det kan gøre Jer glade.
  
  \says{F0} [bryder sammen og kaster sig på knæ foran SSD, hulker
  næsten] Nej. Hvordan gør I det? Tag min usle fysiker-sjæl, men sig
  mig hvordan I gør!
  
  \says{F1 + F2} [Følger efter, griber bedende i hvert et bukseben.]
  
  \says{F2} Vi vil være ligesom datalogerne. Se bare på den sidste fysik-revy!

  \says{F1} Vi vil have penge, magt og succes. Hemmeligheden!
  Hvad er hemmeligheden?  (Er det bunden? Er det fyldet?)
  
  \says{SSD} [Kigger mildt ned på tilbederne og aer så patroniserende
  en af dem over hovedet] OK da. Først og fremmest skal I bare være
  seje.
  
  \says{F*} [finder febrilsk noteshæfter frem, noterer som gale,
  mumler] Være seje!
  
  \says{SSD} Det kan måske virke uoverkommeligt for sådan nogle som
  Jer, men det skal nok gå. Det andet punkt er, at I skal kunne noget,
  som der er nogen, der vil betale mange penge for..
  
  \says{F*} [noterer, mumler] Kunne noget...
  
  \says{SSD} Ja, eller i hvert fald skal I kunne lade som... det gør
  jeg. Det hjælper, hvis man rakker ned på populær software som
  MicroSnot eller legetøjsmaskiner som iMac.
  
  \says{F*} Rakke ned...
  
  \says{F0} [Håbefuldt] Er det virkeligt alt?
  
  \says{SSD} [Griner] Nej, jeg tager røven på Jer. I virkeligheden er
  der et hemmeligt våben, der har givet alle dataloger penge, magt og
  succes!
  
  \scene{\textbf{band}: Trommehvirvel}
  
  \scene{\textbf{teknik}: Tæppet går fra. Bagved sidder SH}
  
  \says{F*} [Meget overraskede] STEPHEN HAWKING?!?!?!?!?!?
  
  \says{SH} [med maskin-stemme] Nej nej nej! STEFFEN Hawking -
  Stephens onde tvillingebror.
  
  \says{F1} Nix! Der er kun en Stephen Hawking \act{Smiler henført}
  
  \says{F2} Ja!
  
  \says{SH} Ja, han er god til at tage rampelyset. Men i virkeligheden
  er det mig, der er genial.
  
  \says{F0} Sniksnak, Stephen Hawking har sagt en masse kloge ting
  
  \says{SH} [smiler] Hvem tror du, har programmeret hans taleboks? I
  virkeligheden er han en savlende grøntsag. Jeg har kodet det hele -
  tal om kunstig intelligens...

  \says{F1} Jamen "Hawkings Univers" er da helt vildt sejt!
  
  \says{F2} Ja, det er for eksempel en bog og sådan, ikk'
  
  \says{SH} I har hørt den med en million aber og Shakespears samlede
  værker? Den bog kom direkte fra /dev/random.
  
  \says{F0} Jamen, teo\em{ri}erne? Han har fundet på en masse ting!
  
  \says{SH} Det var altsammen mig.
  
  \says{F1} Også ormehuller og sorte huller?
  
  \says{F2} Og sorte ormehuller?
  
  \says{SH} Jeg opfandt dem. Jeg kaldte dem symlinks og /dev/null.
  
  \says{F0} [rystet, mere og mere desperat] Og kemi! Syrer og baser?
  
  \says{SH} "All your base are belong to us"

  \says{F0} "Hvad med transistoren?"
  
  \says{SH} "Jeg opfandt den. Jeg kalder den 'Steffen Hawking
  Transistoren'"

  \says{F1} Hvad med computeren?

  \says{F2} Og bip-bip-spil?
  
  \says{SH} "Jeg opfandt det. Jeg kalder det datamat og PDA"

  
  \says{F0} [Tænker lidt, liver op, trumfer] Ha! Men hvad så med
  Holger! Ham kan du i hvert fald ikke toppe! Han har opfundet
  vacuumbomben!!!!
  
  \says{HBN} [Træder ind, skriger uhæmmet:] Hvis man SER!!! På det
  YDERSTE ATOM!!!! På en ELEFANTS HAAAAAAAALE!!!!!!!

  \says{SSD} Nu i nævner det....
  
  \says{JK} [Træder ind, og mumler på halvsvensk:] Om man
  forrrestillllar sik en enorrrrm parrralllel datamaskin, hvorrrpä man
  kjörer programmet detta\ldots
  
  \says{SSD} ...tillad os at præsentere Jürki. Eller som han
  \em{faktisk} hedder: Halger Jürki Beck-Katajainen
  
  \says{F0} [Forsigtigt] Jamen... vil det så sige, at han har lavet
  alt det, som kommer fra Holger?
  
  \says{SSD} [Overdrevet uskyldig, stort smil] Ja ja, det er klart.
  \act{blinker overdrevent konspiratorisk til publikum}
  
  \says{F2} [Bedende] Ååååååårh, hvad. Må vi ikke nok få ham? I har så
  mange...
  
  \says{SSD} [Lader som om han tænker længe og så overgiver sig] OK.
  Tag ham bare... \act{griner ondt til publikum}
  
  \scene{F* trækker afsted med JK. HBN og SSD bliver tilbage, går frem
    foran tæppet, som trækkes for}
  
  \says{HBN} ER de nu også RIGTIGT gået?
  
  \says{SSD} [Spejder efter F*] Ja, de er væk
  
  \says{HBN} [Tager cykelhjelmen af, tørrer sveden af panden, taler nu
  helt normalt] Pyha, tak for hjælpen. Jeg troede aldrig, jeg ville
  slippe af med de fjolser!
  
  \says{SSD} Det var så lidt. Man kan da ikke byde anstændige
  mennesker at omgås den slags.
  
  \says{HBN} Hvordan kan jeg nogensinde takke dig?
  
  \says{SSD} [Tager HBN om skulderen, de går sammen ud af scenen] Jo,
  ser du, vi har jo længe manglet en ordentlig institutbestyrer\ldots
  
  \scene \textbf{teknik}: Men SSD og HBN forlader scenen vil en boss
  komme ind og fortælle om pausen. Træk tæppet for bag vedkommende.

\end{sketch}
\end{document}

%%% Local Variables: 
%%% mode: latex
%%% TeX-master: t
%%% End: 
