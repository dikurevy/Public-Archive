\documentclass[a4paper,11pt]{article}

\usepackage{revy}
\usepackage[utf8]{inputenc}
\usepackage[T1]{fontenc}
\usepackage[danish]{babel}


\revyname{DIKUrevy}
\revyyear{1973}
% HUSK AT OPDATERE VERSIONSNUMMER
\version{1.0}
\eta{? minutter}
\status{Færdig}

\title{En forelæsning i Mat4=Dat0}
\author{?}

\begin{document}
\maketitle

\begin{roles}
\role{F}[] Forelæser
\role{E}[] Elev
\end{roles}

\begin{sketch}
\scene{F kommer ind med en kasse og mappe under armen.  Lægger det på bordet (med ryggen mod tilhørerne)}

\says{F}[vender sig om, smiler bredt] Daav.  I dag skal I så høre
2. del af forelæsningen om mobile databærende medier.  Jeg vil lige
kort minde om, hvad vi hørte om sidste gang.

\scene{Bladrer i mappen, og finder 3 transparenter frem.  Lægger en
  på.}

\says{F} Kan I se?  Skal jeg slukke lyset?  Nå, I ved jo hvad der skal
stå.

\scene{Hiver transparenten væk, lægger i hurtig rækkefølge de to andre
  på.}

\says{F} Så er vi vist fremme hvor vi slap, og kan gå videre.  Det var
hulkort jeg skulle fortælle lidt om.  Et hulkort ser således ud.
\act{Fremdrager et hvidt hulkort} Det kan dog også se således ud.
\act{Et rødt do kommer frem} Eller sådan. \act{De resterende farver}

\says{F} Fælles for disse kort er dimensionerne, det er meget vigtigt,
at alle kort har samme størrelse.  Målene er: (det må jeg hellere
skrive op): bredde 82,5mm, længde 187,3mm.  Materialet er naturligvis
også vigtigt, det skal være manillapapir med en tykkelse på 0,175mm.
Jeg håber, I noterer det ned, der kan jo komme spørgsmål i det, til
multivalgsopgaven.!

\says{F} Disse kort bærer informationen i huller.  En anden form for
kort er de såkaldte dualkort, stregmarkerede kort, hvor man sætter
streger i stedet for at lave huller.  Systemet er ganske enkelt: En
streg sådan, og vi har dette \act{peger}, to streger giver dette
\act{peger igen et ubestemt sted hen}, og endelig tre streger -- dette
\act{peger påny}.  Nu kender I alle systemet, og jeg håber I vil
benytte Jer flittigt af det.  Desværre er kortene blevet trykt skævt,
men hvis I flytter stregen en halv kolonne, skulle det passe.

\says{F} I udartede tilfælde kan hulkort ogs se således ud:
\act{holder et mini-hulkort frem}, men det har kun perifer interesse,
dem vil I aldrig møde, i hvert fald ikke {\em herinde}.

\says{F} Vi må videre med de gængse hulkorttyper.  De opbevares i
kasser af denne type \act{løfter kassen op}, og den har de indvendige
mål: 82,5 * 187,3 * 358,4m.  En hurtig udregning giver \act{skriver på
  tavlen}, at der så kan være 2048 kort i en sådan kasse.  Længden er
naturligvis nøje beregnet, idet 2048 netop er $2^{11}$.

\says{F} Fyldt med uhullede kort vejer en sådan kasse 4,813kg.  Regner
vi med, at der i hvert kort hulles 48 kolonner á 2 huller, vil en
kasse fyldt med hullede kort veje 4,690kg.  Ja, der var et spørgsmål?

\says{E} Hvis man nu regner med, at af de 2048 kort er der 98 røde
jobkort, hvor meget vil kassen så veje?

\says{F} Ja, det var et spændende spørgsmål.  Lad os lige regne en
gang.  Hmm, ja, jo, kassen vil blive ca. 5 gram tungere.  Besvarer det
dit spørgsmål?

\says{E}[skriver] Hmm--øhh\act{ser op} ja ja.

\says{F} Er det nu dual-kort, der er i kassen, bliver situationen jo
en ganske naden.  en kasse med umarkerede kort vejer det samme som en
kasse med uhullede kort, nemlig 4,813kg.  Er kortene derimod
markerede, oad os sige gennemsnitligt 73 streger pr. kort, bliver
kassen 21.7 gram {\em tungere}, hvilket naturligvis skyldes
blyantstregernes vægt.  Ja, du har et spørgsmål igen?

\says{E} Hvilken type blyantstift er lagt til grund for denne
vurdering?

\says{F} Viking nr. 2.  Er der flere spørgsmål?  Nåh ikke.  Jeg vil nu
afslutte med nogle simpe beregninger over gennemsnitshastighed for
udførelse af et monitorkal, som svar på et spøgsmål fra sidste gang.
Det er ikke noget I skal kunne, men jeg bryder mig ikke om at sige
noget, I ikke forstår. \act{Skriver og mumler, vender sig om, står et
  øjeblik tavs og stirrer} Ih, hvor er I stylle, det virker som om I
ikke forstår et klap af det hele.  Nåh tiden er vist gået.  Næste gang
skal I høre om ikke mobile databærende medier.
\end{sketch}
\end{document}
