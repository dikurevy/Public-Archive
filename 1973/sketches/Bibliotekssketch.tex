\documentclass[a4paper,11pt]{article}

\usepackage{revy}
\usepackage[utf8]{inputenc}
\usepackage[T1]{fontenc}
\usepackage[danish]{babel}


\revyname{DIKUrevy}
\revyyear{1973}
% HUSK AT OPDATERE VERSIONSNUMMER
\version{1.0}
\eta{? minutter}
\status{Færdig}

\title{Bibliotekssketch}
\author{?}

\begin{document}
\maketitle

\begin{roles}
\role{A}[Jan] Lille, forlegen fyr
\role{B}[Lissi] Bibliotekar
\role{C}[Arne] Bibliotekar
\role{D}[Mette] Bibliotekar
\end{roles}

\begin{sketch}

  \scene{De tre første vers af sang nr. 1 synges.  I indgangsdøren
    står sangerne, hver med en kaffekande og de to sidste vers af sang
    nr. 1 synges.  Derefter sætter alle sig ned og "`drikker kaffe"'.
    En lille, lidt forlegen fyr (A) kommer ind ad føren, han tror han
    er gået forkert og er ved at gå igen, men spørger så:}

  \says{A} Undskyld, er dette ikke biblioteket?

  \scene{En person der i forvejen sidder og drikker kaffe føler sig
    tvunget til at svare.}

  \says{B} Jo (det er), det står jo udenfor på skiltet!

  \scene{B peger henad mod døren.  A kommer nu helt ind ad døren, ser
    sig meget søgende omkring (det er første gang han er der).}

  \says{C}[spørgende] Er der noget du ikke kan finde?

  \says{A} En datalogs endeligt.

  \says{B} Hvem har skrevet den?

  \says{A} Peter Naur.

  \says{B} Den står... DER.

  \scene{B peger et ubestemmeligt sted hen.  A prøver desperat at
    finde ud af hvor B pegede hen.}

  \says{A}[en lille smule nervøst] Hvor?

  \scene{B rejser sig meget langsomt og går hen til bogen og peger
    præcist på den plads, hvor han tror den står.  B ser sig nu
    forvirret om i lokalet og spøger.}

  \says{B} Hvor er En Datalogs endeligt?  Sidst jeg så den stod den her.

  \says{D} Nå den, jamen den flyttede jeg forleden dag.  Den står på
  den 3. sidste hylde i 2. række, der var mere plads.

  \scene{Bogen bliver nu langt om længe fundet frem og A er på vej ud
    af døren.}

  \says{C} Har du husket at udfylde lånerkortet?

  \says{A} Lånerkortet?

  \says{D} Ja, det der sidder på sidste side i bogen.

  \says{A} Hvad skal der stå på det?

  \says{B} Det står på den røde liste henne ved døren.

  \says{A} Må jeg låne en blyant?

  \says{B} Ja, skidt med det.

  \scene{A finder kortet frem og skriver på det.}

  \says{C} Det var godt, og så skal du bare have en tom lomme.

  \says{A} En tom lomme?  Til hvad?

  \says{D} Til at sætte kortet ind i, selvfølgelig.

  \says{A} Hvor får jeg sådan en?

  \says{B} Derhenne.

  \says{A} Hvad skal jeg gøre med den?

  \says{B} Hen på bogens plads.

  \scene{A får en tom lomme i hånden og lægger kortet ind i den.  C
    tager kort og lomme ud af A's hånd og sætter det på bogens plads.
    A er i mellemtiden på vej ud af døren.}

  \says{B} Hov stop engang.  Har du lånt her før?

  \says{A} Nej.

  \says{B} Jamen så skal du også skrive dig på den grønne liste der
  hænger henne ved døren.

  \scene{A skynder sig hen til døren, skriver sig på listen og farer
    ud af døren.  Alle falder en smule til ro igen, efter at have sat
    en ny låner ind i systmet.}

\end{sketch}
\end{document}
