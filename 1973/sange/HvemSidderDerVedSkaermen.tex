\documentclass[a4paper,11pt]{article}

\usepackage{revy}
\usepackage[utf8]{inputenc}
\usepackage[T1]{fontenc}
\usepackage[danish]{babel}

\revyname{DIKUrevy}
\revyyear{1973}
\version{1.0}
\eta{? min}
\status{Færdig}

\title{Hvem Sidder Der Ved Skaermen}
\author{?}
\melody{``Jens Vejmand''}

\begin{document}
\maketitle

\begin{roles}
\role{S}[] Sanger
\end{roles}

\begin{song}
Hvem sidder der ved skærmen
med strimler om sin hånd
med hulkort op til halsen
og om sin sko et bånd?
Det er såmænd en dat'er
som sidder dagen lang
ved skærmen og forvandler
sit Algol 5 til Slang.

Og vågner du en morgen
i allerførste gry
og ser at skærmen skriver
på ny, på ny, på ny
det er såmænd vor dat'er
som på dig sindssygt glor
og taster vildt og blodigt
ved terminalens bord

Og vandrer du til stuen
hvor alt er fint i stand,
og møder du en stakkel
hvis øjne står i vand, --
det er såmænd vor Dat'er
som kommer dig på tværs
og ikke mer' kan finde
en fejl i line halvfjerds

Og vender du tilbage
i byger og i blæst,
mens aftenstjernen skælver
af kulde fra nordvest
og klinger terminalen
bag ryggen ganske nær
det er såmænd vor dat'er
som endnu sidder dér.

Dog dataskærmen jævned
for ham den slemme vej,
men da det led mod julen,
da sagde skærmen nej;
det var såmænd vor dat'er
hans fingre slap den brat,
de bar ham væk fra stuen
en kold december-nat.

Der er på opslagstavlen
et gammelt grønnet kort
hvor hulningen er ussel
og skriften ganske bort
Det tilhørte vor dat'er
som aldrig gav et gny
men på hans plads ved skærmen
der sidder nu en ny.
\end{song}

\end{document}
