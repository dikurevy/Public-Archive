\documentclass[a4paper,11pt]{article}

\usepackage{revy}
\usepackage[utf8]{inputenc}
\usepackage[T1]{fontenc}
\usepackage[danish]{babel}

\revyname{DikuRevy}
\revyyear{2024}
\version{1.0}
\eta{$3$ minutter}
\status{Næsten Færdig}

\title{Ung Dolk}

\melody{Ung Kniv af Minds of 99}
%angiver originalmelodien på formen 'kunstner: ``titel '

\begin{document}
\maketitle
\section*{Noter}
De to sangere skiftes til at synge linjer, og afslutte hinandens sætninger i omkvædene og slutningen af sangen.

\begin{roles}

\role{S1}[Lise] Sanger 1
\role{S2}[Kristoffer] Sanger 2
\role{X}[Anna Liv] Instruktør
\role{D1}[Asger] Danser 1
\role{D2}[JoJo] Danser 2

\end{roles}
%Liste over roller og deres indehavere.

\begin{props}
    \prop{Rekvisit} FysikRevys Lejerbål
    \prop{Rekvisit} Store Omvendte træer
    \prop{Rekvisit} By i baggrund
\end{props}
%Liste over rekvisitter. Behold teksten [Person, der skaffer],
%indtil det er sikkert, hvem der skal have ansvaret for rekvisitten

\begin{song}

\sings{S1}
Åh, skal vi snitte en pind?
De kan være svære at find' (Medmindre du er ud' i en skov??)
Men hver gang, jeg tror
Det' her, jeg bor
Skal jeg tilbage til byen igen
Med kniv, og dolk
med liv't, i behold
lugter, af bål
elsker, snobrød

\sings{S2}
Tror, jeg drager ud igen
savner skoven, den er min ven
Det' vores jord
Her spejdere bor
Slår et telt op, snitter en pind
Med kniv, og dolk
med liv't, i behold
lugter, af bål
elsker, snobrød

Nyt liv, i skov
ude, på rov
Med kniv, og dolk
med liv't, i behold
tror jeg nok

\sings{S2}
Åh, jeg har snittet en pind
\sings{S1}
De kan være svære at find'

Nu er, jeg klar
Til sno'-, brød på
en pind, på bål
snorbrød, på en pind

Med kniv, og dolk
med liv't, i behold
med liv't, i behold
tror jeg nok
snorbrød, på en pind
på et bål, snorbrød
snorbrød, på en pind, 
på et bål

Snobrød, helt brændt
Snobrød, helt brændt
Snobrød, helt brændt
Snobrød, stadig brændt
Stadig brændt

\scene Lys ned
\end{song}

\end{document}