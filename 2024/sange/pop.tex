\documentclass{article}
\usepackage{revy}
\usepackage[utf8]{inputenc}
\usepackage[T1]{fontenc}
\usepackage[danish]{babel}

\revyname{DIKUrevy}
\revyyear{2024}
\version{1.0}
\eta{3:00}                      
\status{5/7}                   % ÆNDR HÉR

\title{Pop!}
\author{Lotte Bruun, Lise Bruun, et al.}
\melody{Boing! af Nik \& Jay}

\begin{document}                
\maketitle
\section*{Noter}
Bemærk at sidste vers springes over, og der gås direkte til andet omkvæd. Andet vers slutter en takt for tidligt i YouTube-versionen.

Vers kan skrives lidt sjovere. Svær rap. Meget stærkt omkvæd. Åbner en akt. Replikken med håret tilbage er måske ikke fantastisk til Schilling. Omkvædets ``ned-og-ned'' kan lave hype-prep hvor man går ned i squat, og ``op-op'' kan hype publikum med armene. Omkvædet skal måske tunes (fx ``giver \textit{dig} exceptions''), så der er bedre harmoni og frem-og-tilbage. Erstat måske ``som en fjeder'' med ordet ``elevator'' (skud ud til nordfløjen)?

Der er god inspiration at hente fra Nik \& Jay's liveoptrædender.  De er lækre.  De bruger mange ad-libs og publikum-klap over hinanden.  De snakker ind over introen (klar til 3. akt?).  Meget vibe af at have en fest på hele scenebredden, og lige overfor hinanden, meget hip, hop og pop.  De har 3 æraer til kostumedesign.

% https://www.youtube.com/watch?v=rN5BBBkot4s&t=3708s
% -> 3 dansere ind fra første omkvæd. Førførende whip-koreografi.

Når stakken giver overflow, vælter den på scenen!  Sidste omkvæd har harmonering i originalen, prøv at matche det.  Publikum kan evt. synge ``Ååh-nej!'' i omkvædet. Fest og rytme er vigtere end tone.

\begin{roles}
\role{R}[Schilling] Rus
\role{V}[Filippa] Vejleder
\role{X}[Morten] Instruktør
\role{D}[Caitlin]{Danser}
\role{D}[TBD]{Danser}
\role{D}[TBD]{Danser}
\end{roles}

\begin{props}
    \prop{Stak} Flere kasser der let kan stables og væltes. Måske sammensat med snor for at gøre livet nemt. Man kan evt. genbruge dem fra Akt~I, 6.~Samtalen.
\end{props}

% \newpage%
% \begin{sketch}
% \says{A}[råbende] Skriv replik her, eller slet hele sketch-environment.
% \scene{(Her er en scenebeskrivelse. \textbf{A} bevæger sig mod publikum.)}
% \says{A} \act{Her gør en skuespiller noget.}
% \end{sketch}

% Insert divider
% \smallskip\hfil\rule{6cm}{0.1mm}\medskip\par
% \clearpage

\newpage

\begin{song}
[Intro]
\sings{R}%
Whoop! Uh-huh
Whoop! Uh-huh
Pop! Pop!
Whoop! Uh-uh
Yesss---
Okay!

[Vers]
\sings{R}%
Jeg er ny på DIKU, I ser det tydligt,
Har fået håret tilbage, glatbarberet så nydeligt.
Oh yeah, jeg gør det godt,
Der kun drenge omkring mig, men får penge på mit kort

% Bar 13, 14, 15, 16
\sings{R}%
Læser om en stak, det mega nice hva'
Jeg var til forelæsning men jeg tror, jeg sov?
Drøm't om en Scratch kat til forelæser'n river mig i armen igen
Så vild at nogen burde tæmme den!

% Bar 17, 18, 19, 20
% Der skiftes sanger i originalen
\sings{R?}%
Damn, upti-vupti
Musen i hånden og cola fra Coca
Jeg ser det for mig nu: genaflevering
Instruktor observer', sukker skuffet

\sings{V}%
% Kigger på dem, siger stakken er forkert
    Kigger på dem, siger stakken den er forkert
% Jeg råber ``what is this shit,'' tror du, du er ekspert?
    Råber \textit{what} is this \textit{shit}, tror du \textit{du} er eks\textit{pert}?
\sings{R}%
Skal jeg gå, bli', måske bare DNUR?
Nej, jeg slammer videre, når min kode ikke dur

[Bro]
\sings{R}%
Stakken er lang, og instruktor sig mig:
Jeg har alle mine pushes her, det lig'som en leg
Ohh-Ohh.
Men ved du hvor'n man popper dem, ved du hvor'n man popper dem?
Ohh-Ohh.
Men ved du hvor'n man popper dem, ved du hvor'n man popper dem?
Du siger:
\sings{V}%
Baby lad mig pop dem for dig, lad mig pop dem for dig

[Omkvæd]
\sings{V}%
Push! Pop! Stakken den går op og ned som en fjeder
Pop! Pop! Og den går ned og ned og ned og ned og
Push! Push! Stakken stiger, kan næsten ikke mere
\sings{R}%
den kaster mig exceptions, stakken gir mig overflow.

% For at holde vers-boksen sammen med sangen.
\newpage

[Vers]
% Bar 43, 44, 45, 46
\sings{R}%
Stakken var så høj
Når instruktor træder ind, sig'r de bare føj
\sings{V}%
Tænker for mig selv, ``hva' fanden er dét?''
Aldrig har jeg set en stak, der kunne vælte

% Bar 47, 48, 49, 50
Du er selvfø'li' skrald, rus
Forstår slet ik, din kod den er ren sjusk
Så jeg stikker ham en lussing
Sig'r det genaflevering rusling

% Bar 50.4, 52, 53, 54.
Stakken mangler, koden sejler
% Næste linje kan sløjfes, hvis man ovenfor skriver "Stakken den mangler, og din kode den sejler"
Det er skrevet i Scratch
Jeg' uden ord for en stund
\sings{R}%
Det' det objektivt bedste programmeringssprog

% Bar 54.4, 55, 56, 57
\sings{V}%
Smiler for, damn, de er rus
Jeg ka' se dem kæmpe i PoP
De har det godt
Så de kan DNUR når CompSys kommer

[Bro]
\sings{R}%
Stakken er lang, og instruktor sig mig:
Jeg har alle mine pushes her, det lig'som en leg
Ohh-Ohh.
Men ved du hvor'n man popper dem, ved du hvor'n man popper dem?
Ohh-Ohh.
Men ved du hvor'n man popper dem, ved du hvor'n man popper dem?
Du siger:
\sings{V}%
Baby lad mig pop dem for dig, lad mig pop dem for dig

[Omkvæd] $\times$ 2
\sings{V}%
Push! Pop! Stakken den går op og ned som en fjeder
Pop! Pop! Og den går ned og ned og ned og ned og
Push! Push! Stakken stiger, kan næsten ikke mere
\sings{R}%
den kaster mig exceptions, stakken gir mig overflow.

[Outro] \textit{(Optional, ikke skrevet færdig!)}
\sings{V}%
% Du kan boink, du kan bounce, ..-..-
    Du kan \texttt{try}, du kan \texttt{catch}
% Du kan gøre, hvad du vil ..-..-
    Du kan gøre, hvad du vil
% Når beatet det banger én gang til .-..-.-.-
    Når du genafleverer én gang til (i)
% Boing, boing! - -
    PoP, PoP
\sings{R}%
% Aldrig føltes så godt, når du spørg' hvad man vil ..-..-..-..-
    Jeg vil bare lave spil, træt af SU-gæld
\sings{R, V}%
% Nexus, Nik og Jay, Joe n' the Juice baby -.-.--..--.
    Kom så, tredje akt, skål i UP-1!

\end{song}

% Du vil måske opdele i vers, bro (pre-chorus), omkvæd, og kontraststykke (c-stykke), men det er ikke et krav.
% Det vigtige er, at dem, der skal opføre det printede manus, kan finde den information, de søger.
\end{document}