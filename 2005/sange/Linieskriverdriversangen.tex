\documentclass[a4paper,11pt]{article}

\usepackage{revy}
\usepackage[utf8]{inputenc}
\usepackage[T1]{fontenc}
\usepackage[danish]{babel}

\revyname{DIKUrevy}
\revyyear{2005}
% HUSK AT OPDATERE VERSIONSNUMMER
\version{1.0}
\eta{$n$ minutter}
\status{Færdig}

\title{Linieskriverdriversangen}
\author{Uffe og Uffe Productions}
\melody{Dyrene i Hakkebakkeskoven ``Peberkagebagesangen''}

\begin{document}
\maketitle

\begin{roles}
\role{S1}[Ulla] Sanger 1
\role{S2}[Madss] Sanger 2 
\end{roles}

\begin{props}
\prop{}
\end{props}

  
\begin{song}
\scene{2 sangere der skiftes og afbryder hinanden for at rette på den anden}

\sings{S1}Når en skriverdriverskriver
skriver linieskriverdriver
skal en skriverdriver skrive
linieskriverdrivprogram.

For det hedder ikke driver
nej, det' bare nog't man skriver
men bar' fordi Microsoft de skriver
driver bliver driver ikke nødvendigvis korrekt, vel?

\sings{S2}Hvis en lagerlærer lærer
andre lagerlærer's lære
bør der's lagerlærdom lagres
i et lærelagerlag.

Men det er jo så'n med lærer
at de ik' ka' lagre lære
for de kager rundt og tager
jo slet ikke hinanden alvorligt, vel?


\sings{S1}Når processor'n processerer
og processerne bli'r flere
og professor'n ik' ka' 'pere
fler' processer i hans RAM. \act{S1: det hedder internt lager!}

Ja, så må vi allokere
fler' studerende med mere
lager end der's lagerlærer
har i lagerlærerlag.

\sings{S2} NEJ NEJ NEJ!!! Det var sidste vers... Det var \emph{mit} vers!!! \act{stamp stamp}

\sings{S2}Ja, så må vi allokere
fler' studerende med mere
tid til at studere negre
så de ka' forny mit professorat! Ikke sandt?


\sings{S2}Når din vektorlektor nægter
at en sektorvektorhægter
peger på en sektorvektor
er din vektorlektor gal.

For en sektorvektorhægter
hægter kun til sektorvekt're
og det' nytter ik' at lektor
nægter hægters eksistens.

\sings{S1} Eh, du mener\ldots

\sings{S1} For en sektorvektorhægter
hægter kun til sektorvekt're
selvom alle vor's indtægter
mistes når en vektorlektor møder vektorlektorinspektoren, ikke?


\scene{Tæppe}

\end{song}
\end{document}

%%% Local Variables: 
%%% mode: latex
%%% TeX-master: t
%%% End: 

