\documentclass[a4paper,11pt]{article}

\usepackage{revy}
\usepackage[utf8]{inputenc}
\usepackage[T1]{fontenc}
\usepackage[danish]{babel}

\revyname{DIKUrevy}
\revyyear{2005}
% HUSK AT OPDATERE VERSIONSNUMMER
\version{0.6$\epsilon$}
\eta{$n$ minutter}
\status{Færdig}

\title{Undervisere der selv svarer på spørgsmål}
\author{Uffe$^{2}$, André, NP \& marvin}

\begin{document}
\maketitle

\begin{roles}
\role{U}[Uffe] Underviser
\role{C}[Ulla] Censor
\role{S}[Madss] Studerende
\role{S2}[Sandfeld] Anden studerende

\end{roles}

\begin{props}
\prop{Noter (et par ark papir)}
\prop{Eksamensbord}
\prop{2 stole}
\end{props}

  
\begin{sketch}

  \scene{Scenen er et eksamenslokale. Underviser og censor sidder ved det grønne
    bord. Ind kommer den studerende.}

  \says{S}[Skal lige til at sige goddag] \ldots

  \says{U} Og du må så være\ldots Åge Jensen\ldots

  \says{S} Æh\ldots \act{prøver at sige at han ikke hedder Åge}

  \says{U} \ldots og du er så kommet op i\ldots \act{roder lidt i sine
    noter}\ldots Dijkstra.

  \says{S} Øh\ldots \act{prøver at sige at han ikke skal op i Dijkstra}

  \says{U} Nå, Åge, Dijkstra\ldots Hvad er det?

  \says{S}[meget konfus, kommer til at kigge i sine papirer] \ldots

  \says{C}[farer op] Hov! Ikke noget med at stå og glane i noterne hele tiden.
  Hvad tror du de er? Er det en sutteklud? Er du ikke kommet ud af kravlegården
  eller har du altid brug for den slags mentale krykker til din psykiske
  svagheder? Har du\ldots

  \says{U}[afbryder, til C] Tag den nu med ro. Lad nu Åge svare\ldots

  \says{S} Øøøh\ldots jamen jeg er jo\ldots

  \says{U} Jeg forstår godt at du er nervøs. Dijkstra er jo ikke det nemmeste
  spørgsmål du kunne trække. Altså lad os starte ved begyndelsen\ldots Du ved
  Dijkstra\ldots Det er et spørgsmål om at finde korteste vej fra et punkt
  til\ldots

  \says{S} Ah\ldots

  \says{U} \ldots alle andre punkter, ja\ldots Og hvordan gør man så det?

  \says{S} \ldots Jo, ok, man tager\ldots

  \says{C} KANTEN MED LAVEST VÆGT, JA, det ved vi godt, kom nu videre!

  \says{S} Æææhæææhææh\ldots er det ikke den\ldots

  \says{U} Ja, nu du har den første knude, så fortsætter man med den næste og så
  videre, indtil man har dækket det hele, som du jo selvfølgelig ved. Og hvad
  står man så med?

  \says{S} Øhhh\ldots med det mindst\ldots

  \says{U} \ldots udspændende træ, selvfølgelig. Og mere er der vel egentlig ikke
  i Dijkstra, vel?

  \says{S} Æhhh\ldots

  \says{U} \ldots nej, nemlig. Men hvad kan man så bruge sådan et mindst
  udspændende træ til?

  \says{S} Æh jo, man kan\ldots

  \says{C} BRUGE DET TIL AT PLANLÆGGE OPTIMALE NETVÆRKSTOPOLOGIER, JA, klart, det
  ved alle, det står jo i bøgerne, men kunne du ikke komme med et andet eksempel,
  for en gangs skyld?

  \says{S} æææhææhæhææh\ldots man kunne jo også bruge det til\ldots

  \says{C} JA MEN DET VAR SÅ IKKE LIGE DET \emph{JEG} TÆNKTE PÅ, VEL? Kom nu!

  \says{S} \ldots ok, altså, så er det vel\ldots

  \says{U} Nå, vi skal også huske liiiige at komme hele vejen rundt i
  pensum\ldots Så\ldots Hvad kan du fortælle os om bubble sort? Du ved, den der
  hvor elementerne i en liste "bobler" sådan forbi hinanden indtil listen er helt
  sorteret og hvor man bare bliver ved indtil listen ikke ændrer sig længere. Du
  ved den med kvadratisk køretid og konstant forbrug af lager\ldots Kan du ikke
  give os nogle gode grundfakta omkring den?

  \says{S} ææhææh\ldots

  \says{C}[peger på sit ur og kigger på underviseren]

  \says{U} Nå, men vi er vist ved at løbe tør for tid. Ellers noget du vil
  tilføje?

  \says{S} \ldots \ldots\ ldots \ldots

  \says{U} Godt så, gå du blot udenfor et øjeblik.

  \scene{S rejser sig og går ud}

  \scene{tæppe for og fra}

  \scene{S kommer ind igen}

  \says{C} Velkommen ind igen, Åge\ldots

  \says{S} æh, jamen jeg\ldots

  \says{C} \ldots jeg ved ikke helt hvordan jeg skal starte, men\ldots

  \says{U} DET VED JEG!!! DU ER SIMPELTHEN DEN MEST UDUELIGE TÅBELIGT DEBILE DYBT
  UBRUGELIGE UNDSKYLDNING FOR EN STUDERENDE JEG NOGENSINDE HAR HAFT TIL
  EKSAMEN!!!!! Sig mig, hvordan kom du overhovedet ind på datalogi? Vandt du din
  plads på lotteriet for mentalt "udfordrede"? FOR DET FØRSTE så vidste du jo
  INTET omkring Dijkstras algoritme, som jo ligesom var hele hovedspørgsmålet, så
  ALLEREDE DER ryger du jo helt ned under et sekstal. OG SÅ FIK du heller intet
  sagt omkring BUBBLE SORT\ldots hallo! det er fucking BUBBLE SORT!!! HVOR SVÆRT
  KAN DET VÆRE????

  \says{S} \ldots \act{paf}

  \says{U} KOMPLET TÅBELIGE UDUELIGE SINDSSVAGE JEG VED IKKE HVAD\ldots DET ER
  FOLK SOM DIG DER FÅR MIG TIL AT FATTE SYMPATI FOR MANGE AF NAZISTERNES IDÉER!
  HVIS DOKTOR KEVORKIAN VAR HER KUNNE DU FÅ MEDLIDENHEDSDRAB, MEN DET VILLE VÆRE
  EN ALT FOR MILD STRAF FOR DIG. DU ER EN AF DE TYPER DER SKULLE VÆRE SORTERET
  FRA ALLEREDE SOM BEFRUGTET ÆG, NEJ ENDNU TIDLIGERE, DIN FAR SKULLE VÆRE
  KASTRERET!!!!

  \says{C} Årh tag den nu roligt, hans noter er da meget pæne\ldots

  \says{U} AAAAAAAAAAAAAARGHRGHRGRHGRHGRHGRHGRHG!  DE HANDLEDE JO IKKE ENGANG OM
  DIJKSTRA! UD MED DIG, UD AF MIT KLASSELOKALE, UD AF MIN EKSAMEN, UD AF MIT
  KURSUS, UD AF MIT DIKU FOR HELVEDE!!!!!

  \scene{S træder ud foran tæppet idét det bliver trukket for.}

  \says{S}[står og ser paf ud]

  \says{S2}[kommer ind fra siden] Næh hov davs der, har du lige været oppe i
  funktionsprogrammering?

  \says{S} Æh, nej\ldots Det er først i morgen\ldots Jeg skulle bare på
  toilettet\ldots Men der er en der hedder Åge Jensen der vist nok lige har
  dumpet algoritmik.

  \scene{Lys ud}

\end{sketch}
\end{document}

%%% Local Variables: 
%%% mode: latex
%%% TeX-master: t
%%% End: 

