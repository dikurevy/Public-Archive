\documentclass[a4paper,11pt]{article}

\usepackage{revy}
\usepackage[utf8]{inputenc}
\usepackage[T1]{fontenc}
\usepackage[danish]{babel}

\revyname{DIKUrevy}
\revyyear{2005}
% HUSK AT OPDATERE VERSIONSNUMMER
\version{1.0}
\eta{$4$ minutter}
\status{Færdig}

\title{Kohoved kodekaos}
\author{Uhd og Uffefl}

\begin{document}
\maketitle
  
\begin{roles}
\role{HK}[Uhd] Hovedkoder Konrad
\role{CHK}[Ulla] Co-hoved koder Kaj
\role{K1}[Madss] Koder 1
\role{K2}[PeterJo] Koder 2
\end{roles}

\begin{props}
\prop{kaffekande}
\prop{assorteret papir}
\prop{4 stole}
\prop{bord}
\end{props}

\begin{sketch}
\scene{HK, CHK og K1+2 sidder omkring et bord, med HK for bordende og 
ansigt mod publikum.}

\says{HK}[rømmer sig] Velkommen til dette konsoliderende møde. Som I alle ved,
har vores konsortium, Konrad \& Co. modtaget en consession på at kode en virtuel
ko. Mit navn er Konrad og normalt fungerer jeg som hoved-koder, og min kollega
Kaj\ldots

\says{CHK}[entusiastist] Ja, og det er mig. Jeg er vores co-hoved-koder. Vi føler
dog at koordineringen af et så stort projekt har krævet yderligere kræfter. Jeg
vil derfor gerne byde velkommen til vores to kodere \act{nikker mod K1 og K2} som
kommer fra en af vores konkurrenter, og som er købt til at hjælpe med Projekt
Kokoding.

\says{K1}[lavt] Tak tak.

\says{K2} Ja, vi kan knapt vente med at komme i gang.

\says{CHK} Nuvel. Som det første må jeg konstatere at det konceptuelt giver mest
mening at starte med styreenheden til koen -- eller koens hoved, om man vil. Jeg
foreslår at lade Konrad stå for dette.

\says{De andre}[afdæmpet nikkende] Enig enig

\says{HK} Godt, dermed bliver jeg Kohoved hoved-koder for Projekt 
Ko-kodning.

\says{K1} Undskyld, vil det så betyde at vi \act{peger på sig selv og K2} 
bliver konstitueret som Kohoved kodere?

\says{HK} Hmm... ja, det må vi vel konstatere er korrekt.

\says{CHK} Hov hov. Da må jeg insistere på titel at Kohoved Co-hoved 
koder, da man jo nødigt skulle forveksle mig med en \act{let nedladende} 
"almindelig" Kohoved koder.

\says{K2}[fortørnet] Her må jeg lige kommenterer at Hr. Kohoved co-hoved koderen
da ikke kunne kode en kat uden os kohoved kodere!

\says{HK} Ja, og det er nu også meget lidet kollegialt at du kommer til at have
en længere titel end mig. 

\says{K1} [rejser sig] Jeg henter lige en kande kolbekaffe. \act{ud}

\says{CHK} Ha! Det er dog kosteligt komisk. Så Hr. kohoved hoved-koderen er i
konspiration med et konkurrende kompani istedet for kollegaere.

\says{HK} Oprigtigt talt! Beskylder du mig for at være korrupt? Det vil jeg
forelægge til næste bestyrelseskonference!

\says{CHK} Korrupt? Ne, den kobling havde jeg ikke kundgjort. 

\says{K2}[undrende] Kan konsortiet virkeligt klare kokodningsprojektet med så
knapt kompetente kommandanter ved roret?

\says{HK}[afbrydes af K2s kommentar] Ja, lad os nu komme tilbage på ret kurs,
inden der går komplet koks i vores lille kobbel af kodere her.

\says{K1}[kommer hastende ind med en kande kaffe, tydeligt oprørt]
Katastrofe. Kantinevagten er blevet kavlt i et kaffefilter.

\says{HK}[upåvirket] Hmmm... må jeg kondolere? Så må vi klare os uden koffein.

\says{K1}[sætter sig] Javel. Kunne vi kort klarlægge vores \act{peger på sig selv
  og K2} kontante kompensation for kodningen?

\says{HK} Lad os blot sige at den vil være komfortabel. 

\says{CHK}[rejser sig]Lad os så komme i gang.

\scene{de andre rejser sig og begynder at gå ud}   

\says{K1}[lavt til K2] Snakker cheferne så sjovt i samtlige deres samtaler her?

\says{K2} Sandsynligvis. Sikke et sært selskab!    


\end{sketch}
\end{document}

%%% Local Variables: 
%%% mode: latex
%%% TeX-master: t
%%% End: 


