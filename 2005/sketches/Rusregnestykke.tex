\documentclass[a4paper,11pt]{article}

\usepackage{revy}
\usepackage[utf8]{inputenc}
\usepackage[T1]{fontenc}
\usepackage[danish]{babel}

\revyname{DIKUrevy}
\revyyear{2005}
% HUSK AT OPDATERE VERSIONSNUMMER
\version{0.5}
\eta{$n$ minutter}
\status{Færdig}

\title{Rusregnestykke}
\author{Uffe og Uffe}

\begin{document}
\maketitle

\begin{roles}
\role{P1}[Uffe] Dr. Rubbish Quatsch
\role{P2}[Uhd] Dr. Bullocks Bl\"odsinn
\end{roles}

\begin{props}
\prop{Tavle}
\prop{Militærhjelme (2)}
\prop{Solbriller (2)}
\prop{Hvide kitler (2)}
\prop{Beskyttelsesbriller (2)}
\prop{Lommeregnere (2)}
\prop{Navneskilte (2)}
\prop{Små blå mænd (266,04392)}
\prop{Små røde mænd (5,04392)}

\end{props}

  
\begin{sketch}

\says{P1} Hej igen. 

\says{P2}[vinker] Hej!

\says{P1} Vi er blevet inviteret for at forklare jer hvorledes de optagne på datalogi gennemfører...

\says{P2} ..eller ikke gennemfører...

\says{P1} studiet.

\says{P2} Vi starter med de kolde hårde facts: Der starter 200 studerende på datalogi.  Heraf møder 20 aldrig op, dermed er vi nede på 180.

\says{P1} Af de 180 har 60 ikke været på rustur hvilket får over halvdelen til at føle sig udenfor, og medfører et frafald på yderligere 31.

\says{P2} Med 149 studerende starter vi så blok 1. Funktionsprogrammering har obligatoriske ugeopgaver, der med lethed kvæler lysten til datalogi hos de første 27 studerende.

\says{P1} Og Matematik og beregninger sørger for at få yderligere 13 til at falde fra.

\says{P2} Blok 2 starter derfor med hele 109 studerende. En klar forbedring i forhold til tidligere år hvor vi slet ikke kunne måle hvor mange studerende der var tilbage i starten af blok 2.

\says{P1} De fleste studerende glæder sig nu til jul og får derfor ikke fulgt ordentligt med i blok 2. Til gengæld sørger julefrokoster for at stort set alle glemmer at falde fra, mens glæden ved at tage på DIKUski også holder på folk. Alt i alt mistes kun 2,712\% af de resterende studerende i oversættelsen fra blok 2 til blok 3.

\says{P2} Hov, der er et spørgsmål fra salen\ldots

\scene{Kort pause}

\says{P2} ``Hvad mener vi med `oversættelsen' fra blok 2 til blok 3?''

\says{P1} Jo, det er jo klart at mellem to blokke sker en overgang, hvad der i akademikerkredse kaldes en transition. Dette kan betragtes som en translation, i et passende rum. Fra sprogteknologi kender vi begrebet translation som fortolkning eller med passende partiel evaluering ``en oversættelse''.

\says{P2} Nårh\ldots{} Så de studerende møder friske og veludhvilede op til blok 3 hvor Styresystemer og Multiprogrammering har\ldots

\says{P1}[trykker lidt på lommeregneren] \ldots 106,04392

\says{P2} spændte tilhørere i auditoriet. I Algoritmer og datastrukturer finder flere studerende ud af at løsningen på ``korteste vej fra penge til mig''-problemet ikke går via DIKU, hvorfor 34 vælger job frem for studie.

\says{P1}[har kigget undrende på lommeregneren lige siden sidst] Hvad, er 0,04392 studerende? En arm? Eller måske kun en hånd?

\says{P2}[lidt irriteret] Æh\ldots{}

\says{P1}[glad] Nå, vi er under alle omstændigheder nede på 72,04392 studerende. Efter blok 3 vælger SU-kontoret at fjerne støtten for 22 tilfældigt udvalgte studerende som nu må tjene til føden som kodeprostituerede.

\says{P2} Hmmm\ldots{} Faktisk bliver 0,04392 studerende til en hel studerende på næsten \act{finder sin lommeregner frem} næsten 23 år!

\says{P1} Ah! Det må være den hypotetiske specialestuderende jeg har hørt det postuleret der er hos Gregers Koch.

\says{P2} Ja, så der er \act{lommeregner} 50 \act{i kor med P2} ,04392 \act{alene igen} studerende tilbage.

\says{P1} Og her ser vi et problem i vores datagrundlag. Kigger vi
tilbage kan vi nemlig se at hele 70 studerende bestod Dat0 under den
gamle semesterstruktur. Og nu er vi allerede efter blok 3 nede på 50 \act{i kor med P2} ,04392 \act{alene igen} studerende\ldots

\says{P2} Ja, men det må da være en målefejl. Jeg foreslår vi korrigerer med en faktor 2.

\says{P1} Godtaget. Så vi har altså 100 \act{i kor med P2} ,04392 \act{alene igen} studerende efter blok 3.

\says{P2} De kan nu vælge at lade være med at skrive 1. årsprojekt eller at undlade at bestå Databasekurset som har en eksamen der afholdes på en ukendt dag i et hemmeligt lokale med randomiseret pensum.

\says{P1} Under alle omstændigheder fører det til et frafald på 2/3.

\says{P2} Så vi er altså nu nede på 33 \act{i kor med P2} ,04392 \act{alene igen} studerende, der rent faktisk kan gå til eksamen.

\says{P1} Og her har vi jo problemet: blok 4 eksamen er jo ikke overstået endnu i år, så vi har ikke data til denne udregning.

\says{P2} Så synes jeg bare vi skal bruge dem fra sidste år, det er jo sammenlignelige tidspunkter.

\says{P1} Ja, lad mig se \act{slår op i noget} sidste år dumpede 38
studerende Dat0, så det efterlader os med -5 \act{i kor med P2} ,04392
\act{alene igen} studerende.

\says{P2} Æh\ldots{} \emph{minus} 5?

\says{P1} Ja, så vi korrigerer på magisk +5 vis og ender altså med 0
\act{i kor med P2} ,04392 \act{alene igen} studerende der gennemfører første år på DIKU.

\says{P2} Hov, vi korrigerede jo kun for 1 kursus ifht. sidste år. Så vi mangler de 7 andre, der jo nu er.

\says{P1} Nåh, så du mener der må være \ldots 7 gange 38 studerende flere der gennemfører?

\says{P2} Jaja! Så vi har faktisk \act{lommeregner} 266 \act{i kor med P2} ,04392 \act{alene igen} studerende der gennemfører.

\scene{Kort pause mens P1 og P2 indser resultatet}

\says{P1} Fantastisk!

\says{P2} Fænomenalt!

\says{P1} Vi har altså en rekordhøj gennemførselsprocent på 133 \act{i kor med P2} ,04392 \act{alene igen} procent!

\says{P2} Fantastisk!

\says{P1} Fænomenalt!

\says{P2} Vi må konkludere at den ny blokstruktur er en bragende succes.

\says{P1} Nogle spørgsmål fra salen? Nej? Tak for i aften.

\scene{Tæppe}

\end{sketch}
\end{document}

%%% Local Variables: 
%%% mode: latex
%%% TeX-master: t
%%% End: 



