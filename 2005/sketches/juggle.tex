\documentclass[a4paper,11pt]{article}

\usepackage{revy}
\usepackage[utf8]{inputenc}
\usepackage[T1]{fontenc}
\usepackage[danish]{babel}

\revyname{DIKUrevy}
\revyyear{2005}
% HUSK AT OPDATERE VERSIONSNUMMER
\version{1.0}
\eta{$n$ minutter}
\status{Færdig}

\title{Surt show}
\author{peterjo}

\begin{document}
\maketitle

\begin{roles}
\role{J}[PeterJo] Jonglør
\role{H}[Madss] Hjælper med stort krøllet hår
\end{roles}

\begin{props}
\prop{3 jonglørsække med palietter}[]
\prop{2 æbler}
\prop{1 citron}
\end{props}

  
\begin{sketch}

\scene{Ved tæppe står J på scenen.} 

\says{J} Jeg vil fortælle en hverdagshistorie fra DIKU. Men først skal
I lære at jonglere. Lad os se på syntaksen. Jonglering er en uendelig
løkke.  Den grundlæggende algoritme er en afløsning sammen med sin
symmetriske version.

\scene{ Demonstrerer en afløsning}. 

\says{J}Kroppen består af en afløsning og dens symmetriske. Precondition er 2
bolde i højre hånd og 1 i venstre hand. Invarianten er at ingen bolde
tabes. Skulle - mod forventning - en afbrydelse indtræffe og en bold
falde på gulvet, skal forsamlingen sige BUUUH.

\scene{ Taber en bold}

\scene{ BUUUH}

\scene{ Under det følgende jonglerer J en kaskade}

\says{J}En morgen kommer jeg til DIKU, går op ad indgangstrappen og gennem
hall'en. 

\scene{ Skifter til underhåndskaskade} 

\says{J}Det er mandag morgen

\scene{Tager sig til hovedet}

\says{J}og jeg må tage elevatoren op.

\scene{ Skifter til op-ned med venstre hånd}

\says{J}Oppe på 2den sal inde i kantinen ser jeg to studerende diskutere

\says{J} [Skifter til tennis-mønster]

\says{J} [Skifter til kaskade]

\says{J}Jeg sætter mig ved min PC og begynder på at finde den lus i mit
program som irriterede mig hele dagen i går.


\says{J} [Skifter til snap-mønster, 1 snap hver tredie afløsning]


\says{J}Til at begynde med går det lidt trægt.

Men efterhånden går det bedre. Nu er jeg lige ved at have fundet den. 

\says{J} [1 snap hver anden, og derefter hver eneste afløsning]

\says{H} [kommer uset af J ind på scenen og overrasker ham med
sin hilsen] HEJ \ldots


\says{J} [Taber en sæk som bliver liggende på gulvet]

\scene{ Publikum råber BUUUH\ldots}

\says{J} Aaaaarghh. Nu har du ødelagt det hele. Hvor blev min
lus  nu af?

\says{H} Skal jeg hjælpe dig?

\says{J} NEJ! \ldots og dog.  

\scene{ Går hen til H og tager en virtuel lus ud af hans hår.
  Jonglerer med de to bolde og den virtuelle lus, slutter med at gribe
  den. Kigger sig omkring for at finde et sted at anbringe den. sætter
  den tilbage i H's hår}

\says{J} Henter du lidt at spise?

\says{H} [Henter to æbler og en citron. Rækker J to æbler,
  et ad gangen, og derefter - med et satanisk grin - en citron]

\says{J} Jonglerer de to æbler og citronen, bider først forgæves efter
et æble, så forgæves efter citronen. Tager så en bid af den ene æble,
derefter en bid til. Tredie gang bider han i citronen. Tager to æbler,
et i hver hånd så de dækker øjnene og med citronen i munden ser han ud
mod publikum.

\says{H} Surt show

\scene{ Tæppe}

\end{sketch}
\end{document}

%%% Local Variables: 
%%% mode: latex
%%% TeX-master: t
%%% End: 




