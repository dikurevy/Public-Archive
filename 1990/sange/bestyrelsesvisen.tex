\documentclass[a4paper,11pt]{article}

\usepackage{revy}
\usepackage[utf8]{inputenc}
\usepackage[T1]{fontenc}
\usepackage[danish]{babel}


\revyname{DIKUrevy}
\revyyear{1990}
\version{1.0}
\eta{$n$ minutter}
\status{Færdig}

\title{Bestyrelsesvisen}
\author{Ukendt}
\melody{Ukendt}

\begin{document}
\maketitle

\begin{roles}
\role{S}[Jacob] Sanger 
\end{roles}

\begin{song}
    \sings{S} Jeg er mig, der ham
              der er ham, der er mig
              og hvis jeg det er mig - ikke dig.
              Hvis en anden var mig,
              ku' det godt være dig,
              der var mig, hvis jeg ikke var mig.
              Der er klart det er rart
              jeg gik hen i en fart
              og blev ental i første person.
              Det er trist for enhver
              der har fødselsbesvær
              og går rundt og er slet ikke nogen

    \sings{S} Vor bestyrer han styrer
              os alle, så ingen,
              de styrer hvorhen de nu vil.
              Hvis de selv vil bestemme,
              så bli' de kun hjemme,
              eller gør som bestyreren vil.
              Men at være bestyrer,
              til den ringe hyre,
              det er såmænd ikke så sjovt.
              Så, hvis Jens ikke bare
              ku' få lov at bestem'.
              Ja, så var det altså for flovt.

    \sings{S} For at sidde hver uge,
              ja sidde og muge,
              blandt alt hvad de blot sender frem.
              Og at fremsende alt,
              man hver uge har luget,
              og fundet er slet ikke nemt.
              Men at luge og finde,
              og muge og sende,
              Papirene frem til P.JO.
              Er skam bare en byrde,
              men Jens er vor' hyrde
              - og Eric så nu er de to.

    \sings{S} Ja, for nu er det Jul,
              Det er slet ikke Winter.
              Pawel venter nok til det bli'r jul.
              For når julen den kommer,
              er det ikke sommer,
              og så er Sand gået i skjul.
              Men med alle de lærere
              er det sværer' og sværer'
              at klare den nye førstedel.
              Vi har mangen en Koch,
              der ved mere end nok,
              Jørgen er bare anden fidel.

    \sings{S} Vores ny studienævn,
              de tager meget snart hævn,
              førstedelen er stadig for stor.
              Femogtyve studerende,
              de gjorde i bleerne
              og nu er de hjemme hos mor.
              Men far Bertel han siger,
              vi mangler ej piger,
              men ungkarle dem må vi få.
              Så nu skal der studeres,
              og stoffet skal læres,
              hvis du vil på andendel nå.

\end{song}

\end{document}

