\documentclass[a4paper,11pt]{article}

\usepackage{revy}
\usepackage[utf8]{inputenc}
\usepackage[T1]{fontenc}
\usepackage[danish]{babel}


\revyname{DIKUrevy}
\revyyear{1990}
% HUSK AT OPDATERE VERSIONSNUMMER
\version{3} % 3. udkast, 6/6 1990
\eta{$n$ minutter}
\status{Færdig}

\title{Vi havde det hårdt}
\author{Henrik Damborg}

\begin{document}
\maketitle

\begin{roles}
% Medvirkende: Jolanta, Ole, Tine, KDJ
% Rollebesætning ukendt
\role{A}[] Halvgammel, halvskallet 2.-dels-studerende 
\role{B}[] Halvgammel, halvskallet 2.-dels-studerende 
\role{C}[] Halvgammel, halvskallet 2.-dels-studerende 
\role{D}[] Halvgammel, halvskallet 2.-dels-studerende 
\end{roles}

\begin{sketch}

\scene{De fire gamle venner sidder i DIKU-kantinen og sludrer om "De Go'e Gamle Dage".}

\scene{Lys: Lys på alle fire, næsten mørkt omkring dem.}

\says{A} Næh, Ha Ha - Det er virk'li noget isenkram, der VIL noget, den hér nye arbejdsstation.

\says{B} Ja, der er altså ikke noget som en SPARC, vel Bertel?

\says{C} Nej, det har du SANDELIG ret i, Ulrik.

\says{D} HA! - Hvem sku' for ti år siden ha' troet, at vi en dag ville sidde (her) og køre på en SPARC-xxxxxxx [kodenummer] med
         \act{opremsning af vilde, men korrekte data}.

\scene{A får et vemodigt udtryk i øjnene}

\says{A} Næh (SUK!) - Dengang var vi glade, hvis vi bare fik adgang til en god, gammel mikrodatamat.

\says{B} En OTTE-bits-mikrodatamat!

\says{D} Uden HARD-disc!

\says{C} Eller RAM! \act{evt. "Ja, og uden RAM!"}

\scene{De andre kigger let overrasket, A fortsætter efter grinepausen}

\says{A} Med et lille snasket tastatur, hvor halvdelen af tasterne manglede!

\says{D} Når vi kørte på PDP 11/4I, havde vi ikke en gang et tastatur - Vi skrev programmerne på hulkort med en hullemaskine -
         Et ORDENTLIGT Monstrum - hvis den altså virkede; ellers måtte vi selv lave hullerne - med en stoppenål!

\scene{De øvrige måber, B fortsætter i en blanding af vantro og misundelse}

\says{B} Havde I hulkort??? - Vores PDP-11'er havde kun en række kontakter, hvor vi vippede programmerne ind \act{viser med hænderne}
         - sådan BIT for BIT! Det gik langsomt, men vi blev ret gode til maskinkode.

\says{C} Men ved I hva'? - Vi var nu lykkelige dengang, selv om vi havde det hårdt.

\says{A} Ja, FORDI vi havde det hårdt! Som min gamle professor altid sagde: "Det er IKKE LAGER-STØRRELSEN, det kommer an på".

\scene{Pause for latter}

\says{D} Fuld-komment rig-tigt! Jeg var lykkelig dengang, og jeg havde INTET. Vi sad 128 mand i et lille snusket rum og
         deltes om en PDP-8 med 8 k-RAM!

\scene{B er forbavset}

\says{B} SAD I NED?? Vi stod op i en LAANG kø ude i gangen, og så tastede vi en instruktion, hver gang vi passerede maskinen \act{pause}
         Så vi fik lært at lave fejlfrie programmer - \act{med foragt i stemmen} Vi havde sandelig ikke brug for at bevise dem.

\says{C} Jamen, I havde da i det mindste en RIGTIG bygning - I vores skurvogn var der så fugtigt, at vi fik stød på kabinettet - men det tog vi os
         ikke af - vi blev hærdede FORTRAN-programmører.

\says{A} \act{hånligt} Hmpff! Vi drømte om at skrive FORTRAN (-programmer), men det var et uopnåeligt mål (SUK !). Vi måtte skam pænt nøjes med
         at skrive en Relations-database i COBOL!

\says{D} \act{mere hånligt} COBOL - Hah! Vi sad med vores PDP-8 og en sort/hvid-monitor og skrev et CAD/CAM-program i MASKINKODE!

\says{B} \act{overrasket} Sagde du en MONITOR?

\says{D} Ja, netop

\says{B} Hvor Var I Heldige! Vi havde kun et sort-hvidt TV, hvor billedrøret var gået \act{pause} - så vi måtte sende uddata ud via højttaleren.
         \act{pause} - Men vi blev ALTID færdige med rapportopgaverne til tiden, men så stod vi altså også op kl. 5 om morgenen og arbejdede 20 timer i døgnet.

\says{C} Som om det var noget særligt!! Jeg stod op kl. 3 om morgenen, så cyklede jeg fra Roskilde til DIKU, selv om det (altid) øsede ned.
         Hvis vi kom for sent, fik vi tærsk af instruktoren med en 8-tommers diskette \act{pause} - hvis han var i godt humør, altså.
         Så vi havde respekt for vores instruktor. Nutildags bruger de jo halvdelen af tiden på at bage kage.

\says{A} LUKSUS!!! \act{PAUSE - de andre kigger måbende}
         Jeg stod op kl. 12 om natten - så GIK jeg fra Holbæk til DIKU - hele året - i korte bukser! - selv om sneen lå meter-højt \act{pause for latter}
         - men JEG klagede ALDRIG, selv om jeg arbejdede 24 timer i døgnet og kun fik mad hver anden dag. Hvis vi lavede syntaksfejl i COBOL-programmerne
         fik vi pisk af instruktoren med et printerkabel, - men det tog vi ikke skade af ! - vi fik lært at programmere.

\says{D} Nå, og hva' så?
         Næh - vi havde det hårdt! Da jeg lavede kerneopgave, stod jeg op om morgenen kl. 10 om aftenen \act{pause} - en halv time
         FØR jeg gik i seng \act{pause}. Så sad jeg og skrev hulkort 30 timer i døgnet, og når programmet var færdigt, bad vi operatøren køre
         det for os. Hvis han var i godt humør, kunne vi hente udskriften dagen efter. Selv back-up'en tog vi på hulkort -
         der var skam ikke noget, der hed disketter dengang.

\scene{B og C måber}

\says{A} Men prøv at fortælle det til en Dat-nuller i dag, så tror han, du er fuld af løgn!

\says{B,C,D} \act{konstaterene, i munden på hinanden} JA, RIGTIGT!

\says{B} De unge mennesker aner sgu ikke, hvor godt de har det!

\scene{Pause, C begynder at mumle, lyset dæmpes}

\says{C} Ja, jeg blev sendt ud at tjene da jeg var fjorten - for 5 kr om måneden - jeg sov ude i stalden sammen med dyrene - osv.

\scene{Bifald?}

\end{sketch}
\end{document}
