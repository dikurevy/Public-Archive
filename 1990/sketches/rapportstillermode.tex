\documentclass[a4paper,11pt]{article}

\usepackage{revy}
\usepackage[utf8]{inputenc}
\usepackage[T1]{fontenc}
\usepackage[danish]{babel}


\revyname{DIKUrevy}
\revyyear{1990}
\version{1.0}
\eta{$n$ minutter}
\status{Færdig}

\title{Rapportstillermøde}
\author{JT, TV, PS, BQ \& Lud}

\begin{document}
\maketitle

\begin{roles}
% Medvirkende: Harry, Jan, Niels Ull, Rashid
% Rollefordeling ukendt
\role{D}[] Dumgaard
\role{K}[] Kasko Lovacs
\role{J}[] Jonas Nielsen
\role{G}[] Klaus Grue
\role{P}[] Person
\end{roles}

\begin{sketch}

    \scene{Fire rapportstillere kommer ind på scenen med hver deres kop. De sætter sig ved et lettere svinsk bord og sviner det endnu mere til.}

    \says{P} Det er ikke til at holde det her svineri ud, lad os rydde op!

    \scene{Rapportstillerne flytter til et nyt bord, men tager ikke deres kopper med sig.}

    \says{P} \act{Klasker Dumgaard på låret} Nå Dumgaard, hvad har du tænkt dig at stille opgave i?

    \says{D} Øhh hva'!!

    \says{P} Ja, du ved da godt at de skal ha' rapporten i morgen, og instruktorerne skulle have haft den for en uge siden.

    \says{D} Når jeg kun bruger $1\frac{1}{2}$ time på at stille en førstedelsrapport, er det vel rigeligt, at instruktorerne får den til
             gennemlæsning i morgen tidligt.
    
    \says{P} Ok ja - masser af tid. Instruktorerne brokker sig alligevel aldrig, de har jo bestået kurset.

    \says{D} Øh, hvad er det egentlig jeg forelæser i for tiden?

    \says{J} Det ved jeg ikke, men du kunne jo stille dem en Compileropgave, hvor de skal kode mindst $3\frac{1}{2}$ tusinde linier på 3 uger,
             og hvor de skal vildledes af algoritmer med fejl - Det har jeg erfaring for modner de studerende. Hvad synes du Grue?

    \says{G} \act{Stærkt lesbende} Jeg sssynes hellere vi ssskal ssstille en opgave i Lesssp. De ssstuderende ssskal sssimulere et ssspringvand i Lesssp.

    \says{D} Nej jeg tror ikke det er det jeg forelæser i.

    \says{P} Jeg har set et godt forslag fra Hassebasse i systemarbejde. De studerende skal systemer arbejdsgangene i skatteforvaltningen.
             De kommer til at lave en helvedes masse skemaer - 24 timer i døgnet, en hel måned.

    \says{P} Det er der da ingen der gider!

    \says{K} Exactly! It is good for the students to do things they don't like, and which obviously cannot be used for anything.

    \says{P} Det er en skam Gregers ikke er her, han havde en god opgave indenfor SDI - Det var noget med 'stjerneGries'.

    \says{P} Nej det er for langt ude.

    \scene{Alle griner}

    \says{P} Hvad med at lade dem koordinere togtiderne i Europa?

    \says{K} No, no, I understand that. It is not sufficient with too easy exercises. Let the students make a chess program to beat Kasparov\ldots
             - in poore logic! \act{Imens han peger i tindingen}

    \says{P} Det kunne være spændende, at få de studerende til at lave en kerne der kan styre vejret.

    \says{D} Nej, nej jeg har det! De skal lave ordbogsopslag i mikrokode, det er da helt urealistisk.

    \scene{Alle jubler og forlader scenen ledsaget af målscoringsjubel fra lydbordet.}
\end{sketch}
\end{document}
