\documentclass[a4paper,11pt]{article}

\usepackage{revy}
\usepackage[utf8]{inputenc}
\usepackage[T1]{fontenc}
\usepackage[danish]{babel}


\revyname{DIKUrevy}
\revyyear{1990}
\version{1.0}
\eta{$n$ minutter}
\status{Færdig}

\title{Udelelig adgang til Pia}
\author{Ukendt}

\begin{document}
\maketitle

\begin{roles}
% Medvirkende: Jolanta, Ole, Rashid, KDJ
% Rollefordeling ukendt
    \role{A1}[] Amaskulin (vores helt) 
    \role{A2}[] Amaskulin, anden skuespiller (stadig vores helt) 
    \role{B}[] Bmaskulin (vores helts rapportmakker)
    \role{C}[] Cfeminin
\end{roles}

\begin{sketch}

    \says{A1} Du øh\ldots B, det der printerprogram, eller hvad det nu er, hvad er det vi skal lave der?

    \says{B} Bum ba dum ba bum \act{B kigger gennem opgaveteksten}, \act{mumler} her kan variable til udelelig adgang ligge, SKAL KODES.
             Hmm\ldots Jo der står at vi skal sørge for at der er udelelig adgang til pia's.

    \says{A1} Udelelig adgang til Pia's? Det var spøjst. Er du klar over jeg en gang kom sammen med en pige der hed Pia?

    \says{B} Nå? Hvordan det?

    \says{A1} Hvad mener du?

    \says{B} Jah\ldots øhh, du ved. En datalog og en pige. Det er jo ikke ligefrem hverdagskost. Hvordan gik det til at du mødte hende?

    \says{A1} Jo, nu skal du høre \act{lyset dæmpes}. \act{bar stilles stille op} Det hele startede med at jeg var taget i byen sammen med nogle venner,
             og vi var kommet ind på en ret speciel bar, nærmest en space-bar. Den hed vist nok FREJA's sal. Udenfor var det begyndt at THORdne,
             det var lidt SUHRt. Men mens vi sad og hang, skete det med et at hun kom ind gennem døren \act{våd lækker sag C kommer ind på scenen
             og sætter sig ved baren}.

             Mand jeg siger dig, hun havde en lækker struktur. Hun var simpelthen lige min type, og hendes attributter var bare \ldots (soft sectors helt igennem)

    \says{B} VOWWWWW!!!

    \says{A1} Ja ikk'? Jeg var helt vild i varmen, og søgte at få lidt støtte fra vennerne. Men de var helt nedkørte, og trængte til VILE. Men man må jo være VAX,
             og så var jeg jo også kommet dertil i studierne, hvor det var tid til lidt hands-on-experience. Jeg kunne jo ikke sidde og vente på,
             at hun signalerede. Men well\ldots jeg tog en beslutning, pudsede grænsefladen af, og flyttede mit samlede datalager op til baren
             \act{\texttt{movm a0-a6/d0-d7,-(bar)}}

    \says{A2} \act{sætter sig ved siden af C} Hvad baby, er du on-line tonight?

    \says{C} A hva be hva?

    \says{A2} Ja jeg mener, er du til single-user eller multitasking?

    \says{C} Nej må jeg være fri!

    \says{A2} Ja, jeg er heller ikke bundet af omgivelsene.

    \says{C} Det må jeg nok sige. Du går lige til kernen, hvad?

    \says{A2} Ja, der er jo ikke nogen grund til at lave busy-waiting!

    \says{C} HHMMM\ldots søger du noget?

    \says{A2} Who knows, har du lyst til en ROM?

    \says{C} Ja tak!

    \says{A2} Er der nogensinde nogen der har sagt at du er så smuk, at end ikke en VGA-skærm kan gengive det?

    \says{C} Var det et komplement?

    \says{A2} Ja, men jeg er ikke helt sikker på om det var et 1 eller 2 komplement.

    \says{C} Hvaar!!

    \scene{Dyb tavshed (mens der buhes)}

    \says{A2} Du øh, ikke for at være nærgående. Men har du lyst til at se mit hardware?

    \says{A1} Ja, jeg skulle nok have givet hende et waitstate. Hun gik i hvertilfælde i baglås.

    \says{A2} Jaa\ldots øhh jeg mener, trænger du til lidt intern afprøvning?

    \says{C} Det må jeg godt nok sige. Du er ikke særlig inddirekte.

    \says{A2} Ja hvis man ikke gør opmærksom på sig selv kommer man aldrig til.

    \says{C} Aldrig til?

    \says{A2} Ja, uden at afbryde kan man blive helt udsultet.

    \says{C} Hmm. Måske er det tid til proces-skift, du har jo en dejlig høj prioritet, og pointeren ser fin ud.

    \scene{Spot over på A1 og B}

    \says{B} Det må jeg nok sige, hvad skete der så?

    \says{A1} Ja vi tog hjem til der hvor hun BORR, hun satte en disk på, og lavede lidt te. Og så mens anlægget kørte stille i baggrunden \ldots

    \says{B} OJJ. Hvad skete der så?

    \says{A1} Ja så rejste hun sig op, og lukkede for afbrydelser. Hun strippede, og vi begyndte at afstakke tøjet. Shorts og det hele.

    \says{B} Vil det sige at I\ldots øhh\ldots jeg mener\ldots du ved\ldots ikk\ldots

    \says{A1} Ja! Og så startede det. Ja det var jo ikke et peek-show. Jeg nærmede mig hendes kritiske region. Og der var ikke noget med at sove.
              Jeg siger dig hun var som en daemon, en rigtig hex. Jeg var nær blevet hægtet helt af\ldots Men det gik.
    
    \says{B} OJJ

    \says{A1} Ja det siger jeg dig. Der var både intern og extern. Top-down og bottom-up. Og til sidst brugerens. DET var en rigtig dybdeborende afprøvning.

    \says{B} Hvad skete der så?

    \says{A1} Jeg gav hende min adresse, og sagde at hun bare kalde. Vi begyndte at komme sammen, gik i 'grafen, gik i skoven, hun kunne mange små
              jul-elege, med masker, stand-alone, og med objekter. Oh boy, det var en fed tid, det var gensidig rekursion.
              Så en dag mens jeg var nede på værkstedet og få min vogn-retur, så lagde hun en besked, og spurgte om jeg ikke kom over.

    \scene{A2 kommer over hos C}

    \says{A1} Jeg troede jo jeg skulle ha' mere regelstyret udløsning, men jeg siger dig mand. Hun var gået helt i opløsning.

    \says{B} Var det galt?

    \says{A1} Ja, mindst 640x480.

    \says{B} AV for satan.

    \says{A2} Hej min egen fire-seks-og-firs, hvad er der galt?

    \says{C} Jeg har cyklisk overløb!

    \says{A2} HVARR???!!! Vil det sige at\ldots?

    \says{C} JA. Jeg har fået bufferen fyldt!

    \says{A2} Jeg troede du var protected! \ldots Nåhh. Kan du ikke bare tage en abort?

    \scene{C rejser sig op, stikker A2 en på siden af hovedet, og skrider.}

    \says{B} Nå for fanden. Det var så det?

    \says{A1} Ja syg kælling. Hun må ha' fået virus af en eller anden art. Ihverttilfælde trængte hun til en kompleksitetsanalyse.

    \says{B} Du så hende aldrig mere?

    \says{A1} Næhh!

    \says{B} Og hvad skete der bagefter?

    \says{A1} Ikke noget. Lige siden Pia lavede en \texttt{exit}, har der kun været uge-rapport og håndkøring tilbage.

    \says{B} For helvede mand en sørgelig historie. Hold kæft er klokken blevet så mange? Du, vi ses i morgen, ikke?
    
    \says{A1} Joh\ldots

    \says{B} OK. Byte byte.
\end{sketch}
\end{document}
