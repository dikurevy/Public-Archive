\documentclass[a4paper,11pt]{article}

\usepackage{revy}
\usepackage[utf8]{inputenc}
\usepackage[T1]{fontenc}
\usepackage[danish]{babel}


\revyname{DIKUrevy}
\revyyear{1990}
% HUSK AT OPDATERE VERSIONSNUMMER
\version{2.0}
\eta{$n$ minutter}
\status{Færdig}

\title{Førstedelsstafet}
\author{Jacob Marquard}

\begin{document}
\maketitle

\begin{roles}
    % Medvirkende: Harry, Holger, Jacob, Niels Bo, Niels Ull, Ole
    % Rollebesætning ukendt
    \role{S}[] Speaker
    \role{K}[] Kommentator
    \role{L1-4}[] 4 1.delslærere i T-shirts med logo.
\end{roles}

\begin{sketch}

    \says{S} Og vi bringer her et sammendrag fra årets lokale opgør i 1.delsstafet.
             Der er tale om konkurrencen for hold. Årets vinder skal deltage i næste års fakultetskonkurrence.
             Og apropos årets fakultetskonkurrence, hvor det jo kun var 1.årskurser der deltog, og hvor dat0 repræsenterede datalogi.
             Kan du resumere resultatet?

    \says{K} Dat0 vandt den del der handler om dumpeprocent og overforbrug af tid i.fht. normeringen. Mat1 blev nr.2 og Kemi1 vandt
             gennemførselsprocent-konkurrencen. Årets stilistiske vinder blev DatA med 90\% meget utilfredse studerende.
    
    \says{S} Tilbage til årets konkurrence. Kan du forklare hvilke discipliner, der kæmpes i?

    \scene{Der lægges en overhead på}
    \says{K} Der er 4 deldiscipliner.
             \begin{enumerate}
                \item Gennemførselsprocent (lavest mulig)
                \item Dumpeprocent (højest mulig)
                \item Overforbrug af tid i.fht. normeringen (størst mulig)
                \item Stilistisk karakter, hvor der gives point for
                    \begin{enumerate}
                        \item Gnidret skrift, eller for små bogstaver på overheaden
                        \item Rodet tavleorden
                        \item Mumlen
                        \item Skyggen for overheaden
                        \item Tilstræbt dansk udtale.
                    \end{enumerate}
             \end{enumerate}

            Det skal bemærkes at det er særligt svært at holde en lav gennemførselsprocent samtidigt med en høj dumpeprocent, da de har en
            tendens til at modvirke hinanden.

    \says{S} Men tilbage til årets opgør, vi har lavet et kommenteret sammendrag, som vi bringer nu:

    \scene{de 4 personer ved de 4 overheads agerer til}

    \says{S} Vi begynder i september måned, med den første forelæsning, den er psykologisk meget vigtig, specielt for gennemførselsprocent
             og stilistiske points. Og vil du fortælle os hvad der sker?

    \says{K} Ja, dat1P lægger fint ud, mumlen, skyggen, og teoretisk stof, og de studerende klager med det samme - flot stil, det er
             imponerende at se en klage underskrevet af mere end halvdelen af de studerende allerede efter 2. forelæsning.
             Det kunne ikke en gang DatA have klaret!

    \says{S} Og nu efter 3 uger stilles der en rapportopgave i Dat1P, men den ser nem ud?

    \says{K} Ja, den er nem, alt for nem; en trist fejl at lave efter en så perfekt start.

    \says{S} Men se nu Dat0, imens vi har kigget på Dat1P, er Dat0 rykket frem, de spurter afsted!

    \says{K} Ja, og de gennemgår stoffet i et helt utroligt tempo, se det skal nok være godt for gennemførselsprocent,
             overforbrug og ikke mindst stilkarakter. Man kan godt se at der har været en lærer på kursus i overheadvendingsteknik.

    \says{S} Og vi er nået til Dat1Ps ugeopgave, den ser stor ud?
    
    \says{K} Ja, og den skorer da også pænt på overforbrug af tid. Men da en ugeopgave kun er normeret til 4 timer,
             kan en procentuel overskridelse med 200 \% ikke tælle så meget.

    \says{S} Dat2 og Materiel stiller nu deres rapportopgaver - det er flot gået, de er store?

    \says{K} Ja - MEGET STORE - se nu melder de studerende fra Materiel for kun at læse programmel -
             det vil give en fin gennemførselsprocent. Men hov - opgaverne er gode, de kræver et merforbrug af tid på op til 200 \%,
             men de er interessante. Ingen stilistiske points til de to kurser.

    \says{S} Efter første semester må man jo sige at de enkelte kurser ligger fint. Hvordan vil du bedømme de enkelte kursers præstation?

    \says{K} Ja, Dat2 stillede den største opgave, men hæmmes af også at have en 50 \% bedre normering, så alt i alt må Dat1M og Dat2 siges
             at føre med dat0 som nummer 3 og dat1P lige efter.

    \says{S} Og nu går forårssemesteret i gang, dat0 har sænket tempoet, og falder nu helt bagud, samtidigt har Dat1P stillet endnu en overkommelig
             opgave. Hvordan vil du betegne situationen?

    \says{K} Ja, dat1P har taget studienævnet alt, alt for alvorligt.

    \says{S} Nu stilles kerneopgaven af Dat1M, den er som sædvanlig alt for stor.

    \says{K} Ja, det er virkelig Materiels trumfkort, og det giver fine point, men se nu Dat1P forsvarer sig. I Grafteori bruges masser af
             svære matematiske beviser - flot gået.

    \says{S} Men NEJ se nu Dat1M afværger med et krav på ugeseddelen om at KERNEN SKAL KØRE, og Dat1Ps auditorium tømmes.
             Hvad vil du sige om den udvikling?

    \says{K} Ja, det var flot set af 1M og ærgeligt for 1P, det var ellers flotte beviser. Dat2 er desværre også faldet bagud, og prologopgaven
             er stor, men overkommelig.

    \says{S} Resultatet efter de 2 semestre er at Dat1M vinder samlet, samt overforbrug af tid og gennemførselsprocent, mens dat0
             vinder stilkonkurrencen. Det er for tidligt at udtale sig om dumpeprocenterne, her ser det ud til at blive et opgør mellem
             dat1P og Dat2, idet Dat1M kun har dumpet 25\%.

    \says{S} Ja, og har du et godt råd til næste års lærere?

    \says{K} Ja, erstat bøgerne med fotokopier af artikler, uden sammenhæng, og gerne med varierende begrebsapparat, det er vejen frem!

    \says{S} Ja-TAK, og et råd til institutbestyreren med henblik på næste års fakultetskonkurrence.

    \says{K} Ja - gør noget ved MASKINAFDELINGEN, det de præsterer er helt utilstedeligt, systemer der er oppe i en stor del af tiden,
             og hvor printkøer ikke er ude af drift mere end et døgn ad gangen giver kun få muligheder for at vinde næste års fakultetskonkurrence.
             Man har valget mellem at skifte til UNI*C eller lave en aftale med KEMIKERNE at der ikke må bruges urene kemikalier.

    \says{S} JA TAK

\end{sketch}
\end{document}
