\documentclass[a4paper,11pt]{article}

\usepackage{revy}
\usepackage[utf8]{inputenc}
\usepackage[T1]{fontenc}
\usepackage[danish]{babel}


\revyname{DIKUrevy}
\revyyear{1977}
\version{1.0}
\eta{$n$ minutter}
\status{Færdig}

\title{Telefonsketchen}
\author{HVO, SR}

\begin{document}
\maketitle

\begin{roles}
\role{Henrik}[] Henrik
\role{Solveig}[] Solveig
\role{Stemme}[] Stemme
\role{Stemme1}[] Stemme1
\role{Stemme2}[] Stemme2
\role{Stemme3}[] Stemme3
\role{Stemme4}[] Stemme4
\role{Stemme5}[] Stemme5
\role{Stemme6}[] Stemme6
\role{Stemme7}[] Stemme7
\end{roles}


\begin{sketch}

\says{Henrik} Vi er nu klar til `` De ringer, vi for løn for det. ``. Solveig og
jeg sidder klar ved telefonerne, og De er velkommen til at ringe ind på
01-836466 lokal 19.

\says{Solveig} Jeg tror nu vi har den første på linien \act{skrattelyde}.

\says{Henrik} Goddag. De hører DIKU's datamatiske telefonsvarer. Vær venlig at
indtale cifrene fra 0 til 9.

\says{Stemme}  1,2,...,9.

\says{Henrik} Vær venlig at indtale det ønskede lokalnummer med mindst betydende
ciffer først.

\says{Stemme} 2, 3.

\says{Hemrik} Det ønskede lokalnummer eksisterer ikke.

\says{Stemme1} Men det gjorde det da for 5 minutter siden !

\says{Solveig} Undskyld, De har været udsat for en slags transmissionsfejl
hø-hø, om jeg så må sige.

\says{Solveig}[til Henrik] Det er jo først i morgen, vi starter `` Hvem tror De,
De får fat i? Gæt et lokalnummer. ``

\says{Solveig} Ja og hvad var så Deres ønske ?

\says{Stemme1} jeg ville gerne ønske mormor og morfar uendelig løkke.

\says{Solveig} Det kan vi desværre ikke klare her, tak fordi De ringede.

\says{Stemme1} Vi har ellers samlet ind i en tom kantinekasse.

\says{Henrik} Ja ja, farvel.

\says{Solveig} Lad os høre om der er en ny på linien.

\says{Stemme2} Ja hallo, ka' De kende mig? Det er mig, der har kontonummer
7-000-413 i Århus. Jeg skrev engang et program...

\says{Solveig} Kan I også det i Århus ?

\says{Stemme} Det startede med integer i ...

\says{Henrik} Jeg tror Simon har det på terminalen. \act{tastelyde}. Hvad siger
den ? Division by zero attempted .

\says{Stemme2} Det er den, det er den.

\says{Henrik} En af vores små fikse båndaber er ved at finde den frem på
pladelageret, og den bliver spillet på...

\says{Solveig} STOP !! Lad os en gang for alle slå fast, at vi er i gag med ``
De ringer, vi får betaling for det. ``. Hvem har vi nu på tråden ?

\says{Stemme3} Det er bare mig. Jeg ringer fra Vermundsgade.

\says{Solveig} Det var da morsomt. Vi må vist have fat i Danmarkskortet.

\says{Stemme3} Det er lige ovre på den anden side af gården.

\says{Solveig} Nu må vi se, om det skal være et let eller et meget let
spørgsmål. Hvad er klokken ?

\says{Henrik} Det er svært at sige.

\says{Stemme3} Naj, det var ærgerligt.

\says{Solveig} Nu skal De høre. Hvad er Imperialisme ? De får tre muligheder :
1) en stavefejl, 2) en perfiditet, 3) en inskription på portalen til
medarbejderghettoen ?

\says{Stemme3} Hvordan kunne I vide, at det var mig ?

\says{Henrik} Der blev ikke stukket nogen plade i denne omgang. vem har vi nu i
røret?

\says{Stemme4} Det er fru Olsen fra 7. kartoffelrække i Hanherred.

\says{Solveig} Jamen det var da morsomt, fru Olsen. Hvad beskæftiger De Dem med
til daglig ?

\says{Stemme4} Databaser.

\says{Solveig} Så må det lige være et spørgsmål for Dem. Hvor ligger Thulebasen
?

\says{Steme4} Spor 17.

\says{Henrik} Desværre. Bedre held næste gang.

\says{Solveig} Ja, vi er klar med den næste. Hallo .

\says{Stemme5} Ja, hallo. Kan De høre mig ? De er meget langt væk.

\says{Henrik} Så må vi prøve at snakke lidt højere. Kan De høre os nu ?

\says{Stemme5} Nej .

\says{Solveig} Det var godt. Vi har et spørgsmål klar. Hør nu godt efter : Hvad
laver RECKU mest ? Har DE det ?

\says{Stemme5} Næh, det må vist være den berømte klap. Den er gået ned.

\says{Henrik} Rigtigt, den er gået ned. Det var slet ikke så dårligt. De har
vundet en eksamen.

\says{Stemme5} Åh tusind tak fordi jeg fik lov at komme igennem.

\says{Henrik} Ja selv tak. Det er jo ikke mange, der kommer igenem.

\says{Solveig} Næh, det er kun få det lykkes for, desværre.

\says{Stemme5}[Hensat i dyb herykkelse] En eksamen!

\says{Henrik} Ja, der er jo heldigvis mange, der prøver, men vi forsøger at
udvælge så tilfældigt som muligt. Hvem skal vi nu tale med ?

\says{Stemme6} Jeg er Dat-0 studerende.

\says{Solveig} Så er de jo kendt med forholdene her på stedet, så De kan sikkert
sagtens klare spørgsmålet.

\says{Henrik} Hvad er den nye bestyrers valgsprog ?

\says{Stemme6} Det ved jeg sør'me ikke.

\says{Solveig} Tillykke, De har gættet rigtigt. De skylder os en LP.

\says{Stemme6} Jamen...

\says{Henrik} Ja farvel og tak. Og hvem snakker vi nu med ?

\says{Stemme7} Jeg er bare en lille paritetsbit.

\says{Henrik} Det er glimrende, vi skulle jo netop gerne have så mange med som
muligt. Kan frøkenen få øje på klokken.

\says{Solveig} Ja, let.

\says{Henrik} Også spørgsmålet : Hvad vil den tidligere institut-bestyrer helst
huskes for ? 1) at al kommunikation skal foregå i tre eksemplarer, 2) initiering
af nedrivning af DIKU eller 3) at medarbejderghettoen blev skabt ?

\says{Stemme7} Det må være for at klaveret stemmer .

\says{Solveig} Nej, hør nu. Der er de tre muligheder \act{gentages}.

\says{Stemme7} Jo, jeg mener altså stadig, at det er for, at klaveret stemmer.

\says{Solveig og Henrik} Farvel og tak for denne gang. Vi spiller nu -- på
opfordring fra 80 personer -- sangen om databaser. Tak for denne gang. Farvel.

\end{sketch}
\end{document}
