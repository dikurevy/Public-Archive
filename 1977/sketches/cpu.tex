\documentclass[a4paper,11pt]{article}

\usepackage{revy}
\usepackage[utf8]{inputenc}
\usepackage[T1]{fontenc}
\usepackage[danish]{babel}


\revyname{DIKUrevy}
\revyyear{1977}
\version{1.0}
\eta{$n$ minutter}
\status{Færdig}

\title{Computerized Police Unit}
\author{HHK, EM}

\begin{document}
\maketitle

\begin{roles}
\role{C}[] Agent Christensen
\role{K}[] Agent Kvast
\end{roles}


\begin{sketch}

\scene{Scenen består af to skriveborde med front mod publikum. Imellem de to
  borde er der anbragt en skærmterminal. Ved starten af nummeret sidder C. ved
  sit skrivebord. K. kommer ind, på vej hen til sit skrivebord.}

\says{C} Holdt, hvem der? BRUGERID SKRÅSTREG PASORD!!!!!!!

\says{K} En ud af ti filmstjerner vasker sig ikke.\\
Sig Jolly til Ultra Bright.\\
Spis Benti med det hygiejniske indføringshylster.\\
Kurol. Kurol.

\says{C} Ja tak, Roger over, negative. Davs med dig K.

\says{K} Davs med dig C. Nå hvordan går det med konen? \act{sætter sig ved sit
  skrivebord}.

\says{C} Vi har haft noget afet problem i løbet af natten. Vi har fået færten af
en venstreorienteret multinational mafiaring, som har infiltreret nogle af vores
fineste hospitaler. De kalder sig vist ``Det røde System''.

\says{K} C!!!!!!!! Nu er det anden gang i løbet af kort tid du tager
fejl. Sidste gang var det sagen om ``Den rødeTråd''. Husker du ikke det
læserbrev (DIKU-blad 3.9) som Peter Naur skrev fra arresten? Du må lære at
skelne,C. Systemet hedder det røde system fordi man var udgået for andre farver
karton. Og desuden har bogtrykkeren fået tolv år for spionage.

\scene{Telefonen ringer.}

\says{C} Computerized Police Unit, Christensen her med dæknavnet PR2. Er det
Dem, hr. Møller? \act{til K} Det er Orla Isenkram. Hvormed kan jeg være Dem til
tjeneste?

\says{C} Hvad siger De? En farlig mandsperson med hang til det mexikanske
køkken?

\says{C} Har De flere spor i sagen?

\says{C} Aha, der en konstateret svampeangreb i hans træben. Hvad var navnet,
igen?

\says{C} Hr. Z. Z. Kulovski. Ja, et øjeblik, nu skal jeg se hvad vi har på ham
på vores ring-op-på-line-forbindelse.

\scene{C og K går hen til terminalen.}

\says{K} Besynderligt. Er det ikke sært, at der er 512, der hedder
Z. Z. Kulovski.

\says{C} Nå jeg ved snart ikke, det er da et ret almindeligt navn.

\says{K} Jamen de bor allesammen på samme adresse -- GUARD MODE 4040.

\says{C} Nå, det er sikkert en af de der storfamilier. Alle terrorrister bor i
storfamilier.

\says{C}[I telefonen] Ja nu har vi fundet ham frem. Hvad vil De have at vide?
Hvad vi har på ham?

\says{C}[til K]] Hvad betyder ABORT?

\says{K} Det er vist noget med en illegal svangerskabsafbrydelse betalt i
amerikansk valuta.

\says{C}[I telefonen] Ja, her. minister, vi skulle have mere end nok til at
kunne udvise ham.

\says{C} Ja, skulle det være en anden agn \act{lukker røret}.

\scene{Kort pause, hvor de to agenter sidder og bladrer vildt i noget
  leporello.}

\scene{Telefonen ringer.}

\says{K} Ja, goddag hr. Jacobsen, hvordan går det med hørelsen? Et rim på
pædagog? Hvad med typograf?

\says{K} Næ, nej, måske rimer det ikke helt; mendet passer da meget godt, ikke?

\says{K} Ja, naturligvis vil vi da gerne høre Deres sang. \act{C tager sin
  telefon.}

\says{C} Vi synger selvfølgelig med på omkvædet.

\scene{De lytter begge med til sangen. Da omkvædet kommer synger de}
\end{sketch}

\begin{song}
\scene{Melodi: For huun er så uung og så dejlig ser hun ud}

Kooomunisstiisk innfiltration
Kooomunisstiisk innfiltration.

\scene{omkvædet synges efter hver af de to vers}
\end{song}

\begin{sketch}
\says{K} Ja, hr. Jacobsen, vi skal nok lave to vers til på vores
EDB-anlæg. Farvel hr. Jacobsen.

\says{K}[til C] Det må jo være afdelingen for påtaget intelligens, du ved,
Gregers Koch.

\scene{Kort pause. Agenterne bladrer videre i deres leporello.}

\scene{Telefonen ringer.}

\says{C} Hallo, CPU. Christensen speaking.

\says{C} Jamen goddag, fru minister. Hjalp det så med det lille råd vi gav Dem
om RUC? Nå ikke? Ja, ja her i forsvaret må man lære at leve ved visse tab.

\says{C} Nu er vi blevet færdige med kartoteket over deltagerne i
universitetsbesættelsen. Lige et øjeblik. \act{sætter sig hen til skærmen}

\says{C} Vi har to \underline{meget} mistænkelige personer her. Begge har
fornavnet Max. Max Time og Max Page. Åberbart udenlandske konspiratorer.

\says{C} Ja nu kommer der nogle andre navne frem på skærmern. Hvad står der, Ole
GrUnbaum, Peter Wivel, Erik Meling. Og der står oven i købet, at de drak rektors
sherry. Åh,åh. Ja, der \underline{er} faktisk navnene fra 1969. De nye har vi
ikke fået færdige endnu. Hvis de lige vil vente et øjeblik, skal jeg forhøre mig
om, hvor langt det er kommet. \act{går hen til den røde telefon}

\says{C} Hallo, er det DIKU? Må jeg tale med Torben Databoss? Davs, du
gamle. Hvordan går det med dette her studenterregistreringsprojekt?

\says{C} \underline{Hvad} er I ved at have færdig? Arbejdsbeskrivelsen? Jamen,
det var da storartet! Har I ikke et hjørne af databasen? Storartet, lad os se
den! Så må vi lave resten af projektet som datalogisk praktik. \act{I telefonen}

\says{C} Det skulle komme på skærmen nu.

\says{C} AHA, Nå så det er den slags metoder, de anvender! DIVISION BY =
ATTEMPTED. Tænk hvis det var lykkedes!!!!!!!

\says{C} Ja, vi holder os i kontakt. Farvel.

\scene{Begge rejser sig op og synger.}
\end{sketch}

\end{document}
