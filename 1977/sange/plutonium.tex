\documentclass[a4paper,11pt]{article}

\usepackage{revy}
\usepackage[utf8]{inputenc}
\usepackage[T1]{fontenc}
\usepackage[danish]{babel}


\revyname{DIKUrevy}
\revyyear{1977}
\version{1.0}
\eta{$n$ minutter}
\status{Færdig}

\title{Plutonium}
\author{Mikkel}
\melody{``Jomfru Ane''}

\begin{document}
\maketitle

\begin{roles}
\role{S}[Lene] Sanger
\role{K}[] Shubidua-kor
\end{roles}


\begin{song}
\sings{S}%
Inde på DIKU går vi rundt og har det rart
og datalogien -- den lær vi det er klart!

\sings{S}[omkvæd]
Vi assimilerer, vi assimilerer
Den videnskab, (accepterer, accepterer)
Den videnskab, (accepterer, accepterer)

\sings{S}%
Denne sang beretter om en kom'ne datalog
der startede på DIKU og handlede i god tro.

På dat 0 der lærte han at være helt neutral:
for ikke at ha holdning er en datalogs moral.

Hvorfor sku man ha en holdning det er intet plus
vi sælger vores arbejdskraft -- pyt med hvordan den bru's.

På dat 1 der lærte han en hel del om teknik
men stadig ingen holdning det arme men'ske fik.

På dat 2 der lærte han at simulere det
som man kan simulere når man simulere ve'

Endnu da han kom på anden del, så tro'de han
at han skulle se hvordan problemer løses kan.

Nu vud han alt om data, deres væsen, deres brug
man kan da ikke mene at han flere ting ska ku'.

I Superfoss han optimer' profit og sådan nå'd.
Han er så dygtig i sit job, har hus og biler få'd.

Nu må du ikke føle, at DIKU har forsømt.
For ren datalog det blir du, er det ikke skønt?!
\end{song}

\end{document}
