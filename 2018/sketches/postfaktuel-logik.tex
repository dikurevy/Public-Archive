\documentclass[a4paper,11pt]{article}

\usepackage{revy}
\usepackage[utf8]{inputenc}
\usepackage[T1]{fontenc}
\usepackage[danish]{babel}

% LOGIC
\usepackage{mathtools}
\usepackage{boxproof}
\usepackage{tikzsymbols}
\usepackage{wasysym}

\let\implies\rightarrow
\def\elim#1{{{#1}{\cal E}}}
\def\intro#1{{{#1}{\cal I}}}
\newcommand{\prem}[0]{\mbox{præmis}}
\newcommand{\ass}[0]{\mbox{rygte}}
\newcommand{\fact}[0]{\mbox{fakta}}

\newcommand{\rygte}{\rightleftharpoons}
\newcommand{\blamefor}{\hookrightarrow}


\revyname{DIKUrevy}
\revyyear{2018}
% HUSK AT OPDATERE VERSIONSNUMMER
\version{1.0}
\eta{$6$ minutter}
\status{Der er ved at være noget godt}

\title{Postfaktuel logik}
\author{Bette-Mikkel, Niels, Troels, Phillip}

\begin{document}
\maketitle

\begin{roles}
\role{A}[Simon] Andrzej Filinski
\role{X}[Niels] Instruktør
\end{roles}

\begin{props}
  \prop{Lænestol}
  \prop{Bunker af papir}[bare tag nogle fra Andrzejs kontor]
  \prop{Lille guldstatue}
  \prop{Logik-bogen (``Logic in Computer Science'')}[Phillip]
  \prop{AV: Slideshow}[Niels]
  \prop{AV: Slide fra logikbogen}[Phillip]
  \prop{AV: Sammenklipning af tvivlsom TV-logik}[Phillip]
\end{props}


\begin{sketch}

  \scene{Der er halvmørkt på scenen. Midt på scenen står en lænestol, omgivet af høje bunker papir.}

  \scene{En ukendt person (A) sidder i lænestolen med ryggen mod publikum.}

  \scene{På OverTeX vises 10-20 sekunders klip fra Fox News, TV2 News, o.lign., hvor indholdet er meget dumt. Det er klippet sammen på en måde så det ser ud som om at personen i lænestolen zapper rundt mellem forskellige kanaler.}

  \scene{TV'et på OverTeX slukkes. Lys helt op. A rejser sig (eller drejer stolen) og vender sig mod publikum.}

  \says{A} Godaften, og velkommen til første forelæsning i Postfaktuel logik.

  \says{A} Fysikerne ved jo at man kan bruge mange år på at gylpe nogle græske bogstaver ned i en videnskabelig artikel, og så er der nogen som bygger en partikel-accelerator og viser at man tog fejl.

  \says{A} Logik har på lignende vis været ramt af teoretisk isolering.

  \says{A} Vi har brug for empirisk efterprøvning: Hvordan bruger folk formel logik i den virkelige verden?

  \says{A} Det er nu for første gang lykkedes mig at observere den virkelige verden \act{peger i retningen af OverTeX-TV'et}, og jeg har opdateret kursusnoterne til Logik derefter.


\scene{Fra her af kommer der et slide på OverTeX med en masse box proofs. Det er meningen at hver linje i boksbeviset kommer i takt med at A snakker om den.}

{\scene AV: Vis formel.}
\begin{align*}
p \implies \lnot q, q \lor r \vdash p \land q \land r.
\end{align*}

\says{A}[peger på slidet] Lad os tage et eksempel. Vi vil gerne bevise at, givet at $p$ medfører ikke-$q$, og givet at enten $q$ eller $r$ gælder, kan det udledes at $p$ og $q$ og $r$ gælder.

\says{A} Det er nok ikke alle der har haft Logik for nyligt, så lad os lige tage en hurtig opsummering.

\scene{A tager Logik-bogen frem fra stolen og bladrer meget hurtigt igennem den. På HøjTeX kommer der en masse slides meget hurtigt.}

\says{A} Og nu hvor alle er med, kan vi skrive vores præmisser op.

{\scene AV: Følgende boksbevis kommer løbende op på OverTeX.}
\begin{proofbox}
\[
\: p \implies \lnot q \=\prem\\
\: q \lor r \=\prem\\
\[
\: p \= \ass\\
\: \lnot q \= \text{MP}(1, 3)\\
\]
\: \Smiley\ \lnot q \= \intro\Smiley\\
\: \lnot q \= \text{SE}(4, 5)
\]
\end{proofbox}

\says{A} Man kan til enhver tid introducere en indlejret rygteboks. Jeg har i hvert fald hørt folk sige at $p$ er korrekt. Og så kan vi bruge modus ponens af første og tredje linje til at vise $\lnot q$.

\says{A} I dagens politiske klima er det rart at være \textit{imod noget}, så vi kan bruge føles-godt-operatoren på \textbf{ikke}-$q$.

\says{A} Jeg har lavet en regel der siger at hvis der er et rygte om et term $v$, og man godt kan lide $v$, så gælder $v$.

{\scene AV: Vis regel på separat slide.}
\begin{align*}
\rygte v \land \Smiley\ v \vdash v
\end{align*}

{\scene AV: Skift tilbage til boksbeviset.}

\says{A} Vi kalder dette for skepsiseliminering.

{\scene AV: Følgende boksbevis kommer løbende op på OverTeX.}
\begin{proofbox}
\[
\: p \implies \lnot q \=\prem\\
\: q \lor r \=\prem\\
\: \lnot q \=\fact\\
\: \exists? s \=\intro{\exists?}(3)\\
\: s \Sadey \=\intro\Sadey\\
\: s \blamefor \lnot q \=\intro\blamefor(5, 3)\\
\: q \=\text{FN}\\
\]
\end{proofbox}

\says{A} Men vi ville jo bevise $p \land q \land r$ -- og nu er vi kommet frem til $\lnot q$. Det er en såkaldt ufrivillig implikation. Dem kan man komme ud af ved at bruge ``men-hvad-med''-operatoren, som vi gør det med variablen $s$.

\says{A} Nu er $s$ i en ``men-hvad-med''-kontekst. %Den holder ikke forevigt, men nok til at vi kan navigere over i en alternativ fakta.

\says{A} Nu tænker I måske at $s$ hverken er en del af vores præmis eller konklusion. Faktisk lyder $s$ ret \act{frastødt} fremmedt, så vi introducerer et føles-ikke-særligt-godt-udtryk. Den sure smiley skal stå til højre, da det er omvendt boolsk notation.

\says{A} Her træder skyldreglen i kraft: Hvis man har en gammel fakta $v$ og en variabel $w$ der ikke føles god, kan man introducere skyde-skylden-på-operatoren, så vi siger at $w$ er skyld i $v$.

{\scene AV: Vis regel på separat slide.}
\begin{align*}
\text{gammelt}(v) \land w \Sadey \vdash w \blamefor v
\end{align*}

{\scene AV: Skift tilbage til boksbeviset.}

\says{A} Her ville man normalt bruge et modstridsbevis, men i stedet laver vi et såkaldt \textit{mod}-modstridsbevis. Den virker ved at vi gør det \textit{mod}satte af at bevise noget, så vi gør\ldots ingenting. De to ``mod'' cancellerer hinanden ud, og vi står tilbage med et \textit{stridsbevis}! \act{glad}

\says{A} Man kan bruge et stridsbevis til at tilegne sig en modsatrettet fakta fra en fjende, her $s$. Fjende-variablen $s$ mener tydeligvis at $\lnot q$ holder, men \textbf{SÅ SKAL MAN BARE RÅBE HØJT NOK OG LÆNGE NOK TIL MAN FÅR RET OG OVERDØVER MODPARTENS HOLDNING}.

\says{A} Et stridsbevis kaldes derfor også for brug af ``\textbf{f}jendtlig \textbf{n}egation'', eller som de siger i USA: ``fake news''. Og vi har nu vores første fakta, $q$!

{\scene AV: Følgende boksbevis kommer løbende op på OverTeX.}
\begin{proofbox}
\[
\: p \implies \lnot q \=\prem\\
\: q \lor r \=\prem\\
\: q \=\fact\\
\: r \=\text{INDOK}(2, 3)\\
\]
\end{proofbox}

\says{A} Nu skal vi vise at $r$ holder. Det kan vi vise med et \textit{indoktrin}-bevis. Vi ved allerede at $q \lor r$ er sandt.

{\scene AV: Skift til sandhedstabel-slides.}
\begin{center}
\begin{tabular}{r|r|l}
$q$ & $r$ & $q \lor r$\\\hline
T & T & T\\\hline
F & T & T\\\hline
T & F & T
\end{tabular}
\end{center}

\says{A} Når $q$ er falsk, er $r$ nødt til at være sand for at $q \lor r$ holder.

\begin{center}
\begin{tabular}{r|r|l}
$q$ & $r$ & $q \lor r$\\\hline
T & T & T\\\hline
F & \textbf{T} & T\\\hline
T & F & T
\end{tabular}

\begin{align*}
\lnot q, q \lor r \vdash r
\end{align*}
\end{center}

\says{A} Dét er vores basistilfælde!

\says{A} I \textit{indoktrin-skridtet} gentager vi basistilfældet -- at $r$ holder -- indtil det til sidst må være sandt.

\says{A}[gentager sidste sætning] I \textit{indoktrin-skridtet} gentager vi basistilfældet -- at $r$ holder -- indtil det til sidst må være sandt.


\begin{center}
\begin{tabular}{r|r|l}
$q$ & $r$ & $q \lor r$\\\hline
T & T & T\\\hline
F & \textbf{T} & T\\\hline
T & \textbf{T} & T
\end{tabular}

$\downarrow$

\begin{tabular}{r|r|l}
$q$ & $r$ & $q \lor r$\\\hline
T & \textbf{T} & T\\\hline
F & \textbf{T} & T\\\hline
T & \textbf{T} & T
\end{tabular}
\end{center}


{\scene AV: Følgende boksbevis kommer løbende op på OverTeX.}
\begin{proofbox}
\[
\: p \implies \lnot q \=\prem\\
\: q \lor r \=\prem\\
\: q \=\fact\\
\: r \=\fact\\
\(
\: p \=\ass\\
\: \mars p \=\intro\mars(5)\\
\+
\: \lnot p \=\ass\\
\: \female \lnot p \=\intro\female(5)\\
\)
\: p \=\text{QED}\\
\]
\end{proofbox}

\says{A} Nu mangler vi kun at konkludere $p$ som fakta. Her er det nødvendigt med en kombination af flere bevisstrategier.

\says{A} Først starter vi -- med hjælp fra venner -- \textit{både} rygter om $p$ \textit{og} $\lnot p$. Vi kan sige at... Rune synes at $p$ holder, og... Susanne synes at $\lnot p$ holder. Så kan vi introducere de kønnede operatorer. Så undgår vi også at være politisk korrekte -- og dermed ukorrekte.

\says{A} Loven om excluded middle fortæller os at midtsøgende holdninger er \textit{svage}, og at der derfor altid er et ekstremum som er mest rigtigt. Hvis for eksempel en mand siger noget, er det vigtigere end hvis en kvinde siger det. QED, altså, ``Qvinder er dumme''!


  \says{A} Nå, vi er da vist også gået over tid--\act{går sin vej}

  \scene{Sluk bål. Sebbe for. Mammut til frokost.}

\end{sketch}
\end{document}
