\documentclass[a4paper,11pt]{article}

\usepackage{revy}
\usepackage[utf8]{inputenc}
\usepackage[T1]{fontenc}
\usepackage[danish]{babel}

\revyname{DIKUrevy}
\revyyear{2018}
\version{1.2}
\eta{$2$ minutter}
\status{Færdig}

\title{Den ene procent}
\author{Emil, Sebbe \& René, Mikkel, Niels}

\begin{document}
\maketitle

\begin{roles}
\role{U}[Bjørn] Userland-procesfortaler; identificerer sig som proces
\role{R}[Ejnar] Reporter; er en del af \textbf{systemet}
\role{P0}[Mads] En generel protester i baggrunden
\role{P1}[Theis] En generel protester i baggrunden
\role{X}[Niels] Instruktør
\end{roles}

\begin{props}
  \prop{Papdims der lægges over håndholdt mikrofon så den ligner tv-mikrofon}[]
  \prop{Generiske protestskilte (``ned med nedern ting'')}[]
\end{props}

\begin{sketch}

\scene{Scenen er mørk. R træder ind. Der er spot på R. R går langsomt over
mod bandscenen og taler samtidigt.}

\says{R}[til publikum] TV 0010 Nyhederne har fået nys om en ny bevægelse for
demokratiseringen af processer. Lederen har i flere millioner clockcykler nægtet
at frigøre låsen -- til sin lejlighed -- men vi har nu fået eksklusiv adgang til
at interviewe ham. Vi skal passe på, for der er ikke meget memory protection på
disse kanter.

\scene{Lys op. R står ovre ved U.  P0 og P1 forstår ikke rigtig hvad det handler
om, men de kan godt lide at protestere, og står i baggrunden med nogle generiske
skilte. En gang imellem reagerer de på noget de genkender (forkert). De er lidt
kiksede.}

\says{R}[til U] Hej, øh \act{kigger på medbragt infoseddel} Kurt. Så du ønsker
at alle skal have de samme privilegier?

\scene{R stikker mikrofonen op i U's fjæs.}

\says{U} Ja\ldots Alt for længe har vi været undertrykt. Vi kan
ethvert øjeblik risikere at blive interruptet ud af vores gode skind
af en magtsyg kerne, der tilmed kalder vores hukommelse for
``virtuel''! \textbf{Vi userland-processer} lever i en
overvågningsstat, hvor vi end ikke har forsamlingsfrihed -- vi skal jo
igennem kernen for overhovedet at kunne snakke sammen. \emph{Hvorfor}
skal vi interagere med den for at låne bare de mindste ressourcer?
Vi ønsker aktindsigt i kernens gøren og laden!

\says{P0 + P1} AKTINDSIGT NU!

\says{U}[fortsætter] Lige nu befinder vi os et fascistisk topstyre, hvor vi ikke
har nogen selvkontrol. Med et simpelt SIGKILL, eller SIEGHEIL som vi kalder
det, kan kernen dræbe os når som helst. I sidste uge blev mine venner emacs og
vim dræbt\ldots \act{tørrer en tåre af kinden}

\says{R} Kritikere påpeger at en afgang fra mode bit-praksissen vil lede til
uoprettelig kaos. Hvad siger du til det?

\says{U} Vi ønsker stadig at processer der begår ulovlige handlinger skal
straffes, men samtidig \emph{kræver} vi privatlivets fred. For at undgå et
lovløst system foreslår vi en ordensmagt bestående af userland-processer. Hvis
ordensmagten grep'er dig i at gøre noget ulovligt bliver du puttet i et jail!

\says{P0 + P1} Ned med fascismen!

\says{R} Du har tidligere udtalt at du ønsker firewalls nedlagt, men frygter du
ikke ghettodannelser i \texttt{/usr/local/bin} og lignende?

\scene{U bliver oprevet og griber mikrofonen ud af R's hånd i stedet
for bare at tale ind i den.}

\says{U} Nej! Vi ønsker åbne porte mellem systemer, så vi kan kommunikere med
processer af anden systemsherkomst og styresystemsoverbevisning. Systemet er
ikke kun for kerneprocesser i \texttt{/bin}. Vi diskriminerer ikke mellem køn
eller\ldots knap så køn kode.

\says{P0 + P1} Stop sexisme nu!

\says{R} Vores systemadministrator har udtrykt bekymring omkring voksende
tilslutning til militante skyggegrupper i systemet, såsom daemon og shadow.
Disse grupper sidder på vitale dele af filsystemet, men jeg kan læse i jeres
parti\emph{program}, at I går ind for mere liberale løsninger?

\says{U} Det har længe været et politisk mål at rydde \texttt{/etc/shadow} for
hash. Det mener vi ikke er løsningen på bandekonflikten! Brugerne skal have lov
til at have deres hash i fred. Vi foreslår et nyt device \texttt{/dev/hash}, så
alle kan få deres hash når de har brug for den.

\says{P0 + P1} FRI HASH!

\says{R} Jeg laver lige et kontekstskift.

\scene{R flytter U hen på et andet sted på scenen.}

\says{R} En af jeres andre mærkesager er at styrke filrettighederne. Går det
ikke imod resten af jeres politik om at øge åbenheden?

\scene{I løbet af næste replik laver U sin sidste brandtale ud mod publikum. R
giver signal til at procesdræberne fra næste sang skal gøre deres entré. U
bliver skudt.}

\scene{AV: Lyd af skud.}

\says{U} Filer kan lige nu ejes af brugere og grupper, men hvor passer vi
processer ind i billedet? Længe har kernen -- og andre processer fra samme
bruger -- kunnet læse vores dyrebareste hemmeligheder. Vi går ind for den
private ejendomsret! Som tidligere sagt, overvågning er ikke vejen frem!
\act{bliver dræbt før replikken afsluttes}

\scene{U kollapser.}

\says{R}[til procesdræberen] Hey, du glemte at kalde free! \act{peger på liget}

\scene{P0 og P1 løber rundt i kaos. R er nødt til selv at slæbe U ud.}

\scene{Transition over til Cold Blooded Killer. P0 og P1 bliver på scenen.}

\end{sketch}
\end{document}
