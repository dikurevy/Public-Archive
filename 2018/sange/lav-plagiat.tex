\documentclass[a4paper,11pt]{article}

\usepackage{revy}
\usepackage[utf8]{inputenc}
\usepackage[T1]{fontenc}
\usepackage[danish]{babel}


\revyname{DIKUrevy}
\revyyear{2018}
\version{1.0}
\eta{$3.2$ minutter}
\status{Færdig}

\title{Lav Plagiat}
\author{Mikkel Storgaard Knudsen, Christian Kjær}
\melody{Alletiders Nisse: ``Glad i bad''}
% https://www.youtube.com/watch?v=yvEC0wDtAU4

\begin{document}
\maketitle

\begin{props}
  \prop{Pappedatamat}
  \prop{Stole}
  \prop{Bord}
\end{props}

\begin{roles}
\role{S0}[Nathalia] Sanger
\role{S1}[René] Sanger
\role{S2}[Nicklas] Sanger
\role{X}[Caro] Instruktør
\end{roles}

\begin{sketch}
  \scene{META: Da denne sang åbner en akt, og sangen er relativt rolig, har jeg skrevet en indledning på før
    selve sangen, så publikum lige kan samle fokus på scenen inden sangen går i
    gang. Jeg ved ikke hvordan S0-2's roller skal fordeles, så rollerne i
    sketchudkastet er udelukkende placeholders.}

  \scene{SCENE: To studerende sidder og snorksover ind over et bord. På bordet er en datamat}
  \scene{Pludselig bliver de vækket (voldsomt!) af et vækkeur på datamaten. Lys
    op!}

  \says{S1}[overrasket] Jeg sover ikke!
  \says{S2}[forskrækket] Aah!
  \scene{S2 strækker sig}
  \says{S1}[kigger på datamaten] Hvor meget mangler vi at lave?
  \says{S2}[tænker sig om] Altså, hvis vi har alt rigtigt i det vi har lavet
  indtil nu, så har vi \emph{i hvert fald stort set} .. 15 points.
  
  \says{S1} Iiiih.. Så når vi det sikkert slet ikke.
  \says{S2} Jo jo, skal vi ikke bare lede lidt inde på StackOverflow?
  \says{S1} Hmmm..
  \says{S2} Jo jo! Kig her; \act{S2 taster og læser højt} ``how to implement good pipeline,
  how to...''. 

  \scene{Mens S2 taster ting ind i SO, blinker lyset blødt, A/V spiller lyd af stjernestøv der drysser.}
  \scene{S0 springer frem fra bagtæppet med en pop-lyd!}

  \says{S0} Hej med jer, rødder!
  \scene{S0 kigger på sit ur, dernæst på datamaten, og så på uret igen}
  \says{S0}[overdrevet men uoverrasket falsk bekymret] Uuuuhf.. kan I nå alt det
  der inden \act{kigger på sit ur} i eftermiddag?

  \says{S1+S2} nej-jamen..!

  \says{S0} Rolig nu drenge \act{tysser på dem} Det skal vi nok få styr på.
  Jeg har nogle ganske særlige tricks med til jer.
  De har været brugt af studerende i alt fra Inter til AP, fra CoCo til
  CompSys.. ja, selv i nogle bachelorprojekter! \act{S0 peger på tilfældigt
    publikum og blinker til ham/hende.}

  \says{S1} Er det eksamensteknikker?

  \says{S0}[muntert] Det kan man godt kalde det!
\end{sketch}
  
\begin{song}
\sings{S0}
Når du læser til eksamen må du have styr på tiden
ka' du ikke pensum får du smæk
men hvis blot du blander din \LaTeX{} med andre mensk'ers viden
tryller vi bekymringerne væk

Så stop nu med at spekulere
og gå i gang med at kopiere

Lav plagiat, syng ju-bi-du-æh
Det er bare hygge og lidt fjol
\sings{S*}
Lav plagiat, syng ju-bi-du-æh
så får du ganske sikkert tolv

Skrive af og skrive af og skrive ju-bi-du-æh
Hvorfor ha' endnu en ked'lig dag?
Det er så nemt at skrive af

\sings{S1}
Med et studiejob og fredagsbar så kniber det med tiden
Sku' du ik' ha' skrevet en rapport?
Du er knap og nap begyndt men deadline var for længe siden
Her er tricket til at få det gjort

Når du skal TeX'e, hvorfor knokle,
hvis du kan sidde ned og google?

Lav plagiat, syng ju-bi-du-æh
Det er bare hygge og lidt fjol
\sings{S*}
Lav plagiat, syng ju-bi-du-æh
så får du ganske sikkert tolv

Skrive af og skrive af og skrive ju-bi-du-æh
Hvorfor ha' endnu en ked'lig dag?
Det er så nemt at skrive af

\scene{Under mellemspillet hjælper S0 S1 og S2 med at købe en færdig ugeeksamen over nettet.}

\sings{S2}
%Når man skriver af så skal man slette sine referencer,
%det er ganske helt almindeligt pli
%Du kan adde en figur for at forøge dine chancer
%Det må ikke ligne en kopi

%Og hvis du knækker under prøven
%Så hiv beviser ud af røven

Se champignonskyer, der tårner sig op
Huder der sortner som blæk
Og strålingen smelter din feeeede krop
I morgen er verden væk

Og hvis der mangler lidt i sangen
Så plagier du bare en anden

Lav plagiat, syng ju-bi-du-æh
Det er bare hygge og lidt fjol
\sings{S*}
Lav plagiat, syng ju-bi-du-æh
Så får du ganske sikkert tolv

Skrive af og skrive af og skrive ju-bi-du-æh
Hvorfor ha' endnu en ked'lig dag?
Det er så nemt at skrive af

\scene{Lys ned.}
\end{song}

\end{document}
