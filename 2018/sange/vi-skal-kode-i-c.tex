\documentclass[a4paper,11pt]{article}

\usepackage{revy}
\usepackage[utf8]{inputenc}
\usepackage[T1]{fontenc}
\usepackage[danish]{babel}
\usepackage{hyperref}

\revyname{DIKUrevy}
\revyyear{2018}
\version{1.0}
\eta{$4$ minutter}
\status{Færdig}

\title{Vi skal kode i C}
\author{Sebastian Paaske Tørholm, Rene Løwe, Mikkel Storgaard}
\melody{Det Brune Punktum: ``Ud i det blå''}
% (\url{http://www.youtube.com/watch?v=NNwS4Br6F34})

\begin{document}
\maketitle



% præmis: gamle mennesker konkurrerer/diskuterer hvor nederen nye ting er
% catchphrase: baaaaaaaah!

% der er godt nok sket meget på de 19 år

\begin{roles}
\role{C0}[René] C-koder
\role{C1}[Sebbe] C-koder
\role{C2}[Nicklas] C-koder
\role{N0}[Amalie] Ninja
\role{N1}[Romeo] Ninja
\role{X}[Caro] Instruktør
\end{roles}

\begin{sketch}
  \scene{Der sidder tre gamle bitre dataloger ved et busstoppested. Tænk stok,
    skæg, rullekufferter og leverpletter. Jo flere gammel-agtige ting de kan
    have, jo bedre. C2 sidder og læser Jyllands-Posten.}

  \scene{Måske idler de bare indtil publikum har fået ro over sig selv.}

  \says{C0}[kigger op fra avisen] Hrrrmf! Nu er ``machine learning'' åbenbart
  blevet moderne... Skulle det nu være noget særligt..?
  \says{C0}[kigger med i avisen] ``TensorFlow'' i ``Pyyyyyython''? Baaaaaah...
  \says{C0+C2} Python? Baaaaaaaaaah!
  \says{C2}[laver ung stemme, barnligt] wah wah wah se mig, jeg er Python...
  \scene{C1 falder lidt i søvn indover sin stok}
  \says{C2}[fortsat] jeg bruger dynamiske typer
  \says{C1}[Vågner!] Dynamiske typer?? BAAAAAAAAAAAAAAH! \act{kaster sin hat i
    gulvet i vrede}
  \says{C0}[enig, til C1] Ja, baaaaaaah!
  \says{C1}Kan man måske kode kerne i Python?
  \says{C2}Kode kerne i Python??
  \scene{kort kunstpause}
  \says{C2}Nej, det kan man da naturligvis ikke, det er da absurd! Ha ha ha ha!
  \says{C0+C1} Ha ha ha ha!
  \says{C0} Lortesprog! (vigtig at o'et i lortesprog er [o] og ikke [å])
  \scene{C0+C1+C2 griner meget selvsmagende til og med publikum. De må gerne
    smaske lidt.}
  \says{C2} Nej, da jeg var rus.. \act{kigger drømmende ud i luften} i det
  gamle årtusind, var alting smukt.. jeg husker at eksamenerne var sværere..
  \says{C1} Ja.. og øllerne koldere..
  \says{C2} og vi var smukkere..
  \says{C0} Ja.. så vidt jeg husker var det cirka.. sådan her!
  \scene{Musik på! C0+C1+C2 springer op, smider deres skæg og er lækre og gør
    klar til lækker sang!}
\end{sketch}

\begin{song}
\sings{C0} %
    Nu er det sommer.
    Pawel er væk indtil AD næst' år
    så skal der kodes lidt,
    men i hvad? Ja, prøv bare at gæt'.

\sings{C1} %
    Ikke i Java --- det er \emph{objekt}-tivt noget juks
    (Og) typerne i F\# gør det svært at få noget gjort
    Giv mig heller' unions, structs og void pointers

\sings{C*} %
    Vi skal kode i C,
    der hvor alting kan ske
    lad os se hvor beskidt det kan bli'.
    Selv allokere, segfaulte mere
    fordi der ik' var noget at free

\sings{C2} %
    Jeg kører koden
    imod al forventning går det lidt langsomt
    Skal der inlines nu
    eller skal der en makro til?
    Optimerer? Nej, vi præprocesserer!

\sings{C*} %
    Vi skal kode i C,
    der hvor alting kan ske
    lad os se hvor beskidt det kan bli'.
    \#definere \texttt{MAX\_INT} til fire (\#definere synges \emph{hash definere})
    for så kan den bruges som $\pi$.

\sings{C*} %
    Vi skal kode i C,
    der hvor alting kan ske
    lad os se hvor beskidt det kan bli'.
    Dereferere en \texttt{void*} mere (\texttt{void*} synges \emph{void pointer})
    for at se hvad der ligger deri.

\scene{Mellemspil}

\sings{C*} %
    Vi skal kode i C,
    der hvor alting kan ske
    lad os se hvor beskidt det kan bli'.
    Inkrementere det vi definerer
    når \texttt{i = i++ + i;} (synges: \emph{i lig i plus plus plus i})

\sings{C*} %
    Vi skal kode i C,
    der hvor alting kan ske
    lad os se hvor beskidt det kan bli'.
    Omgå compiler, når nu at den fejler
    med lidt inline assembleri.

\end{song}

\end{document}
