\documentclass[a4paper,11pt]{article}

\usepackage{revy}
\usepackage[utf8]{inputenc}
\usepackage[T1]{fontenc}
\usepackage[danish]{babel}
\usepackage{hyperref}

\revyname{DIKUrevy}
\revyyear{2018}
% HUSK AT OPDATERE VERSIONSNUMMER
\version{1.0}
\eta{$0.45$ minutter}
\status{Ikke færdig}

\title{Det er jo frebar!}
\author{René, Sebastian}

\begin{document}
\maketitle

%\begin{roles}
%\role{F}[] Fulderik
%\role{K}[] Kameramand
%\end{roles}

%\begin{props}
%\prop{Rekvisit}[Person, der skaffer]
%\end{props}


\begin{sketch}
\scene{Parodi på \url{https://www.youtube.com/watch?v=8Eb-M1eywpY}. Filmes med mobiltelefon ligesom originalen.}

\scene{Kameraet følger Kasper (K) i kostumet fra Superstring Collider mens han går fordrukkent hen mod et auditorium på HCØ}

\says{P} Er du ikke lidt fuld? Skal jeg ikke hjælpe dig med at finde en sofa, min ven?

\says{K} Nej, det' jo frebar!

\says{P} Er det frebar? Men skal du så ikke hen på Caféen??

\says{K} Hahahaha, det' fre'bar!

\scene{K går videre mod auditoriet}

\says{P} Jeg tror ikke du skal gå derind - der er forelæsning lige nu. Folk sidder jo og arbejder derinde.

\says{K}[afvisende] Neeej, det er jo russer.

\says{P} Men tror du ikke det bliver lidt distraherende med det tøj du har på?

\scene{K tænker kort}

\says{K} Spar tøjet. Spar tøjet.

\scene{K smider kitlen og går ind i auditoriet}

\scene{Kamera panner ned på kitlen, og op igennem auditorievinduet}

\says{P} Ja, det må man nok sige. Hyg dig! \act{stille} For saaatan\ldots

\end{sketch}
\end{document}
