\documentclass[a4paper,11pt]{article}

\usepackage{revy}
\usepackage[utf8]{inputenc}
\usepackage[T1]{fontenc}
\usepackage[danish]{babel}


\revyname{DIKUrevy}
\revyyear{1991}
\version{1.0}
\eta{$n$ minutter}
\status{Færdig}

\title{Velkomstsang}
\author{?}
\melody{Shu-Bi-Dua: ``Danmark''}

\begin{document}
\maketitle

\begin{roles}
\role{S}[?] Sanger
\end{roles}


\begin{song}
\sings{S}%
Når jeg er færdig her, så begynder revyen,
imens jeg synger så går Freja ned.
Og vor's bestyrer han på Coca-cola gror
men han er heller ikke stor.
Kæld'ren får en næsestyver
dem der siger andet lyver
Vor's revy er bar' så god.
Vi har X og meget andet
vores vitser er så vandet.
Vi begynder li'e om lidt.

Se opp' i trappetårnet, der sidder en speaker
om et minut vil han introducer'
imens I klapper taktfast råbende ju-hu.
Hvis I da ikke råber buuh.
Vor revy er noget særligt,
men når jeg skal si'e det ærligt
har jeg glemt det sidste vers.
Jeg kan heller ikke rime
jeg har prøvet det en uge.
Nu er vor revy begyndt!!

Men dette vers det er det sidste i sangen.
Om freja-fejden der siger vi blot:
At ballerup og tvind og bims de får lidt tæsk,
hvis de da ikke får en gang lak!
Dette vers det synges ikke,
af en sanger med lidt hikke.
Det her vers det bliver glemt.
Det kan heller ikke synges,
gulvet det kan ikke gynges.
Derfor er det ikke så slemt.
\end{song}

\end{document}
