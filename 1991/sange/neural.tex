\documentclass[a4paper,11pt]{article}

\usepackage{revy}
\usepackage[utf8]{inputenc}
\usepackage[T1]{fontenc}
\usepackage[danish]{babel}


\revyname{DIKUrevy}
\revyyear{1991}
\version{1.0}
\eta{$n$ minutter}
\status{Færdig}

\title{Neural-sangen}
\author{?}
\melody{``Værket skal ligge...''}

\begin{document}
\maketitle

\begin{roles}
\role{S}[] Sanger
\end{roles}


\begin{song}
\sings{S}%
Vi sad tre-fire stykker med en øl oppe på taget.
Der var mig og et par andre lærer' fra 1P.
Vi snakked' lidt om hvad der sådan skete her på faget,
og om de her neurale net, de viste på TV.
Snakken gik om de der net, det' noget vi skal ha'.
Og det ment' vi ikk' der ku' vær' tvivl om at vi ska'.

Men kurset skal ligg' på Dat-2 har vi tænkt,
eller 0, det synes vi bestemt.
For ha' det her hos os,
nej det vil vi godt nok ikk',
for så bli'r vores kursus nok for klemt.

Videnskaben har bevist, de selv kan tænke -- næsten!
Det var det han sagde, ham der ovre fra Niels Bohr.
Og om et par år, så kan de klare Turing-testen.
Så kan de bli' ansat her, men ud'n eg't kontor.
Og hvad arbejdsløshed angår, ja så mener vi.
At det ikk' kan nå at gi' problemer i vor tid

Men kurset...

Det bli'r skønt når vi får så'n nog'n netværk installeret.
Dem skal vi da snart få trænet i datalogi.
Så ska' vi sgu snart få hele sted't mekaniseret,
så de få der endnu går og forsker kan få fri.
Oh, så skønt vi får det, all' vor's trænsler er forbi.
Vi kan møde klokken elv' og gå når den er ti.

Men kurset...
\end{song}

\end{document}
