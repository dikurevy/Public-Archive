\documentclass[a4paper,11pt]{article}

\usepackage{revy}
\usepackage[utf8]{inputenc}
\usepackage[T1]{fontenc}
\usepackage[danish]{babel}


\revyname{DIKUrevy}
\revyyear{1991}
\version{1.0}
\eta{$n$ minutter}
\status{Færdig}

\title{Lektorsangen}
\author{?}
\melody{``Den gamle gartner''}

\begin{document}
\maketitle

\begin{roles}
\role{S}[] Sanger
\end{roles}


\begin{song}
\sings{S}%
Når jeg ser på små programmer,
dukker glemte traumer frem.
Om min gamle, skøre lektor,
som nu sidder på et hjem.

Jeg skal aldrig, aldrig glemme,
hans kedsommelige stemme.
Han var vis og han var klog,
han had' kigget i en bog.

Kode C og kode Fortran,
bruge U-forstå'ligt sprog.
Lave alting om til LISP,
og finde fejl i hver en krog.

Når jeg læste til eksamen,
blev jeg tavs og ganske stum.
Og mit hoved blev forvandlet,
til det store tomme rum.

Det var strengt for en dat-0'er,
med en viden fuld af huller.
Da'n var barsk og hård og trang,
når den gamle lektor sang.

Give FEM og give NUL-TRE,
give fryg-te-lig kritik.
Slagte løs på alle kræ,
og skabe angst og vild panik.

Når jeg ser på små maskiner,
ka' jeg bli'e no'et så træt,
og jeg tænker på assembler,
og de timer jeg har grædt.

Jeg får stadig frustrationer,
af Motorola-instruktioner.
Når man load'ed koden ned,
blev det ren til-fæl-dig-hed.

Kode orm og kode kerne,
kode uret ligeså.
Kigge på de små processer,
til de pluds'lig gik i STÅ!
\end{song}

\end{document}
