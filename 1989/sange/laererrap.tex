\documentclass[a4paper,11pt]{article}

\usepackage{revy}
\usepackage[utf8]{inputenc}
\usepackage[T1]{fontenc}
\usepackage[danish]{babel}


\revyname{DIKUrevy}
\revyyear{1989}
\version{1.0}
\eta{$n$ minutter}
\status{Færdig}

\title{Lærerrap}
\author{?}
\melody{? (generisk rap-beat)}

\begin{document}
\maketitle

\begin{roles}
  \role{A}[?] Rapper
  \role{A}[?] Rapper
\end{roles}

\begin{song}
  \sings{A}Lærerne på DIKU er en broget flok
  \sings{B}allerførst i spidsen går Gregers Koch

  \sings{A}Valget til studienævnet klarer de nemt
  \sings{B}lodtrækning sørger for at ingen bli'r glemt

  \sings{A}Skelboe sætter tonen med næsen i skyen
  \sings{B}Tucker danser for med pigerne i byen

  \sings{A}Hasse Clausen skriver bøger for bløde bruger'
  \sings{B}Torben Ulrik Zahle skriver baser der duer

  \sings{A}Klaus Grue stilled' opgave; ingen ku' forstå'et
  \sings{B}Frøkjær han snakker, men siger ikke no'et

  \sings{A}Netværk er Klaus Hansens ekspertis
  \sings{B}Gregers Koch holder mest af Gries

  \sings{A}Troldmanden Jones har semantik som passion
  \sings{B}PJo blev dekan; han savnes ikke af no'en

  \sings{A}Kritikeren Naur han er altid på tværs
  \sings{B}DIKU passer ikke i hans univers

  \sings{A}Lærerne på DIKU er en broget flok
  \sings{B}Studenterne på DIKU de har mer' end nok!
\end{song}

\end{document}

