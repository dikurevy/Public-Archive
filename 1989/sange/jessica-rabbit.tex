\documentclass[a4paper,11pt]{article}

\usepackage{revy}
\usepackage[utf8]{inputenc}
\usepackage[T1]{fontenc}
\usepackage[danish]{babel}


\revyname{DIKUrevy}
\revyyear{1989}
\version{1.0}
\eta{$n$ minutter}
\status{Færdig}

\title{Jessica Rabbit}
\author{?}
\melody{Melodien til "`Jessica Rabbit"' eller hvad fanden den nu hedder}

\begin{document}
\maketitle

\begin{roles}
\role{K}[] Sanger
\role{S}[] Sangerinde
\end{roles}

\begin{sketch}
\scene{En koder sidder ved sin maskine, bander over et eller andet (evt. AI-sang før) og siger:}

\says{K} Årh, så vil jeg sgu hellere prøve at få de sidste point i Leisure Suit Larry.

\scene{Han stikker en disk i og begynder at spille.  Ud af skærmen
  (hvis det kan lade sig gøre på en eller anden måde) træder
  sangerinden.}
\end{sketch}

\begin{song}
\sings{S}
Du sidder og spiller hele natten lang
Hvor har du dog fået denne sære trang
Gå dog i byen
Ud i det rigtige liv
På fing'ren ud!
Vis nu lit initiv

Gå med til nogle fester, se hvad der kan ske
Måske du møder en, der har den samme idé
Gå dog i byen

(Mellemspil)

(Evt: Koderen tilbyder hende en eller flere af følgende: En ring, en
rose, en æske chokolade, et pilleglas og et æble (genstande fra
spillet) Hun fejer dem blot til side med et spil, og nusser ham i
håret, knapper hans skjorte lidt op osv.)

Så glem du alt om mig og min HD-grafik
Brug hel're natten på lidt ægte erotik
Gå dog i byen...

\end{song}

\end{document}

