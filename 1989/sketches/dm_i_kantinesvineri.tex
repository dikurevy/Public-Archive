\documentclass[a4paper,11pt]{article}

\usepackage{revy}
\usepackage[utf8]{inputenc}
\usepackage[T1]{fontenc}
\usepackage[danish]{babel}


\revyname{DIKUrevy}
\revyyear{1989}
% HUSK AT OPDATERE VERSIONSNUMMER
\version{1.0}
\eta{$n$ minutter}
\status{Færdig}

\title{DM i kantinesvineri}
\author{?}

\begin{document}
\maketitle

\begin{roles}
\role{K}[] Kommentator
\role{F1}[] Førsteårsstuderende 1
\role{F2}[] Førsteårsstuderende 2
\role{A1}[] Andendelsstuderende 1
\role{A2}[] Andendelsstuderende 2
\role{D}[] Dommer
\role{R}[] Rengøringskone
\role{HD}[] HD'er
\end{roles}

\begin{sketch}

  \says{K} Og vi byder velkommen til de åbne mesterskaber i
  kantinesvineri, der i år afholdes her i de nærmest ideelle
  omgivelser i DIKU's kantine.  Deltagerne er så småt ved at
  indfinde sig...

  \scene{To førsteårsstuderende kommer på scenen med lidt tallerkner,
    tomme kiks, teposer, mm.}

  \says{K} ... her på min højre side har jeg et hold bestående af
  førsteårs datalogistuderende, der overraskende nok har formået at
  kæmpe sig vej frem til finalenn - trods deres ringe erfaring i
  international sammenhæng lægger de en forbavsende entusiasme for
  dagen.  På vej til finalen har de slået så stærke mandskaber som
  Vestforbrændingen og Kommune-Kemi - og hvem husker ikke den
  forrygende semifinalekamp mod det blandede hold fra politiets
  uropatrulje med udvalgte BZ'ere, hvor de førsteårsstuderende måtte
  ud i forlænget svinetid for at få kampen afgjort.

  \scene{To andendelsstuderende kommer på scenen medbringende
    alm. svineri (tallerkner, cigaretter, tom ølflaske, rødbedemad,
    mm.)}

  \says{K} Her på den anden side af mig har jeg her sidste års nummer
  fire: Et blandet hold af andendelsstuderende, der i deres semifinale
  tog revanche fra sidste års nederlag i kvartfinalen til Kemisk Værk
  Køge - kampen blev stoppet før tid på teknisk knock-out da de
  andendels-studerendes fiskefrikadeller brændte på netop under
  røg-alarmen og et hold røg-dykkere med skumsprøjter måtte tilkaldes
  før de andendels-studerende kunne få deres velfortjente sejr.

  \says{K} Og kampen er så småt ved at gå i gang - dommeren kigger på
  sit ur - dommeren er for øvrigt en nuværende forelæser på datalogisk
  institut, han var med på det vindende hold bestående af
  datalogi-forelæsere i 88, men måtte desværre ikke deltage i år på
  grund af de nye regler om at et hold ikke på vinde mere end tre år i
  træk - for øvrigt en skam, da der virkelig var tale om et
  stilskabende og nytænkende hold indenfor kantinesvinerisporten -
  hvem husker ikke den sojakage de bagte i finalen i '86 mod de
  københavnske kloakarbejdere... nå, men dommeren kigger endnu engang
  på sit ur.

  \scene{Dommeren fløjter - deltagerne rumsterer med deres ting.}

  \says{K} Begge holds deltagere lægger forsigtigt ud - de venter
  afventende på modpartens første træk.  De krummer diskret og
  rutinepræget - eller skyldes det blot gammel vane... der kommer
  kampens første træk - de førsteårsstuderende skubber forsigtigt en
  pakke tomme kiks på gulvet - de andendelsstuderende har ikke noget
  modtræk... jo, jo anføreren stiller en tom ølflaske om bag sit ene
  stoleben.

  \says{K} Den tomme ølflaske er et af de klassiske elementer i
  kantinesvinerisporten, og havde det ikke været en offentlig
  turnering kunne den have blevet stående i månedsvis - ja i årevis!
  Hvem husker ikke den Star-flaske der blev fundet sidste år med Peter
  Naur's afslørende fingeraftryk på - den blev dateret ved hjælp af
  kulstof-14 metoden til at være en 1973'er.  Den var i den grad groet
  ind i miljøet at den flyttede med til disse nye omgivelser.

  \says{K} Nå, men jeg fortaber mig... I dette øjeblik forlader de
  unge håb deres stole efterladende en stabel tallerkner og en udsøgt
  samling gamle æbleskrog - iblandt disse skelner jeg et Cox Orange
  fra efteråret 1971 netop indkøbt i HCØ's kantine.

  \says{K} Meeen... Hvad ser jeg ovre på pensionisternes plads... en
  {\em Kernighan \& Ritchie} - den slags smudslitteratur kan dommeren
  da umuligt tillade...

  Og nu, NU ser dommeren det... han fløjter... de andendelsstuderende
  idømmes Dat 1 kursusbog 7... en noget uforståelig dom.

  \says{K} Men hvem er det dog der dukker op på banen?  En HD'er.

  \scene{Dommeren fløjter.}

  \says{K} Denne sportsgren er trods alt ikke for professionelle
  svinere.

  \scene{Dommeren idømmer KB5}

  \says{K} Og ganske rigtigt - han idømmes kursusbog 5 - en hård men
  fortjent dom!

  \says{K} ... og som noget nyt i år to funklende rene kopper... ved
  lodtrækningen blev det afgjort at de førsteårsstuderende skulle have
  denne fordel - læg mærke til hvorledes de som tegn på godt
  sportsmanship viser disse jomfruelige skønhedsåbenbaringer
  frem... men NU, hvad sker der?  ... De andendelsstuderende har
  tilkæmpet sig fordelen ved at skodde i de nye kopper!  Et smukt og
  aggressivt træk!

  \says{K} Men hvad gør de tørstige
  førsteårs-studerende... oversvineren hiver en urtepotte
  frem... smukt returneret.

  \scene{Planten smides over skulderen.}

  \says{K} ... kampen står lige på point nu...

  \scene{Kaffe fra termokande hældes op i urtepotten, men løber ud
    gennem hullet i bunden.}

  \says{K} ... og nu rykker de førsteårsstuderende...

  \scene{En andendelsstuderende åbner et "`tomt"' syltetøjsglas, den
    anden smider en "`rødbedemad"' på gulvet}

  \says{K} ... det er vist bananfluer det der, det må være længe siden
  holdet har været på DIKU, dem har vi jo i forvejen, men i øvrigt en
  smuk detalje med rødbedemaden!!!

  \scene{De førsteårsstuderende slynger to te-poser rundt i luften}

  \says{K} ...uha !  De førsteårsstuderende svarer straks igen...

  \scene{2. delsstuderende vader på maden.}

  \says{K} Bemærk med hvilken elegance rødbedemaden tværes ud over og ned i gulvet.

  \scene{Rødmedemadsvaderen overfaldes af MOR, der slår ham oven i
    hovedet med kost}

  \says{K} ... Men hvad sker der... uha, det er rengøringspersonalet!!
  Dommeren fløjter kampen af og trækker sig tilbage for at ROTERE...

  \says{A1+A2}[råber] MOR!!! Hvor tit skal jeg sige til dig at du ikke
  arbejder her!!!

\end{sketch}
\end{document}
