\documentclass[a4paper,11pt]{article}

\usepackage{revy}
\usepackage[utf8]{inputenc}
\usepackage[T1]{fontenc}
\usepackage[danish]{babel}


\revyname{DIKUrevy}
\revyyear{1989}
% HUSK AT OPDATERE VERSIONSNUMMER
\version{1.0}
\eta{$n$ minutter}
\status{Færdig}

\title{Sikkerhedskopien}
\author{?}

\begin{document}
\maketitle

\begin{roles}
  \role{K}[] Konferencier
  \role{VO1}[] Voiceover
  \role{VO2}[] Kvindelig voiceover
  \role{VO3}[] Endnu en voiceover
  \role{P1}[] Pausefisk
  \role{P2}[] Pausefisk
  \role{P3}[] Pausefisk
\end{roles}

\begin{sketch}

  \scene{Mørke.  Lydeffekt: Forskellige lyde (eksplosioner, skrig,
    knirkende døre, båndsalat, hvad som helst.).  Lys: Herunder
    evt. lys på tom scene, som hurtigt slukkes igen, en enkelt søgende
    spot osv.  Kort sagt: Kaos.  Slutter med mørke.}

  \says{VO1} Vi beklager, men vi har nogle tekniske problemer.  Der vil
  blive en kort pause.

  \scene{Lys på scene.  Pausefiskene kommer ind, og bevæger sig langsomt
    rundt på scenen med fiskemunde og "`svømmetag"'.}

  \says{VO2}[mekanisk] Gå ikke over scenen, der kommer
  pausefisk. \act{Klokkelyd} \act{Gentages i ca. 15 sekunder}

  \scene{Pausefisk ud og konferencier ind.  Lys: Spot på ham.}

  \says{K} Ved et beklageligt uheld er teksterne til 2. akt desværre
  blevet slettet.  Men der er ingen grund til uro.  Vi har et bånd med
  generalprøven.  Er der en superbruger til stedet i salen?

  \scene{Salen svarer antagelig, og her må konferencieren improvisere.
    Hvis salen ikke har påpeget det, råbes der fra kulissen: "`Der
    sidder en ovre i orkestret"' (hvis Kim Høglund er med).}

  \says{K} Mange tak \act{går ud - lyset slukkes}.

  \says{VO3} En arbejdsplads i Danmark.

  \says{Råb}[evt. på bånd] Nej, 5, {\em ikke} 2!

\end{sketch}
\end{document}
