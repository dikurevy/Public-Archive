\documentclass[a4paper,11pt]{article}

\usepackage{revy}
\usepackage[utf8]{inputenc}
\usepackage[T1]{fontenc}
\usepackage[danish]{babel}


\revyname{DIKUrevy}
\revyyear{1989}
% HUSK AT OPDATERE VERSIONSNUMMER
\version{1.0}
\eta{$n$ minutter}
\status{Færdig}

\title{Ethernet-sketch}
\author{?}

\begin{document}
\maketitle

\begin{roles}
  \role{M1}[] TCM
  \role{M2}[] ODML
  \role{S}[] Speaker
\end{roles}

\begin{sketch}

  I begyndelsen var ordet - og ordet var på 4 bytes.

  Nej, for at være seriøs, så sidder NULL på scenen som koder (der har
  netop været "`Barndommens kode"'), og scenen ligger mørk hen - bortset
  fra spot på NULL.  Beaker kommer ind, og stiller sig hen ved NULL med
  den ene hånd på terminalen.  Så kommer indledning.

  \says{S} Nu sidder vores koder og taster.  Han synes, der er lange
  svartider, og det er jo ikke så sært, for når han taster nogle tegn og
  trykker retur, så sker følgende:

  \scene{Lys går op til at dække hele scenen D.v.s. ingen spot.}

  \says{M1+M2} Jeg vil godt tale med Kurt/Egon

  \scene{M1 og M2 sætter sig ned, genererer et tilfældigt tal, og tæller
    til det.  Prøver så igen.}

  \scene{Kort pause}

  \says{M1} Øh, er Kurt derude?

  \says{M2} Ja, her er jeg.  Er der nogen, der vil tale med mig?

  \says{M1} Ja, det er Egon.  Jeg vil godt sende et tegn til dig.  Er du
  klar?

  \says{M2} Ja, jeg er klar.  Send du det bare.

  \says{M1} Her kommer det.

  \says{M1} Nu har jeg sendt det.  Fik du det?

  \says{M2} Ja ja, det ser fint ud.  Paritetschecket er OK.

  \says{M1} Send det lige tilbage, så jeg kan se, om det er det samme,
  jeg sendte til dig.

  \says{M2} OK, er du klar?

  \says{M1} Ja, jeg er klar - send det bare.

  \says{M2} Jeg har sendt det - har du fået det?

  \says{M1} Ja, det ser fint ud.  Det ligner det, jeg sendte afsted.  Nu
  vil jeg sende det næste tegn.  Er du klar?

  \says{M2} Ja, jeg er klar.  Send det bare.

  \says{M1} Her kommer det.

  \says{M1} Nu har jeg sendt det.  Fik du deT?

  \says{M2} Ups, der er paritetsfejl.  Der er noget helt galt.  Hvordan
  klarer vi dette?

  \says{M1} Jo, hvis jeg nu prøver at sende tegnet en gang til.

  \says{M2} Det lyder som en god idé.  Lad os gøre det.

  \says{M1} OK.  Er du klar til at modtage et tegn?

  \says{M2} Jeg er klar.

  \says{M1} Jeg har sendt det - har du modtaget det?

  \says{M2} Jeps.  Paritetschecket er perfekt.

  \says{M1} Jeg vil godt lige have det tilbage igen - for en sikkerheds skyld.

  \says{M2} OK, er du klar?

  \says{M1} Ja, jeg er klar - send det bare.

  \says{M2} Jeg har sendt det - har du fået det?

  \says{M1} Ja, det ser fint ud.  Det ligner det, jeg sendte afsted.
  Men nu har jeg ikke flere tegn at sende.  Tak for denne gang - og hils
  derhjemme.

  \says{M2} Tak i lige måde - det var hyggeligt.

\end{sketch}
\end{document}
