\documentclass[a4paper,12pt]{article}

\usepackage{revy}
\usepackage[utf8]{inputenc}
\usepackage[T1]{fontenc}
\usepackage[danish]{babel}

\revyname{DIKU-revy}
\revyyear{2001}
\version{2.01}
\eta{3,5-7 min}
\status{Sort og goD}
\title{Datalogiens Vugge}

\author{Rune, Uffe Friis Lichtenberg m.fl.}

\begin{document}
\maketitle

\begin{sketch}

\begin{roles}
 \role{N}[Hall] Peter Naur
 \role{A}[Jørgen] Niels Andersen
 \role{L}[Jonas] Lagerbestyrer
 \role{UD}[Uffe] Undtagelse - klædt som en ``Falsk vagt'' og Dekan
 \role{E}[Katrine] Else - i en nederdel lavet af hulkort
 \role{S}[Rune] Søren Mukke - i 50'er version
 \role{Vo}[Jan] Voiceover
\end{roles}

\begin{props}
 \prop{En Planche i fire Graatoner} laves
 \prop{En hulkort nederdel} laves af Katrine H.
 \prop{En ``Falsk'' Skjorte} haves delvist (mangler skjorten
) \prop{En muggert} laves/købes
 \prop{En dekan dragt der kan tages paa i en fart} Ccilie
 \prop{Sort/hvid halvtredser tøj til 6 personer} haves delvist
 \prop{et håndklæde}
 \prop{Naur hat} Bo
\end{props}

\scene{Hvis sketchen er for lang kan Vogon-afslutningen fjernes
  og/eller hele Søren Mukke segmentet, afhængig af det opførende holds
  præferencer.}

\scene{To ældre herrer staar paa scenen (anakronismer er gode). Den ene
er Peter Naur (N). Den anden er hidtil unavngiven (A). Begge herrer
taler paa en overdrevet affektereret maade, i stil med de sort-hvide
smaasketches i Gintbergs Showoff. Sketchen indledes evt. med et skilt
"København anno 1950".}

\says{A} Godaften - hr. Peter Naur - Hvordan skær'n med den
elektroniske databehandling?

\says{N} Tak, skæppeskønt. Ved de hvad? Skulle vi ikke lave et Fag, De
ved, paa Universitetet og kalde det Datalogi?

\says{A} Nej ved De nu hvad, hvad skulle man dog med en Datalog?
Starutten ville jo aldrig kunne faa Arbejde.

\says{N} Næh, men jeg kunne faa et Professorat. \act{pause} Jeg har
arbejdet paa en Enhed til Tidsmaalinger paa min Datamat. Altsaa, den
faar jo sin energi fra en kondicikel...

\scene{Hvis Personen der spiller Naur kan, saa kunne man evt. lade Naur cikle rundt paa en et-hjulet cikel...}

\says{A}[afbryder] Cikler De rundt herinde? Nej, nu maa De passe paa
ikke at *vælte Peter*. De ved, De klokker altid i det.

\says{N} HEUREKA! Jeg kalder Enheden for Clockcikler. Min gode Mand,
De er genial - De skal være Lektor!

\says{A}[glad] Den er mægtig! Men vil de tænke Dem, vores Sekretær
laver stavefejl i vores Hulkort. Hun kan ikke hulle større Enheder end
Oktetter.

\says{N} Er hun *Ord*blind?

\says{A} Aah, dér kom jeg nok til *Kort*.

\says{N} *Hul* i det. Hun skal blot have nogle *dask*, saa *gi'r* hun
sig nok.

\scene{pause}

\says{N} Jeg skulle ellers have indlæst mit nye Spil - jeg tror jeg
vil kalde det Pong - men jeg kan ikke komme i Kontakt med Datamaten.

\says{A} Avra for *Naur*, kan De ikke *pinge Pong*?

\says{N} Næ, min gode Mand, jeg tror saadan noget spilleri er en skidt
Idé. Hvem ved hvad det kan føre ungdommen ud i?

\says{A} Nej, ved de hvad! Blot fordi man laver et uskyldigt Spil med
et fjollet Bat og en firkantet Bold, betyder det jo ikke, at Spil med
Tiden vil udvikle sig til at blive voldelige...

\says{N}[fortsætter sætningen] \ldots og indeholde brutale, blodige
Sekvenser, med \ldots

\says{A}[fortsætter sætningen] \ldots maskinpistoler, \ldots

\says{N}[fortsætter sætningen] \ldots motorsave \ldots

\says{A}[fortsætter sætningen] \ldots og slimede Uhyrer \ldots

\says{N}[fortsætter sætningen] \ldots fra det ydre Rum, \ldots

\says{A+N}[fortsætter sætningen i kor] \ldots der spiser hjerner og vil
indvadere Jorden.

\scene{pause}

\says{A}[griner fjoget] Nej, det har De da evigt ret i.

\says{N} Den slags repræsenterer jo ikke Tidens Værdier.

\says{A} De unge mennesker har jo deres legemsøvelser og deres spejderdyder.

\says{N} Nej, saadan \act{griber i luften og tager "noget" i haanden}
nogle nymodens Værdier vil vi ikke have!

\scene{A forlader scenen, mens N gaar rundt for at vise at han gaar
  et andet sted hen. En lagerbestyrer (L) og en "Falsk vagt" (U)
  kommer ind. U lægger sig ned for fødderne af L. N gaar hen til L.}

\says{N} Ja goddag', jeg vil gerne returnere denne nymodens Værdi.

\says{L} Desværre hr., det kan De ikke. Lageret er fyldt op med
Værdier.

\says{N} Fyldt op?

\says{L} Ja. Lageret er meget *værdifuldt*.

\says{N} Naa. Jamen saa maa jeg jo reklamere. Tag denne her
Fejlbesked.

\says{L} Fejlbesked?

\says{N} Ja. Jeg kunne ikke returnere min Værdi, derfor returnerer jeg
denne fejlbesked istedet. \act{overrækker fejlbeskeden}

\says{L}[kigger paa fejlbeskeden] Hm. Denne Fejlbesked kan jeg ikke
haandtere. Jeg rejser lige en Undtagelse.

\scene{L hjælper U til at rejse sig op}

\says{UD} Goddag. De er nu i Undtagelsestilstand. Maa jeg se
billedidentifikation.

\says{N} De er da ikke nogen Falck-vagt. [peger paa "Falsk" logoet]

\says{L} Hov ikke mere Reklame her.

\says{UD}[prøver at gøre opmærksom paa sig selv] Hør her, jeg erklærer
Undtagelsestilstand!

\says{N} Tage er da ikke særlig ond!

\says{L}[forvirret] Og hvad har det med Elses Tilstand at gøre? Er hun
syg?

\says{N} Ja, hun har faaet en fæl Snue. Hun var uanstændigt klædt.
Hendes Nederdel var nærmest *Hulkort*!

\scene{E kommer ind og vimser omkring med sin hulkorts-nederdel}

\says{UD}[sur] Saa er det godt. Jeg afbryder denne Sketch. Den er alt
for plat!

\says{N}Jeg troede der var lukket for Afbrydelser i
undtagelsestilstand.

\says{L}Jeg synes det er synd at afbryde *for Else*.

\says{N}Nænej, det hedder "If... Then... Else..." ikke "For...
Else..."

\says{L} Forelsket i hvem?

\says{UD} Argh!!!

\scene{UD udvandrer i protest. Ind kommer S med sit sædvanlige
  charmesmil}

\says{S} Godaften, mit navn er Søren Mukke, og jeg er MCP-certificeret
rationaliseringsekspert. Hvor er I glade for at se mig! \act{pause}
[prøver igen] AAh, jeg ved hvad de tænker; Sikke en nydelig mand... Jeg
har netop analyseret Deres Virksomhed, og jeg er naaet frem til, at
denne unge Dame er overflødig.

\says{N} Hvordan! De er jo nys traadt ind af Døren.

\says{S} Hvor er de glad for, at de stillede det spørgsmaal. Jeg har
rationaliseret min egen beslutningsproces. Før i Tiden taenkte jeg mig
om, men det er jo irrationelt! \act{til E} Farvel unge Dame, de er
afskediget og - maa jeg sige dem een ting: hvor er De glad for at
blive fyret af mig.

\says{N}[vred] Nej! Saadan en tølper skal ikke afskedige mine
Medarbejdere.

\says{S}[meget slesk] Se dette letforstaaelige Diagram i fire
Graatoner. \act{rækker ham en Planche}

\says{N}[studerer planchen] Næh... \act{lyser op og ser overbevist ud}

\says{E}[i svime] Ih du milde! Han er vist en værre en! Aah jeg
daaner... \act{daaner}

\says{N} Ja. Det kan jeg jo godt se. \act{bukker sig ned til E og
  tager hendes haand, hun vaagner} Frk. Else, det gør mig ondt, men De
er fyret.

\says{E} Uf! \act{daaner}

\says{N}[belærende] Det hedder ikke "Uf", men "If", Else.

\says{S}[vred] Det hedder altsaa "If... THEN... Else..."! \act{hopper
  og tramper}

\scene{S skubber N ud af scenen. E vaagner op og henvender sig til S}

\says{E} Aah, hvad skal jeg dog gøre?

\says{S} Det skal jeg sige Dem. De skal... \act{uhyggelig musik} gifte
Dem med mig!

\says{E}[forfærdet] Aldrig, hr. Mukke. De er en bølle, en slubbert, en
luskebuks, en bandit, en skurk, en Sjuft og et Skarn. De kan faa een
paa frakken, kan de, ja de kan!

\says{S}[kæk] Kommer den unge Else med *betingede* sætninger? Saa maa
hun jo skæres væk.

\scene{Dekanen kommer ind.}

\says{UD} Stop. Jeg er dekanen. Denne sketch skal skæres væk. Den er dum og den er
plat - og saa skal der gøres plads til en verdensomspændende
informationsmotorvej.

\says{S} Dekan De ikke!

\says{UD} Jo, dekan jeg, for jeg er dekan, saa dekan jeg godt.  Og
nu skal du dø! \act{Tager en muggert frem}

\says{S} Hvad er det?

\says{UD} Det er en muggert, Søren.

\says{S} Hammerfedt\ldots \act{Dør af et akut tilfælde af daarlige
  ordspil (og adskillige slag i hovedet?)}

\says{UD} Dekan be only one!

\scene{En voiceover afbryder Dekanens sejrsrus}

\says{Voiceover med dyb stemme} Revytter, må jeg bede om jeres
opmærksomhed. Dette er \ldots den grumme ITU-dekan fra det ydre rum.
Som I utvivlsomt er klar over, kræver planerne for udviklingen af
landets højere læreranstalter, at DIKU skæres væk for at give plads
til en mere strømlinet uddannelse. Processen vil være overstaaet paa
mindre end to af Jeres \ldots Clock-cIkler.

\scene{Scenen eksploderer i larm og kraftigt lys. TÆPPE. En blaffer med håndklæde træder frem.}

\says{B} I har netop oplevet en kollektiv hallucination. Foregaaende sketch
virkede som nonsens. Det var den ogsaa --- den var et udslag af revyens
usandsynligheds-"drive". Revyen vil nu fortsætte (((uden for
WAP-speed))).

\end{sketch}
\end{document}

