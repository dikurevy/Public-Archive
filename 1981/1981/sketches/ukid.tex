\documentclass[a4paper,11pt]{article}

\usepackage{revy}
\usepackage[utf8]{inputenc}
\usepackage[T1]{fontenc}
\usepackage[danish]{babel}


\revyname{DIKUrevy}
\revyyear{1981}
% HUSK AT OPDATERE VERSIONSNUMMER
\version{0.1}
\eta{$n$ minutter}
\status{Ikke færdig}

\title{UKID}
\author{HVO, TAH}

\begin{document}
\maketitle

\begin{roles}
\role{SPK}[] Speaker
\role{CJ}[] Claus Jensen
\role{ME}[] Mejle Eriksen
\role{AN}[] Anders Nielsen
\role{KG}[] Koch Gregersen
\role{H}[] Hilbert, formand for Fagrådet
\end{roles}

\begin{props}
\prop{Rekvisit}[Person, der skaffer]
\end{props}


\begin{sketch}

  \scene{Transparent på bagtæppet med et spejlvendt billede af DIKU,
    hvor der står UKID.  Scenen er en TV-reportage og speakeren kommer
    ind.}

  \says{SPK} Ja, vi er lige landet på ressourceplaneten UKID, hvor vi
  vil besøge det berømte dalotagiske institut.  Vi har i den anledning
  fået fat på bestyreren Claus Jensen, som vil vise os
  rundt. \act{Præsenterer CJ} Claus, kan du til at begynde med
  fortælle os lidt om instituttet?

  \says{CJ} Jah, instituttet er et af universets ældste, men vi har
  lidt problemer med det dalende studentertal.  Til at begynde med var
  der mange nybagte studenter--brød, som blev optaget, men de sidste
  år har den manglende søgning været så overvældende at lærerne her på
  instituttet overvejer at tvangs-immatrikulere en vis kvote af dette
  års studenter.  Den megen plads her på stedet gør også at vi tit og
  ofte står og råber i de tomme rum.

  \says{SPK} Jeg har fået oplyst at vi ankommer på jeres traditionelle andendelsdag?

  \scene{SPK og CJ går lidt rundt på scenen, bagtæppe er transparenter
    af andendelskurser.}

  \says{CJ} Ja, det er rigtigt.  Her er f.eks. et kursus, som jeg selv
  afholder: Cirkulær programmering. \act{Transparent} Formålet med
  cirkulær programmering er at gennemløbe en given problemstilling
  uendeligt mange gange for "= om muligt "= at finde samtlige
  løsninger.  Jeg tror, at jeg vil lade nogle af mine kolleger selv
  præsenterer deres kurser.

  \says{ME} Go'da, mit navn er Mejle Eriksen, og jeg afholder kursus i
  uformelle sprog.  Alle traditionelle programmeringssprog har en
  meget streng syntax som ikke tillader programmør eller maskine bare
  den mindste smule frihed, hvorimod uformelle sprog tillader en helt
  aden grad af kreativ udfoldelse.  Et eksempel på almindelig syntax
  kunne være {\tt WHILE A DO B}, hvorimor en lignende uformel
  konstruktion ville hedde {\tt CA AS LONG AS A DO B OR SOMETHING LIKE
    THAT}.  Dette sidste må da også siges at være meget mere spændende
  for maskinen, idet det åbner mulighed for maskinel autonomi.

  \says{SPK} Ja tak, jeg tror vi har hørt rigeligt, vi går videre til
  Anders Nielsen som afholder kursis i... \act{afbrydes}

  \says{AN} Jeg er ked af det, men pga. overvældende stor
  studentertilslutning, er jeg desværre ikke i stand til at afholde
  mit kursus: Forværring af rekursive programmer.  Det lader til at
  størsteparten af de studerende har svært ved at se, hvordan deres
  programmer skal kunne laves værre.

  \says{CJ}[til AN] Det var en meget kedelig situation for dig.

  \says{CJ}[til SPK] Dette er virkelig meget usædvanligt.  Normalt er
  det sådan at lærerne må trække lod om de studerende til kurserne.
  Her er simpelthen for mange lærere og for få studerende.  Nå, men
  her træffer vi Koch Gregersen, som vil fortælle os om ægte
  uintelligens.

  \scene{KG optræder som dybt åndssvag.}

  \says{CJ} Det er et af de mere særprægede kurser, men det er meget
  populært, og det er jo også meget aktuelt for mange af vores
  studerende.

  \says{SPK}[Transparent: indre kantine] Vi har nu bevæget os ind i
  kantinen hvor vi møder kantineformanden.  Hvordan fungerer denne
  kantine egentligt?

  \says{KF} Det kører fantastisk godt.  Vi har lige fået præmie af
  sundhedsmyndighederne, fordi her er så rent.  Det skyldes at folk er
  så flinke til at rydde af, vaske op og stille på plads.  Folk er
  også flinke betalere, selvom vi sælger varerne til under indkøbspris
  har vi alligevel et stort overskud.

  \says{SPK} Det lyder jo glimrende.  Har I nogen selvskabelige aktiviteter?

  \says{KF} Ja, vi har en revy hver måned, men det er altid en fiasko,
  teksterne er elendige, og skuespillerne er dårlige, så det er et
  dødsygt foretagende.

  \says{SPK} Det lyder sørgeligt.  Vi vil også tale med en af de
  studerende og i den anledning har vi kontaktet formanden for
  fagrådet.  Hilbert, kan du fortælle os lidt om de studerendes syn på
  instituttet.

  \scene{Transparent med ydre kantine.}

  \says{H}[aggressiv optræden] Vi mener at det kører alt for slaps.
  Her er ingen disciplin.  F.eks. har lærerne på et
  afførstedelskurserne stillet en rapportopgave som ikke strakte sig
  hen over en ferie.  Og han har også stillet et eksamenssæt, som
  kunne læses indenfor de 4 timer som vi har til det.  Vi kræver
  \act{tæller på fingrene}:

  \begin{enumerate}
    \item Større pensum
    \item Flere og længere rapportopgaver
    \item Mindst 1 uløseligt spørgsmål i hvert eksamenssæt
    \item Færre lærere og mere teoretisk stof
  \end{enumerate}

  \says{H}[kalder ud i kulissen] Kom venner, lad os fortælle dem hvad vi vil:
\end{sketch}

Melodi: Another brick in the wall

\begin{song}
Vi kræver fle'r rapportopgaver
vi vil ha' mere teori
eksamenerne skal være sværer'
med uløselige spørgsmål i
Hej lærer' se så at vågn op
!!: Få så fing'ren ud og se at kom' i galop :!!
\end{song}

Verset gentages og en transparent på bagtæppet med teksten.  Prøv at
få salen til at synge med.

\end{document}
