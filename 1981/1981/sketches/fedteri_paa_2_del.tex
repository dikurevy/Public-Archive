\documentclass[a4paper,11pt]{article}

\usepackage{revy}
\usepackage[utf8]{inputenc}
\usepackage[T1]{fontenc}
\usepackage[danish]{babel}


\revyname{DIKUrevy}
\revyyear{1981}
% HUSK AT OPDATERE VERSIONSNUMMER
\version{0.1}
\eta{$n$ minutter}
\status{Færdig}

\title{2. dels sketch}
\author{en forfatter}

\begin{document}
\maketitle

\begin{roles}
  \role{U}[] Udråber
  \role{L}[] Lærer
  \role{S0}[Lone] Studerende
  \role{S1}[Hans] Studerende
  \role{S2}[Hugo] Studerende
  \role{S3}[Trine] Studerende
\end{roles}

\begin{sketch}

  \scene{Der er lys på udråberen og dæmpet lys på læreren.}

  \says{U} Ualmindeligt.  Fantastisk.  Kollosalt.  Overdådigt.  Opulent.
  Ekstra-ordinært.  Eksorbitant.  Virkefeltet er lige blevet udvidet, så
  vi kan nu frigive et skrift projekt.  Ja kom nærmere, kom nærmere,
  mine kære arbejdssomme studenter.  Denne lærer kan tilbyde 10 - jeg
  gentageR: 10 - skriftlige timer meget billigt.  Det er tillige et
  meget enkelt projekt.  Hvem ta'r det?  Var det ikke noget?  \act{Peger
    tilfældigt i salen.}  Nå, hvad så med dig? \act{Peger på et af
    revygruppemedlemmerne i salen.  De tre andre rækker nu også hænderne
    op.}  Og dig? \act{Peger på den ene.}  Og dig? \act{Den næste.} Og
  dig? \act{Den sidste.}  Ja kom nærmere mine kære venner.  HER er plads
  nok.

  \scene{De går over bag læreren.  Lyset på denne er stadig dæmpet, men
    under den følgende monolog forstærkes lyset på læreren, og det er
    ved fuld styrke, når monologen er færdig.}

  \says{U}[mod salen, tager sig til hovedet] Åh, men der er jo kun ET
  projekt!  Hvad skal vi nu gøre? \act{Meget kort kunstpause.} Måske kan
  han alligevel nå at vejlede om ikke 4, så dog 2, hvis han får lidt
  hjælp til de daglige praktiske gøremål, så han får ekstra tid.

  \scene{De studerende puffer lidt til hinanden... "`flyt dig, jeg skal
    stå her, jeg kom først"', og begynder derpå at udføre forskellige
    sysler: Hans manucurerer negle, Trine nulrer nakkemuskler, Lone
    pudser sko og Hugo spiller violon.}

  \says{U} Hvem ved?  Hvem ved?  Måske slet ikke nogen dårlig idé.
  Faktisk genialt.  Måske kunne man oven i købet få dem til at vejlede
  hinanden. \act{Lyset er nu helt tændt.} Nja øh æh bæh, så bliver han jo
  overflødig.  Næh, det er vist ikke nogen god idé.

  \says{U} Se! \act{Læreren rejser sig, strækker, rækker og kigger på
    sine sko og sine negle.  Stikker fingrene i ørerne og ryster dem.}
  Nu rejser han sig.  Der er guf i det her.  Det må jeg hjem og skrive
  et DIKU-blad om!

  \scene{U går ud.  Lys på tronen.  Hugo spiller sagte på violin.
    Trine masserer skuldre.  Lone pudser sko.  Hans manicurerer
    negle.}

  \says{L} Gaaab! \act{Strække, række, klø, kradse.  De 4 studerende
    holder op med, hvad de er igang med, og rykker et skridt baglæns.}

  \says{S0+S1+S2+S3} Iiihhh!

  \says{L} Nå, lad os så se lidt på, hvem der skal have vejledning i
  aritmetisk intervalnummerering med specielt henblik på
  modulalgebraiserede kanoner.

  \says{S0+S1+S2+S3} Ååååh!  \act{I munden på hinanden.} Mig!  Mig!
  Jeg har... Nej mig, miig, jeg vil godt slå...

  \says{L} Ja ja børn, lad os nu se.

  \says{S0+S1+S2+S3} Uuuh!

  \says{L}[tager en notesblok frem og mumler for sig selv] Lone:
  Skopudsning 2 points, studsning af overskæg 6 points. \act{mumler}
  Ialt 26 points.

  \says{S0} Jubii!

  \says{L}[henvendt til Lone] Ja ja, der er andre.

  \says{L} Hans: Manicure af negle 4 points, æbleplukning 4 points
  \act{mumler} Ialt 32 points.  Farvel Lone. \act{S0 går.} Hugo:
  Violinspil 7 points, græsplæne 4 points. \act{mumler} Ialt 55
  points.

  \says{S2} Jubiidubiiilabbelabbe-jæææh!

  \says{S3} Buuuh-huuuh-buuuuuuh!

  \says{L} Hov hov.  Et øjeblik.  Hvad er det, der rinder mig ihu.  Da
  Hugo løb mig en tyr med cykelanhængeren rystede det - og jeg
  vågnede!  30 straffepoints.

  \says{S3} Jubii-osv.

  \says{S2} Buuh-osv. \act{S2 går ud.}

  \scene{L og S3 går langsomt ud.  Han med armene om S3s skuldre.  Hun
    med tilbedende ansigstsudtryk.}

  \says{L} Nu skal du bare høre: Efter opvasken i aften kan vi til
  kaffen begynde at diskutere...

\end{sketch}
\end{document}
