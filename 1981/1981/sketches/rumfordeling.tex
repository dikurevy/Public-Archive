\documentclass[a4paper,11pt]{article}

\usepackage{revy}
\usepackage[utf8]{inputenc}
\usepackage[T1]{fontenc}
\usepackage[danish]{babel}


\revyname{DIKUrevy}
\revyyear{1981}
% HUSK AT OPDATERE VERSIONSNUMMER
\version{0.1}
\eta{$n$ minutter}
\status{Færdig}

\title{Rumfordeling}
\author{?}

\begin{document}
\maketitle

\begin{roles}
\role{O}[] Astronaut
\role{J}[] Astronaut
\end{roles}

\begin{sketch}

\says{O} Højde 220m.

\says{J} Højde 220km.

\says{O} Sæt venstre flip.

\says{J} Venstre flop sat.

\says{O} Sæt automatastronauten til.

\says{J} Astronautautomaten sat til.

\scene{Pause.  Begge strækker sig og gaber.}

\says{J} Hvad skal vi så lave?

\says{O} Kender du den om rummanden og nonnen?

\says{J} Næh...

\says{O} Jo set du, rummanden havde været væk længe, og så kom han til
Venus...

\says{J}[tænksomt] Nå, det er den om rummanden og nonnen?  Den kender
jeg godt.

\scene{Pause.}

\says{J} Hvad skal vi så lave?

\says{O} Har du hørt den sidste nye Pluto-historie?

\says{J} Næh...

\says{O} Jo ser du, Der var en, der ringede ned til kiosken for at
høre, om Anders And var kommet, og der var han.

\scene{Pause.}

\says{J} Hvad skal vi så lave?

\says{O} Nå!  Så er der kun én ting tilbage: Halli-Hallo.

\says{J+O} Jeg vil være den!

\scene{MEGET kort diskussion.}

\says{O} Et ord på 6 bogstaver med H?

\says{J}[hurtigt] Himmel.

\says{O} Et ord med M på 4 bogstaver?

\says{J}[hurtigt] Måne.

\says{O} Et ord på 6 bogstaver med P?

\says{J} Planet.

\scene{Pause.}

\says{B} Hvad skal vi så lave?

\says{O} Vi ku' jo synge lidt, Hilbert!

\says{J} Åh, Sullivan.  Det kan vi da godt.

\scene{De to astronauter sidder i hvert sit rumsæde; under selve
  dialogen bevæger de faktisk kun hovedet.}

\end{sketch}
\end{document}
