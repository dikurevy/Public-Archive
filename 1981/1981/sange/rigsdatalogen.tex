\documentclass[a4paper,11pt]{article}

\usepackage{revy}
\usepackage[utf8]{inputenc}
\usepackage[T1]{fontenc}
\usepackage[danish]{babel}


\revyname{DIKUrevy}
\revyyear{1981}
\version{0.1}
\eta{$n$ minutter}
\status{Færdig}

\title{Rigsdatalogens vise}
\author{Mette}
\melody{Fandens Oldemor}

\begin{document}
\maketitle

\begin{roles}
\role{S}[] Sanger
\end{roles}

\begin{song}
  Vor rigsdatalog, det er faneme mig.
  Jeg har kodet, sid'n I lå i vuggen og skrev.
I sidder dernede betuttet og glor
og tænker "`Hvad var det, der gjorde ham stor"'
\hspace{1cm} Kodetricks og goto og bidtfedteri
\hspace{1cm} samt et umådeholdent forbrug af papir
Skønt jeg offentligt siger: "`Brug tilstandstabel"'
er det andre principper, der gælder mig selv

For hundred' år sid'n gik Charles Babbage omkring
og laved' maskiner, der ku' ingenting.
Fortvivlet bad han Lady Lovelace om råd
og det slog hende, jam'n der mangler jo no't.
\hspace{1cm} Kodetricks og goto og bidtfedteri
\hspace{1cm} du skal se, hvordan bru'n af maskinerne sti'r
Og Charles bed på krogen, for Ada var go'
således fik vi syndefald nummer to!

Von Neumann med fler' ha'd et frygt'ligt besvær
med at regne på, hvo'n så'n en bombe skal vær'.
En syndeflod af tal kan kun svært oversku's
men det lykked's med ENIAC, fordi der ku' bru's
\hspace{1cm} Kodetricks og goto og bidtfedteri
\hspace{1cm} det er så'n, man får krammet på dem man bekri'r
Gudskelov, at sejren blev amerikansk
for hvad hedder "`computer"' på tysk og japansk?

Skønt man havde Assembler, sku' en gør' sig klog
og lave et højniveau datamat-sprog
Designkriteriet hurtigt fremkom,
for hvad er det, enhver programmør synes om?
\hspace{1cm} Kodetricks og goto og bidtfedteri
\hspace{1cm} til den slags er FORTRAN den re-ene svir.
I COBOL er {\tt goto} lidt kedeligt men
man har {\tt ALTER}, så ingen ved hvor man går hen

Men vi kunne sandelig også herhjemm'
Et slag på Dasken, og bi-er kom frem.
Synes ordlængden fyrre dig timmelig vild?
Måske nok, men det gav os en lejlighed til
\hspace{1cm} Kodetricks og goto og bidtfedteri
\hspace{1cm} sådan fik vi implementer't Algol fir'
At det ku' la' sig gøre i fire K bytes
var fordi, vi havde Jørn Jensen's Device!

Nu om dage er nøgleordet Struktur.
Pascal, og bevis for programmerne du'r
men dagen før deadline - og testen går bet
så ved jeg, I alle kun tænker på ét
\hspace{1cm} Kodetricks og goto og bidtfedteri
\hspace{1cm} men I frygter for, hvad instruktorerne si'r.
Lad mig derfor betro jer en hemmelighed:
hvis du vil være stor, så gå til (!) og fedt med
\end{song}

\end{document}

