\documentclass[a4paper,11pt]{article}

\usepackage{revy}
\usepackage[utf8]{inputenc}
\usepackage[T1]{fontenc}
\usepackage[danish]{babel}


\revyname{DIKUrevy}
\revyyear{1981}
% HUSK AT OPDATERE VERSIONSNUMMER
\version{1.0}
\eta{$n$ minutter}
\status{Færdig}

\title{Fremmede Intelligenser}
\author{?}

\begin{document}
\maketitle

\begin{roles}
  \role{1}[] Person i rumdragt
  \role{2}[] Person i rumdragt
\end{roles}

\begin{sketch}

  \scene{To personer i rumdragter sidder (evt. står) i hver sin side
    foran tæppet.  De har hver en kikkert, som de retter ud mod
    publikum.  Publikum bliver betragtet intenst i længere tid.  Der
    er lys på tæppet og slukket i salen.}

  \says{1} Hov, jeg har fundet noget.  Jeg tror det er -- ja, ja det
  er intelligens, men svage, meget svage tegn på intelligens.

  \says{2} Ja, jeg kan svagt se dem.  Gud, hvor er de fjollede.

  \says{1} Jeg stiller lige skarpt og skruer et par dimensioner op.

  \scene{Lyset på tæppet slukkes, og lyset i salen tændes.}

  \says{2} Giv mig lige koordinaterne, så jeg kan se med.

  \says{1} 01-83-64-66.

  \says{2} Ja, nu ser jeg dem bedre.  Kalder du det for intelligens?

  \says{1} Der er jo hundredevis af dem.   De sidder så tæt som kakler på et rumskub.

  \says{2} Gud ved hvad de sidder og venter på?

  \says{1} Hvor er det dumt bare at sidde der.

  \says{2} Ja, det er for åndsvagt.  Der må da foregå andre ting her i rummet.

  \scene{Lyset i salen slukkes, og der kommer lys på scenen.  Tæppet
    går fra, personerne trækker ud i siden, og næste nummer begynder.}

\end{sketch}
\end{document}
