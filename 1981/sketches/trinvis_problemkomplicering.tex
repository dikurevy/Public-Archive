\documentclass[a4paper,11pt]{article}

\usepackage{revy}
\usepackage[utf8]{inputenc}
\usepackage[T1]{fontenc}
\usepackage[danish]{babel}


\revyname{DIKUrevy}
\revyyear{1981}
% HUSK AT OPDATERE VERSIONSNUMMER
\version{0.1}
\eta{$n$ minutter}
\status{Ikke færdig}

\title{Trinvis problemkomplicering}
\author{?}

\begin{document}
\maketitle

\begin{roles}
  \role{M}[] Mand
\end{roles}


\begin{sketch}

  \scene{En mand går rundt med hænderne på ryggen, han bemærker ikke
    straks publikum.  Da han opdager dem udbryder han:}

  \says{M} Åh, der er I! \act{Peger.  Pause.}

  \says{M} Ja, jeg vil gerne sige tak for invitationen til dette
  kollokvium, det er rart at se det store fremmøde.

  \says{M}[med hævet stemme] Emnet er i dag som annonceret: Trinvis problemkomplicering.

  \says{M} Det er en stor personlig tilfredsstillelse for mig at få
  lov til at tale om dette interessante emne, der både interesserer
  mig som forsker, men osse som privatperson, det har nemlig vist sig,
  at jeg faktisk har kunnet anvende teorien indenfor privatsfæren.

  \says{M} Det hele startede med, at jeg fik et kandidatstipendiat, og
  da dette var udløbet, måtte jeg, for at få det forlænget lægge nogen
  forskningsresultater på bordet.

  \says{M} Vel, tænkte jeg.  Her gælder det om at gøre et godt
  indtryk.  Jeg må så vælge et kendt problem, som alle føler sig
  fortrolige med.

  \says{M} Efter lang tids søgen i de førende matematiske tidsskrifter
  fandt jeg endelig en forholdsvis simpel og overskuelig ligning, der
  kunne bruges.

  \scene{$1+1=2$ kommer op på lærredet.}

  \says{M} Mit problem var: Det første skridt i problemstillingen er
  altid det der får de mest vidtgående konsekvenser, derfor gjaldt det
  som at finde nogen fleksible løsninger.

  \scene{$\ln(e)=1 \& \sin^{2}(x)+\cos^{2}(x)+\cos^{2}(x)=1$ kommer på
    lærredet.}

  \says{M} Nu var der endelig lidt stil over ligningen og efterhånden
  var den blevet rimeligt kompliceret.

  \scene{$\ln(e)+\sin^{2}(x)+\cos^{2}(x)=2$ kommer på lærredet.}

  \says{M} Men det var som om der manglede noget.  Måske kunne det
  hjælpe at indføre lidt summation...

  \scene{$\sum_{n=0}^{\infty}1/2^{n}=2$ kommer på lærredet.}

  \says{M} ...så ville det også se mere videnskabeligt ud.

  \scene{$\ln(e)+sin^{2}(x)+cos^{2}(x)=\sum_{n=0}^{\infty}1/2^{n}$ kommer op.}

  \says{M} Efter disse komplicerede overvejelser havde jeg et godt
  grundlag at arbejde videre på.  Men det gik lidt trægt.

  \says{M} Først da en af mine venner fra højspændingslaboratoriet
  fortalte mig, at højspændingsledninger hænger efter kædeligningen,
  kom jeg i tanker om ligningen.

  \scene{$\cosh(y)\cdot\sqrt{1-\tanh^{2}(y)}=1$ kommer op.}

  \says{M} Uhm, lækre transcendente funktioner.  Og nu gik det slag i slag.

  \says{M} Og det varede ikke længe før jeg kom i tanke om den gamle grænseovergang.

  \scene{$\lim_{z\rightarrow\infty}(1+1/z)^{z}=e$ kommer op.}

  \says{M} Dette førte frem til:

  \scene{$\ln(\lim_{z\rightarrow\infty}(1+\frac{1}{z})^{z}+\sin^{2}(x)+\cos^{2}(x)=\sum_{m=0}^{\infty}\frac{\cosh(y)\sqrt{1-\tanh^{2}(y)}}{(\log_{2}(4))^{n}}$}

  \says{M} Et virkeligt smukt udtryk - candy floss for nethinden... Og
  aldeles ikke umiddelbart gennemskueligt.

  \says{M} Nu mente jeg efterhånden at have fundet det helt rigtige
  til min rapport, men for at være sikker på, at det havde det rette
  kompleksitetsniveau, skrev jeg den på fransk.

  \says{M} Men hvad nu hvis der var en franskmand i
  bedømmelsesudvalget?

  \says{M} Løsningen på dette problem fik jeg af nogle polakker som
  havde været med på det polske gymnastiklandshold.  Disse polakker
  (som ellers var ganske flinke og rare mennesker) var hoppet af under
  en europaturne, de var blevet omvendt da de så vestens goder.

  \says{M} Derfor blev de blandt venner kaldt: "`De omvendte polakker"'.

  \says{M} Sådan tænkte jeg, omskriv ligningen til omvendt polsk
  notation, så skulle jeg have dækket mig ind.

  \scene{Ligningen kommer op.}

  \says{M} Og så var den hjemme!  Jeg fik forlænget mit stipendiat:

 \scene{\[1\ 1\ z\ /\ +\ z\ \uparrow\ \lim_{z\rightarrow\infty}\ln\ x\ \sin\ 2
   \uparrow\ x\ \cos\ 2\ \uparrow\ +\ +\ y\ \cosh\ 1\ y\ \tanh\ 2\ \uparrow\ -
   \sqrt{}\ \cdot\ 4\ \log_{2}\ n\ \uparrow\ \sum_{n=0}^{\infty}\]}

\end{sketch}
\end{document}
