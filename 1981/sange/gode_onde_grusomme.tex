\documentclass[a4paper,11pt]{article}

\usepackage{revy}
\usepackage[utf8]{inputenc}
\usepackage[T1]{fontenc}
\usepackage[danish]{babel}


\revyname{DIKUrevy}
\revyyear{1981}
\version{0.1}
\eta{$n$ minutter}
\status{Færdig}

\title{De gode, de onde og de grusomme}
\author{}
\melody{Over tema fra ``De gode, de onde og de grusomme''}

\begin{document}
\maketitle

\begin{roles}
  \role{S}[Skuespiller] Rolleforklaring
\end{roles}

\begin{song}
  På DIKU er vi vokset til at være tusind' folk,
  så lærerne de går og må beskytte sig med folk,
  og også vi studenter ved at noget må der ske,
  man hvad bliver det?
  Lærerne, de forskanser sig på 2. salen,
  "`Vi må ha' fred til forskning"', siger de.
  Men snart ser vi skandalen,
  det' for os studenter de vil være fri!

  Det bedste ville være at få 1. delen ud,
  og efter dem står 2. dels studerende for skud
  for der skal være plads til snak og kaffeslabberas
  bli'r det noget bras?
  Lærerne, de forskanser sig på 2. salen,
  mon ikke de vil dø af kedsomhed?
  Nej, de ta'r blot af totalen
  et par specialestuderende med.

  Men vi studenter vi vil faktisk også være her,
  ja vi vil også bruge instituttet noget mer'.
  Vi stiller samme krav til arbejdspladser og miljø,
  vi vil ikke dø.
  Lærerne, de forskanser sig på 2. salen,
  det må bero på en forglemmelse.
  Men de tror, vi går på halen?
  Nej, på faget har vi medbestemmelse.

  De gode, de onde og de grusomme.
\end{song}

\end{document}

