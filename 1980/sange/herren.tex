\documentclass[a4paper,11pt]{article}

\usepackage{revy}
\usepackage[utf8]{inputenc}
\usepackage[T1]{fontenc}
\usepackage[danish]{babel}


\revyname{DIKUrevy}
\revyyear{1980}
\version{1.0}
\eta{$n$ minutter}
\status{Færdig}

\title{Skabelsesberetning}
\author{EM}
\melody{Farinelli: ``Herren som skabte alt på jord''}

\begin{document}
\maketitle

\begin{roles}
\role{S}[] Sanger
\end{roles}

\begin{song}
Herren som skabte alt på jord
skabte vel også datalo'er
For, som han sagde den gode Gud:
"`Her ser sgu noget rodet ud.
Vi må have skabt en god ekspert,
en der kan klare lidt af hvert,
som aldrig regner noget forkert"'.
Så skabtes datalogen.

Fra dengang af
og til i dag
har vi nu fået set hvad edb betyder.
Nu sidder Han
på himlens rand
og spekulerer på om han fortryder.

Fanden så til fra helvedets port
sagde til sig selv: "`Det går som smurt.
Her må jeg have e nfinger med,
så de kan komme galt afsted"'.
Da var det overløb blev skabt
og decimaler som går tabt,
indeksfejl, og som ærligt skrapt
så skabte fanden {\em go to}

Til Fanden selv
står vi i gæld
for hvert problem vi skaber for hinanden.
Hver ting der kør',
hver programmør
har fået lidt af både Gud og Fanden.
\end{song}

\end{document}

