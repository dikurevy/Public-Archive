\documentclass[a4paper,11pt]{article}

\usepackage{revy}
\usepackage[utf8]{inputenc}
\usepackage[T1]{fontenc}
\usepackage[danish]{babel}


\revyname{DIKUrevy}
\revyyear{1980}
\version{1.0}
\eta{$n$ minutter}
\status{Færdig}

\title{PDP-sang}
\author{Marianne}
\melody{Jeg plukker fløjlsgræs}

\begin{document}
\maketitle

\begin{roles}
\role{S}[] Sanger
\end{roles}

\begin{song}
På vores kursus der sku' vi prøve
at lave kerner til PDP.
Som dataloger så sku' vi øve
os i at køre på så'n et kræ.
Vi fik en hængelås udleveret,
for der sku' køred på god maner.
Sådan blev tiden synkroniseret,
så alle kørsler fik et kvarter.

Vi havde knapt haft al teorien
men gik med krum hals og mod i gang.
De første started' på PDP'en
og kørte løs hele dagen lang.
Via Pascal til det fæle macro
blev end'lig kernerne fler' og fler'
og nu begyndte besværet, ak jo,
for de sku' køres på et kvarter

Vi adressered', manipulered'
med bit\_i tusindvis uden stop.
Nervøsiteten i rummet dirred',
til alle håb om et glemt holdt op.
Og når det skete, at noget lykked's,
en bøn om tid, to sekunder mer',
men uden medføl'se knappen trykked's
præcis på sla'et efter et kvarter

Al denne kerneafprøvning tæred'
for meget på vores PDP.
Den sidste uge - må det ha' været -
man vidste aldrig, hvad der ku' ske.
På tur brød læser og skriver sammen -
midt i en kørsel det pluds'lig sker.
Man konstanterer med højlydt jamren,
at den har ødelagt det kvarter.

Tre dage ventede vi med længsel
til vores tekniker end'lig kom,
mens PDP-stuen var et fængsel,
hvor vi sku' afvente vores dom.
Min kerne kom aldrig til at køre,
det drivprogram voldte mig besvær.
Men til mit forsvar skal alle høre,
det var umuligt på et kvarter.

Nu står jeg her med et fint program,
hvis funktion er enkel og ligetil:
En variabel kan tælles frem
og tilbage, ligesom man det vil.
Om end jeg ikke fik lært så meget,
så ved jeg nu, at på et kvarter
er det et kunststykke ret så speget
at programmere til PDP'er.
\end{song}

\end{document}

