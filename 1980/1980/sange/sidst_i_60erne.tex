\documentclass[a4paper,11pt]{article}

\usepackage{revy}
\usepackage[utf8]{inputenc}
\usepackage[T1]{fontenc}
\usepackage[danish]{babel}


\revyname{DIKUrevy}
\revyyear{1980}
\version{1.0}
\eta{$n$ minutter}
\status{Færdig}

\title{Sidst i 60'erne}
\author{Marianne/EM}
\melody{Så'n var det ikke i halvfemserne}

\begin{document}
\maketitle

\begin{roles}
  \role{S}[] Sanger
\end{roles}

\begin{song}
  Det er en tragisk forskel på DIKU før og nu,
  de glade tresser's gode ånd er faktisk slået itu.
  Så bar' på den begrænsning, der er på anden del,
  at få en plads det kræver fætter Højbens held.

  Så'n var det ikke sidst i tresserne,
  da var der nok af studiepladserne.
  Dumpet, næ tak,
  man blev sgu hevet helt igennem.
  Nu er man svedt,
  hvornår bli'r man smidt ud af lemmen?
  Så'n var det ikke sidst i tresserne,
  da fik man øller af professerne.
  Nu' det alt for tit,
  der er alt for lidt
  af tararabomdiæ.

  Hvert år ved juletide blir vor's budgetter skåret.
  Der er kun tyve kroner, når vi når til efteråret,
  for knap en halv snes tusing er taget af Niels Bohr,
  så vi pante det, vi købte ind i fjor.

  Så'n vra det ikke sidst i tresserne,
  da var der nok i pengekasserne.
  Tusind studenter videnskabens frugt vil spise,
  lærerne har kun kræfter til at undervise.
  Så'n var det ikke sidst i tresserne,
  da var der penge til kongresserne.
  Nu' det alt for tit,
  der er alt for lidt
  af tararabomdiæ.

  Det vælter ind med klager om alt fra stort til småt.
  Der er snart ingen reg'l, som ikke bliver overtrådt,
  og særlig ved eksamen, da blir den gru'lig gal
  på grund af karakterens anden decimal.

  Så'n var det ikke sidst i tresserne,
  man hang sig ik' i petitesserne.
  Før mødtes man i fred og ro og samarbejde,
  nu mødes studienævnet i en blodig fejde.
  Så'n var det ikke sidst i tresserne,
  da var man fæll's om interesserne.
  Nu' det alt for tit,
  der er alt for lidt
  af tararabomdiæ.

  Jeg går til forelæsning hos Naur og hos Søren.
  Jeg sover fra det øjeblik, de træder ind ad døren.
  Det er nu meget godt, men jeg længes frygt'ligt hjem,
  når de ta'r deres gamle transparenter frem.

  Så'n var det osse sidst i tresserne,
  vi faldt i søvn hos forelæserne.
  Reproduktion og multiprogrammeringskerner
  sløver og dræber selv de allerbedste hjerner.
  Så'n var det også sidst i tresserne,
  vi tærsked' langhalm på processerne.
  Nu' det alt for tit,
  der er alt for lidt
  af tararabomdiæ.

  Jeg går til forelæsning hos Naur og hos Søren.
  Jeg sover fra det øjeblik, de træder ind ad døren.
  Det er nu meget godt, men jeg længes frygt'ligt hjem,
  når de ta'r deres gamle transparenter frem.

  Så'n var det osse sidst i tresserne,
  vi faldt i søvn hos forelæserne.
  Reproduktion og multiprogrammeringskerner
  sløver og dræber selv de allerbedste hjerner.
  Så'n var det osse sidst i tresserne,
  vi tærsked' langhalm på processerne.
  Nu' det alt for tit,
  der er alt for lidt
  af tararabomdiæ.

  Revy i vore dage er snart en ked'lig sag.
  Her synes ingen vits for dårlig til at fyre af.
  Vi digter løs på sange, hvis text er så banal,
  at publikums begejstring bliver minimal.

  Så'n var det ikke sidst i tresserne,
  da var der fyld' og klang i basserne.
  Alle pointer drukner i beruset latter,
  næste års sange blir med anonym forfatter.
  Så'n var det ikke sidst i tresserne,
  da grined' man da af finesserne.
  Nu' det alt for tit,
  der er alt for lidt
  af tararabomdiæ.
\end{song}

\end{document}

