\documentclass[a4paper,11pt]{article}

\usepackage{revy}
\usepackage[utf8]{inputenc}
\usepackage[T1]{fontenc}
\usepackage[danish]{babel}


\revyname{DIKUrevy}
\revyyear{1980}
% HUSK AT OPDATERE VERSIONSNUMMER
\version{1.0}
\eta{$n$ minutter}
\status{Færdig}

\title{Datekredaktionen}
\author{BB, BDK}

\begin{document}
\maketitle

\begin{roles}
  \role{F}[] F.Umle
  \role{E}[] E.X.Perten
\end{roles}

\begin{sketch}

  \scene{To personer sidder bag et bord.  Foran hver står et glas vand
    og et navneskilt.  Navnene er F. Umle og E. X. Perten.  Over
    bordet hænger et stort skilt med DATEKREDAKTIONEN.  Som baggrund
    er en 8080 chip projiceret op.}

  \says{F} Goddag og velkommen til Datekredaktionen.  Vi skal idag ha'
  forklaret nogle af det ebd-udtryk, som vi bruger uden rigtigt at
  vide hvad de står for.  E. X. Perten har lovet os at han vil
  fortælle os lidt om dette særprægede område.

  \says{F}[vender sig mod E] Velkommen til, Perten.

  \says{E} Tak, tak tak...

  \says{F} Ja, og inden vi starter må jeg nok hellere fortælle lidt om
  de ting vi har med her i studiet i dag.  Her har vi et lager, her en
  linieskriver, og en skærm.  Ja maskinen derhenne er desværre nede,
  så vi må vente til senere med at demonstrere den.  \act{Peger på det
    forskellige} E. X. Perten vil først fortælle os om filer, er det
  ikke rigtigt Perten?

  \says{E} Jo, en fil er et af de vigtigste redskaber i edb-verdenen.
  Som regel har man samlet flere filer på et lager og for
  overskuelighedens skyld placeres de i dette lager hyppigt på plagt
  eller bånd.  Det er netop så heldigt, at jeg tilfældigvis i dag har
  et bånd med mig, hvorpå er anbragt nogle af de mest tpiske filer.

  \scene{Han tager et bånd frem hvorpå der er fastspændt fem filer: En
    panserfil, en grovfil, en sletfil, en nøglefil og en neglefil.}

  \says{F} Hvad forskel er der så på de filer?

  \says{E} Ja, filer inddeles typisk efter hvor mange rekorder de har
  - eller records, som de siger i staterne.  Panserfilen er her den
  der fører.  Den har ikke mindre end 120 rekorder, og for at nævne de
  mest iøjnefaldende, ser vi at den bl.a. har rekort i længde, i
  bredde og ser man lidt nærmere på den - Ja De kan jo nok ikke se det
  mine damer og herrer, men De min gode Umle - se om panserfilen
  minsandten ikke også har ekord i sporenes bredde.

  \says{F} Ja, fantastisk, men hvad bruges panserfilen så til?

  \says{E} Ja, der kommer De ind på noget centralt Umle, for
  panserfilens største rekord er faktisk ikke rekord i brug, men i
  misbrug.  Det alvorligste misbrug vi har oplevet, er faktisk for
  nylig her på Nørrebro...

  \scene{Mens Perten snakker bliver Umle mere og mere distræt, han
    stirrer på maskinen, der hejses op.}

  \says{F} Nå, ... nå, ja nu er maskinen oppe, skulle vi prøve...

  \scene{Maskinen hejses ned igen, Umle stirrer lidt opgivende på den
    før han forvirret går videre.}

  \says{F} Næh, nu er den nede igen.  Nå, ja, hvad så med grovfilen,
  hvad bruges den til?

  \says{E} Til de rå programmer.

  \scene{Han viser et revyprogram frem, som på en eller anden måde er ufærdigt.}

  \says{E} Se, her har jeg en rå model for programmet til i aften.
  Man ser, at der mangler en side, noget tekst og at det i det hele
  taget er utydeligt.  Når et sådant program er blevet pudset lidt af,
  kan man som regel afgøre om det er anvendeligt.  Det hænder, at det
  viser sig, at programmet er ubrugeligt, og man går så over til at
  anvende sletfilen.  I løbet af få dage skulle man så være blevet
  programmet kvit.  Lykkes dette ikke, kan man overlade filen til en
  båndabe, og så skulle problemet hurtigt blive løst...

  \says{F} De nævnte nogle dyr, båndaber, var det ikke det De kaldte
  dem - hvad er det egentlig for nogle fyre?

  \says{E} Jeg er glad for at de nævner dem, men i forbindelse med
  disse filer vil jeg lige nævnte at de til tider kan være...

  \says{F} Det var de båndaber.

  \says{E} Ja, ja.  Om et øjeblik Umle.  Det kan tage flere timer, ja
  flere dage at få fat på en fil, men...

  \says{F} Jamen, hvad med de båndaber?

  \says{E}[med en opgivende gestus] Jo ser de, båndaber er en slags
  skadedyr som lever på næsten ethvert lager.  Sådan set er det ret
  harmløse dyr, der næsten udelukkende lever af båndsalat, men de
  forårsager ofte, at filer forsvinder, eller at hele bånd går tabt.

  \says{F} Har man ikke forsøgt at udrydde disse båndaber?

  \says{E} Jo, men båndaber er meget udbredte og særdeles svære at
  komme til liv.  Jeg ved at der er gjort adskillige forsøg, men
  efterhånden er de desværre blevet næsten immune overfor den eneste
  kendte gift: ØL.  Det er faktisk utroligt så store doser de kan
  tåle, de bliver nærmest bare mere legesyge.

  \says{E}[henvendt til Umle] Og så kunne jeg måske gøre mine filer
  færdige.

  \says{E}[henvendt til publikum] Ser de, hvis man er bange for at miste en fil kan man få den ud på linieskriveren.

  \scene{Han lægger en fil på hovedet at linieskriveren.}

  \says{F} Ta, tak, skal vi gå over til at tale om det næste EDB-begreb.

  \says{E} Jo, det næste, vi skulle tale om er begrebet "`Et brugbart
  håndtag til et anvendeligt præfix"'.  Og, det er netop så heldigt,
  at jeg tilfældigvis har et brugbart håndtag med i dag.

  \scene{Han dykker ned under bordet og henter en lille kasse frem.
    Op af den tager han et almindeligt dørhåndtag af plastic.}

  \says{E} Og her ser vi så et sådant håndtag.  Bemærk den afrundede
  form, der gør at det ligger godt i hånden.  Læg også mærke til den
  grå farve, der gør at håndtaget kan anvendes ganske diskret.

  \says{F} Mens vi snakker om brugbare håndfix til anvendelige prætag,
  kan du vise os et håndfix, som ikke er anvendeligt?

  \says{E} Det er netop så heldigt, at jeg tilfældigvis har et håndtag
  med der ikke er brugbart.

  \scene{Fra kassen finder han et dørhåndtag magen til det forrige.
    Dette er blot flækket på langs.  Dørhåndtaget fremvises uden ord.}

  \says{F} Hvorfra kommer disse håndtag?

  \says{E} De vokser på et særligt træ, det som vi til dagligt kalder
  syntaks-træ.  Dets korrekte latinske navn er syntiæ takticus.  Det
  er ret svært at oversætte, men på dansk vil det blive noget i
  retning af, ja nærmest ÆH - man kunne måske sige - brugbart håndtags
  træ.

  \says{F} Hvor gror disse syntiæ takt... æh disse syntakstræer?

  \says{E} Det er netop så heldigt, at jeg tilfældigvis har et kort
  over de områder hvor syntiæ takticus gror.

  \scene{På overheaderen kommer en tegning af et ur der bliver savet over.  På tegningen står der med store bogstaver:

    TIDSDELING (eng: time scharing)

    EXPerten står halvt med ryggen til tegningen og begynder at udpege og
    forklare, somon det var den rigtige tegning der var vist.}

  \says{F} Æh... altså kortet... deet...

  \says{E} Min gode Umle, hvad er der nu?

  \says{F}[peger] See...

  \says{E}[vender sig] Det ser ud til vi har haft en teknisk fejl.
  Den vil blive rettet i løbet af kort tid.

  \scene{Op op overheaderen kommer der et landkort.  Det har en
    forbløffende lighed med Sjælland, blot spejlvendt (navne se
    bilag[sic]).  Eksperten peger på dat-o med en pegepind.}

  \says{E} I dette område plantes de første spæde spirer af syntiæ
  takticus.  Nogen egentlig udvikling sker der dog ikke før end de
  overføres til dette område.

  \scene{Han peger på dat-1 området.}

  \says{E} I dette område har de ganske vist meget dårlige
  vækstbetingelser, idet de førende avlere, Neter Paur og Løren
  Saursen, er tilbøjelige til at overdosere gødningen.  Det er årsagen
  til at mange af de små spirer fra dat-0 området ikke kommer videre.
  De få flanter der overlever denne hårhænede behandling bliver
  overført til dat-2 området.

  \scene{Han peger på dat-2 området.}

  \says{E} De heldigste planter er dem der bliver plantet på
  oversætteren.

  \scene{Han udpeger oversætteren}

  \says{E} Her er jorden særligt frugtbar.  Det er på oversætteren at
  træerne bliver færdigudviklet.  Her er det muligt at høste brugbare
  håndtag af en sådan styrke og skønhed, at de er efterspurgt over
  hele landet.

  \says{F} Vil det sige at kun er i disse fire områder man dyrker
  syntakstræer?

  \says{E} Såså min gode Umle, jeg er ikke færdig endnu.  Det er
  sådan, at på store dele af oversætteren, nemlig den del der ligger i
  andendel, er der udbredt misvækst på grund af vanrøgt.  Dette
  skyldes at der er vrøvl med styresystemet.  Det har tendenser til at
  gå i baglås, og det er desværre Løren Saursen der har nøglen.

  Dette resulterer i, at der bliver givet den gødning der er ved
  hånden, i stedet for den gødning der er behov for.  Efterhånden har
  vi kun bifagsområder tilbage.  Det er et goldt ørkenområde, hvor
  syntakstræer overhovedet ikke kan trives.

  \scene{Maskinen hejses op, og Umle bliver ivrig.}

  \says{F} Ja, så er maskinen oppe og det kunne være, vi så til sidst
  skulle demonstrere...

  \scene{Msakinen hejses ned}

  \says{F} Ja, vi var desværre uheldige idag, så vi må vente med
  demonstrationen til næste gang.  Tiden er udløbet, så vi må sige tak
  til seerne og til Perten, fordi han gad at komme herind og gennemgå
  disse vanskelige edb-udtryk.  Tak.

\end{sketch}
\end{document}
