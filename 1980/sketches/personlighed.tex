\documentclass[a4paper,11pt]{article}

\usepackage{revy}
\usepackage[utf8]{inputenc}
\usepackage[T1]{fontenc}
\usepackage[danish]{babel}


\revyname{DIKUrevy}
\revyyear{1980}
% HUSK AT OPDATERE VERSIONSNUMMER
\version{1.0}
\eta{$n$ minutter}
\status{Færdig}

\title{Nyt 2. dels kursus}
\author{BN}

\begin{document}
\maketitle

\begin{roles}
  \role{H}[TAH] Mand
  \role{L}[BN] Lærer
\end{roles}

\begin{sketch}

  \scene{En mand flakker forvirret rundt blandt publikum.}

  \says{H} Undskyld - De har vel ikke set mig?

  Jeg kan nemlig ikke rigtig finde mig selv

  Det er derfor jeg går her og leder.

  Jeg har kun et par minutter tilbage af frikvarteret.

  Jeg er nemlig på kursus, skal jeg sige Dem.

  Personlighedsgenfødselskursus kaldes det.

  Min vejleder siger, at det er noget alle har brug for, selv studerende
  som jeg.

  Det går i al sin enkelthed ud på at nedbryde personligheden, for at en
  ny kan fødes.

  Som Fugl Fønix af sin aske - siger læreren.

  Og så skal det jo nok være rigtigt.

  Det er bare lidt svært at finde sig selv efter sådan en time.

  Og undgå at de andre finder en.

  Men det gør de altid.

  \scene{Læreren ind med det øvrige hold elever.}

  \says{L}[klapper i hænderne] Ja, så er frikvarteret forbi, og vi går
  straks i gang med den næste opgave.  Vi kan begynde med dem.

\says{H}[går op på scenen]  Hvorfor netop mig?

\says{L} Nu ikke så kontrær.  Det er en ganske enkel øvelse.  Fortæl
mig om en helt tilfældig og ligegyldig tildragelse fra Deres barndom.
Den første ting der falder dem ind.

\says{H} Tilfældig?

\says{L} Ja, endelig ikke noget De har gået og ruget over, og som
stadig dukker op i Deres erindring.  Det har vi andre opgaver til.

\says{H} Jamen, det andet har jeg da glemt.

\says{L} Åh, der må sikkert være en eller anden lile pudseløjerlighed,
De kan komme i tanke om.  Gerne noget morsomt.

\says{H} Hvad skulle det være?  \act{leder i hukommelsen} - Jo, der
var for eksempel dengang jeg sad på gelænderet ned til cykelkælderen.
Sammen med naboens Bente.

\says{L} Storartet.  gå videre.

\says{H} Vi sad og åd sukkerstanger - til ti øre stykket.  Det kostede
de dengang.

\says{L}[til holdet] finansielt begavet.

\says{H} Og så morede vi os altså med hvem der kunne få det til at
søge højst, når vi trak sukkerstangen ud af munden med et svup.  Det
kunne Bente.

\says{L} Uhyre interessant.

\says{H} Og så lo jeg og sagde: Det var vel nok et ordentligt knald.
Ja, det var såmænd det hele.

\says{L} Var det første gang De havde hugget penge i tallerkenrækken
til at købe slik for?

\says{H}[krænket] Jeg havde skam ikke hugget nogen penge.  Bente havde
fået sukkerstængerne af sin far.  Fordi hun havde været så flink til
at passe din lillebror.

\says{L}[til holdet] Samvittighedsbetonet erindringsforskydning.
\act{Til ham} Er de sidenhen blevet grebet i tyveri?

\says{H} Aldrig.

\says{L} Eller anden form for kriminalitet?

\says{H} Jeg forsikrer Dem...

\says{L} Jeg ville nu alligevel sætte pris på, om De ville være så
venlig at fremskaffe en straffeattest til næste time.  I mellemtiden
kan vi gå videre med spøgsmålene.  Hvorfor lokkede De Else med ned i
cykelkælderen?

\says{H} Altså - for det første hed hun ikke Else men Bente, og for
det andet...

\says{L} Det med navnet vender vi tilbage til, men De indrømmer altså at have lokket hende med ned i kælderen?

\says{H} Gu' gør jeg ej.  Jeg indrømmer ikke noget som helst.

\says{L}[til holdet] Her har vi fået fat i noget centralt.  Når en
ellers rolig og behersket mand reagerer så voldsomt, er det et
umiskendeligt tegn på, at han er blevet ramt på et ømt punkt.

\says{H} Sådan noget forbandet sludder.

\says{L}[såret] Det var ikke pænt sagt.

\says{H} Jeg gider ikke lege med mere.

\says{L} De vil altså sabotere undervisningen. \act{grædefærdig} De
tror ikke på mine evner til at lære Dem noget.  Jeg må indberette Dem
til institutbestyreren.

\says{H} All right, all right.  Vi gik faktisk ned i
kælderen. \act{til publikum} Hvad skulle jeg ellers sige?

\says{L} Var det Dem der blottede hende, eller gjorde hun det selv?

\says{H} Næ, nu kan det fandengalemig være nok.

\says{L}[inkvisitorisk] Hvem trak bukserne af hende?

\says{H} Hun beholdt dem på.  Hvad i alverden tror De, at vi...

\says{L} Hvad så med knaldet?

\says{H} Der var sgu ikke noget knald.

\says{L} Nu modsiger De Dem selv - ikke også?  \act{Han ser
  appellerende rundt i kredsen, hvor alle nikker dybt alvorligt.}

\says{H} For pokker, mand, der var jo ikke tale om noget knald i den forstand, men...

\says{L} De nøjedes altså med at pille ved hende?

\says{H} Nej, jeg gjorde sgu ikke.  Jeg har aldrig nogensinde rørt ved Bente.

\says{L} Ved Else.

\says{H} Hun heg sgu Bente.

\says{L}[til holdet] En yderst interessant detalje.  Forsøgspersonen
har jo tydeligvis fortrængt navnet Else af sin erindring.  Og hvorfor?
Fordi der ikke findes noget tilsvarende drengenavn.  Hvorimod Bente jo
helt automatisk giver associationer til Bent.  Forstår I, hvad jeg mener?

\act{Holdet nikker alvorligt og ser anklagende på H.}

\says{H} Sig mig engang, er det Deres mening at antyde at...

\says{L} Havde de tidligere mærket homofile tilbøjeligheder?

\says{H}[stammende] Hohohomomo... jeg har ved Gud aldrig...

\says{L} Har De ikke lige påstået, at De ikke rørte ved hende?

\says{H} Det er sgu ikke noget jeg påstår.  Det er den rene og skære sandhed.

\says{L}[langsomt, skærer ud i pap] Så vil de måske også påstå, at De
anser det for normal heteroseksuel adfærd at lade være med at røre ved
en pige, der sutter på en sukkerstand i en mørk kælder?

\says{H} Hvad fanden har sukkerstangen med det at gøre?

\says{L} De forstod altså ikke den enkle symbolik i pigens gestus.  Og
tvang hende derved indirekte til at smide tøjet.

\says{H} Hun smed sgu ikke noget.

\says{L}[triumferende] Ha, så var det altså alligevel Dem, der flåede
det af hende.  Men De var ude af stand til at gennemføre akten, fordi
De følte Dem draget med Deres ven Bent.

\says{H} Jeg har aldrig nogensinde haft e nven med det åndssvage navn.

\says{L} Sese, en uven. \act{til holdet} Det forklarer jo de voldsomme sjælelige komplikationer.  Et dybt tragisk tilfælde.

\scene{Holdet nikker med den dybeste medfølelse.}

\says{H}[ophidset] Er I skingrende sindssyge allesammen?  Jeg er ganske normal og hader både mænd og drenge.  Især når de hedder Bent.  Eller Kurt.

\says{L} Jeg hedder Kurt.

\says{H}[indædt] Derfor.

\says{L}[blidt] Jeg tror, du er forelsket i mig.

\says{H} Hjælp - hjælp \act{flygter ned blandt publikum}.

Jeg ved ikke, om der er en af de damer der vil give mig lejlighed til at bevise, at jeg ikke... nå ikke?

Måske kunne jeg heller ikke bevise det.

Hvad siger?  Hedder du Bent?  Næ, ellers tak.

Nå, det var kun din spøg.  Gudskelov.  For tænk, om det nu vise sig at jeg i virkeligheden var...

\says{H}[fortvivlet]  Hvad er jeg?  Hvem er jeg?

De har pillet mig fra hinanden.

Jeg er hverken eller både og....

\says{L}[begejstret] Han er genfødt.  Ser I det allesammen?  Han er
genfødt!

\end{sketch}
\end{document}
