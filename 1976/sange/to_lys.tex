\documentclass[a4paper,11pt]{article}

\usepackage{revy}
\usepackage[utf8]{inputenc}
\usepackage[T1]{fontenc}
\usepackage[danish]{babel}


\revyname{DIKUrevy}
\revyyear{1976}
\version{1.0}
\eta{$?$ minutter}
\status{Færdig}

\title{To lys på en væg}
\author{?}
\melody{Otto Brandenburg: ``To lys på et bord''}

\begin{document}
\maketitle

\begin{roles}
\role{S}[] Sanger
\end{roles}

\begin{song}
  2 lys på en væg
  3 små jobs, der blev væk,
  4 opgaver ingen forstod,
5 studenter med fler',
6 subsekunder til hver,
blev betalt med vort hjerteblod

2 lys på en væg,
een for snot, een for skæg,
een der lyser, og een der er slukt.
Når den lyser så rød,
er maskinen kun død,
og den grønne bli'r aldrig brugt.

(Mellemspil)
Hvordan er det sket?  Har vi slet ikke set,
at de lys brændte hutigt ned?
Den røde er tændt!
Vores håb, det e brændt!
Hvordan skal det dog vare ved?!

En ond parodi
på et talknuseri
det er alt, hvad der findes i dag.
Og de lys og de ord,
som jeg kom med i fjor,
har de glemt midt i tidens jag.

2 lys på en væg
3 små jobs, der blev væk,
4 opgaver ingen forstod.
5 studenter med fler',
6 subsekunder til hver,
blev betalt med vort hjerteblod.
\end{song}

\end{document}

