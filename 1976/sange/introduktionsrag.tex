\documentclass[a4paper,11pt]{article}

\usepackage{revy}
\usepackage[utf8]{inputenc}
\usepackage[T1]{fontenc}
\usepackage[danish]{babel}


\revyname{DIKUrevy}
\revyyear{1976}
\version{1.0}
\eta{$?$ minutter}
\status{Færdig}

\title{Vestre Hastværk Introduktions Rag}
\author{em}
\melody{Maple Leaf Rag}

\begin{document}
\maketitle

\begin{roles}
  \role{S}[?] Sanger
\end{roles}

\begin{song}
  Oh, vær velkommen ver ærede gæst
  ud her til byen og til vores fest.
  Der soldes
  og fjolles.
  Pokerspiller,
  tørvetriller,
  flygelspiller,
  frække friller.
  Sæt jer ned og lyt så til musikken
  I kan sikkert li' den
  når de gi'r den gas.
  Læg nu rigtigt mærke til lyrikken
  der er vitser i den
  der får smilet på plads.

  I vor Wild West by er stemningen høj
  og her i baren er der skæg og halløj.
  Hver mand er
  blandt venner.
  Apoteker,
  hurtigtrækker,
  kistesnedker
  og en kvæker.
  Der er plads til alle, små og store,
  der er sgu folklore
  på drengen i nat.
  Syngepi'rne gi'r den hele armen
  sparer ikke på charmen
  når de først ta'r fat.

  (B stykket)
  Og der danses
  no'en er med i menuetten
  andre griber kastagnetten
  og er ellevilde med den.
  Og fra baren
  smiler hver en ølkrussvinger
  der er hundred' tyve tønder øl igen pr. mand
  fulde i bogstaveligste forstand.
  Og der spilles
  der er gang i huskvartetten
  særlig ham på klarinetten
  og ham på flygeltaburetten.
  Folk de klapper
  og tramper med i takt når Ole
  svinger trommestikkerne og rigtig gi'r den gas
  med ragtime og andet gammel jazz.

  Og har du penge så ta' et spil kort
  det er her ved bordet men spil ikke for hårdt.
  De skyder
  hver snyder.
  Spark i bagen,
  slag på hagen,
  knuste nyrer,
  skrappe fyre.
  Men er du blog nogenlunde fred'li
  så er de alle red' te'
  at gi' dig en hånd.
  Tag og put revolveren ned i lommen
  vi synes ikke om en
  cowboy med en gun.

  Når det er hverdag så slider vi i 'ed
  det er et hårdt job med lang arbejdstid.
  Vi aser
  og maser.
  Rens nu stalden!
  Fang så kal'en!
  Stak nu høet!
  Spred så møget!
  Men når solen så går ned i vesten
  så sadler vi jo hesten
  og rider til byen
  Så får vi no'en drinks ned under vesten
  så vakler alle næsten
  ja sikke et syn.

Men alt er ikke lutter fred og idyl
for nogen vil slås for der's fornøjelses skyld.
Pas på dem!
Til våben!
Folk de skyder
en forbryder
rammer næsten
altså præsten.
Fest forvandles til en heftig fejde
der vil modarbejde
uskyldigt halløj.
Så er det sgu så som så med hyggen
revolverløb i ryggen
er ikke en spøg.
...
Revolverløb i ryggen
er ikke en spøg.
\end{song}

\end{document}

