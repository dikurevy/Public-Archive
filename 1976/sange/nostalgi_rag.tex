\documentclass[a4paper,11pt]{article}

\usepackage{revy}
\usepackage[utf8]{inputenc}
\usepackage[T1]{fontenc}
\usepackage[danish]{babel}


\revyname{DIKUrevy}
\revyyear{1976}
\version{1.0}
\eta{$?$ minutter}
\status{Færdig}

\title{Vestre Hastværks Nostalgi Rag}
\author{hhk, em}
\melody{The Entertainer}

\begin{document}
\maketitle

\begin{roles}
  \role{S}[] Sanger
\end{roles}

\begin{song}
  Her er en sang med lidt nostalgi
  fra dengang vesten var fuld af tju-bang.
  De gamle dage er nu forbi
  men før de glemmes, så husk er der var engang
  da kortspil, slagsmål og skyderi
  var ej forbudt af et stift reglement.
  Der var cowboy'er te'
  som vi får ej mer' at se
  det er en tid, der for altid forsvandt.

  For da teknikken kom
  da slog mimikken om
  fra smil til sure miner, død og salmesang - åh hvilken gru -
  Det gode fællesskab
  det blev til fælles tab
  da tangenter fortrængte fjer og pen
  Det hule tomme sug
  fra apparaters bug
  forkyndte nye tiders fødsel, nu struktur
  hvor lamper blinker døgnet rundt i hundredevis
  et teknokratens paradis

  Kom med mig til den gamle by
  kom og genskab det gamle miljø
  Lyt, klaveret kan hør's på ny
  som dets sjæl aldrig rigtig ville dø.
  Lad os bryde teknikkens bånd
  lad os ombringe vor plageånd.
  Et tryk på denne kontakt
  så er det hele fuldbragt
  og så kan livet begyndet på ny.
\end{song}

\end{document}

