\documentclass[a4paper,11pt]{article}

\usepackage{revy}
\usepackage[utf8]{inputenc}
\usepackage[T1]{fontenc}
\usepackage[danish]{babel}


\revyname{DIKUrevy}
\revyyear{1976}
\version{1.0}
\eta{$?$ minutter}
\status{Færdig}

\title{Drikkevise}
\author{?}
\melody{Blueberry Hill og omegn}

\begin{document}
\maketitle

\begin{roles}
\role{S}[] Sanger
\role{D}[] Datamaten
\end{roles}

\begin{song}
Jeg fik en lille
på Blueberry Hille
en drink eller til
gjorde mig li'som fri.

Men det blev en dille
på Blueberry Hille
for alt sku' laves om
dengang edb kom.

Men glæden forsvandt da
promillens gesandt sa':
Hvis der skulle bælles,
så skulle der tælles!

Med tælleren på
gik glæden i stå
hver rød elefant
gik hen og forsvandt.

Nu skal man ha'
et kontokort, ja
for hver eneste sug
bli'r det klippet itu.

Og efter det sidste glas
hvor man sidder lidt tungt på sin plads
og hvor det er slut med at køb'
De'r overløb!

Og nu er det slut
med at blive budt
på whisky og rom
og en lille vom

Hvor er min rus
mit bredfyldte krus
ølhanen drej's om
for min konto er tom.

Maskinen skal bæres væk
den er blevet bareks skræk
for alt det der tælleri
vil jeg helst være fri.

Jeg ku' bli' fuld
for en halv pose guld
men nu bli'r lommen tom
for et enkelt glas rom.

Jeg rejser mod vest
både mig og min hest
for at finde et sted
hvor vi kan drikke i fred.
\end{song}

\begin{sketch}
  \says{D} Vil ejeren af hest nr. P*RUHEST.HYP melde sig.  De kan sgu
  ikke få det store øg ned i den lille pose.
\end{sketch}

\end{document}

