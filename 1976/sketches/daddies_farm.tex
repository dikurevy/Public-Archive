\documentclass[a4paper,11pt]{article}

\usepackage{revy}
\usepackage[utf8]{inputenc}
\usepackage[T1]{fontenc}
\usepackage[danish]{babel}


\revyname{DIKUrevy}
\revyyear{1976}
% HUSK AT OPDATERE VERSIONSNUMMER
\version{1.0}
\eta{$?$ minutter}
\status{Færdig}

\title{Daddies farm}
\author{?}

\begin{document}
\maketitle

\begin{roles}
\role{F}[] Cowboy-flygtning
\role{2C}[] 2. cowboy
\role{3C}[] 2. cowboy
\role{4C}[] 2. cowboy
\role{1F}[] Første formand
\role{2F}[] Anden formand
\end{roles}

\begin{sketch}

\scene{Almindelig barstemning.  Pludselig}

\says{F}[ind, ser sig skræmt omkring, skjuler sig i lokalet.]
\act{Umiddelbart efter kommer to formænd ind -- med pisk}

\says{1F} Hvor fanden er han, løb han ikke her ind?

\scene{De ser sig omkring.  F er gang på gang ved at blive
  afsløret, men redder sig altid i sidste øjeblik.  Alle bargæsterne
  forholder sig passive og tavse.  Efter et stykke tids eftersøgning.}

\says{2F} Nå, får vi ham ikke her i juni, så ta'r vi ham sgu til august.

\scene{De går ud, og F kommer frem.}

\says{F} Pyha, det var på et hængende hår.  Der havde de nær fået mig.

\says{2C} Hvor kommer du fra, cowboy?  Og hvem er de mænd, der jager dig?

\says{F} Ved du ikke det?  De to kommer ude fra Daddies Farm.  Der har
jeg været cowboy ude.  Det var sgu et helvede.  Der va måske nok en
masse interessant arbejde derude, men hele den måde, vi skulle gå frem
på, var ikk etil at holde ud.  Og ikke nok med det, der var ikke kun
én herre, der var sgu tre!  Mens jeg arbejdede derude, arbejdede jeg
hos den første herre på Daddies Farm.  Dad. 0.  Det var på
Stensgården.  Sikke et job.  Der var elleve hundrede og ti forskellige
jobs, der skulle laves.  Og så havde man ansat så mange cowboys, at vi
stod og vadede på hinanden, hver gang vi skulle til at lave noget.
Næ, jeg har såmænd ikke noget imod at lave noget.  Det er ikke det,
der trykker.  Men, tænk jer, hver gang man havde lavet det mindste, ja
så var det ikke nok bare at sige "`Nu har jeg lavet det"'.  Man skulle
også {\em bevise}, at man havde lavet det.  Og det selvom det var
ligetil at se.  Ja, jeg siger jer, at jeg tilbragte mere tid med at
overbevise dad. 0 om, at jeg havde lavet noget, end jeg tilbragte med
at lave noget.  Sådan er det derude.  Og alt muligt blev man sat til
at lave.  Vi skulle fandeme rende rundt i træerne, og gud nåde og
trøste den, der ikke havde fået vendt hvert eneste blad.  Men tror I
det var nok?  Næ, der var også forskrifter for, i hvilken rækkefølge
vi skulle vende bladene.  Der var næsten ingen ende på det.

Og alle de køer, der skulle passes.  Pyha.  Der skulle puttes noget
ind i den ene ende, og så skulle der tages noget fra i den anden ende.
Det var sgu ikke altid, det lugtede ens.  Fy for satan siger jeg jer.

Nå, det var vel ikke værre, end når høsten skulle stakkes.  Det var en
ordentlig grebfuld oveni, før der kom nogle andre for at tage noget
fra.  Jeg kan huske en gang, hvor der ikke kunne være mere i stakken.
Der var en, der fik den geniale idé, at lave et forskydeligt loft.
Men så kom dad. 0 farende.  Det skulle der ikke være noget af her,
sagde han, vi skulle lige fra starten have indset, hvor meget plads
der skulle bruges.  Har I hørt magen til stædighed.  Jeg er helt
sikker på, at om 100 år vil de skrive af grin af den slags.

Nå, det var også i orden, jeg kan indrette mig under lidt af hvert, og
jeg er ikke bange for at arbejde.  Men jeg kan fandeme ikke holde ud
at udflde de åndssvagt arbejdssedler.  Side op og side ned.  Et af
stederne på arbejdssedlen skulle vi skrive, hvorfor vi nu lavede det,
som vi altså lavede, og ikke noget andet som vi altså ikke lavede, og
alle de ting i nærheden som vi selvfølgelig slet ikke gad lave.  Havde
de bare givet os bedre besked, kunne vi have sparet alt det der.  Et
andet sted skulle vi fortælle om arbejdsgangen, sådan noget med hvilke
slags værktøj, man har brugt og så videre.  Hvem fanden kan huske det
bagefter?  Nå, man blev jo efterhånden udlært digter.  Men det værste
var næstsidste afsnit.  Det var det rene fantasi.  Der digtede man om,
hvor godt og brugbart det hele var.  Og skulle man bruge en vindmølle
i stedet for en mødding, så vender man bare arbejdstegningen om!  Det
skal nok virke, og det skrev man så i sidste afsnit.

\scene{Flere cowboys er kommet til.}

\says{3C} Ja, jeg har hørt ,at hjælpeformændene derude er de cowboys,
der er blevet bidt af køerne.  Sikken en personale-politik.

\says{3C} Tror du det er noget, så skulle du fanden gale mig prøve at
være hyret hos en af de andre daddies derude.  Dad. 1.  Det er det
sted ,man kalder "`den halte trekløver"'.  Der får du lov at arbejde.
Kan du forestille dig noget værre end nogen, der slæber dig gennem ild
og møg, mens de går og rabler vittigheder af sig.  Vorherre bevar's.

Tænk dig, noget af det første vi skulle lave var at sige ligesom et
lommetærskeværk.  Kan du forestille dig, hvor åndssvagt det så ud, når
100 cowboys står på række og brummer "`Halløjsa, der har vi nok en
lille bid mere, så må vi jo hellere stoppe og kigge lidt på den.
Hovsadasse, der har vi jo hele to mere, så er det nok på tide at sende
noget af det gamle videre."'

I kan sige, hvad I vil, men det var det eneste sted, hvor der var så
meget jag og ræs, at man blev tvunget til at lave en masse ting
samtidigt.  Men ikke nok med det.  Dad. 1 var så bange for, at vi
skulle støde sammen, så hver gang vi kom hjem fra marken, skulle vi
stille os et stykke fra døråbningen og stå og råbe "`Nu vil jeg gerne
komme, er det i orden med jer andre?"'  Og tænkte man sig ikke
omhyggeligt om, så kunne man stå og råbe det i kor hele dagen.  Det
var helt overflødigt.  Og det sprog de brugte.  Jeg plejede jo at råbe
"`Se jer for"', men de råbte "`semafor"'.  Sikke en dialekt.

Men første gang jeg så min lønseddel, orw hvor blev jeg glad.  Hold
kæft en kæmpeløn.  Er du da mør, et kæmpetal.  36 cifre.  Det viste
sig nu senere at være ren svindel.  Jeg blev så rasende, at jeg
knaldede dem allesammen en på bærret, så kom jeg selvfølgelig i
brummen.  Det er den eneste gang i mit liv, hvor jeg har været glad
for at kunne håndtere en fil.

\end{sketch}
\end{document}
