\documentclass[a4paper,11pt]{article}

\usepackage{revy}
\usepackage[utf8]{inputenc}
\usepackage[T1]{fontenc}
\usepackage[danish]{babel}


\revyname{DIKUrevy}
\revyyear{1976}
% HUSK AT OPDATERE VERSIONSNUMMER
\version{1.0}
\eta{$?$ minutter}
\status{Færdig}

\title{Dillegencen}
\author{KK}

\begin{document}
\maketitle

\begin{roles}
\role{A}[] Person
\role{B}[] Person
\end{roles}

\begin{sketch}

  \scene{To personer sidder over for hinanden på hver sin bænk.  På
    trommen antydes fjerne hovslag.  De to gynger i takt med; de er i
    en DILLIGENCE.  De læser i hver sin bog.  Pludselig kigger A op
    over bogen på B.  Læser videre.  Lidt efter gør B det samme.  Så
    siger}

\says{A} Hvad er det du læser?

\says{B} "`A Concise Survey of Computer Methods"'

\scene{Skilt ind: DS: "`En Konfus Surdej af Komputer Metoder"'}

\says{A} Må jeg se \act{tager bogen.  Læser:} "`Må ikke hjemlånes.  Kun til brug på instituttet"'. \act{Ler kunstigt} Ha, ha.

\says{B}[endnu kunstigere] Ha, ha, ha.

\says{A+B}[yderst kunstigt] Ha, ha, ha, ha.

\says{A} Hvor har jeg set det før?  Åh, jo.  Her \act{tager sin egen,
  læser:} "`Må ikke hjemlånes.  Kun til brug på instituttet"'.

\scene{B ler voldsomt og larmende.}

\says{A} Ha -- øh. \act{Pause} Hvad griner du af?

\says{B}[halvkvalt af grin] Så er vi jo fra samme sted!

\says{A} Hvad!  Er du fra Sigurdstown?

\says{B} Ja.

\says{A} Fra DIKU \act{udtales engelsk: [di ai kej ju]}.

\says{B} Ja!

\says{A+B} Arh!  Kom i mine arme \act{falder hinanden om halsen}!

\says{A} Det må vi have en lille een på \act{tager en lommelærke frem.
  De drikker}.

\says{B} Sigurdstown, øh.  Hvem er der?

\says{A} Jah, der er jo Pnautilus.

\says{B} Øh -- Pnautilus?

\says{A} Ja -- Pnau"= \act{pause} -- tilus \act{pause}.

\says{B} Hvem er det?

\says{A} Nå, ja.  Så er der jo Abe Colerani.

\says{B}[henrykt genkendende] Åh, Colerani!

\says{A} Jah, og Tuz Datkabale.

\says{B} Tuz Datkabale, ja.  Den store!  \act{Interesseret:} Hvad går
Datkabale og FORSKER i -- for tiden?

\says{A} Studienævn -- han skal se om han kan finde en sag -- efter det nye arkivsystem.

\says{B} Og så er der jo --

\says{A} Du kender nok dem allesammen?

\says{B} Ja, da.  Da jeg var i Sigurdstown var jeg jo ansat dér.

\says{A} Dér?  Er det sandt?

\says{B} Ja, på universitetet.

\says{A} Det var jeg osse!

\says{A+B} Arh, kom i mine arme \act{falder hinanden om halsen}!

\says{A} Det må vi ha' en lille een på \act{tager lommelærken frem.  De drikker}

\says{A} Ved hvilket kursus?

\says{B} EDB7.

\says{A} Det var jeg osse!

\says{A+B} Arh, kom i mine arme \act{falder hinanden om halsen}!

\says{A} Det må vi ha' en lille een på \act{De drikker}

\says{B} Forårs-- eller efterårssemestret?

\says{A} Forår.

\says{B} Nå ja, jeg var i efteråret. \act{pause}

\says{A+B} Så er det derfor vi to aldrig har set hinanden!!

\says{A} Det må vi ha' en lille een på!

\scene{De drikker.  Dilligencen standser.}

\says{B} Nå jeg skal af her.  Jeg skal hente min hest her.  Det er
sådan en lille islænder.  En Mikro.  Men det er en dejlig hest.

\says{A} Mikro?  Men så er du den EDB7-lærer der forsvandt i Midtvesten sidste vinter?

\says{B} Ja"=øh--

\says{A} EDDY SWENSSON, I PRESUME!  (Griber hans hånd, højtideligt)

\says{B} Det er mig!

\says{A+B}[ud] Det må vi ha' en lille een på!

\scene{Tæppe}

\end{sketch}
\end{document}
