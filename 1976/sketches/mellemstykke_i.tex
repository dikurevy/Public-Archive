\documentclass[a4paper,11pt]{article}

\usepackage{revy}
\usepackage[utf8]{inputenc}
\usepackage[T1]{fontenc}
\usepackage[danish]{babel}


\revyname{DIKUrevy}
\revyyear{1976}
% HUSK AT OPDATERE VERSIONSNUMMER
\version{1.0}
\eta{$?$ minutter}
\status{Færdig}

\title{Mellemstykke I}
\author{?}

\begin{document}
\maketitle

\begin{roles}
\role{C1}[] Cowboy
\role{C2}[] Cowboy
\role{C3}[] Cowboy
\role{C4}[] Cowboy
\role{B}[] Bartender
\end{roles}

\begin{sketch}

\says{C1}[ind] En dobbelt.

\says{B} Den kan du vel selv tage.  Det er velike meningen, jeg skal
betjene alle og enhver, bare fordi andre ikke gider og jeg har barvagt
4. dag i træk.

\says{C2}[ind, tager et glas, sætter sig ved spillebordet] Vil du være
med til en makro?

\scene{C1 og C2 sætter sig, spiller agurk med {\em 5} kort, mens}

\says{C3}[ind for at deltage] Hvad kommer man med på?

\says{C1} Halvdød.

\says{C3} Så må jeg have en forhånd. \act{efter at have deltaget en
  enkelt runde}: Jeg dør \act{skyder sig}.

\says{C4}[ind] Hvad kommer man med på?

\says{C2} Pas lidt på dit helbred, hva?  Du kan vel se, der allerede
ligger.

\says{C1} Ja, det er altid så trist med selvmord.

\says{C4} Selvmord?

\says{C2} Ja, han fik en selvdød.

\scene{Musikken har måske hyggespillet, men slår i hvert fald over til
  can-can indledning.}

\says{C4}[til C3] Se at komme på benene.  Nu kommer pigerne.

\scene{C3 rejser sig.}

\end{sketch}
\end{document}
