\documentclass[danish]{article}
\usepackage{revy}
\usepackage[utf8]{inputenc}
\usepackage{babel}
\usepackage{a4wide}

\title{D.S.O.I.S.}
\author{Tøserne og Uffe}

\version{7} % HUSK AT AJOURFØRE VERSIONSNUMMER!!
\status{færdig (kan dog have bedre punchline)} % ...OG STATUS!!
\eta{7 min.}

\revyyear{2000}

\begin{document}
\maketitle

\begin{roles}
  \role{V1}[Maiken] En veninde
  \role{D}[Katrine] Den sidste veninde (Datalone)
  \role{F1}[Niels] En lækker fyr, endnu en lækker fyr og den sidste lækre fyr
  \role{K}[Peter WN] Karl Koder -- en datalogfyr
  \role{H}[Uhd] Håndværker
  \role{G}[Andre] Megakæk gameshowvært type
\end{roles}

\begin{sketch}

\scene En overkæk gameshowvært type (G) træder ud foran tæppet og
introducerer sketchen.

\says{G}[superoverkækt] Mine herrer og damer -- ja I bliver jo flere -- det
er DIKU-revyen en ære at præsentere: "Den Store Omvendte Interaktive
Scoresketch"

\scene G træder ud til siden, tæppet bliver trukket fra og vi ser en bar.
Der sidder tre veninder (V1 og D) og hænger en lækre fyr (F1), en datalog (K(datalog)) og en håndværker (H).

\says{V1} Altså jeg fik en stor buket blomster af min kæreste i går!

\says{D} Hvorfor har jeg ikke en kæreste.

\says{V1} Hvornår finder du dig en, Datalone?

\says{D} Øøøh?

\says{V1} Der er da masser på dit studie!

\scene Datalone sukker og ruller med øjnene, gaber osv.

\says{V1}[peger på F1] Hvad med ham der sidder derovre? Han er da lækker, scor ham!

\says{D} Det kan jeg da ikke

\says{V1} Jo, jo

\says{D} Okay! \act{går over til ham}

\says{D}[kælent] Heeej

\says{F1}[viser stor interesse] Hejsa, hvad laver en lækker pige som dig
her?

\says{D} Jeg prøver at score

\says{F1}[lægger armene om hende] Nå da, så er du kommet til den rette

\says{D} Hvad laver du til dagligt? \act{fniser}

\says{F1} Jeg er lige blevet færdig som fysioterapeut

\says{D}[hiver lidt i sine skuldre] Jeg har så ømme skuldre, kan du ikke
lige
massere mig her \act{peger på skuldrene}

\says{F1}[begynder at massere] Nå hvad laver du så?

\says{D} Jeg læser datalogi \act{meget glad og smilende}

\says{F1}[fjerner hurtigt sine hænder fra pigen og rejser sig op og går]
Ad!! \act{Skifter trøje med ryggen vendt mod publikum. Vender sig om og går hen mod Datalone}

\scene Imens kigger Datalone desperat på sine veninde som ser forvirrede ud og
peger glædeligt på en fyr som kommer gående mod hende. Husk at holde pause, så F1 kan nå at skifte tøj

\says{F2} Hey, du ser enormt sexet ud, må jeg byde på en øl?

\says{D} Øh, ja tak!

\says{F2} Hvad drikker du?

\says{D}[smiler] Nykøbing Blå eller Coca Cola!

\says{F2}[hånligt med lidt skræk i stemmen] Du læser datalogi ik'?
\act{skynder sig at gå. Skifter trøje igen.}

\says{D}[ser sig godt omkring. Nu er hun virkelig desperat, og går hen mod F3]
Jeg vil gøre alt, hvad du vil have mig til!

\says{F3}[Kommer gående hen mod D. Ser hånligt på hende] Du læser datalogi ik'? \act{vender sig om og går}

\says{D}[råber] Men jeg har en hobby!!!

\says{K}[prikker pigen på skuldrene] Hej, jeg læser datalogi. \act{Håndværker ind på scenen.}

\says{D}[himmelvendte øjne] Åh, nej! \act{går over til en håndværker. Tager
ham under armene og skal til at gå ud}

\says{K}[afbryder] Arhj hør nu her tøser. Vi ved jo godt hvad det
går ud på... Se, nu gi'r jeg lige fri bar for os allesammen med mit
forgyldte dankort \act{flasher sit dankort} for jeg tjener urimeligt mange
penge...

\scene Billedet ``fryser''. Ind på scenen kommer G og er (naturligt nok)
meget kæk.

\says{G}[supermegaoverkækt] Og nu mine herrer og damer -- ja I bliver jo
flere -- kommer det øjeblik vi har ventet på. For det er jo "Den Store
Omvendte Interaktive Scoresketch".

\says{G}[skaber spænding] Skal datalogpigen score datalogfyren og risikere
plug'n'play og koldstart? Så vil datalogfyren jo få allertiders nat i
selskab med... \act{med stor røst} EN KVINDE!!!

\says{G}[mere hys hys] Eller skal hun knalde håndværkeren for sort arbejde?
Dermed vil datalogfyren jo få allertiders nat i selskab med... \act{med stor
røst} ET DATAMATSPIL!!!

\says{G}[kæk igen] Ja publikum. Nu er det op til jer. I bestemmer. Som I det
gode gamle Grækenland er reglerne simple: der råbes i to omgange og først
når jeg har talt til 3. Først: alle der ønsker at datalogpigen scorer
datalogfyren -- når jeg har talt til 3 så råber I "SEX". Okay? 1... 2...
3...

\says{Publikum}[med stor iver naturligvis] SEX! \act{og mon ikke også der er
enkelte der råber "smid tøjet" og "fisse"?}

\says{G} Okay... okay... Godt. Og så: alle der ønsker at datalogpigen scorer
håndværkeren -- når jeg har talt til 3 så råber I "QUAKE". Okay? 1... 2...
3...

\says{Publikum}[med stor iver naturligvis] QUAKE! \act{og mon ikke også der
er enkelte der råber "smid tøjet" og "få jer et liv"?}

\scene ------------------------------------------------------------------------------

\scene Alternativ slutning 1: hvis der bliver råbt "SEX" højest

\says{G}[smart] Allright! Vi har hermed fået afgjort at datalogerne rent
faktisk burde være i stand til at formere sig -- I har på demokratisk vis
tilkendegjort at SEX er mere interessant end QUAKE.

\says{G}[hys hys] Lad sketchen fortsætte.

\scene Billedet ufryser (afrimes?)

\scene Veninderne trækker sig lidt til siden og viske tisker. Griner
derefter hysterisk.

\says{D} Arj altså Karl Koder -- lige pludselig synes vi du er helt vildt lækker.

\says{K}[griner fjoget] Hej, sig det lige igen. \act{a la
løvernes konge}

\says{V1} Må vi ikke liiige se dit dankort. \act{tager kortet fra ham} Ej sikke
et pænt billede...\act{kunstpause, og K lyser op} ...er det din bror? \act{K
  falder ned igen}

\says{K}[spørger nervøst] Vil I med på webmotel?

\says{V1} Øhh..ih jo. Vi køber lige nogle øl på dit dankort først!
\act{smiler, og skubber K væk}

\scene Veninder går med dankortet hen til håndværkeren, hvorefter de sammen
går dansende ud og synger ``Håndværker sex det er fedt - smæk!''

\scene K står tilbage alene på scenen og er lidt betuttet. Tæppe for.

\says{G}[voice-over] Datalog... Find dig et nyt studie... Eller find dig i
hvad som helst!

\scene ------------------------------------------------------------------------------

\scene Alternativ slutning 2: hvis der bliver råbt "QUAKE" højest

\says{G}[overrasket] Aaaaha! Det er DIKU-revyen en fornøjelse at spille her
i aften, for hvad der tydeligvis må blive den sidste generation af
dataloger -- I har på demokratisk vis tilkendegjort at QUAKE er mere
interessant end SEX.

\says{G}[hys hys] Lad sketchen fortsætte.

\scene Billedet ufryser (afrimes?)

\says{V1, D}[i kor] Ihh altså Karl Koder lige pludselig synes vi bare du er
helt vild lækker...

\scene K forlader scenen med de to tøser under armen(e). Håndværkeren går ud
foran tæppet mens det bliver trukket for.

\scene Håndværkeren står tilbage alene på scenen. Han er lidt uberørt af
situationen.

\says{H}[kigger på publikum, skeptisk] Tror I selv på den? 

\says{G}[voice-over] På DIKU lider 81\% af mændene af grænsepsykotiske
vrangsforestillinger forårsaget af sexuel underernæring. \act{pause} Datalog...
Find dig et nyt studie... Eller find dig i hvad som helst!

\end{sketch}

\end{document}
% Local Variables: 
% mode: latex
% TeX-master: t
% End: 
