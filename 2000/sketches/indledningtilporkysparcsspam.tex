\documentclass[danish]{article}
\usepackage{revy}
\usepackage[utf8]{inputenc}
\usepackage{babel}
\usepackage{a4wide}

\title{Indledning til SPAM!}
\author{Theo}

\version{1} % HUSK AT AJOURFØRE VERSIONSNUMMER!!
\status{færdig} % ...OG STATUS!!
\eta{3 min.}

\revyyear{2000}

\begin{document}
\maketitle

\begin{roles}
  \role{S}[Mike] studerende
  \role{AF}[Fehår] adfærdsforsker
  \role{K1}[Andre] Kor
  \role{K2}[Adam] Kor
\end{roles}

\begin{props}
  \role{flipover} Til flowchart?
\end{props}

\begin{sketch}

\scene{På scenen står et bord med en terminal placeret i en vinkel så
publikum ikke kan se den, en post-kasse, der kan sige en "lyd" (dette er
ambolt-lyden fra sangen), når der kommer et brev, evt. kan den også hejse et
lille flag, lyse med en lampe eller noget lignende i takt til musikken -
nåhja, og en stol. Evt. en maskine på bordet med skiltet "Gorky SPARC" på.}

\scene{Tæppet går fra. Der er helt mennesketomt, ikke så meget lys, en
person kommer løbende ind}

\says{S}[sætter sig ned ved datamaten, febrilsk, grøn spot]
Fuck, fuck, jeg kan lige nå det, forelæseren er kun lige ankommet til
auditoriet, og så, logge ind, \ldots, og password, \ldots, kom nu, kom nu,
kom nu!!!!! Fuck, fuck!

\scene{Postkassen siger "kling", lyser og gør ved\ldots}

\says{S}Yeah!!!!! (pause) Naiiij!!!! Pis!

\scene{S smadrer maskinen, banker keyboardet ned i bordet, generelt meget
aggresiv}

\says{AF}[voice-over] STOP STOP!! (alt fryser, al lys
kommer på) Vi må vist lige forklare hvad der sker her...

\scene{AF kommer ind på scenen}

\says{AF} Ja, som I jo nok kan se er der tale om et meget frustreret
menneske her. Og det vil jeg gerne forklare jer lidt om her i aften. Men
først skal jeg lige have hjælp af Sutte og Rulle\ldots, hmmm.

\scene{Kan ikke finde Rulle og Sutte i sin taske}

\says{AF}[henter en flip-over] Ja, da Sutte og Rulle ikke er her i aften,
bliver jeg nødt til at få lidt hjælp fra jer, publikum.

\scene{Flipper flip-overen, hvorpå der står flg.\ tekst:}

\begin{song}
\sings{Publikum} Åh åh Åh-oh
     Åh-åh Åh-oh Åh
     Åh Åh Åh-oh
     Åh-åh Åh-oh Åh (eller noget i den stil)
\end{song}

\says{AF} Hvis jeg må bede bandet om at starte\ldots

\scene{Postkasse (med ambolt-lyden), trommehvirvler og bas starter, så
publikum kan finde tonelejet}

\says{AF} Ja, nu skal I bare synge med, I skal bare være klar på at stoppe
når jeg viser jer dette stop-skilt.

\scene {\ldots{} med pegepind og sang startes publikum op! Når publikum er
kommet op at køre med masser af "højere" tilråb osv. bliver de stoppet med
stop-skiltet og sangen begynder, sangen bliver naturligvis sunget af den
studerende, til sidst kunne man afsløre hele omkvædet så publikum kunne
synge med på både åh-åh-stykket og omkvædet to gange??}

\end{sketch}

\end{document}
% Local Variables: 
% mode: latex
% TeX-master: t
% End: 
