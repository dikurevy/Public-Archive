\documentclass[a4paper,11pt]{article}

\usepackage{revy}
\usepackage[utf8]{inputenc}
\usepackage[T1]{fontenc}
\usepackage[danish]{babel}

\revyname{DIKUrevyens 50 års jubilæum}
\revyyear{(2023)}
\version{0.1}
\eta{$0.5$ minutter}
\status{Ikke færdig}

\title{Unittest shoutout}
\author{Bosserne}

\begin{document}
\maketitle

\begin{roles}
\role{B}[Elbo] Boss, iklædt revytrøje
\role{U}[Torben] Fornavn Efternavn
\role{V}[Sean] Fornavn \emph{Mellemnavn} Efternavn
\role{X}[Niels] Instruktør
\end{roles}

\begin{sketch}

\scene{U træder ind på scenen, clipboard i hånden og kigger sig omkring. U stiller sig hen på lokation 0.}

\says{U} Mit navn er Fornavn Efternavn og jeg skal præsentere en sketch!

\says{U} Jeg kan bekræfte, at der til start af sketchen er publikum, når jeg står på lokation 0. Så jeg siger hej publikum!

\scene{U hiver en lap papir op af baglommen og læser op fra den}
\says{U} Har I hørt den om ham der skiftede fra vim til emacs?
\scene{U fniser for sig selv og lægger sedlen væk.}

\says{U} Antallet af jokes fortalt: 1. Jeg tester nu, at publikum stadigvæk er her på lokation 0.
\scene{U kigger ud på publikum.}

\says{U} Efter én joke, kan jeg bekræfte, at publikum stadigvæk er på lokation 0.

\scene{V træder langsomt og opgivende ind på scenen}

\says{V} Goddag. Mit navn er Fornavn Mellemnavn Efternavn og jeg SKAL være med i denne test.

\scene{V træder hen til lokation 1}

\says{B} Jeg er nu på lokation 1, og vil fremføre joke 0.

\scene{B hiver en lap papir op af baglommen og læser opgivende op fra den}
\says{V} Har I hørt den om ham der skiftede fra vim til emacs?

\scene{U dør af grin}

\scene{B kommer løbende ind på scenen}

\says{B} STOP STOP STOP! Jeg glemte at vi skal holde en tale!

\scene{V er lettet over at unit-test-sketchen er stoppet}

\says{B} Velkommen til DIKUrevyens 50-års jubilæum. For 50 år siden
    slæbte en gruppe dataloger et klaver ud på en mark, og afholdt den første
    revy. Siden da har meget materiale været opført:

    Der er blevet programmeret på hulkort, i terminalrummet, alene i kantinen,
    til alarmens sirenesang, i C, Emerald, SML og Scratch.

    Gennem flytninger, reformer, autoforelæsninger og høje dumpeprocenter
    har revyen været der, for at bringe glæde og samle de studerende.

    \scene{TODO: færdiggør denne?}

\says{B}[til U/V] Nå, men, undskyld afbrydelsen! I kan bare fortsætte.

\scene{V opgivende, U glad}

\says{U}[gladt] Jeg går nu på lokation 2, for at fremføre joke 0.

\says{V} Nej, stop. Torben, jeg gider simpelthen ikke være med i revyen, hvis
jeg skal være med i den her sketch igen\ldots

\scene{V begynder at tage sit kostume af.}

\says{V} Hér, hold mine ting - jeg sætter mig ned til publikum; så må I klare jer
uden mig.

\scene{V giver mikrofon og kostume til U}

\says{U} Ah øv, jeg havde ellers lige formuleret en grundig testplan - den burde ikke have taget mere end 45 minutter\ldots Nå, men velkommen til DIKUs 50 års jubilæumsrevy!

\scene{Lys ned}

\end{sketch}
\end{document}
