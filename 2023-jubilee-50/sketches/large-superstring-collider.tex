% Oprindeligt filnavn: largesuperstringcollider.tex
\documentclass[a4paper,11pt]{article}
\usepackage{revy}
\usepackage[utf8]{inputenc}
\usepackage[T1]{fontenc}
\usepackage[danish]{babel}
\usepackage{graphicx}
\revyname{DIKUrevyens 50 års jubilæum}
\revyyear{(2018)}
\version{1.0}
\eta{$2$ minutter}
\status{Færdig}
\title{Large Superstring Collider}
\author{Troels, Phillip}
\begin{document}
\maketitle
\begin{roles}
  \role{F}[Eva] Forsker
  \role{N1}[Elbo] Ninja til at bære overhead-projektor
  \role{N2}[Sebbe] Ninja til at bære overhead-projektor
  \role{X}[Simon] Instruktør
\end{roles}
\begin{props}
  \prop{Diasshow}
\end{props}

\begin{sketch}

\scene{Forskertype kommer ind på scenen}

\says{F} Jeg har de sidste 20-30 år ledt et hold af nogle af de
dygtigste forskere på DIKU, for at udvikle den datalogiske
superstrengsteori!

\says{F}[ned til publikum] Ser I, alt i verden kan beskrives ved
hjælpe af strenge, og nogle af disse strenge har særlige
karakteristika, der gør at vi kalder dem
\textit{super}strenge. \act{Væk fra publikum} Dét er videnskab!

\says{F} Her ses eksempler på typiske, omend meget korte,
superstrenge, som vi har været i stand til at isolere:

\scene{Dias med: supermario, superuser, superdrinks, superstreng, superduperstreng osv....}

\says{F}[ned til publikum] Men idet der findes uendeligt mange
strenge, \act{næste} kan det være et problem at filtrere netop disse
\textit{super}strenge fra.

\says{F}[farer videre til en fjerm publikummer] Vi har derfor over de
sidste 10 år bygget en maskine, \act{næste publikummer} der er i stand
til at afgøre om en \textit{vilkårlig} streng er en
...\textit{super}streng!

\says{F} Vores første prototype var en lineær \textit{super}-DFA.

\includegraphics[width=10cm]{large-superstring-collider/super1.png}

\says{F} Dét er videnskab!

\says{F} Vi havde dog problemer med at få den til at blive hurtig
nok. \act{Ned til publikum} Ser du, der er uendeligt mange
superstrenge, \act{næste} og de fleste af dem er enoooormt lange,
\act{næste} og de bevægede sig ret langsomt igennem DFAen.

\says{F}[Væk fra publikum] Derfor udvidede vi vores DFA til en
cirkulær NFA, med en accelerator-ring af epsilon-transitioner, så vi
kan accelerere strengene helt op til lysets hastighed.

\includegraphics[width=10cm]{large-superstring-collider/super2.png}

\says{F}[Ned til publikum] Vi håber i disse superstrenge at kunne
finde bevis på eksistensen af den flygtige
suf\textit{fiks}-boson, \act{næste} som giver suffiks til alle strenge
i universet. \act{Væk fra publikum} Det' videnskab!

\says{F}[Ned til publikum] Men selv om vi nu kan \textit{finde}
\textit{super}strenge, hvordan skal vi så undersøge dem? \act{Næste}
Jo, vi udvider vores NFA til at være en \textit{super}streng-collider
ved at lade to strenge køre i modsat retning.

\includegraphics[width=10cm]{large-superstring-collider/super3.png}

\says{F} Altså, hvis de andre må forske ved at smadre ting sammen og
kigge i resterne, så må vi vel også!  Det \textit{er} jo videnskab!

\scene{Lys ud}

\end{sketch}

\end{document}

%%% Local Variables:
%%% mode: latex
%%% TeX-master: "loadingbar"
%%% End:
