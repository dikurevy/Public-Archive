% Oprindeligt filnavn: hri.tex
\documentclass[a4paper,11pt]{article}

\usepackage{revy}
\usepackage[utf8]{inputenc}
\usepackage[T1]{fontenc}
\usepackage[danish]{babel}

\revyname{DIKUrevyens 50 års jubilæum}
\revyyear{(2013)}
\version{2.0}
\eta{$3$ minutter}
\status{Færdig}

\title{Human-Reality Interaction}
\author{Ejnar, Troels, Søren, Nana, Ronni, NB}

\begin{document}
\maketitle

\begin{roles}
    \role{F1}[NB] Forsker 1
    \role{F2}[Niels] Forsker 2
    \role{S1}[Mads] Studerende 1
    \role{S2}[Sia] Studerende 2
    \role{X}[Simon] Instruktør
\end{roles}

\begin{sketch}

  \scene{Lys op.}

  \says{F1} Og så sagde han, at den kørte O$(n \lg n)$.

  \scene{F1 og F2 griner - de studerende kommer over}

  \says{S2} Vi er altså ikke tilfredse med undervisningen. Vi føler
  lidt, at der kun undervises i teori uden forbindelse til virkelige
  problemer.

  \says{S1} Ja, vi er blevet helt ude af stand til at begå os i den
  analoge virkelighed.

  \says{F1} Pjat, datalogien rummer alle svarene på hvordan man...

  \says{F2} ...hvordan man på optimal vis begår sig i hverdagen.

  \says{F1 + F2} Kig i bøgerne.

  \says{S2} Jamen vi har så mange bøger, at der næsten ikke er plads i
  tasken!

  \says{F2} Trivielt! Du skal blot pakke bedre.

  \says{F1} Det er Knapsack problemet.

  \says{F2} Rygsæk! Det er da teori brugt i virkeligheden.

  \says{F1} Hører I slet ikke efter til forelæsningerne?

  \says{S1}[lidt forlegen] Jeg kunne ikke lige nå rundt til alle
  forelæsninger!

  \says{F1} Det kunne du jo bare have sagt - så skal du bruge en
  omrejsende dørsælger.

  \says{F2} Traveling salesman.

  \says{F1} Det er jo vigtigt at løse det mere generelle problem!

  \says{F2} Det burde I vide!

  \says{F1} Har I slet ikke prioriteterne i orden!

  \says{S2} Og hvordan vil du have, at vi ordner vores prioriteter.

  \says{F1} Jo det er trivielt at ordne sine prioriteter - med en effektiv
  sortering...

  \scene{De studerende står ved siden af hinanden, og ser blankt ud i luften}

  \says{F1}[fortsætter uden at tabe tempo] ... ja, hvis nu for
  eksempel man skulle ordne jer efter højde, bygger vi bare et træ af
  jer og farver den ene rød.

  \says{F2} Nej nej, vi skal bare lægge dem i en stor hob, så er de automatisk
  sorterede.

  \says{F1} Eller, vi skiller dem ad og så sætter vi dem sammen igen.

  \says{F2}[bevæger sig væk fra de studerende og går ud af en tangent]
  Vi kunne også tage det ene element fra og partitionere resten. Og
  partitionere igen. Og igen!

  \says{F1} Nej, jeg har det! Vi skal bruge 10 spande...

  \says{S1} I forstår jo slet ikke hvordan verden hænger sammen! Og det smitter af
  på os.

  \says{F2 + F1} Hvordan?

  \says{S1} Algoritmer er det eneste vi mangler på bacheloren.

  \says{S2} Men nu dumper vi det igen og bliver smidt ud.

  \says{S1} Og så var alle de andre kurser jo spild af tid!

  \says{F2} Så skulle I jo bare dumpe algoritmer fra starten.

  \says{F1+F2} Det er jo et grådigt problem!

  \scene{Lys ned.}
\end{sketch}
\end{document}
