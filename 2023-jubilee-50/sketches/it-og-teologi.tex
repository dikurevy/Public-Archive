% Oprindeligt filnavn: It_og_teologi.tex
\documentclass[a4paper,11pt]{article}

\usepackage{revy}
\usepackage[utf8]{inputenc}
\usepackage[T1]{fontenc}
\usepackage[danish]{babel}

\revyname{DIKUrevyens 50 års jubilæum}
\revyyear{(2011)}
\version{1.0}
\eta{$8.5$ minutter}
\status{Færdig}

\title{IT og Teologi}
\author{Troels, Spectrum}

\begin{document}
\maketitle

\begin{roles}
  \role{P1}[Torben M] Præst
  \role{P2}[Troels] Præst
  \role{R}[Marcus] \texttt{root} (kun stemme)
  \role{FK}[Sejer] Foo / Kain / Programmør 1
  \role{BA}[Kim] Bar / Abel / Programmør 2
  \role{U3}[Lukas] Programmør 3
  \role{U4}[Johnson] Programmør 4
  \role{X}[Simon] Instruktør
\end{roles}

\begin{props}
  \prop{Bord (skal kunne skjule mappedatamaterne)}[]
  \prop{To stole}[]
  \prop{To Apple-mappedatamater}[]
  \prop{Noget til at repræsentere babelstakken}[]
  \prop{Prædikestol}[]
  \prop{Donald Knuth's \emph{The Art of Computer Programming}}[?]
  \prop{A3-papstykke 1: Nogle få tegn ML-kode: \texttt{fun foo n = ...}}
  \prop{A3-papstykke 2: Nogle få tegn ML-kode: \texttt{fun bar n = ...}}
  \prop{A3-papstykke 3: Virkelig grim børnetegning}
  \prop{A3-papstykke 4: Endnu en grim børnetegning}
  \prop{Papstykke ($\geq 1 \ \textnormal{m}^2$) formet som en bølge med teksten ``SYN'' mange gange}
  \prop{Papstykke (ca. 1 m langt) der ligner siden af en båd med teksten ``ACK''}
  \prop{Fem stykker pap med programnavne; \texttt{sed}, \texttt{awk}, \texttt{grep}, \texttt{troff} og \texttt{tar}}
  \prop{Krave og præstekjole}
  \prop{2 togaer}
  \prop{Tyl til bar}
\end{props}

\begin{sketch}
  \scene{Præsteklædt Forelæser foran tæppet.}

  \says{P1} Vi er samlet i auditoriet til fjerde forelæsning efter Trinitatis.
  \scene{Forelæser går ud til siden/prædikestolen og tager TAoCP frem}
  \says{P2} Dagens pensum kommer fra evangelisten Knuth's første bog - opstartsberetningen.

  \says{P1} På dag 0 skabte \texttt{root} MIPS.

  \says{P2} På dag 1 slackede \texttt{root}.

  \says{P1} På dag 2 skrev \texttt{root} assembleren.

  \says{P2} På dag 3 slackede \texttt{root}.

  \says{P1} På dag 6 søgte \texttt{root} om dispensation.

  \says{P2} På dag 7 tog \texttt{root} på Caféen?.

  \says{P1} På dag 9 kom \texttt{root} hjem.

  \says{P2} På dag 12 mistede \texttt{root} sin SU.

  \says{P1} På dag 13 skabte \texttt{root} instituttet.  Kun 7 dage over normeret.

  \says{P2} Instituttet var dengang tomt og øde, og \texttt{root}s hånd
  hvilede på tastaturet.

  \says{P2} \texttt{root} sagde:

  \says{R} Lad der være lys.

  \says{P2} og kørte kantinens persiennerne op, og der blev lys.

  \scene{Alt for kraftigt lys på scenen.}

  \says{P2} \texttt{root} så at lyset var ondt, og kørte dem ned igen

  \scene{Behageligt lys på scenen.}

  \says{P1} \texttt{root} skabte nu den første datalog i sit billede, og
  kaldte ham Foo, og satte ham i kantinen, for at han skulle kode.

  \scene{F træder ind på scenen, klædt som hulemand.}

  \says{P1} Så bad \texttt{root} datalogen navngive systemprogrammerne, men
  datalogen var asocial og gad ikke snakke, og hans grynten blev til
  programmernes navne.

  \scene{Programmer føres på en eller anden måde forbi Foo, som
    blot grynter og vifter dem væk.}

  \says{F} sed, awk, grep, troff, tar.

  \says{P2} \texttt{root} så at uden en gruppepartner ville datalogen aldrig
  blive social og skabte den første bifagsstuderende, og kaldte ham Bar.

  \scene{B træder ind på scenen, klædt lidt humanistisk men stadig
  gammel-bibelsk. B har en Apple-maskine med, som fanger F's interesse.}

  \says{R} "Den forbudne frugt!  En vederstyggelighed i mit åsyn!"

  \scene{Torden og lynild.}

  \says{P2} Som straf bortviste \texttt{root} datalogerne fra evig studietid;
  kun otte år skulle de have til at gennemføre en femårig uddannelse.

  \says{P1}
  Tiden gik, og Foo og Bar blev sidenhen rusvejledere og de
  fik russerne Kain og Abel.

  \scene{F og B forlader scenen.  K og A går ind på scenen, klædt som lidt mere civiliserede oldtidsmennesker.}

  \says{P1} Engang afleverede Abel en opgave skrevet i SML til
  \texttt{root}.

  \scene{A løfter papstykke med ML-kode til himmels og laver en lille dans.}

  \says{P1} Også Kain afleverede en opgave, skrevet i VB.

  \scene{K løfter børnetegning til himmels og laver en meget dum og primitiv dans.}

  \says{P1} \texttt{root} godkendte Abels aflevering...

  \scene{Lys på A. A jubler.}

  \says{P1}
  ... men Kain blev sendt til genaflevering.

  \scene{Hurtige lysglimt på K.  Tordenbrag.  K taber sin tegning på
    jorden.  K bliver først bange, dernæst vred og ryster sin næve mod
    himlen.}

  \says{R} DNUR.

  \says{P2} Til eksamen havde Abel ikke sat filrettighederne på sin
  hjemmemappe ordentligt...

  \scene{A tager det andet papstykke med ML-kode og stiller det midt på scenen.}

  \says{P2} og Kain overskrev Abel's aflevering.

  \scene{K placerer en børnetegning oven på ML-koden.}

  \says{P2} Da sagde \texttt{root}:

  \says{R til K} "Hvad er det, du har gjort?  Fra nu af når I koder
  skal I selv deallokere jeres hukommelse, typefejl skal ikke blive
  opdaget af oversætteren, og jeres programmer skal evigt korrumpere
  deres data ved kørsel!"

  \says{P2} Således skabte \texttt{root} de første C-programmører.

  \scene{K og A forlader scenen.  U1, U2, U3 og U4 kommer ind.}

  \says{P1} Mange år senere forsøgte datalogerne at knække
  \texttt{root}s løsen ved at bygge verdens største rekursive program,
  Babelstakken.  \texttt{root} blev vred, og for at forhindre dem i at
  samarbejde forbandede han hver datalog med sit eget sprog.

  \says{U1} Eureka!  Vi skal bruge Java til at konvertere vor XML til stack traces!

  \says{U2} Visual Basic til at spore IP-adresser!

  \says{U3} SML til at genere russerne!

  \says{U4} Den skal køre Gentoo!

  \scene{Alle high-fiver.}

  \says{P1} \texttt{root} så, at flere programmeringssprog ikke løste
  nogen problemer, og var nødsaget til at sende en SYN-flood.

  \scene{SYN-papstykket kommer ind på scenen, båret af en mand skjult bag det}

  \says{P1} Men selv dette slog fejl, for datalogerne reddede sig i
  land ved at bygge verdens største pakke: Noa's ACK.

  \scene{U1-U4 ``bygger'' båden ved at tage båd-papstykket, der ligger fladt på scenen, og rejse det op.}

  \says{P1} Dette var dagens pensum.

  \scene{Tæppe for, ikke lys ned.  Spot på P1 og P2.}

  \says{P2} I oversætterens, linkerens, og loaderens navn:

  \says{P2} Lad os alle kode \act{Får alle til at rejse sig og bøjer hovedet}:

  \says{P1+P2} Vi forsager Java og alle dens klasser og al dets syntaks.

  \says{P1} Og Swing.

  \says{P2} Og navngivningskonventionerne.

  \says{P1} Og downcasting!

  \says{P2} Fraværdet af sumtyper.

  \says{P1} De usunde subtyping-regler for arrays.

  \says{P2} Ja, og Oracle's licensbetingelser!

  \says{P1} Og jeg er heller ikke just begejstret for den virtuelle
  maskine.

  \scene{P1 fatter sig og fortsætter.}

  \says{P1+P2} Vi tror på Nils Andersen, den førsteuddannede,
  kursusnoternes skaber.

  \says{P1+P2}
  Vi koder i ML\\
  hans modersmål\\
  formaliseret af Milner\\
  nedskrevet af Tofte\\
  implementeret af Ken Friis\\
  forældet, dødt og begravet\\
  overtrumfet af Haskell\\
  på DIKU opstanden fra de døde\\
  nu brugt på IP\\
  skrevet på Nils Andersens transparenter\\
  hvorfra det skal komme at straffe såvel russer som gengangere\\
\\
  Vi koder i Emacs\\
  den objektivt bedste editor\\
  For kommandoerne er hurtigere, genvejstasterne bedre\\
  GUI'en pænere, og mailklient indbygget\\
  End of File
\end{sketch}
\end{document}

%%% Local Variables:
%%% mode: latex
%%% TeX-master: t
%%% End:
