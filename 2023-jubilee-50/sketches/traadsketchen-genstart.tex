\documentclass[a4paper,11pt]{article}

\usepackage{revy}
\usepackage[utf8]{inputenc}
\usepackage[T1]{fontenc}
\usepackage[danish]{babel}

\revyname{DIKUrevyens 50 års jubilæum}
\revyyear{(2009)}
\version{1.5}
\eta{$0.1$ minutter}
\status{Færdig}

\title{Trådsketchen-genstarten}
\author{Alle dem fra Trådsketchen}

\begin{document}
\maketitle

\begin{roles}
    \role{RH}[Amira] Rødhætte
    \role{U}[Niels] Store stygge ulv
    \role{G3}[Brandt] Tredje lille gris
    \role{B}[Elbo] Revyboss
    \role{X}[Niels] Instruktør
\end{roles}

\begin{sketch}

\scene{Lys op.  Der står parat til det første rigtige ekstranummer, HURRA KODEN
SEJLER, som kommer direkte efter denne reprise.  Foruden denne opsætning står
også RH, U og G3 fra Trådsketchen i akt 1 på scenen, samt G3's hus.  De står i
samme position som de endte med i sketchen.}

\scene{Rødhætte banker på den låste dør}
\says{RH} Hallåå Bedste, luk mig ind, lissom!!
\says{G3} Nej, jeg lukker dig ikke ind!
\says{U} \act{Pust} \act{Prust}!

\scene{Rødhætte banker på den låste dør}
\says{RH} Hallåå Bedste, luk mig ind, lissom!!
\says{G3} Nej, jeg lukker dig ikke ind!
\says{U} \act{Pust} \act{Prust}!

\scene{B kommer ind og slukker sketchen.  TODO: Skal der være en slukknap her?}

\scene{DIREKTE SKIFT TIL HURRA KODEN SEJLER.  IKKE NOGET LYS NED/LYS OP.}

\end{sketch}
\end{document}
