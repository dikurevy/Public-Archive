\documentclass[a4paper,11pt]{article}

\usepackage{revy}
\usepackage[utf8]{inputenc}
\usepackage[T1]{fontenc}
\usepackage[danish]{babel}


\revyname{DIKUrevy}
\revyyear{1984}
\version{0.1}
\eta{$n$ minutter}
\status{Ikke faerdig}

\title{Afslutning}
\author{Ukendt}
\melody{Kunstner: ``Originaltitel''}

\begin{document}
\maketitle

\begin{roles}
\role{S}[LOL] Sanger
\role{E0}[JT] Engel
\role{E1}[IB] Engel
\role{E2}[CH] Engel
\role{E3}[CE] Engel
\end{roles}

% Måske er det denne sang:

\begin{song}
\scene{2. dels stræber sangen (Mel. Liste Q)}

Andendelen med stækkede vinger
Svajer på den knækkede gren
Kom vi lister hen og tager os et bifag
Med bifag i lommen klarer vi os nemt
Arbejdstager, skøger og andet godtfolk
Øje for øje og tand for tand
Nu er det på tide at rydde op på DIKU
Kom så bifagsstridsmænd lad os vise hvad vi kan

Omkvæd:
Vi har patent på den evige visdom
Halleluja for  den gode idé
Vi prædiker mådehold for vores kammerater
Med Bertel på vor side uuhu

Rusken i træet og håber det bedste
Af andeldelskurser er det nok
For med bifag i orden og Mat D bestået
Marcherer vi mod frelsen i samlet flok
Bifag er livets nådegave
Skynd dig snup dit bifag men gør det kvikt
Glem ikke DIKU vil gerne gi' dig sparket
Hvis du ikke stræber nok på din første del

Omkvæd

Afslutning:
Politk er lort det mener vi
Lad de andre ordne institut og økonomi
For os gør det ingen forskel
Om andre brænder når vi går fri

\end{song}

\end{document}

