\documentclass[a4paper,11pt]{article}

\usepackage{revy}
\usepackage[utf8]{inputenc}
\usepackage[T1]{fontenc}
\usepackage[danish]{babel}


\revyname{DIKUrevy}
\revyyear{1984}
\version{0.1}
\eta{$n$ minutter}
\status{Færdig}

\title{Indledning}
\author{HVO}
\melody{Strawbs: ``Part of the union''}

\begin{document}
\maketitle

\begin{roles}
\role{S0}[KB] Sanger
\role{S1}[FSS] Sanger
\role{E}[CH] Engel
\role{D}[TL] Djævel
\role{N}[J] Naur
\role{B}[JM] Bertel
\end{roles}

\begin{sketch}

\scene{En person kommer frem foran tæppet, som trækkes fra mens han snakker.}

\says{S} Mes dames, Messieurs\\
  Bien venue - au revy\\
  \act{slår ud med hånden mod pianisten}\\
  Maestro - l'ouverture

\scene{Musikken går i gang.}

\end{sketch}

\begin{song}

\sings{S} God aften, bien venue
  til 84-revy
  mes dames, messieurs
  sig ej adjø
  vi ønsker jer ``salut''
  Big Bertel's watching you
  især her på DIKU
  men hans falkeblik
  og rammer ingen nu

\sings{S}[omkvæd] Slå gækken løs, nu starter revyen
  Vær ej nervøs, nu starter revyen
  Og kys din tøs/knøs, nu starter revyen
  Hele natten lang, hele natten lang

\scene{Mellem andet og tredje vers spilles (efter det alm. omkvæd) et omkvæd m. klaversolo, hvorunder en engel, en djævel, en Naur og en Bertel kommer ind.  Alle disse synger med på de 2 sidste omkvæd.}

\sings{S}
  Vi er alle i samme båd
  Her er ingen gode råd
  Men at gå sin vej
  Og begrave sig
  Det kan ikke nytte noget
  Luk blot mange nye ind
  Vi må dele sol og vind
  Der er ungefähr
  Hundred fyrre her
  Så 120 det er ingenting

\scene{Omkvæd.}

\sings{S}
  En professor har sagt stop
  Det er slut, han holder op
  Vil du forske - så flyt
  Vort miljø er dødt
  Flere andre følger stop
  Instituttet folkes af
  Vi har hejst det halve flag
  Men hvor de gamle faldt
  Er der ny overalt
  De vil rejse sig i dag

\scene{Omkvæd.}

\scene{Omkvæd.}

\end{song}

\end{document}

