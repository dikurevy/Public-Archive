\documentclass[a4paper,11pt]{article}

\usepackage{revy}
\usepackage[utf8]{inputenc}
\usepackage[T1]{fontenc}
\usepackage[danish]{babel}


\revyname{DIKUrevy}
\revyyear{1984}
\version{0.1}
\eta{$n$ minutter}
\status{Færdig}

\title{Kære DIKU}
\author{HVO}
\melody{Kræn Bysteds: ``Solen er så rød mor''}

\begin{document}
\maketitle

\begin{roles}
\role{P}[LOL] Pigen
\role{K0}[JT] Kor
\role{K1}[JN] Kor
\role{K2}[LJ] Kor
%role{F}[Skuespiller] Skummel fyr
\role{ES}[L] Edda
\role{S1}[JC] Studerende
%role{S2}[Skuespiller] Studerende
%role{S3}[Skuespiller] Studerende
\end{roles}

\begin{song}
\scene{En pige (P), der har læst bifag i et stykke tid, kommer til DIKU for at gøre sig færdig.  Men der er så mærkeligt\ldots}

\sings{P} Tænk her virker så mærkværdigt, der er lissom noget galt
  Alle færdes så stilfærdigt, her bli'r næsten ikke talt
  Der er slet ingen lærere at se
  Mon jeg er gået forkert, er DIKU flyttet, ak og ve
  Det er to et halvt år siden at jeg viste mig her sidst
  Og i mellemtiden har jeg taget bifag ganske vist
  Nu vil jeg gerne ha' mit hovedfag
  Med kurser og speciale, kandidat en skønne dag

\sings{P}[omkvæd] Kære DIKU, er du gået helt i stå
  Hvad tænker du dog på
  Du virker træt og grå
  Og uden sjæl
  Slået ihjel

\end{song}

\begin{sketch}

\scene{En skummelt udseende fyr kommer snigende, griber fat i armen på pigen.}

\says{F} Jeg har nogen point, er du interesseret?  Hurtigt.

\says{P} Point?? Jeg forstår ikke\ldots

\says{F} Er du kommet ind med fir'toget?  Andendelspoint!  Men de er Varme.

\says{P} Andendelspoint?  Mener du \ldots timer?

\says{F} Nej, POINT.  For at få timer skal du ha' point.  De er gode nok, jeg har dem fra en fyr, hvis fætters kone hedder det samme som en, der læser matematik og har taget bifag, det kan garanteret ikke opdages.  Jeg skal ikke ha' ret meget for dem.

\scene{F sidder E, skynder sig ud.}

\end{sketch}

\begin{song}

\sings{P} Hvad er det, jeg så en lærer, bare hun kan gøre noget
  Folk de virker meget sære, jeg må spørge hende om råd
  Undskyld, har du tid et øjeblik
  Jeg kommer lige fra bifag i kinesisk og etik
  Har du bifag, søde milde, stå dog ikke bare der
  Du skal sandelig få kurser hos vor studiekommisær
  Halløjsa, har I hørt det, kom og se
  Hun har taget bifag, var det ikke en idé

\end{song}

\begin{sketch}

\scene{3 laset udseende studerende (gerne med træben, stok og mørke briller etc.) flokkes om pigen.}

\says{S1} Kan du huske mig?  Vi var på ruskursus sammen, ikke osse?  Jeg har altid godt kunnet lide dig, og \ldots

\says{S2} Overlad et par hundrede point til en falleret evighedsstudent.

\says{S1} \ldots og jeg kan huske, du var hjemme hos min mor og få kager og kaffe og \ldots

\scene{S3 har fat i et sammenrullet papir, der stikker op ad baglommen hos pige.}

\says{P} Hov, lad være, hvad \ldots

\says{S3} Åhr, må jeg ikke nok \emph{holde} dit bifagsbevis i hånden?  Eller bare \emph{se} det?

\says{S2} \ldots eller bare 50 point måske?

\scene{Gennes væk af ES.}

\end{sketch}

\begin{song}

\sings{P}[omkvæd; gentages] De ku'
  Men vi kan
  Det er dig det gælder, brug dog din forstand

\sings{P} Nu' det SLUT, vi gider ikke mere
  Vi må stoppe før eksploderer
  Vi har sunget vores sange nu
  Der er ikke mer' totalgakgak
  Men den kommer lige et par gange endnu
  Og så er der bare at sige

\scene{Omkvæd.}

\scene{Så er der guitarsolo mens alle i revygruppen kommer ind, tager hinanden i hænderne og bukker og sådan noget.  Derefter:}

\sings{P}[omkvæd] Forbi
  Forbi
  Det er tid til DIKUsommerfesteri
  Vi kan
  Ej mer'
  Så revyen den er SLUT

\scene{Stopper brat.}

\end{song}

\end{document}

