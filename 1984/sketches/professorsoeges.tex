\documentclass[a4paper,11pt]{article}

\usepackage{revy}
\usepackage[utf8]{inputenc}
\usepackage[T1]{fontenc}
\usepackage[danish]{babel}


\revyname{DIKUrevy}
\revyyear{1984}
% HUSK AT OPDATERE VERSIONSNUMMER
\version{0.1}
\eta{$n$ minutter}
\status{Færdig}

\title{Professor Søges}
\author{HVO, Jens}

\begin{document}
\maketitle

\begin{roles}
\role{O}[HVO] Opråber
%role{P1}[Skuespiller] Person 1
%role{P2}[Skuespiller] Person 2
%role{M0}[Skuespiller] Musiker 1
%role{M1}[Skuespiller] Musiker 2
\role{N}[J] Naur
\role{B}[JM] Bertel
\role{D}[TL] Djævlen
\role{PBH}[LJ] Per Brinch Hansen
\end{roles}

\begin{props}
\prop{PDP11/20}[Person, der skaffer]
\prop{Stor (meget stor) pil}
\end{props}

\begin{sketch}

\scene{Opråber (O) kommer ind som sandwich ude blandt publikum.}

\says{O} Professor søges, professor søges!

\scene{O kigger sig (stadig råbende) om, generer et par publikummer etc.  Kommer efterhånden op til scenen og trækker selv tæppet fra i begge sider.  Råber lidt mere, kigger sig om etc.  2 personer kommer ind bærende PDP11/20 imellem sig.  Henvender sig til opråberen.}

\says{P1} Her!
\says{O} Hvad skal jeg med den antikvitet?
\says{P2} Ja, sagde du ikke lige selv: Processor søges?
\says{O} Idioter!  Ud!
\scene{O peger.  Råber lidt mere.  2 af musikerne bærer en stor (meget stor) pil hen over scenen.}
\says{O} Hvad er det, I har der?
\says{M0} En vektor.
\says{O} En lektor?  Det dur ikke jeg skal bruge en professor.
\scene{M* bærer pilen ud på gangen.  Naur ind.}
\says{N} Undskyld, jeg hører, De søger en professor, det er netop så heldigt at jeg er professor og søger mig et nyt job.
\says{O}[Kigger lidt op og ned af N] Kan De Regne?
\says{N} Hvabehar?  Regne?
\says{O} Ja, kan de regne?  Jeg søger en professor til meteorologisk institut.
\says{N} Regne? Jamen - det har man da maskiner til.  Jeg husker tydeligt--
\scene{Afbrydes af de to musikere der kommer tilbage, bærende PDP11/20 med pilen ovenpå, og stille den på et bord (uden at sige noget).  Musikerne går på plads.}
\says{N} Åh, lige hvad jeg altid har ønsket.
\scene{N piller lidt ved panelet.  I mellemtiden har O taget sandwichskiltene af og begynder at hoppe tåbeligt om bordet.}
\says{N} Hvad i alverden laver De dog?
\says{O} Regnedans, hvad ellers?
\scene{Noget tromme tam tam.  N og O hopper et par gange rundt om bordet.  En transparent med simple regnestykker vises og trommerne holder op.}
\says{N} Milde Gier!  Det virker!
\says{O} Selvfølgelig.  Vi betjener os af de mest moderne videnskabelige metoder.  Jeg ved godt, det lyder helt hen i vejret, men det er beregnet...
\scene{Afbrydes af Bertel (B), der er kommet ind.}
\says{B} Må jeg se Deres bevilling!
\says{N+O}[i kor] Bevilling?
\says{B} Ja, bevilling! \act{til O} Har De bevilling til en professor?
\says{N}[til publikum] Gid fanden havde ham Bertel!
\scene{Fanden (D) kommer ind, tager Bertel (B) i kraven og slæber ham ud.}
\says{O} Flot klaret!
\says{N} Nåh, jah, mjoh.  \emph{Har} De bevilling?
\says{O} Selvfølgelig.  \act{hiver et stykke papir op fra lommen, folder det omstændigt ud og læser op}  Bevilling på et styk professor.  Underskrevet: Rektor Tor A. Bak.
\says{N} Tor A. Bak?  Jamen - den er jo forældet mand!  Altså, hør nu her:  For det første er det 100 år og en sommer siden Tor A. Bak var rektor, for det andet er det Bertel, der bestemmer den slags nu.  Din bevilling dur slet ikke!
\says{O} Bertel!  Åh gid fanden havde ham!
\scene{F kommer ind med B.}
\says{D} Jeg vil ikke have ham!
\scene{B tager N og O ved hver deres øre.}
\says{B} Nå, I troede lige I var sluppet af med mig, hvad?  Skidt knægte.  Næh - såden noget småskravl kan jeg ikke bruge til mit regnecenter.
\says{O} Av!  Slip mit øre.
\says{N} Ha!  Selv med hjælp ovenfra kan du jo ikke få andre.
\says{B}[slipper ørerne på N og O] Jo netop!  Jeg har henvendt mig i God's own country - regnedansens fædreland - og har fået tilbudt en kapacitet, en af fagets tunge drenge.  Han er vist god for en GIGA FLOP i sekundet.
\scene{D kommer ind med PBH i kraven.}
\says{D} Ham her vil jeg heller ikke ha' - han forsøgte at tvinge mig til at købe 6 IBM-PC'er.
\says{B} Åh - det var godt - velkommen til.  Nu skal jeg lige af med de to her, så må jeg høre, hvordan du vil ha' indrettet dit ny center.
\scene{B genner O og N ud mens PBH synker prustende men tilfreds ned i skrivebordsstol på hjul.}
\says{PBH} Jamen, hvad har jeg egentligt?  Nu har jeg prøvet at snakke med dem enkeltvis og det er blev jeg bare forvirret af.  Der er jo lige så mange meninger som der er lærere.
\says{B} Jeg har det, vi tager på overraskelsesvisit i cirkus DIKUTSKI.

\end{sketch}

\end{document}
