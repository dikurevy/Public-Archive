\documentclass[a4paper,11pt]{article}

\usepackage{revy}
\usepackage[utf8]{inputenc}
\usepackage[T1]{fontenc}
\usepackage[danish]{babel}


\revyname{DIKUrevy}
\revyyear{2013}
\version{1.0}
\eta{$3$ minutter}
\status{Færdig}

\title{Bobbelsortering}
\author{Troels, Phillip, Jenny, Nana, Brainfuck, Andreas, Simon}
\melody{Liva Weel: ``Lad det boble''}

\begin{document}
\maketitle

\begin{roles}
  \role{I}[Troels] Instruktør
  \role{P}[Sebastian] Pawel Winter
  \role{S}[Jenny] Kæk studerende
  \role{N}[Daniel] Ninja-kridt
\end{roles}

\begin{props}
  \prop{7 plader med tallene $0--6$, der skal kunne monteres på
    scene-elementerne og flyttes rundt under sangen.}[]
\end{props}

Meget af denne sang bygger på scenespil hvor en bobbelsortering
håndkøres.  Forestil jer at K står og illustrerer algoritmen, så giver
teksten mere mening.

\begin{sketch}
  \says{Pawel} Sig mig engang, sig mig engang, Kan De nu virkelig
  bevi-i-ise det, hva'?  De ser mig ikke videre velbegavet ud

  \says{S} Hvad man ikke har i hovedet må man have på tavlen!  Tavlen
  har jeg her, kridtet har jeg her, men det aller-vigtigste...

  \says{Pawel} Og hvad er det?

  \says{S} \ldots har jeg her.

  \says{Pawel} Aha! Algoritmen!  Og kan De så forklare den?

  \says{S} Først l vi have en tabel.  Vi fylder den med tal.
  De har det godt, i deres små kasser, lunt og godt.

  \scene{Tallene er \texttt{3 5 0 2 6 4 1}}.

  \says{S} Så sætter vi vor finger ved første kasse.  Hvis dennes
  indhold er større end sidemandens, bytter vi om på dem.\act{Der bobles}  Ha ha, se
  hvor de bobler!
\end{sketch}

\scene{Musik begynder.}

\begin{song}
  \sings{S}
  Ret din peger mod dit datasæt
  følg nu med, lad vær' at blive træt
  algoritmen listen traverserer
  data vi sorterer ret

  Tag to tal og se på deres værdi
  er det mindre, bytter vi fordi
  facit søges, korrekthed øges
  så'n er datalogi

  Lad dem boble, boble, boble,\act{Bobles}
  bytte om på de her to\act{Bobles}
  blot programmet terminerer,\act{Bobles}
  kan jeg sagtens her bestå\act{Bobles}

  Når man først sit emne trækker
  og hukommelsen den strejker
  håber gavmildheden rækker
  bare manglerne er få

  Jeg hehøver aldrig lære fler'
  hvis jeg kan bevise denne her
  andre algoritmer de er svære
  siger vores lærebog

  Når engang jeg bobler helt på plads
  er al tiden gået er mit sats
  min metode er langsom kod' (for)
  Pawels spørgsmål gør nas

  Lad dem boble, boble, boble,\act{Bobles}
  bytte om på de her to\act{Bobles}
  vi kan se det terminerer\act{Bobles}
  ergo pensum jeg forstår\act{Bobles}

  Når I nu om lidt voterer\act{Bobles}
  skal I husk' hvad jeg præsterer\act{Bobles}
  algoritmen terminerer
  så lad mig nu bestå!
\end{song}

\end{document}

