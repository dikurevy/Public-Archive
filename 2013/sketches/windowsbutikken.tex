\documentclass[a4paper,11pt]{article}
\usepackage{revy}
\usepackage[utf8]{inputenc}
\usepackage[T1]{fontenc}
\usepackage[danish]{babel}
\revyname{DIKUrevy}\revyyear{2013}
% HUSK AT OPDATERE VERSIONSNUMMER
\version{1.0}\eta{3 minutter}\status{Færdig}
\title{Windowsbutikken}
\author{Troels, Phillip}
\begin{document}
\maketitle
\begin{roles}
 \role{I}[KØK] Instruktør
  \role{R}[Ejnar] Repræsentant for DIKU-revyen
  \role{S}[Mikkel] Sælger
  \role{K}[Troels] Kunde
\end{roles}
\begin{props}
  \prop{Roulade.}
  \prop{Butiksdesk til Windowsbutikken.}
  \prop{Generel dekoration til Windowsbutikken (logoer, kasser, osv)}
\end{props}
\begin{sketch}
\scene{Der kommer en velklædt repræsentant fra DIKUrevyen ind på scenen.}

\says{R} Vi i DIKUrevyen bliver ofte kritiseret for at høste billige point, ved
ukritisk at svine Microsoft til, samtidig med at vi skamroser Linux. Vor
forfattere har derfor lagt sig ekstra i selen, for at vende denne stereotyp
om, og skrive en sketch der er mere venlig stemt overfor Windows, og måske
lave lidt sjov med de mangler der nu engang kan være i Linux.

\says{R} Vi håber I vil nyde resultatet af vores arbejde.

\scene{R går ud. Tæppe fra.}

\scene{Windows-butik. Der står en sælger bag disken og en kunde foran
  disken. De er begge enormt glade. De har cykelhjelme, albue- og
  knæ-beskyttere på, og bukserne trukket alt for højt op. De ser ud som om at
  de har lidt for mange X-kromosomer. De taler med slørede og dybe stemmer og
  har underbid (ligesom Morten og Peter). Kunden har en roulade under armen,
  som han vifter med, når han snakker.}

\says{K} Det kan godt være, at man ikke skal genstarte det der Ubuntu så meget,
men det er kun fordi at efter jeg opdaterede det, så vil det slet ikke
starte igen!

\says{S:} Du er kommet til det rigtige sted, kammerat!, hvis du vil have fikset
din datamat! \act{Kigger sultent på rouladen.}

\says{K} \act{Går over til disken.} Ja, tilbage til Windows for
mig. Jeg har ikke noget imod at betale for kvalitet, når blot
supporten er i orden. Min tid er for værdifuld, til problemer som jeg
kan betale mig fra.

\says{S:} Hvad skal du så bruge?

\says{K} Hvad kan jeg få for rouladen?

\says{S:} Jaa.. Det kan vi vel finde ud af bagefter. Hvad vil du have?
\act{Kigger stadig på rouladen}

\says{K} Har du det der Visual Studio? Jeg kan nemlig godt lide \act{tæller stavelser
på fingrene}: ud-vik-lings-værk-tøj'r der kan håndtere de store kodebaser,
som mine en-ter-pri-se-kund-er har oparbejdet gennem mange år.

\says{S:} Ja, det er faktisk imponerende at Microsoft har så god
bagudkompatibilitet, at man kan opgradere sine maskiner uden at være
bekymret for at ens software holder op med at virke. Den hello-world jeg
oversatte her til morges på min Ubunut, kan ikke køre længere.

\says{K} Ja, du skal nok genoversætte din kerne.

\scene{K og S griner lystigt}

\says{S:} Jeg burde køre Gentoo.

\scene{K og S griner videre}

\says{K} Windooooows!!! \act{Slår armene op i vejret.}

\says{S:} Windooooows!!!

\scene{K og S danser rundt i butikken. S gafler rouladen og tager en
  bid, og har herefter flødeskum i hele fjæset.}

\says{K} Hov, lad os prøve Linux-dansen i stedet for!

\says{S:} segfault! \act{Falder om.}

\says{K} Kernel Panic! \act{Falder om}

\scene{K og S ligger begge på scenen. S tager en bid af rouladen. Tæppe
  for, lyd ud.}

\end{sketch}
\end{document}
