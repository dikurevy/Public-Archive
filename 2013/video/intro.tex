\documentclass[a4paper,11pt]{article}

\usepackage{revy}
\usepackage[utf8]{inputenc}
\usepackage[T1]{fontenc}
\usepackage[danish]{babel}
\usepackage{hyperref}

\revyname{DIKUrevy}
\revyyear{2013}
\version{1.01}
\eta{$8$ minutter--$8$ minutter $45$ sekunder}
\status{I gang med optagelser}

\title{Revyintro 2013: Hacky Pointer}
\author{Introgruppen et al}

\begin{document}
\maketitle

\begin{sketch}

Manuskriptet ligger indtil videre på
\url{http://harlemklub.dk/wiki/doku.php?id=nyeintroer:harrypotter}; bruger:
harlem, løsen: horn

Der er ikke blevet lavet råklip endnu, men alle video- og lydfiler er
tilgængelige på \url{http://hongabar.org/~niels/pw/hackypointer-media/}
(bruger og løsen samme som på wikien).

Nogle af de mest interessante klip er
\begin{itemize}
\item \url{http://hongabar.org/~niels/pw/hackypointer-media/06-03-2013/video/00010.MTS}
\item \url{http://hongabar.org/~niels/pw/hackypointer-media/13-03-2013/video/00010.MTS}
\item \url{http://hongabar.org/~niels/pw/hackypointer-media/13-03-2013/video/00014.MTS}
\item \url{http://hongabar.org/~niels/pw/hackypointer-media/13-03-2013/video/00015.MTS}
\item \url{http://hongabar.org/~niels/pw/hackypointer-media/14-03-2013/video/00021.MTS}
\item \url{http://hongabar.org/~niels/pw/hackypointer-media/14-03-2013/video/00028.MTS}
\item \url{http://hongabar.org/~niels/pw/hackypointer-media/21-03-2013/video/00033.MTS}
\item \url{http://hongabar.org/~niels/pw/hackypointer-media/21-03-2013/video/00038.MTS}
\item \url{http://hongabar.org/~niels/pw/hackypointer-media/21-03-2013/video/00044.MTS}
\end{itemize}

\end{sketch}
\end{document}
